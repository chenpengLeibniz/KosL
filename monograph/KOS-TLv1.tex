\chapter{知行逻辑}\footnote{本章的主要内容参考“KOS-TL (Knowledge Operation System Type Logic)”\cite{Chenpeng2026}}

To Know is To Act.

知道即行动。


知识表征与推理是逻辑的一个重要的应用领域,然而伴随着大规模数据整合与复杂决策系统的普及,知识表征与推理的研究对象正从静态知识库逐步转向持续演化的知识操作体系中。在面向持续研发的知识操作中概念建模或本体一致性验证不再是核心的关注点,其核心挑战转换为如何在事件驱动、状态演化与强工程约束的环境中,对知识进行可执行、可追溯且可验证的操作。



\section{面向知识操作的逻辑新需求}

面对知识操作领域的新需求,现有主流逻辑框架(尤其是以描述逻辑(Description Logic, DL)及其语义网实现(如 OWL)为代表的形式体系)在理论假设与语义结构上,与上述需求之间存在根本性不匹配。具体体现在如下几个方面。

\begin{enumerate}[label=(\arabic*)]

    \item 静态语义与动态操作之间的张力

    描述逻辑以静态模型论语义为基础,其核心推理问题围绕概念可满足性、概念包含与实例判定展开。这一范式默认知识是对“世界可能状态”的描述,而非对“系统运行状态”的刻画。相比之下,知识操作中的核心对象是事件、操作与状态转移,其基本问题不再是某一断言是否在某个模型中为真,而是某一操作是否可被合法执行,以及其执行后系统状态如何演化。

    描述逻辑的模型论语义以静态解释结构为核心,其基本目标是刻画“世界在逻辑上可能是什么样”。概念被解释为个体域上的集合,角色被解释为二元关系,推理问题主要围绕可满足性、概念包含与实例判定展开。这种语义结构天然适合于分类学、本体工程和术语推理。

    然而,在知识操作领域所面对的应用场景中,知识并非一个静态集合,而是一个随时间持续演化的状态系统。核心问题不再是“某一断言是否在某个模型中为真”,而是“某个事件是否已经发生、某个状态是否已经被更新,以及这些变化将触发哪些新的事实”。描述逻辑并未将事件与状态转移视为一等逻辑对象,其对动态过程的支持只能通过外部机制或重述化(reification)方式间接实现,这在工程上代价高昂且语义不透明。

    \item 开放世界假设与操作语义的根本冲突

    描述逻辑所坚持的开放世界假设(Open World Assumption)与知识操作中普遍采用的封闭或半封闭世界语义存在根本冲突。在工程实践中,缺失信息往往被视为异常状态或操作失败的依据,而非逻辑上的“未知”。

    描述逻辑坚持开放世界假设(Open World Assumption, OWA),即未知并不等价于假。这一假设在语义网与开放知识环境中具有合理性,但在知识操作中却往往成为障碍。在企业治理、风险控制、合规审计等场景中,“缺失记录”本身即构成一种负面信息或异常状态。

    知识操作系统语义更接近封闭世界假设:未出现的事实被视为未发生的事件,未满足的约束被视为系统错误。这种以封闭世界与可执行约束为核心的语义取向,使得描述逻辑的模型存在性推理难以直接服务于实际系统运行。

    \item 概念语义与名义类型语义的差异

    描述逻辑中的概念语义是外延性的,其成员资格可通过推理动态决定;而在实际知识操作中,类型更多承担的是名义性与约束性的角色。对象是否属于某一类型,并非通过逻辑蕴涵推导,而是系统在数据摄取与操作阶段必须满足的前提条件,其直接决定了操作是否合法、流程是否可继续,而类型错误表现为系统不可执行状态,而非单纯的推理失败。这种类型语义更接近于程序语言与操作系统中的类型系统,而非传统本体逻辑。

    \item 推理目标的根本转移

    最后,描述逻辑的推理目标主要是逻辑蕴涵的证明,而知识操作中的“推理”则更接近于规则驱动的事实物化(materialization),即在给定当前数据状态下,哪些新事实应当被立即生成、存储并参与后续计算\footnote{这种“推理”更接近于规则驱动的这种推理方式与 Datalog、触发器以及操作语义更为接近,而非以证明论或模型论为中心的经典逻辑推理。}。推理的结果并非仅用于回答查询,而是直接改变系统的可观察状态,并对后续操作产生约束。这种推理模式在经典逻辑框架中缺乏直接的形式化刻画。

\end{enumerate}

在知识操作领域,问题并不在于逻辑系统是否足够强大,而在于其是否能够原生支持事件、时间、状态变化与可执行规则。传统描述逻辑在静态知识表示方面依然具有不可替代的价值,但其逻辑假设与语义结构并不适合直接作为知识操作系统的内核。

上述差异揭示了一个关键的理论空白:传统的逻辑体系尚未为“知识作为可操作对象”的系统提供一个统一的形式基础。这直接催生了对一种新逻辑系统的需求:该系统应当以事件与状态转移为核心对象,采用类型化与操作语义驱动的推理方式,内建类型化约束以刻画操作合法性,支持规则驱动的状态演化,并在可靠性与可终止性之间取得工程可行的平衡。

针对这一理论缺口,我们提出并研究一种新的形式逻辑体系———知行逻辑(Knowledge Operation System Type Logic,简称KOS-TL)。KOS-TL 旨在以直觉类型论为基础,引入事件化与操作语义,使逻辑系统能够原生刻画知识的理解、操作与状态更新过程,从而为知识操作系统提供一个可验证、可执行且可扩展的逻辑内核。

\section{KOS-TL 的分层形式化定义}

为了将逻辑严谨性与操作表达能力结合,知行逻辑(KOS-TL)被定义为一个\emph{分层形式化系统}。分别包括三层形式化定义:\emph{Core (核心)}、\emph{Kernel (内核)}与\emph{Runtime(运行时)},每一层在逻辑角色与语义承诺上均有所区分(如表\ref{tab:kos-layers} 所示)。

\begin{table}[htbp]
\centering
\caption{KOS-TL 层次结构概述}
\label{tab:kos-layers}
\begin{tabular}{llp{4cm}p{3cm}l}
\toprule
\textbf{层次} & \textbf{形式化名称} & \textbf{核心职责} & \textbf{逻辑对象} & \textbf{判定性} \\
\midrule
L0  Core & 静态真理层 (Logic) & 定义“什么是合法的”。基于依存类型论(ITT)确立类型构造与约束。 & 类型 $T$、证明项 $p$、命题 $P$ & 强可判定 \\
L1  Kernel & 动态转换层 (Dynamics) & 定义“如何改变”。引入小步操作语义,处理事件对状态的转换。 & 状态 $\sigma$、事件 $e$、转移 $\to$ & 局部可判定 \\
L2  Runtime & 环境演化层 (System) & 定义“如何运行”。处理外部 I/O、时间线挂载、非确定性输入。 & 队列 $Q$、外部源 $\mathcal{E}_{\mathrm{env}}$ & 半判定 \\
\bottomrule
\end{tabular}
\end{table}

知行逻辑的分层原则如下:
\begin{itemize}
    \item 逻辑合法性仅由核心层决定;
    \item 操作正确性由内核层强制;
    \item 系统演化在运行时层实现。
\end{itemize}

这种分层设计保证即便在开放、非确定性的系统演化中,核心逻辑的正确性仍被保持。

\subsection{L0: Core(核心层)—— 静态逻辑论域}

从知识操作系统的全局视野审视,KOS-TL 的核心层超越了传统意义上的数据库模式(Schema)范畴。它作为整个系统的逻辑宪法,承担着将模糊的现实世界业务规则精准转化为严密数学约束的核心使命。其设计目标在于:在处理大规模动态数据流的同时,通过元物理规范确保每一比特的操作均具备逻辑完备性与可溯源性。

\subsubsection{核心层需求分析与逻辑构造}

在构造性表达需求方面,传统系统往往仅侧重于数据的静态存储,而在高安全性与高可信场景下,系统必须过渡到对“知识”的存储。基于直觉主义类型论(ITT),每一条知识项被定义为一个依存对 $\Sigma(d:D).P(d)$,其中数据 $d$ 与证明其满足业务本体的凭证 $P(d)$ 强耦合。这种设计从源头上消除了“无根数据”,确保所有进入内核的知识均经过构造性验证。

物理与逻辑的深度对齐是另一项关键需求。通过依存类型(Dependent Types)的表达力,系统能够将工业或金融领域的物理定律与合规性约束内生化,而非依赖外部的临时逻辑判断。这种“非法状态不可表示”(Make Illegal States Unrepresentable)的范式,确保任何违反公理的操作尝试在类型检查阶段即告失败,从而维持系统的稳态。

此外,为应对高频合规要求,系统引入了计算反射性(Computational Reflexivity)。通过要求核心层描述自身的规约规则,系统在执行每一小步逻辑演化($\beta, \iota, \delta$ 规约)时均能自动合成等价性证明 $\text{Id}(t, t')$。这种全路径自动化审计将传统的“事后追查”提升为“运行时的实时形式化验证”。最后,通过定义精化算子($\textsf{elab}$)模板,核心层为运行时(Runtime)信号提供了语义提升的基准,建立了从物理比特(Bits)到逻辑真理(Truth)的唯一映射路径。

\subsubsection{总体架构描述}

核心层被架构为一个基于依存类型论的强校验微内核,主要由以下三个功能模块驱动:其一,类型构造器与本体管理器作为系统的“立法者”,负责将业务领域本体转化为类型系统中的目(Sorts)与依存类型结构,明确合法对象的边界。其二,规约引擎作为系统的“推理机”,负责处理逻辑项的细粒度演化,通过执行函数应用与结构拆解计算知识演化后的逻辑稳态。其三,类型检查器作为系统的“守门人”,执行双向类型检查(Bidirectional Type Checking),确保所有进入内核层(Kernel Layer)的操作均满足预先验证的正确性(Correct-by-construction)。

\subsubsection{关键设计决策}

核心层的技术选型与架构决策反映了对逻辑严密性与工程可行性的权衡。

\begin{enumerate}[label=(\arabic*)]

    \item 依存类型论(MLTT)对一阶逻辑(FOL)的替代

    传统知识库基于 FOL 或描述逻辑(DL),导致逻辑断言与具体数据脱节。核心层选择依存类型论,利用 $\Sigma$ 类型实现数据与约束的原子化封装。这一决策解决了知识操作中“证据缺失”的顽疾,从架构级别强制执行物理约束。

    \item 计算反射性的引入与内生审计

    针对传统审计滞后且易被篡改的风险,核心层将规约规则建模为逻辑处理对象。每当状态变更,系统自动合成同一性证明(Identity Proof)。这种设计使审计行为转化为自动化的类型检查过程,只要证明链完整,系统行为即具绝对合规性。

    \item 双轴世界体系与命题收缩(Prop-Shrinking)

    为解决形式化证明带来的计算开销问题,核心层设计了多级世界($\mathcal{U}_i$)用于复杂建模,并引入专门的证明空间(Prop)。基于证明无关性(Proof Irrelevance)原则,系统在完成严苛校验后通过类型擦除(Erasure)技术收缩证明细节。这一决策实现了逻辑深度与工程效能的平衡,支持大规模实时知识流的高效处理。

    \item 从指称语义向操作语义的范式转移

    传统逻辑侧重静态真值,而 KOS-TL 核心层通过定义逻辑项的小步规约(Small-step reduction)规则,将知识操作定义为规约过程。这一决策确保了从核心层到内核层的转换是无损且确定的,逻辑推演步与物理计算步实现了高度同构。
\end{enumerate}

核心层的决策与传统的知识库架构的对比如表\ref{tab:KosCorecomparison}所示。

\begin{table}[h]
\caption{传统架构与KOS-TL Core 层决策比较}
\label{tab:KosCorecomparison}
\centering
\begin{tabular}{|l|l|l|}
\hline
\textbf{设计维度} & \textbf{传统架构决策} & \textbf{KOS-TL Core 层决策} \\
\hline
知识载体 & 数据库记录 + 外部校验 & 代码带有依存证明的逻辑项 (Terms) \\
\hline
约束触发 & 运行时拦截 / 业务代码 if-else & 类型检查 (Type Checking) \\
\hline
演化动力 & 数据库事务 (Transactions) & 逻辑规约 (Reduction) \\
\hline
信任根源 & 系统管理员权限 / 日志记录 & 不可篡改的数学证明链 \\
\hline
\end{tabular}
\end{table}

\subsection{L1: Kernel(内核层)—— 操作语义论域}

在 KOS-TL 系统的分层架构中,内核层承担着将核心层定义的静态真理转化为动态演化动力的中枢职能。若将核心层比作“宪法”,则内核层即为“行政中枢”与“动力引擎”。其核心目标在于通过形式化规约机制,确保知识系统在处理高频业务流与时间流时,维持逻辑上的确定性与演化上的连续性。

\subsubsection{内核层需求建模:动态演化与状态确定性}

针对复杂知识操作系统,内核层的设计首要解决知识状态的“坍缩”与“生成”问题。

首先是因果追溯需求,系统要求每一项知识状态的变更必须具备显式的因果关联,即所有变更必须溯源至特定的精化事件(Event)。其次是状态一致性需求,尤其在分布式或并发环境下,必须保证状态迁移的原子性,从逻辑层面根除冲突状态的产生。最后是实时性与进度保证(Progress Guarantee),系统必须确保在接收到合法输入后,能够逻辑确定地演化至下一个稳步,避免因非确定性导致的未定义行为或逻辑死循环。

\subsubsection{设计方法论:小步操作语义与状态机模型}

内核层拒绝采用传统的黑盒式批处理,而是建立在形式化小步操作语义(Small-step Operational Semantics)的基础之上。

其方法论核心在于状态三元组模型。内核将系统抽象为 $\sigma = \langle \mathcal{K}, \mathcal{TS}, \mathcal{P} \rangle$,其中 $\mathcal{K}$ 代表当前经校验的知识真理集,$\mathcal{TS}$ 为逻辑时钟与物理锚点的耦合,$\mathcal{P}$ 则为已精化但尚未执行的事件队列。内核层本质上是以“事件”为核心,基于事件的本体论地位\cite{Davidson1967},事件被定义为状态转移算子(Events as Transitions):每一类事件均对应一个显式的转换算子,该算子通过同步调用核心层的判定结果,驱动旧状态 $\sigma$ 向新状态 $\sigma'$ 的单调演化。

\subsubsection{总体架构描述}

作为介于逻辑基座(Core)与物理环境(Runtime)之间的“逻辑路由器”,内核层由三个核心模块协同工作:

\begin{itemize}
\item 事件队列管理器(Event Queue Manager):负责接收来自运行层精化后的事件包 $\langle e, p \rangle$,并执行严格的定序(Sequencing)与依赖冲突检测。

\item 演化调度器(Evolution Scheduler):驱动系统的核心演化循环。该模块采用“Peek-Verify-Reduce-Confirm”流程,即在消费事件前调用核心层验证前置条件,执行规约操作,并校验后置证明。其操作对象始终为逻辑项(Terms)而非底层物理比特。

\item 状态镜像库(State Mirror):维护逻辑上的最新“真值视图”,为运行层的状态查询及核心层的语境验证提供一致性上下文。
\end{itemize}

\subsubsection{关键设计决策}

\begin{enumerate}[label=(\arabic*)]

    \item 强顺序提交与异步精化的解耦

    针对物理信号高频产生与逻辑校验(深层证明)耗时之间的矛盾,内核层采取“异步精化、顺序提交”的决策。运行层可并行执行信号精化,但内核层坚持顺序提交(Sequential Commit)。这一决策确保了因果链条的唯一性,使知识演化轨迹呈现为确定性的线性路径,规避了复杂的逻辑分支回溯风险。

    \item 闭环演化:前置验证与后置证据合成

    为确保操作后的系统状态持续符合本体定义,内核层确立了“前置验证、后置合成”的闭环机制。在执行操作前,内核强制验证核心层提供的 $Pre(e)$;操作完成后,自动触发核心层合成 $Post(e)$ 的新证明。该决策保证了系统始终在“已证明为真的状态”之间迁移,消除了逻辑真空期。

    \item 逻辑时间与物理时间的松耦合机制

    针对分布式网络中物理时间先后与因果逻辑不一致的痛点,内核层维护独立的逻辑序。物理时间戳仅作为证据锚点,系统仅在物化阶段(Runtime)进行同步。此决策利用逻辑证明中的时间证据进行重排序,从根源上解决了“幻觉时序”问题,保障了\textbf{因果序(Causality Order)}的绝对保真。

    \item 确定性规约(Deterministic Reduction)

    内核层严禁非确定性(Non-deterministic)选择。若某一事件在规约过程中指向多个可能的演化分支,该事件将在核心层被判定为类型错误。这一决策极大增强了系统的可预测性:在相同的初始状态与事件序列下,内核层将产生数学上唯一的、可复现的知识视图。

\end{enumerate}

\subsection{L2: Runtime(运行时层)—— 系统演化论域}

在 KOS-TL(知行逻辑)的体系结构中,运行层被定义为逻辑世界与非确定性物理世界之间的关键“锚点”。若将核心层视为系统的“宪法”,内核层视为“行政中枢”,则运行层即为系统的“感官、肢体与物理载体”。其核心使命在于解决形式逻辑在复杂工程环境落地过程中的“最后一公里”问题,实现逻辑语义与底层物理信号的无缝衔接。

\subsubsection{需求建模:物理保真与环境精化}

运行层的设计旨在应对物理世界与逻辑抽象之间的根本张力。

首先是信号提升需求,物理世界产生的原始位流(Raw Bits)缺乏内在语义,运行层需将其“提升”为具备逻辑内涵的事件对象。其次是资源映射需求,逻辑层输出的抽象状态更新最终必须精准落实于磁盘比特、内存条目或硬件控制器的电平状态。最后是实时性与并发需求,在处理高频传感器信号时,系统必须在确保低延迟摄入的同时,不破坏内核层维持的因果顺序一致性。

\subsubsection{设计方法论:双向映射中的精化与物化}

运行层的方法论核心在于建立逻辑语义与物理资源之间的双向精化关系。

\begin{enumerate}[label=(\arabic*)]
    \item 精化(Refinement/Elaboration)方法论

    系统引入 $\textsf{elab}$ 算子实现信号的提升。与传统的语法解析(Parsing)不同,$\textsf{elab}$ 是一个证明构造过程:它对照核心层的本体模板,为原始信号 $s$ 寻找逻辑证据 $p$,从而将其转换为符合内核接口的带证明事件 $\langle e, p \rangle$。这一过程确保了进入系统的所有输入均具备形式化的合法性依据。

    \item 物化(Materialization)方法论

    通过 $\mathcal{M}$ 算子,运行层负责将内核层中抽象的知识状态 $\mathcal{K}$ “降解”为具体的物理形态。例如,将一个逻辑上的“转账成功”断言物化为数据库中的 ACID 事务或区块链上的交易条目。物化机制确保了逻辑结论在物理世界中的保真执行。
\end{enumerate}

\subsubsection{总体架构描述}

作为负责“跨境交互”的枢纽,运行层由以下关键组件构成:

\begin{itemize}
\item 精化引擎 (Elaborator):连接物理 I/O 设备,负责监听外部中断及传感器数据流,并执行双向精化,将非确定性信号锚定为确定的逻辑事件。

\item 物理存储管理器 (Physical Storage Manager):屏蔽底层介质差异,管理数据库及内存映射,确保物化操作的原子性与持久性。

\item 调度中继 (Scheduler Relay):作为内核顺序演变的缓冲器,负责多线程信号的并发摄入与保序入队。
\end{itemize}

\subsubsection{关键设计决策}

\begin{enumerate}[label=(\arabic*)]
    \item 基于“精化”的输入过滤决策

    针对传统系统直接读取变量易受环境干扰(脏数据)的问题,运行层规定所有输入必须通过 $\textsf{elab}$ 算子。若信号无法在核心层构造出合法证明 $p$,则判定为无效输入并立即丢弃。该决策建立了逻辑防火墙,保证了内核层不被非预期信号污染。

    \item 原子化提交栅栏 (Atomic Commit Fence)

    为解决“知行不一”(即逻辑更新成功但物理写入失败)的问题,运行层引入了类似两阶段提交的栅栏机制。只有当物理层返回写入确认(ACK)后,逻辑时钟 $\mathcal{TS}$ 才正式向前推进。这一决策实现了物理存储与逻辑真理的精确同步。

    \item 资源抽象与多后端插拔支持

    为适应从嵌入式设备到云端集群的差异化环境,运行层将物化算子 $\mathcal{M}$ 设计为可插拔的后端。该决策实现了逻辑便携性:核心层与内核层的逻辑公理保持不变,通过更换运行层后端即可实现跨平台的语义一致性运行。

    \item 确定性轨迹重放与灾后自愈

    基于物理环境的不可靠性,运行层完整记录所有精化事件的原始轨迹。利用内核层确定性规约的特性,系统在故障后可通过重放(Replay)事件流,在逻辑上重新推演并修复物理配置。这一决策为系统提供了极强的逻辑健壮性与自愈能力。
\end{enumerate}

整个KOS-TL系统的层次结构及稳定性保障如表\ref{tab:KOSHiera}所示。

\begin{table}[htbp]
\centering
\caption{系统层次结构与稳定性保证}
\label{tab:KOSHiera}
\begin{tabular}{llll}  % l: 左对齐;p: 固定宽度段落
\toprule  % 上粗线(需booktabs包)
层次 & 核心关注点 & 核心算子 & 稳定性保证 \\
\midrule  % 中线
L0: Core & 知识的静态结构 & $\Pi, \Sigma, \textsf{Prop}$ & 类型检查 (Type Checking) \\
L1: Kernel & 基于事件的状态转移 & $\textsf{STEP}, \textsf{Ev}$ & 逻辑证明 (Proof Verification) \\
L2: Runtime & 现实与逻辑的映射 & $\textsf{elaborate}, \mathcal{M} $ & 事务一致性 (Transactional Consistency) \\
\bottomrule  % 下粗线
\end{tabular}
\end{table}

\section{KOS-TL 核心层 (Core Layer):静态逻辑基础}

核心层(Core Layer)是 KOS-TL 系统的“形式宪法”,基于直觉依赖类型论(Intuitive Dependent Type Theory, IDTT)。其核心任务是定义知识的静态结构、逻辑约束及其合法性证明,为上层执行提供不可篡改的逻辑根基。它不随时间改变,只负责定义什么是“合法的构造”。

\subsection{语法 (Syntax)}


\subsubsection{1. 论域($\mathcal{D}_{Core}$)}

核心层的论域($\mathcal{D}_{Core}$)世界由双轴构成,一个是数据轴,另一个是逻辑轴。

\begin{itemize}
    \item \textbf{双轴世界 (Universes)}:
    \begin{itemize}
        \item \textbf{计算轴 ($\mathcal{U}_i$)}:遵循谓述性。$\mathcal{U}_0$ 包含基础目(Sorts),$\mathcal{U}_{i+1}$ 包含 $\mathcal{U}_i$ 作为元素。用于建模具有物理效应的数据。
        \item \textbf{逻辑轴 ($\textsf{Type}_i$)}:遵循谓述性,但其底层 $\textsf{Prop} : \textsf{Type}_1$ 具有非谓述性。$\textsf{Type}_i$ 用于建模逻辑谓词空间及元逻辑规则。
        \item \textbf{层级关系}:
        \begin{itemize}
            \item $\textsf{Prop} : \textsf{Type}_1, \quad \textsf{Type}_i : \textsf{Type}_{i+1}$
            \item $\mathcal{U}_i : \mathcal{U}_{i+1}, \quad \mathcal{U}_i : \textsf{Type}_{i+1}$ (计算宇宙可作为逻辑讨论的对象)
            \item $\textsf{Prop} \hookrightarrow \mathcal{U}_1$ (命题可以嵌入数据轴)
        \end{itemize}
    \end{itemize}
    \item \textbf{基础目 (Sorts)}:$\textsf{Val}$(原子值)、$\textsf{Time}$(时间点标量)、$\textsf{ID}$(唯一标识符)。
    \item \textbf{知识对象 (Objects)}:所有的依存记录类型(Dependent Record Types)的实例。
\end{itemize}

\subsubsection{2. 语法}

\begin{itemize}

\item \textbf{类型构造 (Types)}:

$$\begin{aligned}
A, B ::= & \ \textsf{Prop} \mid \textsf{Type}_i \mid \mathcal{U}_i & (\text{Universes}) \\
& \mid \textsf{Val} \mid \textsf{Time} \mid \textsf{ID} & (\text{Base Sorts}) \\
& \mid \Pi(x:A).B \mid \Sigma(x:A).B \mid A + B \mid \textsf{Id}_A(a, b) & (\text{Constructors})
\end{aligned}$$

\item \textbf{项构造 (Terms)}:

$$
\begin{aligned}
t, u ::= & \ x \mid \lambda x.t \mid t\,u & (\Pi\text{ Intro/Elim}) \\
& \mid \langle t, u \rangle \mid \textsf{split}(t, x.y.u) & (\Sigma\text{ Intro/Elim}) \\
& \mid \textsf{inl}(t) \mid \textsf{inr}(t) \mid \textsf{case}(t, x.u, y.v) & (+\text{ Intro/Elim}) \\
& \mid \textsf{refl} & (\text{Id Intro})
\end{aligned}
$$

\item \textbf{判断式 (Judgments)}:
\begin{itemize}
    \item $\Gamma \vdash A : \mathcal{S}$ :表示 $A$ 是一个合法类型,其中 $\mathcal{S} \in \{ \textsf{Type}_i, \mathcal{U}_i \}$。
    \item $\Gamma \vdash t : A$ :表示 $t$ 是类型 $A$ 的合法实例(对于数据轴)或合法证明(对于逻辑轴)。
    \item $\Gamma \vdash A \equiv B$ 及 $\Gamma \vdash t \equiv u$ :表示类型或项在计算上等价(转换规则)。
\end{itemize}
\end{itemize}

为了保持核心层的简洁性,$\textsf{Prop}$ 中的逻辑运算通过 $\mathcal{U}$ 中的类型构造算子实现。具体的语义映射关系如表\ref{tab:correspondenceofpro2type}所示。

注:关于 $\textsf{Prop}$ 的特殊性。KOS-TL Core 遵循 \textbf{证言忽略 (Proof Irrelevance)} 原则:对于任何 $P : \textsf{Prop}$,若 $p, q : P$,则在语义层面上 $\llbracket p \rrbracket = \llbracket q \rrbracket$。这意味着在制造业看板中,我们只关心“安全属性是否被证明”,而不关心证明的具体推导路径。这确保了逻辑层不占用额外的运行内存。

\begin{table}[h]
\centering
\caption{逻辑命题与类型的同构}
\label{tab:correspondenceofpro2type}
\begin{tabular}{|l|l|l|}
\hline
\textbf{逻辑命题 ($\textsf{Prop}$)} & \textbf{类型构造 ($\mathcal{T}$)} & \textbf{项构造 (Terms)} \\
\hline
全称量词 $\forall x:A.\, P(x)$ & 依存乘积 $\Pi(x:A).P$ & $\lambda x.\, p$  \\
\hline
存在量词 $\exists x:A.\, P(x)$ & 依存求和 $\Sigma(x:A).P$ & $\langle a, p \rangle$ \\
\hline
逻辑蕴含 $P \to Q$ & 函数空间 $P \to Q$ & $\lambda p.\, q$ \\
\hline
逻辑合取 $P \wedge Q$ & 积类型 $P \times Q$ & $\langle p, q \rangle$ \\
\hline
逻辑析取 $P \vee Q$ & 和类型 $P + Q$ & $\textsf{inl}(p) / \textsf{inr}(q)$ \\
\hline
\end{tabular}
\end{table}

根据项与类型的构造规则,KOS-TL 核心层的类型集合 $\mathcal{T}$ 由以下规则归纳定义。

\begin{enumerate}[label=(\arabic*)]

    \item 基础规则 (Base Rules)

    $$\begin{aligned}
        & \textsf{Prop} : \textsf{Type}_1 \quad (\text{逻辑轴起点}) \\
        & \textsf{Type}_i : \textsf{Type}_{i+1} \quad (\text{逻辑宇宙累积}) \\
        & \mathcal{U}_i : \mathcal{U}_{i+1} \quad (\text{数据宇宙累积}) \\
        & \textsf{Prop} \hookrightarrow \mathcal{U}_1 \quad (\text{提升规则:命题可以作为数据处理,这是一个隐式强制类型转换(Coercion)})
    \end{aligned}$$

    $\textsf{Prop} \hookrightarrow \mathcal{U}_1$ 是一种单向嵌入,允许我们将证明作为对象嵌入到数据记录(如 $\Sigma$ 类型)中,但 $\mathcal{U}_i$ 中的普通数据不能直接作为命题进行逻辑推导。

    原子类型: $\textsf{Val} \in \mathcal{T}, \textsf{Time} \in \mathcal{T}, \textsf{ID} \in \mathcal{T}$。

    \item 依存乘积构造 (Pi-types / Dependent Products)

    \begin{itemize}
        \item \textbf{逻辑/计算混合规则}:

        $$\frac{\Gamma \vdash A : \textsf{Type}_i/\mathcal{U}_i \quad \Gamma, x:A \vdash B : \textsf{Prop}}{\Gamma \vdash \Pi(x:A).B : \textsf{Prop}} (\text{非谓述性})$$

        \item \textbf{纯 Universe 规则}:

        $$\frac{\Gamma \vdash A : \textsf{Type}_i \quad \Gamma, x:A \vdash B : \textsf{Type}_j}{\Gamma \vdash \Pi(x:A).B : \textsf{Type}_{\max(i, j)}} (\text{谓述性})$$
    \end{itemize}
    $\textsf{Prop}$ 具有一种特殊的性质,称为非谓述性 (Impredicativity)。无论 $A$ 的层级多高,只要 $B$ 属于 $\textsf{Prop}$,那么 $\Pi(x:A).B$ 通常仍然属于 $\textsf{Prop}$。意义:这允许我们对“无限的对象”进行逻辑判定。例如,“对于所有 $\mathcal{U}_1$ 中的类型,它们都满足安全属性 $P$”,这个判定本身依然只是一个简单的 $\textsf{Prop}$ (真或假),而不会爆炸成一个超级复杂的类型。

    \item 依存求和构造 (Sigma-types / Dependent Sums)

    为了防止逻辑悖论(类似 Girard's Paradox),$\Sigma$ 类型在 KOS-TL 中必须是谓述性的。若 $A$ 是一个类型,且在变量 $x:A$ 的假设下 $B$ 是一个类型,则:

    $$\frac{\Gamma \vdash A : \textsf{Type}_i \quad \Gamma, x:A \vdash B : \textsf{Type}_j}{\Gamma \vdash \Sigma(x:A).B : \textsf{Type}_{\max(i, j)}}$$

    注:若 $A, B \in \mathcal{U}$,则结果在 $\mathcal{U}$ 中;若涉及证明提取,其最高层级受逻辑轴 Universe 约束。

    依存求和构造建模知识对象。$\Sigma$ 类型是 KOS-TL 的核心,它强制要求数据 $x$ 必须关联一个证明项 $p:B(x)$。

    \item 和类型构造 (Sum Types / Disjoint Union)

    若 $A$ 和 $B$ 分别是合法类型,则它们的析取和(不交并)也是一个类型:

$$\frac{\Gamma \vdash A : \mathcal{U} \quad \Gamma \vdash B : \mathcal{U}}{\Gamma \vdash A + B : \mathcal{U}}$$

  和类型建模逻辑中的“析取($\vee$)”关系。在制造业场景中,它用于建模“互斥状态”或“备选路径”。例如,一个任务的状态要么是 $\textsf{Success}$,要么是 $\textsf{Failure}$。

    \item 等价类型构造 (Identity Types)

    若 $A$ 是一个类型,且 $u, v$ 是类型 $A$ 的两个项,则:

    $$\frac{\Gamma \vdash A : \mathcal{U} \quad \Gamma \vdash u:A \quad \Gamma \vdash v:A}{\Gamma \vdash \textsf{Id}_A(u, v) : \textsf{Prop}}$$

    等价类型构造建模知识的等价性,它执行“因果追溯”和“状态回滚”时判断两个事实是否一致的逻辑基础。
\end{enumerate}


在 KOS-TL Core 中,$\textsf{Prop}$ 是一个特殊的论域,专门用于处理逻辑断言。与普通的 $\mathcal{U}_i$ 不同,它在 $\Pi$ 构造下表现出非谓述性。

\begin{Definition}{非谓述 $\Pi$ 构造规则}

对于任意层级的类型 $A : \mathcal{U}_i$,若在假设 $x:A$ 下 $B$ 是一个命题,则其全称量词(或函数空间)依然映射回最小的命题世界:

$$\frac{\Gamma \vdash A : \mathcal{U}_i \quad \Gamma, x:A \vdash B : \textsf{Prop}}{\Gamma \vdash \Pi(x:A).B : \textsf{Prop}}$$

\textbf{逻辑闭环要点}:这意味着命题的复杂度不随其量词覆盖范围的扩大而提升等级。这一性质允许我们在不触发 Universe 爆炸的前提下,对全量数据(即使是 $\mathcal{U}_k$ 层级的对象)进行一致性断言。
\end{Definition}


\begin{Definition}{ Universe 提升与包含规则 }

为了支撑双轴语义的闭环,系统引入以下隐式转换:
\begin{itemize}
    \item \textbf{计算到逻辑的观测}:

    $$\frac{\Gamma \vdash A : \mathcal{U}_i}{\Gamma \vdash A : \textsf{Type}_{i+1}}$$

    这意味着任何计算类型都可以被当作逻辑命题的讨论对象(例如在 $\textsf{Type}$ 层级讨论 SensorData 的代数性质)。

    \item \textbf{命题的计算嵌入}:

    $$\frac{\Gamma \vdash P : \textsf{Prop}}{\Gamma \vdash P : \mathcal{U}_1}$$

    这允许逻辑证明项 $p:P$ 被包装进 $\Sigma$ 记录中,作为实时计算系统的输入(即“带有证明的数据包”)。
\end{itemize}

\end{Definition}

\subsubsection{3. 判定规则 (Judgmental Rules)}

为了确保上述构造在逻辑上是良构的,KOS-TL Core 遵循以下推导规则。

\begin{enumerate}[label=(\arabic*)]

    \item 依存乘积 ($\Pi$-Types)
    \begin{itemize}
        \item 引入规则 (Introduction Rules)

        对于 $\Pi$类型,其构造(引入)规则为:

        $$\frac{\Gamma, x:A \vdash t : B}{\Gamma \vdash \lambda x:A. t : \Pi(x:A).B}$$

        在当前的上下文 $\Gamma$ 中,如果我们假设有一个类型为 $A$ 的变量 $x$,并且能够构造出一个类型为 $B$ 的项 $t$。那么我们可以构造一个 $\lambda$ 抽象(即函数),它的类型就是 $\Pi(x:A).B$。

        \item 消解规则 (Elimination Rules)

        对于 $\Pi$类型,有一个通用规则 $f$(类型为 $\Pi(x:A).B$)和有一个具体的对象 $a$(类型为 $A$)。将 $f$ 应用于 $a$(记作 $f\,a$),得到的结果类型是 $B[a/x]$。结果的类型中,所有的 $x$ 都被替换成了具体的值 $a$。

        $$\frac{\Gamma \vdash f : \Pi(x:A).B \quad \Gamma \vdash a : A}{\Gamma \vdash f\,a : B[a/x]}$$
    \end{itemize}

    \item 依存求和 ($\Sigma$-Types)

    \begin{itemize}
        \item 引入规则 (Introduction Rules)

        对于 $\Sigma$ 类型,其构造(引入)规则为:

        $$\frac{\Gamma \vdash a : A \quad \Gamma \vdash b : B[a/x]}{\Gamma \vdash \langle a, b \rangle : \Sigma(x:A).B}$$

        引入规则体现了 “构造即证明”,只有当你能提供满足 $B(a)$ 的证据 $b$ 时,知识对象才能被创建。

        \item 消解规则 (Elimination Rules)

        对于 $\Sigma$ 类型,我们定义通用的依存消解算子 $\textsf{split}$。它允许通过匹配对结构来构造依赖于该对整体的目标项。

        $$\frac{\Gamma \vdash p : \Sigma(x:A).B \quad \Gamma, x:A, y:B \vdash t : C[\langle x, y \rangle / z]}{\Gamma \vdash \textsf{split}(p, x.y.t) : C[p/z]}$$

        传统的投影算子可以作为 $\textsf{split}$ 的特例被定义:
        \begin{align*}
            \textsf{proj}_1(p)\text{(左投影)} & \equiv \textsf{split}(p, x.y.x) \\
            \textsf{proj}_2(p)\text{(右投影)} &\equiv \textsf{split}(p, x.y.y)
        \end{align*}
    \end{itemize}

    \item 和类型 ($A + B$)
    \begin{itemize}
        \item 引入规则 (Introduction Rules)

        引入规则定义了如何制造一个类型为 $A+B$ 的对象。它有两个分支,分别对应“左选择”和“右选择”。

        $$\frac{\Gamma \vdash a : A}{\Gamma \vdash \textsf{inl}(a) : A + B} \quad \frac{\Gamma \vdash b : B}{\Gamma \vdash \textsf{inr}(b) : A + B}$$

        如果有一个 $A$ 类型的证据 $a$,可以通过标签 $\textsf{inl}$(In-Left)把它包装成 $A+B$ 类型。同理,如果有一个 $B$ 类型的证据 $b$,通过 $\textsf{inr}$(In-Right)也能包装成 $A+B$。

        \item 消解规则 (Elimination Rules)

        消解规则定义了当拿到一个 $A+B$ 类型的对象时,如何安全地使用它。由于不知道它内部到底是 $A$ 还是 $B$,必须准备好两套方案。

        $$\frac{\Gamma \vdash s : A + B \quad \Gamma, x:A \vdash t:C \quad \Gamma, y:B \vdash u:C}{\Gamma \vdash \textsf{case}(s, x.t, y.u) : C}$$

        $s : A + B$是输入。$\Gamma, x:A \vdash t:C$是方案一。如果 $s$ 最终证明是 $A$ 类型的,把里面的数据取出来给 $x$,然后按逻辑 $t$ 算出一个结果,结果类型是 $C$。$\Gamma, y:B \vdash u:C$是方案二。如果 $s$ 是 $B$ 类型的,把数据给 $y$,按逻辑 $u$ 同样算出一个 $C$ 类型的结果。$\textsf{case}(s, x.t, y.u) : C$表明无论 $s$ 走哪条路,最终都能得到一个类型为 $C$ 的确定的结果。
    \end{itemize}

    \item 转换规则 (Conversion Rule)

    $$\frac{\Gamma \vdash t : A \quad \Gamma \vdash A \equiv B \quad \Gamma \vdash B : \mathcal{S}}{\Gamma \vdash t : B}$$

    该规则确保了如果项 $t$ 在 $\mathcal{U}_1$ 中合法,且 $\mathcal{U}_1 \hookrightarrow \textsf{Type}_2$ 成立,则 $t$ 自动具备在更高层级被观测的合法性。
\end{enumerate}

\subsubsection{4. 规约规则}

我们使用 $\to$ 表示单步规约(One-step reduction),使用 $\twoheadrightarrow$ 表示多步规约(计算的闭包)。

\begin{enumerate}[label=(\arabic*)]

\item \textbf{函数规约 ($\beta$-reduction)}

针对 $\Pi$ 类型构造($\lambda$ 抽象):
$$\frac{}{\Gamma \vdash (\lambda x:A.t)\,u \to t[u/x]}$$

\item \textbf{依存记录规约 ($\iota$-reduction)}

针对 $\Sigma$ 类型的结构化消解。这是核心修正点,通过 \textsf{split} 算子直接解构配对:
$$\frac{}{\Gamma \vdash \textsf{split}(\langle u, v \rangle, x.y.t) \to t[u/x, v/y]}$$

在这种定义下,传统的投影规约可以作为特例自然导出:
\begin{itemize}
    \item \textbf{左投影}:$\textsf{proj}_1(\langle u, v \rangle) \equiv \textsf{split}(\langle u, v \rangle, x.y.x) \to u$
    \item \textbf{右投影}:$\textsf{proj}_2(\langle u, v \rangle) \equiv \textsf{split}(\langle u, v \rangle, x.y.y) \to v$
\end{itemize}

\item \textbf{和类型规约 ($\iota$-reduction)}

针对 $+$ 类型的分支判定。通过 $\textsf{case}$ 算子对标签进行匹配:
$$\frac{}{\Gamma \vdash \textsf{case}(\textsf{inl}(u), x.t, y.v) \to t[u/x]} \quad \frac{}{\Gamma \vdash \textsf{case}(\textsf{inr}(w), x.t, y.v) \to v[w/y]}$$

\item \textbf{等价项规约 ($\iota$-reduction)}

当判定项已归约为 $\textsf{refl}$ 时,等价判定自动消解。
\end{enumerate}

为了支持工程实践中的模块化定义与局部变量,系统定义了以下辅助转换规则。与核心规约($\beta, \iota$)不同,这些转换通常在类型检查器的\textbf{等价性判定(Conversion Check)}阶段按需触发,不计入核心逻辑步。

\begin{itemize}
\item \textbf{全局展开 ($\delta$-conversion)}:\\
  若项 $c$ 在上下文 $\Gamma$ 中被定义为 $c := t:A$,则在判定等价性时允许展开:\\
  \[
  \frac{(c := t:A) \in \Gamma}{\Gamma \vdash c \equiv_\delta t}
  \]\\
  意义:允许系统识别别名(Alias)与其原始定义在逻辑上是同一对象。

\item \textbf{局部绑定展开 ($\zeta$-conversion)}:\\
  针对局部定义的 \texttt{let} 结构,其语义等价于立即代换:\\
  \[
  \Gamma \vdash (\textsf{let } x = u \textsf{ in } t) \equiv_\zeta t[u/x]
  \]\\
  意义:在不增加函数调用($\beta$)开销的前提下,支持项的局部复用。

\item \textbf{外延等价性 ($\eta$-conversion)}:\\
  为了保证函数的一致性,系统支持函数外延性判定:\\
  \[
  \Gamma \vdash \lambda x:A.(f\,x) \equiv_\eta f \quad (x \notin \text{FV}(f))
  \]\\
  意义:确保函数抽象与直接引用的行为一致,支持函数式编程范式。
\end{itemize}


“$\delta$ 与 $\zeta$ 确保了系统具有定义透明性 (Definitional Transparency),即名称的引用不改变逻辑项的语义本质。”

\begin{Definition}{等价判定 (Definitional Equality)}
KOS-TL 的判断等价关系 $\equiv$ 是由上述所有规约($\beta, \iota$)和转换($\delta, \zeta, \eta$)生成的最小等价关系(满足对称性、传递性与同余性)。
\end{Definition}



\subsection{核心层逻辑性质}

\begin{Definition}{范型(Normal Form)}

一个项 $t$ 被称为处于 \textbf{范型}(记作 $t \in \textsf{NF}$),当且仅当它不包含任何 \textbf{Redex}(可归约式)。即对于 KOS-TL Core 中定义的归约关系 $\to$,不存在项 $t'$ 使得:
\[
t \to t'
\]
\end{Definition}


\begin{Definition}{强范型(Strong Normalization,$\textsf{SN}$)}

一个项 $t$ 被称为是\textbf{强范型}的,当且仅当不存在从 $t$ 出发的无限规约序列。即所有可能的规约路径 $t \to t_1 \to t_2 \dots$ 均在有限步内终止于某个正规形式(Normal Form)。

\end{Definition}


\begin{Definition}{可还原性集合$\textsf{Red}_A$}

对于任意类型 $A$ 和项 $t : A$,可还原性集合 $\textsf{Red}_A \subseteq \textsf{Val}_A$ 通过对 $A$ 的结构归纳定义如下:

\begin{enumerate}[label=(\arabic*)]

    \item \textbf{世界类型情况 (Universe Rules)}

    \begin{itemize}
        \item $A \equiv \mathcal{U}_i$
        \[ t \in \textsf{Red}_{\mathcal{U}_i} \iff t \in \mathcal{RC} \land \textsf{level}(t) < i \]
        其中 $\mathcal{RC}$ 是所有满足 CR 属性(SN, 稳定性, 中性项构造)的项集集合。

     \item $A \equiv \textsf{Type}_i$
        \[ t \in \textsf{Red}_{\textsf{Type}_i} \iff t \in \mathcal{RC} \land \textsf{level}(t) < i \]

        \item \textbf{跨轴约束}:
        由于 $\textsf{Prop} : \textsf{Type}_1$,则 $\textsf{Prop} \in \textsf{Red}_{\textsf{Type}_1}$。
        由于 $\textsf{Val} : \mathcal{U}_0$,则 $\textsf{Val} \in \textsf{Red}_{\mathcal{U}_1}$。
    \end{itemize}

    \item 基类型情况
    \begin{itemize}

        \item $A \equiv \textsf{Val}$

    $t \in \textsf{Red}_A \iff t \in \textsf{SN} \land \exists c \in \textsf{Const}_\mathbb{N}.\ t \to^* c$

        \item $A \equiv \textsf{Time}$

    $t \in \textsf{Red}_A \iff t \in \textsf{SN} \land t \text{ 表征合法时间戳或时长(即 } t \to^* \mathsf{timestamp}(n) \text{ 或 } t \to^* \mathsf{duration}(n) \text{ 对于 } n \in \mathbb{N}\text{)}$

    \end{itemize}
    \item 构造类型情况
    \begin{itemize}
        \item $A \equiv \Pi(x : B).C$

     $t \in \textsf{Red}_A \iff \forall u \in \textsf{Red}_B.\ (t \  u) \in \textsf{Red}_{C[u/x]}$

        \item $A \equiv \Sigma(x : B).C$

    $t \in \textsf{Red}_{\Sigma(x:B).C} \iff t \twoheadrightarrow \langle u, v \rangle \land u \in \textsf{Red}_B \land v \in \textsf{Red}_{C[u/x]}$

     $t \in \textsf{Red}_A \iff t \in \textsf{SN} \land \left( t \to^* \mathsf{inl}(u) \implies u \in \textsf{Red}_B \right) \land \left( t \to^* \mathsf{inr}(v) \implies v \in \textsf{Red}_D \right)$

        \item $A \equiv \textsf{Id}_B(a, b)$

     $t \in \textsf{Red}_A \iff t \in \textsf{SN} \land \left( t \to^* \mathsf{refl}(w) \implies w \in \textsf{Red}_B \land a \equiv_B w \land b \equiv_B w \right)$
    \end{itemize}
\end{enumerate}

其中:
\begin{itemize}
\item $\textsf{Const}_\mathbb{N}$ 是自然数常量集合(数值常量)。
\item $\to^*$ 表示 $\beta\eta$-归约的多步闭包(multi-step $\beta\eta$-reduction)。
\item $\equiv_B$ 表示在类型 $B$ 下的可还原性等价:$p \equiv_B q \iff \exists v \in \textsf{Red}_B.\ (p \to^* v \land q \to^* v)$。
\item 对于 $\textsf{Time}$,需预定义“合法时间戳或时长”的精确语义(如 $\mathsf{timestamp}(n)$ 表示 Unix 时间戳 $n$,$\mathsf{duration}(n)$ 表示 $n$ 毫秒时长),以确保形式化。
\end{itemize}

 需要指出的是$\textsf{Prop}$ 的解释不依赖于其所在的世界等级。
\end{Definition}


\begin{Definition}{中性项(Neutral Term)}

在 KOS-TL Core 层中,一个项 $t$ 被称为 \textbf{中性项},当且仅当它满足以下两个条件之一:
\begin{enumerate}[label=(\arabic*)]
\item $t$ 是一个变量 $x$。
\item $t$ 的头部(Head)是一个变量,且该项正处于被消去(Elimination)的过程中,但无法进行进一步的归约。
\end{enumerate}
\end{Definition}

类型$A$ 的可还原集 $\textsf{Red}_A$必须满足三个关键的饱和属性(Saturation Properties),通常称为 $CR1,CR2,CR3$。
\begin{Definition}{可还原集 $\textsf{Red}_A$的饱和属性}
\begin{enumerate}[label=(\arabic*)]
    \item \textbf{CR 1 (包含性)}

    若 $t \in \textsf{Red}_A$,则 $t \in \textsf{SN}$(即 $t$ 必须先是强规范化的)。
    \item \textbf{CR 2 (稳定性)}

    若 $t \in \textsf{Red}_A$ 且 $t \to t'$,则 $t' \in \textsf{Red}_A$。
    \item \textbf{CR 3 (中性项构造)}

    若 $t$ 是一个中性项,且 $t$ 的所有单步归约项 $t'$ 都在 $\textsf{Red}_A$ 中,则 $t \in \textsf{Red}_A$。所有变量都是中性项。
\end{enumerate}
\end{Definition}

\begin{Lemma}{替换引理(Substitution Lemma)}
\label{lemma:KOSsubstitution}

若 $\Gamma, x:B, \Delta \vdash t : A$ 且 $\Gamma \vdash u : B$,则 $\Gamma, \Delta[u/x] \vdash t[u/x] : A[u/x]$。

其中, $\Delta$通用上下文,用以处理 $x$ 之后定义的依赖于 $x$ 的变量。在简单情况下,$\Delta$ 为空。
\end{Lemma}

\begin{proof}

依据类型推导树的结构进行归纳证明。根据 $t$ 的构造规则,分以下几种核心情况讨论:
\begin{enumerate}[label=(\arabic*)]

\item \textbf{变量情况 (Variable)}\\
假设 $t$ 是一个变量 $y$。

\begin{itemize}
\item \textbf{子情况 1:}$y = x$\\
  根据推导规则,此时 $A = B$。\\
  我们需要证明 $\Gamma, \Delta[u/x] \vdash x[u/x] : B[u/x]$。\\
  由于 $x[u/x] = u$,且前提已知 $\Gamma \vdash u : B$。根据上下文弱化规则(Weakening),可在 $\Gamma$ 后增加 $\Delta[u/x]$,故结论成立。

\item \textbf{子情况 2:}$y \neq x$\\
  此时 $y$ 必须在 $\Gamma$ 或 $\Delta$ 中定义。\\
  若 $y \in \Gamma$,则 $y[u/x] = y$ 且 $A[u/x] = A$(因为 $\Gamma$ 中的类型不依赖 $x$),结论显然。\\
  若 $y \in \Delta$,则 $y[u/x] = y$,其类型为 $A[u/x]$,这正是 $\Delta[u/x]$ 中对应的声明。
\end{itemize}

\item \textbf{$\Pi$-类型引入 ($\lambda$-抽象)}\\
假设 $t = \lambda y:C. M$,且 $A = \Pi(y:C). D$。\\
推导最后一步为
\[
\frac{\Gamma, x:B, \Delta, y:C \vdash M : D}{\Gamma, x:B, \Delta \vdash \lambda y:C. M : \Pi(y:C). D}
\]

\begin{itemize}
\item 应用归纳假设:对 $M$ 使用归纳假设(此时上下文为 $\Delta, y:C$):
  \[
  \Gamma, \Delta[u/x], y:C[u/x] \vdash M[u/x] : D[u/x]
  \]

\item 构造结论:应用 $\Pi$-引入规则:
  \[
  \Gamma, \Delta[u/x] \vdash \lambda y:C[u/x]. M[u/x] : \Pi(y:C[u/x]). D[u/x]
  \]
  这等价于 $(\lambda y:C. M)[u/x] : (\Pi(y:C). D)[u/x]$。
\end{itemize}

\item \textbf{$\Pi$-类型消去 (应用)}\\
假设 $t = (f \, v)$,其中 $\Gamma, x:B, \Delta \vdash f : \Pi(y:C). D$ 且 $\Gamma, x:B, \Delta \vdash v : C$。

\begin{itemize}
\item 归纳假设 1:$\Gamma, \Delta[u/x] \vdash f[u/x] : (\Pi(y:C). D)[u/x]$。
\item 归纳假设 2:$\Gamma, \Delta[u/x] \vdash v[u/x] : C[u/x]$。
\item 组合:应用消去规则:
  \[
  (f[u/x] \, v[u/x]) : D[u/x][v[u/x]/y]
  \]
  根据代换的可交换性,上述类型等价于 $D[v/y][u/x]$,即 $A[u/x]$。
\end{itemize}

\item \textbf{$\Sigma$-类型构造 (配对)}

假设 $t = \langle t_1, t_2 \rangle$,且 $A = \Sigma(y:C). D$。

\begin{itemize}
\item 由归纳假设,$t_1[u/x] : C[u/x]$。
\item 由归纳假设,$t_2[u/x] : D[t_1/y][u/x]$。
\item 根据规则构造出 $\langle t_1[u/x], t_2[u/x] \rangle : (\Sigma(y:C). D)[u/x]$。
\end{itemize}

\item \textbf{$\Sigma$-类型消去 (结构化消解)}

假设 $t = \textsf{split}(p, x.y.u)$,且最后推导步为:

$$\frac{\Gamma, z:B, \Delta \vdash p : \Sigma(x:A).B \quad \Gamma, z:B, \Delta, x:A, y:B \vdash u : C(\langle x, y \rangle)}{\Gamma, z:B, \Delta \vdash \textsf{split}(p, x.y.u) : C(p)}$$

(此处 $z:B$ 是我们要替换的变量)
\begin{itemize}
    \item \textbf{归纳假设 1}:对 $p$ 应用归纳假设,得到 $\Gamma, \Delta[v/z] \vdash p[v/z] : (\Sigma(x:A).B)[v/z]$。
    \item \textbf{归纳假设 2}:对消解体 $u$ 应用归纳假设(此时上下文多了 $x, y$),得到:$$ \Gamma, \Delta[v/z], x:A[v/z], y:B[v/z] \vdash u[v/z] : C(\langle x, y \rangle)[v/z] $$
    \item \textbf{构造结论}:重新应用 $\Sigma$-消去规则($\textsf{split}$ 规则):

    $$ \Gamma, \Delta[v/z] \vdash \textsf{split}(p[v/z], x.y.u[v/z]) : C(p)[v/z] $$

    由于 $\textsf{split}(p, x.y.u)[v/z] = \textsf{split}(p[v/z], x.y.u[v/z])$,结论成立。
\end{itemize}


\item \textbf{$+$-类型引入 (注入)} \\
假设 $t = \textsf{inl}_D(s)$,且 $A = C + D$($\textsf{inr}$ 情况对称)。
\begin{itemize}
    \item \textbf{前提}:已知 $\Gamma, x:B, \Delta \vdash s : C$ 且 $\Gamma, x:B, \Delta \vdash D : \mathcal{U}$。
    \item \textbf{归纳假设}:对 $s$ 有 $s[u/x] : C[u/x]$;对类型 $D$ 有 $D[u/x] : \mathcal{U}$。\item \textbf{构造结论}:应用 $+$-引入规则:$$ \Gamma, \Delta[u/x] \vdash \textsf{inl}_{D[u/x]}(s[u/x]) : C[u/x] + D[u/x] $$即 $(\textsf{inl}_D(s))[u/x] : (C + D)[u/x]$。
\end{itemize}


\item \textbf{$+$-类型消去 (分支判定)}\\
假设 $t = \textsf{case}(s, y.t_1, z.t_2)$,且最后一步推导为:$$\frac{\Gamma', s:C+D \quad \Gamma', y:C \vdash t_1 : A \quad \Gamma', z:D \vdash t_2 : A}{\Gamma' \vdash \textsf{case}(s, y.t_1, z.t_2) : A}$$(其中 $\Gamma'$ 简记为 $\Gamma, x:B, \Delta$)
\begin{itemize}
    \item \textbf{归纳假设 1}:对判定项 $s$,有 $\Gamma, \Delta[u/x] \vdash s[u/x] : C[u/x] + D[u/x]$。
    \item \textbf{归纳假设 2}:对左分支 $t_1$(此时上下文增加 $y:C$),有 $\Gamma, \Delta[u/x], y:C[u/x] \vdash t_1[u/x] : A[u/x]$。
    \item \textbf{归纳假设 3}:对右分支 $t_2$ (此时上下文增加 $z:D$),有 $\Gamma, \Delta[u/x], z:D[u/x] \vdash t_2[u/x] : A[u/x]$。
    \item \textbf{组合}:应用 $+$-消去规则($\textsf{case}$ 规则):$$ \Gamma, \Delta[u/x] \vdash \textsf{case}(s[u/x], y.t_1[u/x], z.t_2[u/x]) : A[u/x] $$结论成立。
\end{itemize}

\item \textbf{恒等类型 (Identity)}\\
若 $t = \textsf{refl}_a$,则 $A = \textsf{Id}_C(a, a)$。

\begin{itemize}
\item 由归纳假设,$a[u/x] : C[u/x]$。
\item 直接应用构造规则得到 $\textsf{refl}_{a[u/x]} : \textsf{Id}_{C[u/x]}(a[u/x], a[u/x])$,即 $A[u/x]$。
\end{itemize}
\end{enumerate}
\end{proof}


\begin{Lemma}{可还原基本引理}
\label{lemma:KOSreduction}

设 $\Gamma = \{x_1:A_1, \dots, x_n:A_n\}$ 是一个良构上下文。若 $\Gamma \vdash t : C$,且存在一个可还原替换 $\gamma = [u_1/x_1, \dots, u_n/x_n]$ 使得对于所有 $i$,均有 $u_i \in \textsf{Red}_{A_i}$。
则替换后的项 $t[\gamma]$ 必定满足:
\[
t[\gamma] \in \textsf{Red}_C
\]
\end{Lemma}

\begin{proof}

依据类型推导树的结构进行归纳证明。
\begin{enumerate}[label=(\arabic*)]

    \item \textbf{变量规则 (Variables)}

    若推导为 $\Gamma \vdash x_i : A_i$。

    根据替换定义,$x_i[\gamma] = u_i$。由前提 $u_i \in \textsf{Red}_{A_i}$,命题显然成立。

    \item \textbf{$\Pi$-类型引入 ($\lambda$-抽象)}

    若 $\Gamma \vdash \lambda x:A.M : \Pi(x:A).B$。\\
    需证:对于任意 $u \in \textsf{Red}_A$,$((\lambda x:A.M)[\gamma] \, u) \in \textsf{Red}_{B[u/x]}$。

    \begin{enumerate}[label=(\alph*)]
        \item 该项 $\beta$-归约为 $M[\gamma, u/x]$。
        \item 由于 $u \in \textsf{Red}_A$ 且 $\gamma \in \textsf{Red}_\Gamma$,则 $(\gamma, u/x)$ 是语境 $(\Gamma, x:A)$ 下的一个可还原替换。
        \item 根据归纳假设,$M[\gamma, u/x] \in \textsf{Red}_B$。
        \item 由于可还原集合对逆向归约封闭,故 $(\lambda x:A.M)[\gamma] \in \textsf{Red}_{\Pi(x:A).B}$。
    \end{enumerate}

    \item \textbf{$\Pi$-类型消去 (应用)}

    若 $\Gamma \vdash (f \, v) : C$,其中 $C = B[v/x]$ 且 $\Gamma \vdash f : \Pi(x:A).B, \Gamma \vdash v : A$。
    \begin{enumerate}[label=(\alph*)]
        \item 由归纳假设,$f[\gamma] \in \textsf{Red}_{\Pi(x:A).B}$。
        \item 由归纳假设,$v[\gamma] \in \textsf{Red}_A$。
        \item 根据 $\textsf{Red}_{\Pi}$ 的定义:若一个项属于 $\Pi$ 类型的可还原集,则它作用于任何属于参数类型可还原集的项,其结果必属于结果类型的可还原集。
        \item 因此,$(f[\gamma] \, v[\gamma]) \in \textsf{Red}_{B[v[\gamma]/x]}$。
        \item 由于 $(f[\gamma] \, v[\gamma]) = (f \, v)[\gamma]$ 且 $B[v[\gamma]/x] = C[\gamma]$,结论成立。
    \end{enumerate}


    \item \textbf{$\Sigma$-类型引入 (配对)}

    若 $\Gamma \vdash \langle a, b \rangle : \Sigma(x:A).B$。

    \begin{enumerate}[label=(\alph*)]
        \item 由归纳假设,$a[\gamma] \in \textsf{Red}_A$。
        \item 由归纳假设,$b[\gamma] \in \textsf{Red}_{B[a[\gamma]/x]}$。
        \item 根据 $\textsf{Red}_{\Sigma}$ 定义,分量均可还原,则 $\langle a, b \rangle[\gamma] \in \textsf{Red}_{\Sigma(x:A).B}$。
    \end{enumerate}

    \item \textbf{$\Sigma$-类型消去 (split 算子)}

    若 $\Gamma \vdash \textsf{split}(p, x.y.t) : C$,其中 $\Gamma \vdash p : \Sigma(x:A).B$。

    \begin{enumerate}[label=(\alph*)]
        \item \textbf{归纳假设}:由归纳假设知,$p[\gamma] \in \textsf{Red}_{\Sigma(x:A).B}$。这意味着 $p[\gamma]$ 强规范化且最终归约为某个配对 $\langle u, v \rangle$,其中 $u \in \textsf{Red}_A, v \in \textsf{Red}_{B[u/x]}$。
        \item \textbf{规约分析}:根据 $\iota$-规约,$\textsf{split}(p[\gamma], x.y.t[\gamma]) \twoheadrightarrow t[\gamma, u/x, v/y]$。
        \item \textbf{应用归纳假设}:由于 $(\gamma, u/x, v/y)$ 是语境 $(\Gamma, x:A, y:B)$ 下的合法可还原替换,对 $t$ 应用归纳假设得:$$ t[\gamma, u/x, v/y] \in \textsf{Red}_C $$
        \item \textbf{封闭性}:利用可还原集合对逆向 $\iota$-规约的封闭性(CR 3 属性的推论),原始项 $\textsf{split}(p, x.y.t)[\gamma]$ 亦属于 $\textsf{Red}_C$。
    \end{enumerate}

    \item \textbf{$+$-类型引入 (注入)}

    若 $\Gamma \vdash \textsf{inl}(a) : A + B$($\textsf{inr}$ 同理)。
    \begin{enumerate}[label=(\alph*)]
        \item 由归纳假设,$a[\gamma] \in \textsf{Red}_A$。
        \item 根据 $\textsf{Red}_{A+B}$ 的定义(通常由中性项和注入项的性质定义):由于 $a[\gamma]$ 是可还原的,则其注入项 $\textsf{inl}(a[\gamma])$ 在 $+$-类型的可还原体系下也是可还原的(利用逆向归约封闭性)。
        \item 故 $\textsf{inl}(a)[\gamma] \in \textsf{Red}_{A+B}$。
    \end{enumerate}

    \item \textbf{和类型消去 (Case Analysis)}

    若 $\Gamma \vdash \textsf{case}(t, x.M, y.N) : C$。
    \begin{enumerate}[label=(\alph*)]
        \item 由归纳假设,$t[\gamma] \in \textsf{Red}_{A+B}$。
        \item $t[\gamma]$ 将归约为 $\textsf{inl}(u)$ 或 $\textsf{inr}(v)$。假设为 $\textsf{inl}(u)$,则 $u \in \textsf{Red}_A$。
        \item 此时 $\textsf{case}$ 项归约为 $M[\gamma, u/x]$。
        \item 根据归纳假设,$M[\gamma, u/x] \in \textsf{Red}_C$。同理可证 $\textsf{inr}$ 情况。
    \end{enumerate}

    \item \textbf{恒等类型引入 ($\textsf{refl}$)}

    若 $\Gamma \vdash \textsf{refl}_a : \textsf{Id}_A(a, a)$。

    \begin{enumerate}
        \item 由归纳假设,$a[\gamma] \in \textsf{Red}_A$。
        \item 显然 $a[\gamma] \cong a[\gamma]$ 且 $\textsf{refl} \in \textsf{SN}$。
        \item 故 $\textsf{refl}_{a[\gamma]} \in \textsf{Red}_{\textsf{Id}_A(a[\gamma], a[\gamma])}$。
    \end{enumerate}

    \item \textbf{$\delta$-规约与局部定义 ($\textsf{let}$)}
    若 $\Gamma \vdash \textsf{let } x=u \textsf{ in } t : C$。
    \begin{enumerate}[label=(\alph*)]
        \item 由归纳假设,$u[\gamma] \in \textsf{Red}_A$。
        \item 构造扩展替换 $\gamma' = [\gamma, u[\gamma]/x]$。由于 $u[\gamma]$ 可还原,$\gamma'$ 是良构语境下的可还原替换。
        \item 由归纳假设,$t[\gamma'] \in \textsf{Red}_C$。
        \item 因为 $(\textsf{let } x=u \textsf{ in } t)[\gamma] \to t[\gamma, u[\gamma]/x]$,根据可还原集合对逆向 $\zeta$-规约的封闭性,结论成立。
    \end{enumerate}
\end{enumerate}
\end{proof}

\begin{Theorem}{KOS-TL内核的强范型(Strong Normalization for KOS-TL Core)}

设 $\Gamma$ 是一个良构语境。若项 $t$ 满足 $\Gamma \vdash t : A$,则 $t$ 是强范型(即 $t \in \textsf{SN}$)。

这意味着从 $t$ 出发的任何归约序列 $t \to t_1 \to t_2 \dots$ 都是有限的。
\end{Theorem}

\begin{proof}

证明参考Tait-Girard 方法\cite{Girard1972}\cite{Girard1989}。证明的核心逻辑是利用基本引理将“类型合法性”转化为“可还原性”,再利用可还原项必强规范化的性质。

\textbf{第一步:引入恒等替换 (Identity Substitution)}

对于语境 $\Gamma = \{x_1:A_1, \dots, x_n:A_n\}$,我们构造一个特殊的替换 $\gamma_{id}$:
\[
\gamma_{id} = [x_1/x_1, \dots, x_n/x_n]
\]
为了应用基本引理,我们需要证明 $\gamma_{id}$ 是一个可还原替换。这意味着对于每一个变量 $x_i$,必须证明其自身属于可还原集合,即 $x_i \in \textsf{Red}_{A_i}$。

\textbf{第二步:变量的可还原性 (Reducibility of Variables)}

根据可还原性集合(Girard's $\mathcal{RC}$)的性质,所有 $\textsf{Red}_A$ 集合都满足以下两个关键属性:
\begin{enumerate}
\item CR 1 (SN 包含性):若 $t \in \textsf{Red}_A$,则 $t \in \textsf{SN}$。
\item CR 3 (中性项属性):若 $t$ 是中性项(即变量或变量的应用,且无法进一步归约)且其所有单步归约项都在 $\textsf{Red}_A$ 中,则 $t \in \textsf{Red}_A$。
\end{enumerate}
由于变量 $x_i$ 是基础的中性项且没有归约式,根据 CR 3,推导出 $x_i \in \textsf{Red}_{A_i}$。\\
因此,$\gamma_{id}$ 满足基本引理的前提条件。

\textbf{第三步:应用基本引理 (Application of Fundamental Lemma)}

由于前提 $\Gamma \vdash t : A$ 成立,且 $\gamma_{id}$ 是可还原替换,根据基本引理\ref{lemma:KOSreduction}:
\[
t[\gamma_{id}] \in \textsf{Red}_A
\]
由于 $t[\gamma_{id}]$ 在语法上等同于项 $t$ 本身,我们得到:
\[
t \in \textsf{Red}_A
\]

\textbf{第四步:结论导出}

根据可还原集合的属性 CR 1(所有属于可还原集合的项都是强规范化的):
\[
t \in \textsf{Red}_A \implies t \in \textsf{SN}
\]
至此,证明完毕。
\end{proof}

\begin{Theorem}{类型保持性(Subject Reduction)}

在 KOS-TL Core 中,归约操作不改变项的类型。若 $\Gamma \vdash t : A$ 且 $t \to t'$,则 $\Gamma \vdash t' : A$。
\end{Theorem}

\begin{proof}

基于替换引理\ref{lemma:KOSsubstitution},结构归纳证明如下。

\begin{enumerate}[label=(\arabic*)]
    \item \textbf{主要归约情形:$\beta$-归约}

    考虑最基本的归约步 $(\lambda x:B.t)u \to t[u/x]$:
    \begin{itemize}
        \item 由前提 $\Gamma \vdash (\lambda x:B.t)u : A$ 可知,必存在类型 $B$,使得 $\Gamma \vdash \lambda x:B.t : \Pi(x:B).A'$ 且 $\Gamma \vdash u : B$。
        \item 根据 $\Pi$-引入规则的反转,可知 $\Gamma, x:B \vdash t : A'$。
        \item 应用代换引理,直接得出 $\Gamma \vdash t[u/x] : A'[u/x]$。
    \end{itemize}

    \item \textbf{$\Sigma$-类型规约:$\iota$-规约}

    考虑情形 $\textsf{split}(\langle u, v \rangle, x.y.t) \to t[u/x, v/y]$:
    \begin{itemize}
        \item \textbf{前提}:由 $\Gamma \vdash \textsf{split}(\langle u, v \rangle, x.y.t) : C$ 可知:
        \begin{enumerate}[label=(\alph*)]
            \item $\Gamma \vdash \langle u, v \rangle : \Sigma(x:A).B$
            \item $\Gamma, x:A, y:B \vdash t : C'$ (其中 $C$ 实际上是 $C'[\langle u, v \rangle / z]$)
        \end{enumerate}
        \item \textbf{推导}:从 (a) 依据引入规则反转,得 $\Gamma \vdash u : A$ 且 $\Gamma \vdash v : B[u/x]$。
        \item \textbf{应用}:对 $t$ 连续应用两次代换引理(先换 $x$ 后换 $y$),直接得到:$$ \Gamma \vdash t[u/x, v/y] : C'[\langle u, v \rangle / z] $$ 类型保持一致,结论成立。
    \end{itemize}

\item \textbf{和类型归约:$\iota$-归约}

考虑情形 $\textsf{case}(\textsf{inl}(u), x.M, y.N) \to M[u/x]$:
\begin{itemize}
    \item 由前提 $\Gamma \vdash \textsf{case}(\textsf{inl}(u), x.M, y.N) : A$ 可知:
    \begin{enumerate}[label=(\alph*)]
        \item $\Gamma \vdash \textsf{inl}(u) : B + C$
        \item $\Gamma, x:B \vdash M : A$ 且 $\Gamma, y:C \vdash N : A$。
    \end{enumerate}
    \item 从 (a) 根据 $+$-引入规则的反转,可知 $\Gamma \vdash u : B$。
    \item 应用代换引理(引理\ref{lemma:KOSsubstitution}),由 $\Gamma, x:B \vdash M : A$ 和 $\Gamma \vdash u : B$ 可得 $\Gamma \vdash M[u/x] : A[u/x]$。
    \item 若 $A$ 不依赖于判定项,则 $A[u/x] = A$,结论成立。(对于依赖类型情形,代换同样保持类型的一致性)。
\end{itemize}

\item \textbf{局部定义归约:$\zeta$-归约}

考虑情形 $\textsf{let } x = u \textsf{ in } t \to t[u/x]$:
\begin{itemize}
    \item 由前提 $\Gamma \vdash \textsf{let } x = u \textsf{ in } t : A$ 可知,必存在类型 $B$ 使得 $\Gamma \vdash u : B$ 且 $\Gamma, x:B \vdash t : A$。
    \item 这是一个标准的代换引理应用场景。根据引理\ref{lemma:KOSsubstitution},直接推导得 $\Gamma \vdash t[u/x] : A[u/x]$。
\end{itemize}

\item \textbf{定义展开归约:$\delta$-归约}

考虑情形 $c \to t$,其中 $(c := t:A) \in \Gamma$:
\begin{itemize}
    \item 由前提 $\Gamma \vdash c : A$ 可知,$c$ 是在上下文中声明的常量。
    \item 根据 $\delta$-规约的定义,标识符 $c$ 的类型与其定义体 $t$ 的类型在 $\Gamma$ 中是完全一致的。
    \item 故由 $\Gamma$ 的良构性直接得 $\Gamma \vdash t : A$。
\end{itemize}

\item \textbf{同余情形 (Congruence Cases)}

若归约发生在子项中,例如 $t = f\,u \to f'\,u$(其中 $f \to f'$):
\begin{itemize}
    \item 根据归纳假设,$f'$ 保持了 $f$ 的类型 $\Pi(x:B).A$。
    \item 重新应用 $\Pi$-消去规则,整体项的类型依然为 $A[u/x]$。
    \item 同理可证所有其他构造(配对、投影、注入等)在同余规约下的类型保持性。
\end{itemize}

\end{enumerate}

    由于所有基本的计算归约($\beta, \iota$ 等)均满足类型保持,且归约关系 $\to$ 在上下文构造下是封闭的,通过对归约关系的结构归纳,证明定理对所有归约步成立。
\end{proof}


我们将类型 $A$ 解释为一个由项组成的集合 $\llbracket A \rrbracket$。这些集合必须满足前文提到的可还原性候选者(CR)性质。

\begin{Definition}{类型语义}

\begin{itemize}
\item $\llbracket \textsf{Val} \rrbracket = \{ t \mid t \in \textsf{SN} \land t \text{ 最终归约为数值常量} \}$
\item $\llbracket \textsf{Time} \rrbracket = \{ t \mid t \in \textsf{SN} \land t \text{ 最终归约为合法时间戳} \}$
\item $\llbracket \Pi(x:A).B \rrbracket = \{ f \mid \forall u \in \llbracket A \rrbracket, (f \, u) \in \llbracket B \rrbracket [u/x] \}$
\item $\llbracket \Sigma(x:A).B \rrbracket = \{ p \mid p \twoheadrightarrow \langle u, v \rangle \land u \in \llbracket A \rrbracket \land v \in \llbracket B \rrbracket[u/x] \}$
\item $\llbracket A + B \rrbracket = \{ t \mid t \in \textsf{SN} \land (t \twoheadrightarrow \textsf{inl}(u) \Rightarrow u \in \llbracket A \rrbracket) \land (t \twoheadrightarrow \textsf{inr}(v) \Rightarrow v \in \llbracket B \rrbracket) \}$
\item $\llbracket \textsf{Id}_A(a, b) \rrbracket = \{ \textsf{refl} \mid a, b \in \llbracket A \rrbracket \land a \simeq_{Red} b \}$。其中 $\simeq_{Red}$ 表示它们规约到相同的范式。
\end{itemize}
\end{Definition}



解释函数 $\llbracket t \rrbracket_\rho$ 负责将带变量的语法项转换为其对应的语义值。在强规范化证明中,这种“解释”通常就是代换(Substitution)操作。

\begin{Definition}{项的语义解释}

设 $\rho$ 是一个从变量到语义值的映射(赋值)。
\begin{itemize}
\item $\llbracket x \rrbracket_\rho = \rho(x)$ (直接从环境中读取赋值)
\item $\llbracket \lambda x:A. M \rrbracket_\rho = \text{一个函数 } v \mapsto \llbracket M \rrbracket_{\rho[x \mapsto v]}$
\item $\llbracket f \, a \rrbracket_\rho = \llbracket f \rrbracket_\rho (\llbracket a \rrbracket_\rho)$ (函数作用)
\item $\llbracket \langle a, b \rangle \rrbracket_\rho = (\llbracket a \rrbracket_\rho, \llbracket b \rrbracket_\rho)$ (语义配对)
\item $\llbracket \textsf{refl} \rrbracket_\rho = \textsf{refl}$ (常量解释为其自身)
\end{itemize}
\end{Definition}

为了使 $\llbracket t \rrbracket_\rho \in \llbracket A \rrbracket$ 成立,赋值 $\rho$ 必须是“合法的”。

\begin{Definition}{命题语义的逻辑闭环(Logical Closure of Prop)}

在语义模型 $\mathcal{M}$ 中,$\textsf{Prop}$ 的解释 $\llbracket \textsf{Prop} \rrbracket$ 被定义为所有满足以下性质的项集 $S$ 的集合:
\begin{enumerate}[label=(\arabic*)]
    \item \textbf{SN 性}:$S \subseteq \textsf{SN}$。\item \textbf{CR 属性}:$S$ 对归约和逆向归约封闭,且包含所有中性项。

    \item \textbf{非谓述语义算子}:对于任意集合 $X$ 和函数 $F: X \to \llbracket \textsf{Prop} \rrbracket$,交集运算定义为:

    $$ \llbracket \Pi(x:A).B \rrbracket_\rho = \bigcap_{u \in \llbracket A \rrbracket_\rho} \{ f \mid (f \, u) \in \llbracket B \rrbracket_{\rho[x \mapsto u]} \} $$
\end{enumerate}

\textbf{闭环修正}:由于 $\llbracket \textsf{Prop} \rrbracket$ 包含了它自身构造出的所有 $\Pi$ 类型解释(通过对可还原候选者 $\mathcal{RC}$ 的交集运算),这保证了即使 $A$ 是无限大的 Universe,其映射后的结果依然留在 $\llbracket \textsf{Prop} \rrbracket$ 预定义的集合内。
\end{Definition}

\begin{Definition}{合法赋值}

称赋值 $\rho$ 满足语境 $\Gamma$(记作 $\rho \models \Gamma$),当且仅当对于 $\Gamma$ 中的每一个绑定 $(x:A)$,都有:
\[
\rho(x) \in \llbracket A \rrbracket_\rho
\]
\end{Definition}


\begin{Theorem}{语义可靠性}
\label{Theorem:KosSoundness}

若 $\Gamma \vdash t : A$ 且 $\rho \models \Gamma$,则 $\llbracket t \rrbracket_\rho \in \llbracket A \rrbracket_\rho$。
\end{Theorem}

\begin{proof}

对推导树 $\Gamma \vdash t : A$ 的结构进行归纳。由于 $\textsf{Prop}$ 的非谓述性,其可还原候选者($\mathcal{RC}$)的构造基于 Girard 的分层候选集方法,而非简单的 Tarski 风格语义。

\textbf{A. 变量规则 (Variable)}

假设推导为
\[
\frac{(x:A) \in \Gamma}{\Gamma \vdash x : A}
\]

\begin{itemize}
\item 证明:由前提 $\rho \models \Gamma$ 可知,环境中每个绑定的变量其赋值必属于该类型的解释。因此 $\rho(x) \in \llbracket A \rrbracket_\rho$。
\item 由于 $\llbracket x \rrbracket_\rho = \rho(x)$,结论 $\llbracket x \rrbracket_\rho \in \llbracket A \rrbracket_\rho$ 成立。
\end{itemize}

\textbf{B. $\Pi$-类型引入 ($\lambda$-抽象)}

假设推导最后一步为
\[
\frac{\Gamma, x:A \vdash M : B}{\Gamma \vdash \lambda x:A. M : \Pi(x:A). B}
\]

\begin{itemize}
\item 目标:证 $\llbracket \lambda x:A. M \rrbracket_\rho \in \llbracket \Pi(x:A). B \rrbracket_\rho$。
\item 根据 $\Pi$ 的语义定义,需证:对任意 $u \in \llbracket A \rrbracket_\rho$,有 $(\llbracket \lambda x:A. M \rrbracket_\rho \cdot u) \in \llbracket B \rrbracket_{\rho[x \mapsto u]}$。
\item 由项的解释定义,$(\llbracket \lambda x:A. M \rrbracket_\rho \cdot u) \to_\beta \llbracket M \rrbracket_{\rho[x \mapsto u]}$。
\item 由于 $u \in \llbracket A \rrbracket_\rho$ 且 $\rho \models \Gamma$,则新赋值 $\rho' = \rho[x \mapsto u]$ 满足 $\rho' \models (\Gamma, x:A)$。
\item 应用归纳假设:$\llbracket M \rrbracket_{\rho'} \in \llbracket B \rrbracket_{\rho'}$。
\item 由于可还原集合对逆向归约封闭(CR 属性),故原应用项亦属于该集合。
\end{itemize}

\textbf{C. $\Pi$-类型消去 (应用)}

假设推导最后一步为应用规则:
\[
\frac{\Gamma \vdash f : \Pi(x:A).B \quad \Gamma \vdash u : A}{\Gamma \vdash f \, u : B[u/x]}
\]

\begin{itemize}
\item 证明:由归纳假设,$\llbracket f \rrbracket_\rho \in \llbracket \Pi(x:A).B \rrbracket_\rho$ 且 $\llbracket u \rrbracket_\rho \in \llbracket A \rrbracket_\rho$。
\item 根据 $\Pi$ 类型语义的定义,函数 $\llbracket f \rrbracket_\rho$ 作用于任何属于 $\llbracket A \rrbracket_\rho$ 的元素,其结果必属于 $B$ 的解释。
\item 因此 $\llbracket f \rrbracket_\rho(\llbracket u \rrbracket_\rho) \in \llbracket B \rrbracket_{\rho[x \mapsto \llbracket u \rrbracket_\rho]}$。
\item 根据替换引理,该集合即为 $\llbracket B[u/x] \rrbracket_\rho$。
\end{itemize}

\textbf{D. $\Sigma$-类型引入 (配对)}

假设推导为:
\[
\frac{\Gamma \vdash a : A \quad \Gamma \vdash b : B[a/x]}{\Gamma \vdash \langle a, b \rangle : \Sigma(x:A). B}
\]

\begin{enumerate}
\item 由归纳假设,$\llbracket a \rrbracket_\rho \in \llbracket A \rrbracket_\rho$。
\item 由归纳假设,$\llbracket b \rrbracket_\rho \in \llbracket B[a/x] \rrbracket_\rho$。
\item 根据 $\Sigma$ 的语义定义:$\llbracket \Sigma(x:A). B \rrbracket_\rho = \{ (u, v) \mid u \in \llbracket A \rrbracket_\rho, v \in \llbracket B \rrbracket_{\rho[x \mapsto u]} \}$。
\item 结合替换引理 $\llbracket B[a/x] \rrbracket_\rho = \llbracket B \rrbracket_{\rho[x \mapsto \llbracket a \rrbracket_\rho]}$,可知 $\llbracket \langle a, b \rangle \rrbracket_\rho$ 的两个分量完全符合定义。
\end{enumerate}

\textbf{E. 依存求和消解 ($\Sigma$-消解 / split)}

假设推导的最后一步应用了 $\Sigma$ 消解规则:
\[
\frac{\Gamma \vdash p : \Sigma(x:A).B \quad \Gamma, x:A, y:B \vdash t : C(\langle x, y \rangle)}{\Gamma \vdash \textsf{split}(p, x.y.t) : C(p)}
\]
我们需要证明:若 $\rho \models \Gamma$,则 $\llbracket \textsf{split}(p, x.y.t) \rrbracket_\rho \in \llbracket C(p) \rrbracket_\rho$。

\begin{enumerate}
\item \textbf{语义前提推导} \\
根据归纳假设(Inductive Hypothesis):\\
\begin{itemize}
\item 对于项 $p$,有 $\llbracket p \rrbracket_\rho \in \llbracket \Sigma(x:A).B \rrbracket_\rho$。
\item 根据 $\Sigma$ 类型的语义定义,存在 $u \in \llbracket A \rrbracket_\rho$ 且 $v \in \llbracket B \rrbracket_{\rho[x \mapsto u]}$,使得 $\llbracket p \rrbracket_\rho$ 在语义上等价于对 $(u, v)$。
\end{itemize}

\item \textbf{构造合法赋值} \\
定义一个新的赋值 $\rho' = \rho[x \mapsto u, y \mapsto v]$。\\
\begin{itemize}
\item 由于 $u \in \llbracket A \rrbracket_\rho$,则 $\rho[x \mapsto u] \models (\Gamma, x:A)$。
\item 由于 $v \in \llbracket B \rrbracket_{\rho[x \mapsto u]}$,且 $\rho[x \mapsto u]$ 满足前置上下文,则 $\rho' \models (\Gamma, x:A, y:B)$。
\end{itemize}

\item \textbf{应用归纳假设} \\
对消解体 $t$ 应用归纳假设:\\
\[
\llbracket t \rrbracket_{\rho'} \in \llbracket C(\langle x, y \rangle) \rrbracket_{\rho'}
\] \\
根据项的语义解释定义:\\
\[
\llbracket \textsf{split}(p, x.y.t) \rrbracket_\rho = \llbracket t \rrbracket_{\rho[x \mapsto \pi_1(\llbracket p \rrbracket_\rho), y \mapsto \pi_2(\llbracket p \rrbracket_\rho)]}
\] \\
代入 $\llbracket p \rrbracket_\rho$ 的分量 $u$ 和 $v$,得:\\
\[
\llbracket \textsf{split}(p, x.y.t) \rrbracket_\rho = \llbracket t \rrbracket_{\rho'}
\]

\item \textbf{类型的一致性 (Conversion)} \\
为了完成证明,必须确保结果所属的集合一致。\\
根据依存类型的代换性质:\\
\[
\llbracket C(\langle x, y \rangle) \rrbracket_{\rho'} = \llbracket C \rrbracket_{\rho[p' \mapsto (u, v)]}
\] \\
由于 $\llbracket p \rrbracket_\rho \simeq (u, v)$(在可还原性等价意义下),且 $\textsf{Red}$ 集合对计算等价性封闭:\\
\[
\llbracket C \rrbracket_{\rho[p' \mapsto \llbracket p \rrbracket_\rho]} = \llbracket C(p) \rrbracket_\rho
\] \\
因此,$\llbracket t \rrbracket_{\rho'} \in \llbracket C(p) \rrbracket_\rho$,结论成立。
\end{enumerate}

\textbf{F. 关于 $\iota$-规约的语义保值性 (Reduction Invariance)}

为了支撑上述步骤,我们需要证明 $\iota$-规约步不改变语义属性。对于 $\Sigma$ 类型:
\[
\textsf{split}(\langle u, v \rangle, x.y.t) \to_\iota t[u/x, v/y]
\]
\begin{enumerate}
\item \textbf{语义一致性}:根据定义,左侧的解释 $\llbracket \textsf{split}(\langle u, v \rangle, x.y.t) \rrbracket_\rho$ 会展开为 $\llbracket t \rrbracket_{\rho[x \mapsto \llbracket u \rrbracket_\rho, y \mapsto \llbracket v \rrbracket_\rho]}$。

\item \textbf{替换引理}:根据替换引理(Lemma \ref{lemma:KOSsubstitution}),右侧项的解释 $\llbracket t[u/x, v/y] \rrbracket_\rho$ 与上述展开形式在语义上完全一致。

\item \textbf{结论}:由于两者的语义解释在集合论意义下是同一个元素,且 $\textsf{Red}_C$ 满足 CR 2(稳定性),规约后的项依然保持在对应的可还原集合中。
\end{enumerate}

\textbf{F. $+$ 类型引入 (注入)}

假设推导最后一步为左注入规则(右注入 $\textsf{inr}$ 同理):

$$\frac{\Gamma \vdash a : A \quad \Gamma \vdash B : \mathcal{U}}{\Gamma \vdash \textsf{inl}_B(a) : A + B}$$

\begin{itemize}
    \item \textbf{证明}:由归纳假设,$\llbracket a \rrbracket_\rho \in \llbracket A \rrbracket_\rho$。
    \item 根据和类型的语义定义:$\llbracket A + B \rrbracket_\rho = \{ \textsf{inl}(u) \mid u \in \llbracket A \rrbracket_\rho \} \cup \{ \textsf{inr}(v) \mid v \in \llbracket B \rrbracket_\rho \} \cup \textsf{Neutral}$。
    \item 由于 $\llbracket \textsf{inl}_B(a) \rrbracket_\rho = \textsf{inl}(\llbracket a \rrbracket_\rho)$,且已知 $\llbracket a \rrbracket_\rho \in \llbracket A \rrbracket_\rho$。
    \item 按照集合构造,$\textsf{inl}(\llbracket a \rrbracket_\rho)$ 显然属于 $\llbracket A + B \rrbracket_\rho$ 的左分支定义部分。
\end{itemize}

latex\textbf{G. 和类型消去 ($+$-消解 / case)}

假设推导的最后一步应用了 $+$ 消解规则:
\[
\frac{\Gamma \vdash s : A + B \quad \Gamma, x:A \vdash t : C \quad \Gamma, y:B \vdash u : C}{\Gamma \vdash \textsf{case}(s, x.t, y.u) : C}
\]
我们需要证明:若 $\rho \models \Gamma$,则 $\llbracket \textsf{case}(s, x.t, y.u) \rrbracket_\rho \in \llbracket C \rrbracket_\rho$。

\begin{enumerate}
\item \textbf{分支前提分析} \\
根据归纳假设(IH):\\
\begin{itemize}
\item 对于判定项 $s$,有 $\llbracket s \rrbracket_\rho \in \llbracket A + B \rrbracket_\rho$。
\item 根据 $A+B$ 的语义定义,$\llbracket s \rrbracket_\rho$ 必须强规范化(SN),且最终归约为 $\textsf{inl}(a)$ 或 $\textsf{inr}(b)$ 形式。
\end{itemize}

\item \textbf{分支讨论 (Case Analysis)} \\
我们需要分两种语义路径讨论:\\
\begin{itemize}
\item \textbf{路径一:左注入 ($\textsf{inl}$)} \\
  \begin{enumerate}
  \item 假设 $\llbracket s \rrbracket_\rho \twoheadrightarrow \textsf{inl}(a)$,由定义知 $a \in \llbracket A \rrbracket_\rho$。
  \item 构造赋值 $\rho_x = \rho[x \mapsto a]$。由于 $a \in \llbracket A \rrbracket_\rho$,则 $\rho_x \models (\Gamma, x:A)$。
  \item 对左分支项 $t$ 应用归纳假设:$\llbracket t \rrbracket_{\rho_x} \in \llbracket C \rrbracket_{\rho_x}$。
  \item 由于 $C$ 在此规则中不依赖于 $s$ 的具体值(简单消解情形),$\llbracket C \rrbracket_{\rho_x} = \llbracket C \rrbracket_\rho$。
  \end{enumerate}

\item \textbf{路径二:右注入 ($\textsf{inr}$)} \\
  \begin{enumerate}
  \item 假设 $\llbracket s \rrbracket_\rho \twoheadrightarrow \textsf{inr}(b)$,由定义知 $b \in \llbracket B \rrbracket_\rho$。
  \item 构造赋值 $\rho_y = \rho[y \mapsto b]$。则 $\rho_y \models (\Gamma, y:B)$。
  \item 对右分支项 $u$ 应用归纳假设:$\llbracket u \rrbracket_{\rho_y} \in \llbracket C \rrbracket_{\rho_y}$。
  \item 同理,$\llbracket u \rrbracket_{\rho_y} \in \llbracket C \rrbracket_\rho$。
  \end{enumerate}
\end{itemize}

\item \textbf{语义解释的统一} \\
根据 $\textsf{case}$ 算子的语义定义:
\[
\llbracket \textsf{case}(s, x.t, y.u) \rrbracket_\rho =
\begin{cases}
\llbracket t \rrbracket_{\rho[x \mapsto a]} & \text{if } \llbracket s \rrbracket_\rho \twoheadrightarrow \textsf{inl}(a) \\
\llbracket u \rrbracket_{\rho[y \mapsto b]} & \text{if } \llbracket s \rrbracket_\rho \twoheadrightarrow \textsf{inr}(b)
\end{cases}
\]
无论 $\llbracket s \rrbracket_\rho$ 塌陷到哪个分支,结果都属于 $\llbracket C \rrbracket_\rho$。

\item \textbf{逆向归约封闭性 (CR 3 应用)} \\
由于 $\textsf{case}(s, x.t, y.u)$ 是通过规约($\iota$-reduction)到达 $\llbracket t \rrbracket_{\rho_x}$ 或 $\llbracket u \rrbracket_{\rho_y}$ 的,根据可还原集合的 CR 3 属性(以及对逆向规约的封闭性),原始的 $\textsf{case}$ 项本身也必然属于可还原集合 $\llbracket C \rrbracket_\rho$。结论成立。
\end{enumerate}

\end{proof}

可靠性意味着“凡是能证明的,都是真的”。在定理\ref{Theorem:KosSoundness}中,“能证明的”是类型判断 $\Gamma \vdash t : A$,“真的”是项在语义模型中的归属 $\llbracket t \rrbracket \in \llbracket A \rrbracket$。它保证了 KOS-TL Core 的语法构造没有脱离其逻辑语义。

\begin{Theorem}{一致性定理(Consistency Theorem)}

不存在项 $t$ 使得 $\emptyset \vdash t : \bot$。则称系统为一致的。
\end{Theorem}

\begin{proof}

该证明的核心思想是:语法推导不能逃脱语义边界。我们分三个阶段展开论证。

\textbf{第一阶段:利用强范型(Syntactic Normalization)}

根据 KOS-TL Core 的强规范化定理,如果存在一个项 $t$ 满足 $\emptyset \vdash t : \bot$,则 $t$ 必然可以规约到一个范式(Normal Form) $t_{nf}$,且类型保持不变:
\[
\emptyset \vdash t_{nf} : \bot
\]
在空上下文中,范式只能是构造项(Constructor)。然而,根据 $\bot$ 类型的定义,它没有引入规则(Introduction Rules),即不存在任何构造算子能产生 $\bot$。 这意味着在语法层面上,$t_{nf}$ 是不存在的。

\textbf{第二阶段:语义可靠性映射(Semantic Soundness)}

为了使论证在数学上无懈可击,我们使用解释模型 $\mathcal{M}$。

\textbf{步骤 1:建立映射}\\
解释函数 $\llbracket \cdot \rrbracket$ 将语法世界(Types/Terms)映射到语义世界(Sets/Elements)。

\begin{itemize}
\item 对于任何类型 $A$,其解释 $\llbracket A \rrbracket$ 是一个集合。
\item 对于任何项 $t : A$,其解释 $\llbracket t \rrbracket$ 必须是集合 $\llbracket A \rrbracket$ 中的一个元素。
\end{itemize}

\textbf{步骤 2:空类型的特殊性}

在定义 $\bot$ 的语义时,我们将其映射为数学上的绝对空集:
\[
\llbracket \bot \rrbracket = \emptyset
\]
这是合理的,因为逻辑上的“伪”在模型论中对应的就是没有任何见证者(Witness)的状态。

\textbf{步骤 3:应用语义可靠性}

根据语义可靠性定理\ref{Theorem:KosSoundness}:
\[
\text{若 } \Gamma \vdash t : A, \text{ 则对所有合法赋值 } \rho, \llbracket t \rrbracket_\rho \in \llbracket A \rrbracket
\]
在空上下文 $\emptyset$ 下,不需要任何赋值 $\rho$,直接得到:
\[
\llbracket t \rrbracket \in \llbracket \bot \rrbracket
\]

\textbf{第三阶段:归谬与矛盾(Reduction to Absurdity)}

\begin{enumerate}[label=(\arabic*)]
\item 由上述推导得:$\llbracket t \rrbracket \in \emptyset$。
\item 根据集合论的外延公理与空集公理:$\forall x, x \notin \emptyset$。
\item 判定:$\llbracket t \rrbracket \in \emptyset$ 与 $\forall x, x \notin \emptyset$ 构成了直接的逻辑矛盾。
\item 回溯:由于语义解释和集合论公理是预设正确的,矛盾的源头只能是假设“存在项 $t$”。
\end{enumerate}

结论:假设不成立,$\neg \exists t, \emptyset \vdash t : \bot$。一致性得证。
\end{proof}

在 KOS-TL Core 这种基于柯里-霍华德同构的系统中,逻辑一致性等价于证明空类型(Empty Type)是不可居住的(Uninhabited)。

推论:逻辑一致性 (Consistency)

由于 KOS-TL 核心层满足强规范化(SN)且具有类型保持性(Subject Reduction),且系统中不存在空类型 $\bot$ 的构造子,因此不存在项 $t$ 使得 $\vdash t : \bot$。这证明了核心层作为“形式宪法”在逻辑上是无矛盾的。

在类型论的严谨定义下,KOS-TL Core 层是绝对可判定的(Decidable)。Core 层作为整个系统的数学基石,必须在理论上保证所有基本操作(类型检查、等价判定)在任何情况下都能在有限步内停机。

\begin{Definition}{合流性(Confluence)}

对于任意的 Core 层项 $M, N, P$,如果存在规约路径使得 $M \twoheadrightarrow N$ 且 $M \twoheadrightarrow P$,则必然存在一个项 $Q$,使得 $N \twoheadrightarrow Q$ 且 $P \twoheadrightarrow Q$。

\end{Definition}

设 $\to$ 为单步规约关系(包含 $\beta, \delta, \zeta, \eta$ 规约),$\twoheadrightarrow$ 为其传递闭包(多步规约)。合流性在逻辑上意味着所有的分叉最终都会汇合。

\begin{Theorem}{KOS-TL Core 合流性定理}

KOS-TL Core 层的所有良构项均满足合流性,且具有唯一的范式(Unique Normal Form)。
\end{Theorem}
\begin{proof}

我们采用 Tait-Martin-L\"{o}f 平行规约法 (Parallel Reduction) 结合强规范化性质进行证明。

\textbf{步骤 A:定义平行规约 ($\Rightarrow$)}

为了处理单步规约无法覆盖的“分叉同时规约”问题,我们定义一种平行规约关系 $\Rightarrow$,它是单步规约 $\rightarrow$ 的推广,允许同时规约项的多个子部分。具体地,$\Rightarrow$ 是最小关系,满足以下规则:
\begin{enumerate}
\item 若 $M \rightarrow M'$(单步 $\beta$-规约),则 $M \Rightarrow M'$。
\item 对于任意规约上下文 $C[\cdot]$,若 $M \Rightarrow M'$,则 $C[M] \Rightarrow C[M']$。
\item 并行应用规约:
$$\frac{M \Rightarrow M' \quad N \Rightarrow N'}{(\lambda x. M) N \Rightarrow M'[x := N']}$$
其中 $[x := N']$ 表示捕获避免的替换。
\item 对于构造子(如 $\Pi x : A. M$),允许同时规约域和体:
$$\frac{A \Rightarrow A' \quad M \Rightarrow M'}{\Pi x : A. M \Rightarrow \Pi x : A'. M'}$$
($\Sigma$ 和 $\mathsf{Id}$ 类型类似)。
\end{enumerate}

\textbf{步骤 B:证明钻石性质 (Diamond Property)}

我们证明平行规约 $\Rightarrow$ 满足钻石性质:若 $M \Rightarrow N$ 且 $M \Rightarrow P$,则存在 $Q$ 使得 $N \Rightarrow Q$ 且 $P \Rightarrow Q$。
证明通过对 $M$ 的结构进行归纳:
\begin{itemize}
\item \textbf{基础情形}:$M$ 是变量或原子,则 $N = P = M$,取 $Q = M$。
\item \textbf{归纳情形}:假设对所有真子项成立。

\textbf{情形:$M = (\lambda x. M_1) M_2$}:由 $\Rightarrow$ 定义,$N$ 和 $P$ 源于不同子规约分叉。归纳假设应用于 $M_1$ 和 $M_2$,存在 $Q_1, Q_2$ 使得子项汇合,则 $Q = Q_1[x := Q_2]$。
\textbf{情形:构造子(如 $\Pi x : A. M$)}:KOS-TL Core 的构造子正交(无重叠规约规则),归纳应用于 $A$ 和 $M$,汇合为 $\Pi x : A''. M''$。$\mathsf{Id}$ 类型类似,无内部冲突。
\end{itemize}
由正交性和类型守恒,规约无批判对,确保钻石性质。

\textbf{步骤 C:从平行规约推导到多步规约}

多步规约 $\twoheadrightarrow$ 是 $\Rightarrow$ 的反射-传递闭包:$M \twoheadrightarrow N$ 当且仅当存在链 $M = M_0 \Rightarrow M_1 \Rightarrow \cdots \Rightarrow M_k = N$。
由 Newman 引理,若 $\Rightarrow$ 满足钻石性质且系统强规范化(SN,无无限链),则 $\twoheadrightarrow$ 满足合流性:对于 $M \twoheadrightarrow N_1$ 和 $M \twoheadrightarrow N_2$,存在 $Q$ 使得 $N_1 \twoheadrightarrow Q$ 和 $N_2 \twoheadrightarrow Q$。SN 确保弱规范化,钻石蕴涵局部合流。
\textbf{步骤 D:唯一范式证明}
假设 $M$ 有两个范式 $N_1$ 和 $N_2$(不可再规约)。由合流性,存在 $Q$ 使得 $N_1 \twoheadrightarrow Q$ 和 $N_2 \twoheadrightarrow Q$。 但 $N_1, N_2$ 是范式,故 $N_1 = Q = N_2$。由 SN,每项有范式,故唯一。
\end{proof}

KOS-TL Core 的判定性主要由以下两个性质支撑:
\begin{enumerate}
\item 类型检查的可判定性:给定 $\Gamma, t, A$,判定 $\Gamma \vdash t : A$ 是否成立。
\item 等价判定的可判定性:给定 $\Gamma, t, u$,判定 $t \equiv u$(在当前上下文下两项是否逻辑等价)是否成立。
\end{enumerate}

Core 层作为整个系统的数学基石,必须在理论上保证所有基本操作(类型检查、等价判定)在任何情况下都能在有限步内停机。

\begin{Theorem}{Core 层可判定性定理}

KOS-TL Core 语言的类型检查问题和项的等价判定问题是可判定的。
\end{Theorem}

\begin{proof}

该证明建立在强规范化(Strong Normalization, SN)和合流性(Confluence)的基础之上。

\textbf{1. 归约序列的有限性}

根据 Tait-Girard 方法证明的强规范化定理,KOS-TL Core 中的任何良构项 $t$ 不存在无限长的规约序列。这意味着从任何项开始,通过 $\beta, \delta, \zeta$ 等规约步,必然在有限步内达到唯一的范式 $\textsf{nf}(t)$。

\textbf{2. 等价判定的算法化}

判定 $t \equiv u$ 的过程可以转化为:
\begin{enumerate}
\item 将 $t$ 规约至范式 $t^*$。
\item 将 $u$ 规约至范式 $u^*$。
\item 比较 $t^*$ 和 $u^*$ 是否在字面上(Syntactically)完全一致。
\end{enumerate}
由于步骤 1 和 2 保证在有限时间内完成,且步骤 3 是简单的符号匹配,因此 $t \equiv u$ 是可判定的。

\textbf{3. 类型检查算法的递归停机}

类型检查器在处理项 $t$ 时,按照其语法结构进行递归:
\begin{itemize}
\item 对于应用项 $(f \, a)$,检查 $f$ 的类型是否为 $\Pi$-类型,并判定 $a$ 的类型是否匹配。
\item 类型匹配过程中调用上述的等价判定算法。
\end{itemize}

由于项的构造是有限的,且等价判定是可判定的,整个递归过程必然停机。
\end{proof}



\subsection{应用示例:质量异常知识建模}

在制造业场景中,一个“合格批次”不仅仅是一个数据记录,它必须包含质检通过的证据。

\begin{itemize}
\item \textbf{类型定义}:

$$\textsf{QualifiedBatch} \equiv \Sigma(b : \textsf{BatchID}). \Sigma(res : \textsf{Result}). \textsf{Id}_{\textsf{Result}}(res, \textsf{Pass})$$

\item \textbf{逻辑释义}:
该类型要求任何实例必须包含一个批次 ID、一个质检结果,以及一个证明该结果等于  的恒等项。
\item \textbf{构造尝试}:
若批次  的质检结果为 ,则由于类型  是空的(无构造子),系统在核心层将拒绝实例化该对象,从而在逻辑层阻止了不合格品进入后续流程。
\end{itemize}


\begin{Example}{构造一个在安全范围内的温度读数}

(1)声明原子谓词(作为公理或基本判定)

在 Core 层的初始化上下文中,我们需要引入量纲判定谓词(is\_unit):

$$\Gamma \vdash \textsf{is\_unit\_Celsius} : \Pi(v: \textsf{Val}). \textsf{Prop}$$

此外,引入比较谓词构造 $\textsf{is\_safe}$

$$\Gamma \vdash L, H : \textsf{Val}$$$$\Gamma \vdash \textsf{is\_safe} \equiv \lambda v. \textsf{And}(L \le v, v \le H) : \textsf{Val} \to \textsf{Prop}$$

(2)构造 $Temp$ 类型

利用 $\Sigma$-类型构造规则:

$$\frac{\Gamma \vdash \textsf{Val} : \mathcal{U} \quad \Gamma, v:\textsf{Val} \vdash \textsf{is\_unit\_Celsius}(v) : \textsf{Prop}}{\Gamma \vdash \Sigma(v: \textsf{Val}). \textsf{is\_unit\_Celsius}(v) : \mathcal{U}}$$

此时,$Temp \equiv \Sigma(v: \textsf{Val}). \textsf{is\_unit\_Celsius}(v)$ 正式成为一个合法的类型。

(3)在此基础上构造 $QualifiedTemp$

 现在,我们再叠加逻辑安全谓词 $\textsf{is\_safe}$:

 $$QualifiedTemp \equiv \Sigma(t: Temp). \textsf{is\_safe}(\textsf{proj}_1(t))$$

(4)构造对象实例

  假设 $v = 25$:$p_{unit} : \textsf{is\_unit\_Celsius}(25)$$p_{range} : \textsf{is\_safe}(25)$$obj = \langle \langle 25, p_{unit} \rangle, p_{range} \rangle$

\end{Example}

\section{KOS-TL 内核层 (Kernel Layer):状态演化与操作语义}

内核层(Kernel Layer)是 KOS-TL 的动力学核心。它基于核心层的静态类型系统,引入了时间维度与状态转移机制,通过受控的事件驱动实现知识库从状态  向  的原子演化。

\subsection{语法 (Syntax)}

KOS-TL内核层引入了“动态”的概念,论域从静态对象扩展到了状态转移轨迹。它是连接逻辑与执行的桥梁。

KOS-TL内核层可以表征为一个三元组:

$$\langle \Sigma, \textsf{Ev}, \Delta \rangle$$

\begin{Definition}{状态 ($\Sigma$)}

    内核层状态($\Sigma$)定义为一个配置三元组:

    $$\Sigma \equiv \langle \mathcal{K}, \mathcal{TS}, \mathcal{P} \rangle$$

\begin{enumerate}[label=(\arabic*)]
    \item 知识库 ($\mathcal{K}$ - Knowledge Base)

    $$\mathcal{K} = \{ (id_i, t_i, A_i) \mid \Gamma_{Core} \vdash t_i : A_i \}$$

    它存储了系统当前所有已证实的真理(Facts)
    \item 逻辑时钟 ($\mathcal{TS}$ - Logical Clock)

    基于 Core 层基础目 $\textsf{Time}$。它不仅仅是一个计数器,而是状态全序关系的度量。

    $$\mathcal{TS} : \textsf{Time} \quad \text{满足单调性规则:} \Sigma \to \Sigma' \implies \mathcal{T}' > \mathcal{T}$$

    \item  挂起队列 ($\mathcal{P}$ - Pending Events)

    一个由受限的事件(Events)组成的有序序列。

\end{enumerate}
\end{Definition}


\begin{Definition}{事件 $\textsf{Ev}$ }

事件 $\textsf{Ev}$ 是一个在全局上下文 $\Gamma$ 下良构的五元组:

$$\textsf{Ev} \equiv \langle \textsf{Args}, \textsf{Pre}, \textsf{Op}, \textsf{Post}, \textsf{Prf} \rangle$$

其中各组件的类型限定与逻辑语义为:
\begin{enumerate}[label=(\arabic*)]
    \item $\textsf{Args}$ (参数集):

        $$\textsf{Args} : A$$(其中 $A \in \mathcal{U}_{Core}$)。

        从外部世界(Runtime 层)摄入的实例化数据,如 sensor\_value 或 transaction\_amount。
    \item $\textsf{Pre}$ (前置条件谓词):

        $$\textsf{Pre} : \textsf{Args} \to \Sigma \to \textsf{Prop}$$

        一个依存命题,定义了该事件在当前状态 $\Sigma$ 下必须满足的逻辑前提。它可以引用当前知识库 $\mathcal{K}$ 或逻辑时间 $\mathcal{TS}$。
    \item $\textsf{Op}$ (操作算子):

        $$\textsf{Op} : \textsf{Args} \to \Sigma \to \Sigma$$

        核心的演化函数。它描述了如何生成新状态 $\Sigma'$。
        \begin{itemize}
            \item $\mathcal{K} \to \mathcal{K}'$:知识项的增减。
            \item $\mathcal{TS} \to \mathcal{TS} + \Delta t$:逻辑时钟的步进。
            \item $\mathcal{P} \to \mathcal{P}'$:挂起意图队列的更新(消耗自身,或派生新意图)。
        \end{itemize}

    \item $\textsf{Post}$ (后置约束/不变式):

    $$\textsf{Post} : \Sigma' \to \textsf{Prop}$$

定义转换后必须满足的安全准则(Safety Properties),如“能量守恒”、“账户非负”或“时钟单调性”。

    \item $\textsf{Prf}$ (证明项):

$$\textsf{Prf} : \textsf{Pre}(\textsf{Args}, \Sigma)$$

这是事件的“通行证”。在 Runtime 层将信号精化为事件时,必须构造出前置条件成立的构造性证明。如果没有有效的 $\textsf{Prf}$,内核将拒绝执行该事件。

\end{enumerate}
\end{Definition}


\begin{Definition}{转移记录 ($\Delta$) }

为了支撑因果追溯,$\Delta$ 需要记录时钟的跳变:

$$\Delta \subseteq \Sigma \times \textsf{Ev} \times \Sigma$$

一次典型的转移记录项:

$$\delta = \langle \langle \mathcal{K}, \mathcal{TS}, \mathcal{P} \rangle \xrightarrow{e} \langle \mathcal{K}', \mathcal{TS}', \mathcal{P}' \rangle \rangle$$

$\forall \langle \Sigma \xrightarrow{e} \Sigma' \rangle \in \Delta, \quad \Sigma'.\mathcal{T} > \Sigma.\mathcal{T}$。
这在理论上锁定了“时间箭头”,保证了知识演化的不可逆性。

每一个新注入 $\mathcal{K}'$ 的知识项 $ku_{new}$ 都会隐式携带当前的 $\mathcal{T}'$。

转移发生时,$e$ 从 $\mathcal{P}$ 中出队,执行 $\textsf{Op}$ 后,其结果合并入 $\mathcal{K}'$。
\end{Definition}

\begin{Definition}{演化确定性 (Evolutionary Determinism)}

给定状态 $\Sigma$ 与事件 $e$,若存在满足操作语义的 $\Sigma'$,则其范式(Normal Form)在内涵等价意义下是唯一的。
\end{Definition}

\subsection{操作语义}

内核层的演化遵循“Small-step 操作语义”。令$\Sigma$为系统的配置(Configuration),其状态转移规则定义为:

\begin{equation} \frac{ e = \textsf{head}(\Sigma.\mathcal{P}) \quad \Gamma, \Sigma.\mathcal{K}, \Sigma.\mathcal{TS} \vdash p : \textsf{Pre}(\textsf{Args}_e, \Sigma) \quad \Sigma' = \textsf{Op}(\textsf{Args}_e, \Sigma) \quad \Sigma' \vdash p' : \textsf{Post}(\textsf{Args}_e, \Sigma') }{ \langle \Sigma, e \rangle \longrightarrow_{KOS} \Sigma' }
\end{equation}

其中:
\begin{itemize}
    \item 意图触发条件 ($e = \textsf{head}(\Sigma.\mathcal{P})$)

    明确了事件 $e$ 的来源。转移不是随机发生的,而是由挂起队列 $\mathcal{P}$ 的头部事件驱动。保证了演化的序属性,即内核按照意图队列的逻辑顺序进行调度。

    \item 环境感知的证明判定 ($\Gamma, \Sigma.\mathcal{K}, \Sigma.\mathcal{T} \vdash p$)

    显式列出了证明项 $p$ 所依赖的上下文。前置条件的证明不仅依赖全局上下文 $\Gamma$,还必须与当前知识库 $\mathcal{K}$ 中的事实以及逻辑时间 $\mathcal{TS}$ 保持一致。这落实了“只有在正确的时间和事实基础上,事件才能发生”的逻辑。

    \item 参数化算子应用 ($\Sigma' = \textsf{Op}(\textsf{Args}_e, \Sigma)$)

    引入了 $\textsf{Args}_e$。强调了转移是基于事件携带的具体参数(来自 Runtime 层的精化)对当前三元组配置的整体变换。

    \item 后置约束的自洽性 ($\Sigma' \vdash p' : \textsf{Post}(\textsf{Args}_e, \Sigma')$)

    明确了 $\textsf{Post}$ 是在新状态 $\Sigma'$ 下进行的判定。这定义了逻辑提交(Commit)的硬门槛。如果演化后的状态无法满足其安全不变式(例如:时钟没有步进,或者知识库出现了不一致),则该判定式不成立,转移规则失效。
\end{itemize}

该语义规定了一次有效的知识转移必须同时满足“前提可证”与“结果合规”。若任何一个条件无法在核心层得到证明,状态将保持不变(即执行回滚)。

内核层不负责处理物理失败的重试策略,它仅定义了“逻辑上合法的演化轨迹”。任何未能通过 Post 校验的物理尝试,在 Kernel 层表现为“未发生的转移”,从而在逻辑层强制实现了事务的原子性。

\begin{Example}{规约算例展示}
\label{exam:KOSKernel-reduction}

 假定如下的场景。传感器数据融合假设系统中有两个独立的传感器 $ku_1$(温度)和 $ku_2$(湿度)。我们需要一个合并函数 combine,将它们封装成一个“环境状态”对象。

\begin{enumerate}

    \item 基础类型与谓词定义

    目标类型$Env \equiv \Sigma(t:Temp).(Humi)$,环境状态是一个包含温湿度的依存对。其中:

\centerline{$Temp \equiv \Sigma(v:\textsf{Val}). \textsf{is\_T}(v)$}

\centerline{$Humi \equiv \Sigma(v:\textsf{Val}). \textsf{is\_H}(v)$}

    \item 具体的知识项(实例)
[[
$ku_1 = \langle 25, p_T \rangle : Temp$

$ku_2 = \langle 60, p_H \rangle : Humi$
]]

    \item 合并函数($\Pi$-类型)

    定义一个接收温度和湿度,并返回环境对象的函数:

$$\textsf{combine} \equiv \lambda t:Temp. \lambda h:Humi. \langle t, h \rangle$$

    其类型为$\Pi(t:Temp). \Pi(h:Humi). Env$。
\end{enumerate}

现在我们展示将 $\textsf{combine}$ 应用于 $ku_1$ 和 $ku_2$ 的规约过程。这通常发生在 Kernel 接收到两个信号并试图更新全局状态时。

步骤 1:构造初始应用项在 Kernel 的控制流中,产生了一个待规约的项:

$$(\textsf{combine} \,\, ku_1) \,\, ku_2$$

步骤 2:第一次 $\beta$-规约(替换温度)

根据 $\beta$-规约规则 $(\lambda x. M) N \to M[N/x]$:

$$(\lambda t. \lambda h. \langle t, h \rangle) \,\, ku_1 \to \lambda h. \langle ku_1, h \rangle$$

函数“吃掉”了温度数据,变成了一个“等待湿度输入”的特化函数。

步骤 3:第二次 $\beta$-规约(替换湿度)

$$(\lambda h. \langle ku_1, h \rangle) \,\, ku_2 \to \langle ku_1, ku_2 \rangle$$

湿度数据被填入,生成了一个完整的配对。

步骤 4:展开与结构归约($\iota$-规约)

如果系统需要进一步提取其中的原始数值(例如为了执行 analyze),则会发生 $\iota$- 规约:

$$\textsf{proj}_1(\langle ku_1, ku_2 \rangle) \to ku_1 = \langle 25, p_T \rangle$$

$$\textsf{proj}_1(\textsf{proj}_1(\langle ku_1, ku_2 \rangle)) \to 25$$

上述规约过程都伴随类型判定过程,根据类型保持性 (Subject Reduction),整个规约过程中的每一项都必须是良构的:

起始项:$(\textsf{combine} \,\, ku_1) \,\, ku_2$ 具有类型 $Env$。

中间项:$\lambda h. \langle ku_1, h \rangle$ 具有类型 $\Pi(h:Humi). Env$。

最终项:$\langle ku_1, ku_2 \rangle$ 具有类型 $Env$。

在这个算例中,规约操作完成了从 “逻辑意图”(如何合并数据)到 “逻辑事实”(已合并的数据对象)的转化。在 Core 层看来,$(\textsf{combine} \,\, ku_1) \,\, ku_2$ 与 $\langle ku_1, ku_2 \rangle$ 是判断等价(Judgmentally Equal)的。它们是同一个真理的不同表达形式。在 Kernel 层看来,规约是一次计算求值。它消耗了 CPU 周期,将两个分散的内存指针合并到了一个新的 $\Sigma$ 结构体中。
\end{Example}

KOS-TL 通过静态类型语义在逻辑执行之前构建“防火墙”,类型不匹配拦截 (Type Mismatch Interception)发生在规约之前。根据核心层的判定规则,如果项(Term)无法通过类型检查,它就永远不会被推入 Kernel 的规约引擎。

假设我们有合并函数 combine,它期待一个湿度对象 $Humi$:$$\textsf{combine} : \Pi(t:Temp). \Pi(h:Humi). Env$$现在,Runtime 层错误地捕获了一个压力信号 $p : Press$,并试图执行合并:

$$(\textsf{combine} \,\, ku_1) \,\, p$$

Kernel 调用 $\Pi$-消解规则(Application Rule):

$$\frac{\Gamma \vdash f : \Pi(h:Humi).Env \quad \Gamma \vdash p : A}{\Gamma \vdash f\,p : Env [p/h] \quad (\text{要求 } A \equiv Humi)}$$

内核尝试判定 $Press \equiv Humi$。

$Humi \equiv \Sigma(v:\textsf{Val}). \textsf{is\_H}(v)$

$Press \equiv \Sigma(v:\textsf{Val}). \textsf{is\_P}(v)$

由于谓词 $\textsf{is\_H} \neq \textsf{is\_P}$,类型合一(Unification)失败。

因此,项 $(\textsf{combine} \,\, ku_1) \,\, p$ 被判定为非良构(Ill-typed)。规约引擎拒绝执行 $\beta$-规约,系统状态 $\sigma$ 保持不变,同时触发一个类型错误异常。

\begin{Example}{因果回溯分析 (Causal Backtracking Analysis)}

    基于例\ref{exam:KOSKernel-reduction},当系统发现合并后的结果虽然“类型正确”,但“逻辑异常”(例如 $Env$ 的值超出了安全范围)时,就需要利用 Id 类型(等价类型) 进行因果回溯。

    假设我们已经得到了合并对象 $obj = \langle ku_1, ku_2 \rangle$,但 analyze 谓词判定其非法。

    回溯过程遵循以下逻辑归约:

\begin{enumerate}[label=(\arabic*)]

    \item 解构对象 (Deconstruction)

    通过 Core 层的投影算子 $\textsf{proj}_i$,将复合对象拆解回原始证据:

$$t = \textsf{proj}_1(obj) \quad h = \textsf{proj}_2(obj)$$

    \item 等价性验证 (Identity Verification)

    内核构造一个等价性声明,要求证明当前数据与输入源是一致的:

$$\textsf{Id}_{Temp}(t, ku_1) \wedge \textsf{Id}_{Humi}(h, ku_2)$$

    如果 refl(自反性证明)无法在此处构造,说明合并过程中发生了计算错误或内存污染。

    \item 定位根因 (Root Cause Localization)

    回溯分析函数 analyze 会沿着规约步逆向搜索。在 KOS-TL 中,这表现为对证明项的检查:检查 $ku_1$ 的右投影 $\textsf{proj}_2(ku_1)$,即温度安全证明 $p_T$。 若 $p_T$ 校验失败,则判定:根因在于传感器 1 的输入数据。若 $p_T$ 校验通过,则判定:根因在于合并函数 combine 的逻辑演算。
\end{enumerate}
\end{Example}

\subsection{通用算子}

\subsubsection{1. 状态投影算子 (State Projection Operators)}

投影算子定义了从复杂依存对(Dependent Pairs)中提取组件的逻辑。

\begin{itemize}

\item \textbf{知识提取算子 ($\textsf{get\_K}$)}
\begin{itemize}
\item \textbf{Core 类型}: $\textsf{get\_K} : \Pi(\sigma : \Sigma). \textsf{Set}(\textsf{Facts})$
\item \textbf{算子定义}: $\textsf{get\_K} \equiv \lambda \sigma. \textsf{proj}_1(\sigma)$
\item \textbf{说明}: 该算子利用第一投影提取知识库 $\mathcal{K}$。在 Core 层中,它确保了返回的集合项皆为良构的类型实例。
\end{itemize}
\item \textbf{时钟读取算子 ($\textsf{now}$)}
\begin{itemize}
    \item \textbf{Core 类型}: $\textsf{now} : \Pi(\sigma : \Sigma). \textsf{Time}$
    \item \textbf{算子定义}: $\textsf{now} \equiv \lambda \sigma. \textsf{proj}_1(\textsf{proj}_2(\sigma))$
    \item \textbf{说明}: 提取三元组中间项 $\mathcal{T}$。该算子是所有时序逻辑判断(如“合同是否过期”)的基础。
\end{itemize}
\end{itemize}

\subsubsection{2. 意图调度算子 (Intention Scheduling Operators)}

调度算子通过递归列表操作来管理意图队列 $\mathcal{P}$。

\begin{itemize}
\item \textbf{意图压入算子 ($\textsf{schedule}$)}
\begin{itemize}
\item \textbf{Core 类型}: $\textsf{schedule} : \Pi(\sigma : \Sigma). \Pi(e : \textsf{Ev}). \Sigma$
\item \textbf{算子定义}: $\textsf{schedule} \equiv \lambda \sigma. \lambda e. \langle \textsf{get\_K}(\sigma), \textsf{now}(\sigma), \textsf{append}(\textsf{proj}_2(\textsf{proj}_2(\sigma)), e) \rangle$
    \item \textbf{说明}: 该算子构造一个新的 $\Sigma$ 实例。其核心是在队列末尾追加一个符合 $\textsf{Ev}$ 类型限制的五元组。
\end{itemize}\end{itemize}

\subsubsection{3. 演化控制算子 (Evolution Control Operators)}

这是驱动系统向前演化的核心,涉及计算与一致性判定的融合。
\begin{itemize}
\item \textbf{时钟步进算子 ($\textsf{tick}$)}
\begin{itemize}
\item \textbf{Core 类型}: $\textsf{tick} : \Pi(\sigma : \Sigma). \Sigma$
\item \textbf{算子定义}: $\textsf{tick} \equiv \lambda \sigma. \langle \textsf{get\_K}(\sigma), \textsf{now}(\sigma) + 1, \textsf{consume}(\sigma) \rangle$
    \item \textbf{说明}: 它不仅增加时间计数,通常还伴随着当前事件的消耗,代表逻辑步进的完成。\end{itemize}
    \item \textbf{知识合一算子 ($\textsf{unify}$)}
\begin{itemize}
    \item \textbf{Core 类型}: $\textsf{unify} : \Pi(\sigma : \Sigma). \Pi(f : \textsf{Fact}). \Sigma$
    \item \textbf{算子定义}: $\textsf{unify} \equiv \lambda \sigma. \lambda f. \textsf{if } \textsf{is\_consistent}(\textsf{get\_K}(\sigma), f) \text{ then } \langle \textsf{get\_K}(\sigma) \cup \{f\}, \textsf{now}(\sigma), \dots \rangle \text{ else } \sigma$
    \item \textbf{说明}: 这是最复杂的算子。它在合并新事实前,利用 Core 层的判定规则验证 $f$ 与现有 $\mathcal{K}$ 的逻辑相容性(Consistency)。
\end{itemize}
\end{itemize}

\subsubsection{4. 因果追溯算子 (Causal \& Trace Operators)}

利用等价类型(Identity Type)进行深层审计。

\begin{itemize}
\item \textbf{等价校验算子 ($\textsf{verify\_id}$)}
\begin{itemize}
\item \textbf{Core 类型}: $\textsf{verify\_id} : \Pi(\sigma_1 : \Sigma). \Pi(\sigma_2 : \Sigma). \textsf{Type}$
\item \textbf{算子定义}: $\textsf{verify\_id} \equiv \lambda \sigma_1. \lambda \sigma_2. \textsf{Id}_{\Sigma}(\sigma_1, \sigma_2)$
\item \textbf{说明}: 返回一个命题类型(Prop)。若要在 Kernel 中执行,必须提供构造性证明(如 $\textsf{refl}$),用以验证两个配置在逻辑上是否为同一个真理。
\end{itemize}
\end{itemize}

当上述 Core 层算子被 Kernel 层调用时,其执行遵循以下规约路径:
\begin{enumerate}
\item \textbf{参数替换 ($\beta$- 规约)}: 将 Kernel 层的当前状态实例(如 $\sigma_{current}$)代入算子的 $\lambda$-项。
\item \textbf{结构展开 ($\iota$-规约)}: 投影算子 $\textsf{proj}$ 提取三元组中的具体组件。
\item \textbf{状态物化}: 规约得到的最终项(如新的 $\Sigma'$)被存入内核存储,成为下一个循环的输入。
\end{enumerate}

\begin{Definition}{算子的终止性 (Termination)}

由于 Core 层基于强规范化(Strong Normalization)的计算模型,所有内核通用算子在有限步内必将终止并给出结果。这从理论上避免了内核在处理状态转移时发生“死循环”。
\end{Definition}

\subsection{内核层逻辑性质}

在 KOS-TL 内核架构中,状态保持性(Preservation),也常被称为类型保持性(Subject Reduction)在状态机维度的扩展。它确保了系统在执行环境演化或事务提交时,逻辑上的“良构性”(Well-formedness)不会因为数据的流入而坍塌。

\begin{Theorem}{状态保持性 - Preservation}

设 $\Gamma$ 为系统全局上下文。若内核状态 $\Sigma$ 是良构的(记作 $\Gamma \vdash \Sigma \,\, \textsf{ok}$),且存在一个由事件 $e$ 触发的转移步骤 $\Sigma \xrightarrow{e} \Sigma'$,则转移后的新状态 $\Sigma'$ 依然是良构的:
\[
\Gamma \vdash \Sigma \,\, \textsf{ok} \quad \wedge \quad \Sigma \xrightarrow{e} \Sigma' \implies \Gamma \vdash \Sigma' \,\, \textsf{ok}
\]
其中,良构性 $\Sigma \,\, \textsf{ok}$ 定义为:对于状态中包含的所有事实项 $ku_i \in \Sigma$,均存在类型 $A_i$ 使得 $\Gamma \vdash ku_i : A_i$,且 $\Sigma$ 满足一致性 $\Sigma \not\vdash \bot$。
\end{Theorem}

\begin{proof}
我们根据事件 $e$ 的性质,通过对转移算子的结构归纳法进行证明。

\textbf{1. 内部演算步 (Internal Computation):}

若 $e$ 对应内核内部的规约(如表达式化简 $\beta$-reduction),则 $\Sigma' = \Sigma$ 且仅涉及控制项 $ctrl$ 的变化。
\begin{itemize}
\item 根据 Core 层的 \textbf{Subject Reduction} 定理:若 $\Gamma, \Sigma \vdash ctrl : A$ 且 $ctrl \to ctrl'$,则 $\Gamma, \Sigma \vdash ctrl' : A$。
\item 由于 $\Sigma$ 本身未发生改变,其良构性 $\Gamma \vdash \Sigma \,\, \textsf{ok}$ 自动保持。
\end{itemize}

\textbf{2. 外部事实注入 (Fact Injection):}

若 $e$ 对应向知识库注入新事实 $ku_{new}$,则转移由 $\textsf{unify}(\Sigma, ku_{new})$ 定义。
\begin{itemize}
\item \textbf{类型预检}:转移的前提是 $ku_{new}$ 必须通过类型检查,即 $\Gamma, \Sigma \vdash ku_{new} : A_{new}$。
\item \textbf{一致性冲突处理}:
\begin{itemize}
\item \textit{分支 A (相容)}:若 $\Sigma \cup \{ku_{new}\} \not\vdash \bot$,则 $\Sigma' = \Sigma \cup \{ku_{new}\}$。根据弱化引理 (Weakening Lemma),原有的事实项在更大的语境下依然保持良型。
\item \textit{分支 B (冲突)}:若 $\Sigma \cup \{ku_{new}\} \vdash \pi : \bot$,内核不会直接合并,而是构造 $\Sigma' = \Sigma \cup \{ \textsf{Invalidated}(ku_{new}, \pi) \}$。
\end{itemize}
\item 在两种分支下,$\Sigma'$ 都不包含可直接推导出的 $\bot$,且所有元素都拥有对应的构造证明。因此,$\Gamma \vdash \Sigma' \,\, \textsf{ok}$。
\end{itemize}

\textbf{3. 环境消解步 (Elimination Step):}

若 $e$ 对应消去规则的应用(例如从 $\Sigma$-类型事实中提取投影项)。
\begin{itemize}
\item 设 $\Sigma$ 中存在 $\langle a, p \rangle : \Sigma(x:A).B$。转移步产生 $a : A$。
\item 根据 $\Sigma$-消去规则的语义可靠性(Soundness),投影出的项 $a$ 的类型 $A$ 是预定义的且合法的。
\item 这种操作只是对既有良构知识的展开,不会引入不一致性,故 $\Sigma'$ 保持良构。
\end{itemize}

\textbf{结论:} 综上所述,无论何种转移事件 $e$,新状态 $\Sigma'$ 均能保持逻辑上的良构性与一致性。
\end{proof}

在 KOS-TL Kernel 层中,确定性 (Determinism) 是确保分布式共识(Consensus)和逻辑可追溯性的基石。在依存类型系统下,确定性不仅意味着计算结果的一致性,还意味着规约路径的合流性(Confluence)。

\begin{Theorem}{内核演算确定性(Determinism of Kernel Evolution)}

设 $\textsf{Op}$ 为内核状态转移函数,$\Sigma$ 为当前良构的内核状态,$e$ 为触发事件。若转移规则定义为 $\Sigma' = \textsf{Op}(\Sigma, e)$,则对于相同的输入对 $(\Sigma, e)$,产出的新状态 $\Sigma'$ 在逻辑等价意义下是唯一的:
\[
\forall \Sigma, e, \Sigma'_1, \Sigma'_2: (\Sigma \xrightarrow{e} \Sigma'_1 \wedge \Sigma \xrightarrow{e} \Sigma'_2) \implies \Sigma'_1 \equiv \Sigma'_2
\]
其中 $\equiv$ 表示内涵相等(Intensional Equality),即两者的范式(Normal Form)完全一致。
\end{Theorem}

\begin{proof}
证明基于 KOS-TL Core 的纯函数性质与强规范化演算的合流性,分为以下三个维度展开:

\textbf{1. 算子的纯函数性 (Purity of Operators):}

内核中的所有转移算子(如 $\textsf{unify}$, $\textsf{subst}$, $\textsf{eval}$)均定义为核心层(Core Layer)中的项。
\begin{itemize}
\item 在核心层理论中,所有构造子均满足 \textbf{计算一致性}。给定相同的输入 $\rho$(赋值环境),解释函数 $\llbracket \textsf{Op} \rrbracket_\rho$ 是一个数学意义上的单值函数。
\item 既然 $\textsf{Op}$ 不依赖于任何外部隐式状态或随机源,其映射关系 $\Sigma \times e \to \Sigma'$ 在函数式语义下是确定的。
\end{itemize}

\textbf{2. 强规范化与合流性 (Confluence):}

由于 KOS-TL 具有强规范化(Strong Normalization)性质,根据 \textbf{Church-Rosser 定理},该演算系统具有合流性。
\begin{itemize}
\item 即使在规约 $\Sigma \xrightarrow{e} \Sigma'$ 的过程中存在多个可选的规约红式(Redex),合流性保证了无论采取何种规约顺序(Reduction Strategy),最终得到的范式 $\textsf{nf}(\Sigma')$ 是唯一的。
\item 因此,虽然物理内存中的中间步可能略有差异,但逻辑层面的状态(即能够参与后续推导的事实集合)是唯一的。
\end{itemize}

\textbf{3. 冲突解决的确定性 (Deterministic Conflict Resolution):}

在处理 $e$ 导致的一致性冲突时,内核的分支判定逻辑:
\begin{itemize}
\item \textsf{unify} 算子按照类型推导规则的优先级(Typing Rule Priority)进行穷举搜索。
\item 矛盾证明项 $\pi$ 的构造遵循标准的搜索算法(如统一化算法 Unification Algorithm)。在给定的搜索空间内,第一个被找到的最小证明项是确定的。
\item 分支 A 或分支 B 的选择完全由“是否存在证明项 $\pi$”这一逻辑真值决定,不具备非确定性选择(Non-deterministic Choice)。
\end{itemize}

\textbf{结论:} 综上所述,由于算子的纯函数定义及底层演算系统的合流性,KOS-TL 内核的状态转移具有严格的确定性。

\end{proof}


进度性 (Progress) 是确保内核实时响应能力和鲁棒性的核心数学基石。它保证了逻辑自愈引擎在任何时刻都不会陷入“计算死胡同”。

\begin{Theorem}{内核进度性 (Kernel Progress)}

设 $\mathcal{C} = \langle \Sigma, \textsf{Ev}, \Delta \rangle$ 为一个良构的 KOS-TL 内核配置,其中状态 $\Sigma = \langle \mathcal{K}, \mathcal{TS}, \mathcal{P} \rangle$ 且转移记录 $\Delta$ 满足时间单调性。若 $\mathcal{C}$ 满足全局类型分配且当前活跃事件 $\textsf{Ev}$ 在上下文 $\Sigma$ 下是良构的,则必有以下之一成立:
\begin{enumerate}
\item[\textbf{PP1.}] \textbf{逻辑稳态 (Logical Quiescence)}:

$\textsf{Ev} = \textsf{null}$ 且挂起队列 $\mathcal{P} = \emptyset$。此时系统所有因果链均已在 $\Delta$ 中物化,计算暂时终止。
\item[\textbf{PP2.}] \textbf{因果推进 (Causal Advancement)}:

存在一个新的配置 $\mathcal{C}' = \langle \Sigma', \textsf{Ev}', \Delta \cup \{ \delta \} \rangle$,使得系统通过以下规约步之一向前推进:
\begin{itemize}
\item \textbf{执行步 (Execution)}:若 $\textsf{Ev} = e \neq \textsf{null}$,则执行 $\textsf{Op}$ 产生新状态 $\Sigma'$,并生成转移记录 $\delta = \langle \Sigma \xrightarrow{e} \Sigma' \rangle$。
\item \textbf{激活步 (Activation)}:若 $\textsf{Ev} = \textsf{null}$ 且 $\mathcal{P} = e_0 :: \mathcal{P}_{rest}$,则通过提取算子将队列首位事件激活。
\end{itemize}
\end{enumerate}
\end{Theorem}
\begin{proof}
对当前活跃项 $\textsf{Ev}$ 的构造及队列 $\mathcal{P}$ 的状态进行分类讨论:

\textbf{1. 活跃事件的演算 (Calculus of Active Events)}

当 $\textsf{Ev} = e \equiv \langle \textsf{Args}, \textsf{Pre}, \textsf{Op}, \textsf{Post}, \textsf{Prf} \rangle$ 时,由于配置良构,存在证明项 $\textsf{Prf}$ 满足 $\textsf{Pre}(\textsf{Args}, \Sigma)$。根据核心层(Core Layer)的强规范化性质:

\begin{itemize}
\item $\textsf{Op}$ 作为一个全函数(Total Function),对于输入对 $(\textsf{Args}, \Sigma)$ 必有定义的输出值 $\Sigma'$。
\item 根据消解规则的完备性,后置谓词 $\textsf{Post}(\Sigma')$ 在 $\Sigma'$ 构建完成后是可判定的。
\end{itemize}

因此,执行算子必能产生一个新的转移项 $\delta$,从而使系统向 PP2 转换。

\textbf{2. 队列动力学 (Queue Dynamics)}

若 $\textsf{Ev} = \textsf{null}$,系统检查挂起队列 $\mathcal{P}$:
\begin{itemize}
\item 若 $\mathcal{P} = e_0 :: \mathcal{P}_{rest}$,根据内核的操作语义,存在一个良定义的“入队- 出队”转换,将 $\textsf{Ev}$ 更新为 $e_0$。此步骤不改变 $\mathcal{K}$,但改变了系统的动能分配。
\item 若 $\mathcal{P} = \emptyset$,则系统满足 PP1 描述的稳态条件。
\end{itemize}

\textbf{3. 时间箭头约束 (Temporal Arrow Constraint)}

在所有转移 $\Sigma \xrightarrow{e} \Sigma'$ 中,单调性规则 $\Sigma'.\mathcal{TS} > \Sigma.\mathcal{TS}$ 确保了转移项 $\delta$ 的唯一性。由于转移记录集 $\Delta$ 是单调递增的合集,系统排除了逻辑循环(Cycles)的可能性。根据依存类型论的规范形式引理(Canonical Forms Lemma),在良构的 $\Sigma$ 下,没有任何 $\textsf{Op}$ 能够产生类型不匹配的挂起。综上所述,良构的 KOS-TL 内核配置始终具备向下一步演化的能力,直到所有事件被清空。

\end{proof}

\begin{Theorem}{演化相容性 (Evolutionary Consistency)}

设良构的内核配置为 $\mathcal{C} = \langle \Sigma, \textsf{Ev}, \Delta \rangle$,其中状态 $\Sigma = \langle \mathcal{K}, \mathcal{TS}, \mathcal{P} \rangle$。若其满足:
\begin{enumerate}[label=(\arabic*)]
\item \textbf{状态合法性}:对于所有 $(id_i, t_i, A_i) \in \mathcal{K}$,均有 $\Gamma_{Core} \vdash t_i : A_i$ 且 $\mathcal{K} \not\vdash \bot$。
\item \textbf{因果完备性}:活跃事件 $\textsf{Ev}$ 携带有效的证明项 $\textsf{Prf} : \textsf{Pre}(\textsf{Args}, \Sigma)$。
\end{enumerate}

若内核执行小步演化 $\mathcal{C} \xrightarrow{step} \mathcal{C}'$,则新配置 $\mathcal{C}' = \langle \Sigma', \textsf{Ev}', \Delta \cup \{ \delta \} \rangle$ 依然保持状态合法性与全局逻辑一致性。
\end{Theorem}

\begin{proof}

我们通过对配置转换的原子性质进行分类讨论来展开证明。

\begin{enumerate}[label=(\arabic*)]
\item \textbf{计算规约步(原子项演化)}

若规约仅涉及 $\textsf{Ev} \to \textsf{Ev}'$(如证明项的内涵简化或参数代换),而知识库 $\mathcal{K}$ 保持不变:
\begin{itemize}
    \item \textbf{一致性继承}:由于 $\mathcal{K}$ 未发生变化且已知 $\mathcal{K} \not\vdash \bot$,一致性自然保持。
    \item \textbf{主项规约 (Subject Reduction)}:根据 KOS-TL Core 的元理论,依存类型项的规约保持其类型。由于 $\textsf{Ev}$ 在 $\Sigma$ 下良构,规约后的 $\textsf{Ev}'$ 依然满足原有的类型签名。
\end{itemize}

\item \textbf{状态转换步(核心演化算子)}

当执行 $\textsf{Op}$ 导致 $\mathcal{K} \to \mathcal{K}'$ 时,系统执行算子 $\textsf{commit}(\mathcal{K}, ku_{new})$。我们对新知识项 $ku_{new}$ 及其证明进行一致性分析:

\textbf{情况 A:单调扩张 (Monotonic Expansion)}

\begin{itemize}
    \item \textbf{弱化引理应用}:若 $ku_{new}$ 与现存知识无冲突,根据构造性逻辑的弱化引理 (Weakening Lemma),原有 $\mathcal{K}$ 中的所有证明在 $\mathcal{K}' = \mathcal{K} \cup \{ku_{new}\}$ 中依然有效。
    \item \textbf{安全闭环}:由于 $\textsf{Ev}$ 包含 $\textsf{Post} : \Sigma' \to \textsf{Prop}$,内核在物化 $\mathcal{K}'$ 前会强制检查后置约束。若检查通过,则 $\mathcal{K}' \not\vdash \bot$ 得到形式化保证。
\end{itemize}

\textbf{情况 B:潜在冲突检测 (Conflict Mitigation)}

\begin{itemize}
    \item \textbf{逻辑隔离}:若 $ku_{new}$ 引入了逻辑矛盾(即存在 $\pi : \mathcal{K}, ku_{new} \vdash \bot$),内核的防护机制将阻止直接合并。
    \item \textbf{否定引入构造}:内核转而构造 $\textsf{absurd}(ku_{new}, \pi)$ 并存入知识库。在语义模型中,这相当于将冲突转化为对输入信号的证伪结论。由于矛盾被包裹在否定构造子中,它无法作为消除规则的前提,从而保护了全局一致性。
\end{itemize}

\item \textbf{时间箭头与转移记录的约束}

为了证明演化的轨迹是合法的,内核利用 $\Delta$ 构造因果证据 $\delta$:
\begin{itemize}
    \item \textbf{时钟单调性证明}:每一次演化必伴随 $\mathcal{TS}' > \mathcal{TS}$。这证明了 $\Sigma'$ 不是 $\Sigma$ 的简单循环,而是逻辑上的单调后继。
    \item \textbf{物化归纳}:系统状态 $\Sigma_n$ 的合法性可以通过归纳法溯源至初始空状态 $\Sigma_0$:
    \[
    \Sigma_n = \textsf{Apply}(\delta_n, \textsf{Apply}(\delta_{n-1}, \dots \Sigma_0))
    \]
    每一项 $\delta_i = \langle \Sigma_{i-1} \xrightarrow{e_i} \Sigma_i \rangle$ 都包含了对 $\textsf{Pre}$ 的验证和 $\textsf{Post}$ 的满足,确保了演化链条的每一环都对齐了逻辑基座。
\end{itemize}
\end{enumerate}
\end{proof}

这个证明解释了 KOS-TL 如何处理现实世界中的“脏数据”(如银行的错误流水、传感器的误报):逻辑防火墙:演化相容性确保了任何试图破坏系统一致性的数据,都会在 $\textsf{unify}$ 阶段被转换成“关于矛盾的证据”,而不是让系统本身变得矛盾。双向同步的安全性:在您之前提到的“全球供应链”例子中,即使底层数据库被非法篡改(产生冲突数据),演化相容性也会强制内核生成一个 Refute 项,从而在本体视图层保持逻辑的纯净。

这两大性质共同界定了 KOS-TL Kernel 的运行边界:

进度性 (Progress) 保证了内核的活锁自由(Livelock Freedom)。只要逻辑是良构的,内核分析程序就一定能跑下去,给出分析结果。

演化相容性 (Evolutionary Consistency) 保证了内核的运行安全性(Runtime Safety)。它确保了内核知识库在动态运行过程中,永远不会退化为一个自相矛盾的废弃系统。

\begin{Theorem}{局部可判定性定理(Local Decidability Theorem)}

给定内核状态 $\Sigma$、新事实 $ku_{new}$ 以及搜索边界 $\Delta = \{depth, fuel\}$,内核算子 $\textsf{unify}(\Sigma, ku_{new}, \Delta)$ 的执行过程是可判定的。
\end{Theorem}

\begin{proof}
证明通过对搜索空间和规约步数的双重归纳完成:

\textbf{1. 搜索空间的有限性 (Finite Search Space):}

由于内核限制了 $depth$(递归深度),证明搜索树被强制修剪为有限高度。在每一层级,合一(Unification)候选者的数量由上下文 $\Gamma$ 中的变量数决定,这同样是有限的。

\textbf{2. 规约步数的强制停机 (Forced Termination via Fuel):}

引入 $fuel$ 参数(计算能量)。每执行一次 $\beta$-规约或 $\delta$-展开,消耗一个单位的 $fuel$。
\begin{itemize}
\item 算法在每次操作前检查 $fuel > 0$。
\item 既然 $fuel$ 是自然数且随步数严格递减,根据良序原理,计算必然在有限步内要么得到范式,要么耗尽 $fuel$。
\end{itemize}

\textbf{3. 结果集的完备性:}

当计算停止时:
\begin{itemize}
\item 若范式匹配,返回 \textbf{True}。
\item 若发现结构冲突(如构造子不匹配),返回 \textbf{False}。
\item 若因 $depth$ 或 $fuel$ 耗尽而停止,返回 \textbf{Unknown}。
\end{itemize}
由于算法在所有路径上都保证停机,因此该过程是可判定的。
\end{proof}

\subsection{应用示例:质量异常溯源派生}

在制造业场景中,当检测到批次缺陷时,内核层自动触发溯源逻辑。

\begin{itemize}
\item \textbf{事件定义}:设$e_{trace}$ 为溯源事件。
\begin{itemize}
\item $\textsf{Pre}$:状态中存在某批次的缺陷报告项 $r : \textsf{DefectReport}$。
\item $\textsf{Op}$:根据生产日志,寻找与该批次关联的设备异常记录 $s : \textsf{EquipmentStatus}$。
\item  $\textsf{Post}$:生成并物化一个因果链知识对象 $cc : \textsf{CausalChain}$。
\end{itemize}
\item \textbf{演化过程}:
一旦 Runtime 将缺陷报告精化并存入$\Sigma$,内核层通过上述方程式发现 $p: \textsf{Pre}$成立,自动执行$\textsf{Op}$。系统从“已知有缺陷”的状态小步迁移到“已知缺陷原因”的更高熵状态。
\end{itemize}


\begin{Example}{跨国合规转账事件 ($e_{transfer}$)}

假设当前系统状态为 $\sigma$,包含账户 $A$ 和 $B$ 的余额。我们要定义一个从 $A$ 向 $B$ 转账金额 $v$ 的事件。

1. 状态定义 ($\sigma \in \Sigma$)

状态 $\sigma$ 是一个知识快照,包含:

$Balance(A, \sigma) = 1000$

$Balance(B, \sigma) = 500$

2.事件具体构造 ($e_{transfer} : \textsf{Ev}$)

 根据你的 $\textsf{Ev}$ 定义,该事件由三部分组成:

 前提条件 ($\textsf{pre}$):

 $$\textsf{pre} \equiv (Balance(A, \sigma) \ge v) \land \textsf{IsVerified}(A)$$

 (解释:$A$ 的余额必须足够,且 $A$ 必须通过了实名认证。)

 动作算子 ($\textsf{act}$):

 $$\textsf{act}(\sigma) \equiv \sigma [Balance(A) \gets Balance(A) - v, Balance(B) \gets Balance(B) + v]$$

 (解释:这是一个函数,描述了状态如何改变:$A$ 减钱,$B$ 加钱。)

 后置变换自证 ($\textsf{post\_prf}$):这是一个证明项,它保证了:对于任何满足 $\textsf{pre}$ 的状态 $\sigma$,执行 $\textsf{act}$ 后的新状态一定满足守恒定律($Sum_{after} = Sum_{before}$)。

 $$\textsf{post\_prf} : \Pi(\sigma:\Sigma). \textsf{pre} \to \textsf{Correct}(\textsf{act}(\sigma))$$

 3. 执行过程:小步转移 ($\Delta$)

 当内核尝试执行这个转账时,会发生以下判定过程:

 类型检查:内核首先验证 $\Gamma \vdash e_{transfer} : \textsf{Event}$。

 如果开发者写的 $\textsf{act}$ 逻辑有误(比如只减钱不加钱),那么 $\textsf{post\_prf}$ 将无法构造,该事件在编译阶段就会被拒绝。

 触发转移:输入当前快照 $\sigma$ 和事件 $e$。

 $$\textsf{STEP}(\sigma, e_{transfer}) \to \sigma'$$

 生成转移记录 ($\Delta$):产生一个三元组 $(\sigma \xrightarrow{e_{transfer}} \sigma')$。这条记录被永久存入演化轨迹$\Delta$ 中。
\end{Example}


\section{KOS-TL 运行层 (Runtime Layer):环境交互与信号精化}

运行层(Runtime Layer)是 KOS-TL 逻辑系统与物理世界之间的边界。它负责处理非确定性的外部信号、管理计算资源、调度事件队列,并将内核层生成的逻辑状态持久化为物理存储。

\subsection{语法 (Syntax)}

运行层状态由配置 $Cfg$ 描述,它将逻辑内核嵌入到物理宿主中:

\begin{equation}
Cfg \equiv \langle \Sigma, Q_{raw}, \textsf{Env}, \mathcal{M} \rangle
\end{equation}

其中:
\begin{itemize}
    \item $\Sigma$ —— 逻辑内核状态 (Logical Kernel State)

    这是系统的“大脑”在逻辑层面的当前形态。它包含知识库 $\mathcal{K}$、逻辑时钟 $\mathcal{T}$ 和挂起意图队列 $\mathcal{P}$。它代表了系统当前认为“真实”且“经过证明”的所有逻辑事实。运行层通过观察 $\Sigma$,决定下一步应该对外执行什么动作,或者如何响应外部信号。

    \item $Q_{raw}$ —— 物理原始信号队列 (Raw Signal Queue)

    这是系统的“感知输入缓冲区”。存放的是来自外部物理世界(如传感器、网络包、用户点击)但尚未被逻辑化的原始二进制或文本数据。信号的到达是随机的。这里的信号还没有对应的逻辑证明(Proof),只是“脏数据”。它是 $\textsf{elab}$(精化算子)的原料。系统会从 $Q_{raw}$ 取出信号,尝试将其提升(Promote)为内核可理解的事件。

    \item $\textsf{Env}$ —— 物理运行时环境 (Physical Environment)

    这是系统所在的物理宿主上下文。包含逻辑层无法直接感知、但运行层必须掌握的物理资源:(1)物理时钟 ($T_{wall}$):类似现实世界的墙上时间,用于处理超时判定。(2)I/O 句柄:数据库连接池、网络套接字、硬件寄存器地址。(3)计算资源:内存状态、线程池负载等。在执行 $\textsf{elab}$ 时,$\textsf{Env}$ 提供了构造逻辑证明所需的物理证据(例如:传感器自检成功的状态位)。

    \item $\mathcal{M}$ —— 物化映射与存储 (Materialization \& Storage)

    $$\mathcal{M} : \mathcal{K} \to \textsf{PhysicalStorage}$$

    这是系统的“记忆体”和“物理投影”。$\mathcal{M}$ 既代表物理存储(如磁盘上的数据库),也代表逻辑项到物理表示的映射规则。它的作用包括:(1)投影:将内核层抽象的“依存对知识”映射为数据库中的“行、列、索引”。(2)持久化:确保 $\Sigma$ 中的逻辑演化结果被安全地写入非易失性存储。(3)外部一致性:保证物理世界看到的状态(如屏幕上显示的余额)与逻辑内核 $\Sigma$ 保持同步。

\end{itemize}
%
%调度主循环(Scheduler)的逻辑应体现在驱动内核算子上:
%\begin{alltt}
%while $Q_{raw}$ is not empty:
%    s = pop($Q_{raw}$)
%    match elab(s, Env):
%        case (e):
%        // 调用内核层定义的转移规则
%            if $\langle \Sigma, e \rangle \longrightarrow_{KOS} \Sigma'$:
%                $\Sigma = \Sigma'$
%            Commit_to_Storage($\mathcal{M}$, $\Sigma$)
%        case None:
%            Log_Refinement_Failure(s)
%\end{alltt}


运行层引入了精化算子(Elaborator),其语法功能是将“脏数据”转化为核心层可理解的“构造项”:

$$\textsf{elab} : \textsf{RawSignal} \to \textsf{Env} \to \textsf{Option} \left( \sum_{e : \textsf{Ev}} \textsf{Pre}(e, \Sigma) \right)$$

精化算子是实现“信号逻辑化”的网关。其核心任务是为外部数据补充证明项:
\begin{equation}
\textsf{elab}(s, \textsf{Env}) =
\begin{cases}
\textsf{Some}(\langle e, \pi \rangle) & \text{若能构造 } \pi : \textsf{Pre}(e, \Sigma) \\ \textsf{None} & \text{否则}
\end{cases}
\end{equation}


精化过程包括:
\begin{itemize}
\item \textbf{信号解析}:将外部 JSON/二进制流解析为基础排序值($\textsf{Val}, \textsf{ID}$)。
\item \textbf{证明构造}:根据当前$\textsf{Env}$ 自动尝试构造前置条件的逻辑证明项$p$。
\item \textbf{时间锚定}:将物理接收时间映射为核心层的$\textsf{Time}$ 类型。
\end{itemize}

\subsection{运行语义 (Runtime Semantics)}

运行层的演化呈现为一种“异步驱动的小步转移”,其核心规则为“精化-提交”循环:

运行层的语义规则必须包含外部环境的更新和存储的持久化:
\begin{equation}
 \frac{ s = \textsf{head}(Q_{raw}) \quad \textsf{elab}(s, \textsf{Env}) = \textsf{Some}(\langle e, \pi \rangle) \quad \langle \Sigma, e, \pi \rangle \longrightarrow_{KOS} \Sigma' \quad \textsf{Persist}(\Sigma', \mathcal{M}) = \textsf{Success} }{ \langle \Sigma, s :: Q, \textsf{Env}, \mathcal{M} \rangle \xRightarrow{\textsf{commit}} \langle \Sigma', Q, \textsf{Env}', \mathcal{M}' \rangle }
\end{equation}
这里,$\mathcal{M} \vdash \Sigma' \Downarrow \mathcal{M}'$ 表示新状态 $\Sigma'$ 被成功“降解”(Down-cast)并物化到物理介质 $\mathcal{M}'$ 中。

\subsection{逻辑性质 (Logical Properties)}

在 Runtime 层,我们将每一个执行动作(Action)建模为一对 $e = (t, p)$,其中 $t$ 是目标命题(任务),$p$ 是其对应的证明项。

\begin{Definition}{因果依赖序}

设 $\mathcal{E}$ 为系统中所有可能的执行项集合。定义因果依赖关系 $\prec_{L} \subseteq \mathcal{E} \times \mathcal{E}$:若在 Core 层中,命题 $t_2$ 的构造项中包含对 $t_1$ 的引用(即 $t_1$ 是 $t_2$ 的前提),则称 $e_1 \prec_{L} e_2$。

\end{Definition}

\begin{Definition}{Runtime 执行序列}

执行序列 $S = [e_1, e_2, \dots, e_n]$ 是一个全序集,代表了 Runtime 层实际处理数据的物理时间顺序。
\end{Definition}


\begin{Theorem}{因果序一致性 (Causal Ordering Consistency)}

对于任意 Runtime 执行序列 $S$,若该序列被内核接受(Accepted),则对于 $S$ 中任意两个执行项 $e_i$ 和 $e_j$,必须满足:
$ \text{若 } e_i \prec_{L} e_j, \text{ 则在序列 } S \text{ 中 } e_i \text{ 必须先于 } e_j \text{ 被完成规约。} $
若物理网络导致 $e_j$ 先于 $e_i$ 到达,Runtime 必须阻塞 $e_j$ 的执行直到 $e_i$ 的证明项补齐。
\end{Theorem}
\begin{proof}

我们通过反证法(Proof by Contradiction)结合 Core 层的类型检查机制进行证明。

\textbf{步骤 1:假设存在逆序执行。}

假设 Runtime 接受了一个违反因果序的序列 $S'$,其中存在 $e_j$ 在 $e_i$ 之前完成执行,且已知 $e_i \prec_{L} e_j$。

\textbf{步骤 2:核心层约束映射。}

根据 $\prec_{L}$ 的定义,在 Core 层中,$e_j$ 的证明项校验依赖于 $e_i$ 的存在。其类型检查规则如下:

$$ \frac{\Gamma \vdash p_i : T_i \quad \Gamma, x:T_i \vdash p_j : T_j}{\Gamma \vdash \langle p_i, p_j \rangle : \Sigma(x:T_i).T_j} $$

这意味着,要判定 $e_j$ 合法,内核必须在上下文 $\Gamma$ 中已经包含 $e_i$ 的证明项。

\textbf{步骤 3:Runtime 状态演化。}

Runtime 的状态由上下文序列 $\Gamma_t$ 表示。在执行 $e_j$ 时,其算子为 $\textsf{check}(\Gamma_{current}, e_j)$。
\begin{itemize}
\item 若 $e_i$ 未执行,则 $e_i \notin \Gamma_{current}$。
\item 此时,根据 Core 层的 \textbf{范围确定性 (Scope Determinism)},$e_j$ 中对 $e_i$ 的引用将产生一个“未定义变量”错误或“自由变量”逃逸。
\item 类型检查器将返回 $\textsf{Fail}$。
\end{itemize}

\textbf{步骤 4:阻塞机制的必然性。}

由于 KOS-TL 的 Runtime 强制执行 \textbf{类型安全栅栏 (Type-Safe Fence)},任何校验失败的操作无法改变 $\Sigma$ 事实库的状态。为了使执行继续,Runtime 调度器必须挂起 $e_j$,将其放入待处理池(Pending Pool),并发出 $\textsf{Requirement}(e_i)$ 信号。

\textbf{步骤 5:结论。}

只有当 $e_i$ 到达并成功规约进入 $\Gamma$ 后,$e_j$ 的上下文才满足校验条件。因此,最终被“接受”的序列必然满足因果序。
\end{proof}

基于仿真(Simulation)理论,KOS-TL Runtime 具备两个核心逻辑性质:精化保真性 (Refinement Fidelity) 和 实时可观察性 (Observational Adequacy)。

\begin{Theorem}{精化保真性(Refinement Fidelity)}

设 $\mathcal{S}$ 为物理硬件状态空间(如 FPGA 寄存器或传感器读数集合),$\mathcal{D}_{Core}$ 为逻辑论域。定义精化函数 $\mathcal{E} : \mathcal{S} \to \mathcal{D}_{Core}$。若 Runtime 捕获物理状态 $s \in \mathcal{S}$ 得到 $ku = \mathcal{E}(s)$,则满足:
\begin{enumerate}
\item \textbf{合法性 (Well-formedness)}:存在类型 $A \in \mathcal{U}$,使得 $\Gamma \vdash ku : A$ 恒成立。
\item \textbf{模拟一致性 (Simulation Consistency)}:对于物理迁移 $s \xrightarrow{hw} s'$,存在仿真关系 $R \subseteq \mathcal{S} \times \mathcal{D}_{Core}$ 使得:

$$(s, ku) \in R \implies \exists ku' . (s', ku') \in R \land (ku \xrightarrow{small}^* ku' \lor \textit{Invalidated}(ku'))$$

\end{enumerate}
\end{Theorem}

\begin{proof}
我们通过构造仿真关系 (Simulation Relation) 并结合硬件抽象层(HAL)的规约进行证明:

\textbf{1. 构造仿真关系 $R$:}
定义关系 $R$ 如下:

$$(s, ku) \in R \iff (\textsf{val}(ku) = \textsf{measure}(s)) \land (\textsf{proof}(ku) \models \textsf{Inv}_{HW}(s))$$

其中 $\textsf{measure}(s)$ 是对物理信号的量化,$\textsf{Inv}_{HW}(s)$ 是由硬件电路(如冗余校验位或看门狗状态)强制执行的物理不变性。

\textbf{2. 合法性映射证明:}
根据 TL-Lang 的运行时精化规则,$\mathcal{E}(s)$ 的构造式为:

$$\mathcal{E}(s) \triangleq \langle \textsf{quantize}(s), \textsf{synthesize\_witness}(s) \rangle$$

由于 $\textsf{synthesize\_witness}$ 是由硬件描述语言(HDL)定义的确定性算子,它根据硬件寄存器状态 $\textsf{Reg}_{status}$ 直接映射为 Core 层的引入项(Introduction Rules)。根据 $\Sigma$ 类型的构造原则,只要硬件信号在物理量程内,总能构造出一个良构的项 $ku$。若信号超出量程,精化函数根据完备性定义,将映射至预定义的错误类型项,依然保持良构。

\textbf{3. 模拟一致性证明:}
对物理状态迁移 $s \xrightarrow{hw} s'$ 进行分类讨论:
\begin{itemize}
\item \textbf{情况 A:合规迁移。}
若 $s'$ 满足所有硬件安全约束,则精化函数 $\mathcal{E}$ 会提取新的状态位并合成新的证明项 $p'$。由于硬件层保证了 $s'$ 是由 $s$ 经由合法逻辑门变换而来,在 Core 层,对应的映射项 $ku'$ 必然可以通过内核规约步(如 $\beta$ 或 $\iota$ 规约)从 $ku$ 演化而来,从而维持了仿真关系。
\item \textbf{情况 B:非法/异常迁移。}
若物理迁移导致状态违背了 $\textsf{Inv}_{HW}$(例如传感器断线),硬件状态位会发生翻转。此时精化映射 $\mathcal{E}(s')$ 无法构造出原始类型 $A$ 的项,转而构造出 $\textsf{Invalidated}(ku')$。这种从“正常项”到“失效项”的转变,在 Kernel 层体现为结论的非单调翻转,符合 KOS-TL 处理冲突的语义,仿真关系依然在“错误处理”的维度上得到保持。
\end{itemize}
综上所述,精化过程保证了物理世界的任何有效变动都能在逻辑世界找到对应的真理表示。
\end{proof}

性质讨论:对“软硬件鸿沟”的弥合精化保真性(Refinement Fidelity)解决了传统嵌入式系统中最危险的**“认知失调”**问题:物理真实性:它保证了你在屏幕(或本体层)看到的“压力正常”,不仅仅是一个 UI 上的数字,而是由硬件寄存器状态数学推导出来的结论。安全性下沉:通过这个性质,KOS-TL 的安全性不仅停留在软件逻辑,而是通过精化函数 $\mathcal{E}$ 直接“锚定”在了 FPGA 的物理门电路上。

\begin{Theorem}{实时可观察性(Observational Adequacy)}

设 $\textit{ctrl} \in \mathcal{D}_{Core}$ 为内核生成的逻辑控制项,其类型为指令集 $\textsf{Cmd}$。设 $\mathcal{G} : \textsf{Cmd} \to \Pi^*$ 为指令生成器,将逻辑项映射为硬件指令序列 $\pi$。若 $\Gamma \vdash \textit{ctrl} : \textsf{Cmd}$ 且逻辑层断言 $\textit{ctrl}$ 满足性质 $\phi$,则:

$$\forall s \in \mathcal{S}, \quad (\textit{ctrl} \vdash \phi) \implies (\textsf{Exec}(\mathcal{G}(\textit{ctrl}), s) \models \textsf{Refine}^{-1}(\phi))$$

其中 $\textsf{Refine}^{-1}(\phi)$ 是逻辑性质 $\phi$ 在物理状态空间 $\mathcal{S}$ 中的谓词解释。
\end{Theorem}

\begin{proof}
我们采用 Hoare 逻辑 (Hoare Logic) 与 代数精化演算 (Refinement Calculus) 结合的方法进行证明:

\textbf{1. 构造映射关系:}
定义逻辑谓词 $\phi$ 与物理状态谓词 $P$ 之间的映射 $\mathcal{M} : \textsf{Prop} \to \mathcal{P}(\mathcal{S})$。
根据 精化保真性 (Refinement Fidelity) 的逆映射,若 $\textit{ctrl}$ 的语义目标是使系统进入 $\phi$ 状态,则其对应的底层寄存器状态必须满足 $P = \textsf{Refine}^{-1}(\phi)$。

\textbf{2. 逆向推导 (Backward Derivation):}
对 $\textsf{Cmd}$ 类型的构造进行归纳:
\begin{itemize}
\item \textbf{基础操作(原子指令):}
若 $\textit{ctrl}$ 是一个原子操作(如 $\textsf{SetValve(open)}$),其在 TL-Lang 的底层规约中被映射为特定的机器码序列 $\pi_a$。根据硬件抽象层(HAL)的 Hoare 三元组定义:

$$\{s \in \mathcal{S}\} \, \pi_a \, \{s' \in \textsf{Refine}^{-1}(\phi)\}$$

由于 $\mathcal{G}$ 在构建时已通过了基于 HAL 公理的静态验证,指令生成的正确性由 HAL 的完备性保证。

\item \textbf{复合操作(序列与分支):}
若 $\textit{ctrl}$ 由多个子项复合而成,则根据 Hoare 逻辑的组合规则:
若 $\{P\} \, \pi_1 \, \{Q\}$ 且 $\{Q\} \, \pi_2 \, \{R\}$,则 $\{P\} \, \pi_1; \pi_2 \, \{R\}$。
由于 $\textsf{Cmd}$ 类型在 Core 层满足\textbf{强规范化},生成的指令序列 $\pi$ 长度有限且路径确定,不存在逻辑层未定义的副作用。

\end{itemize}

\textbf{3. 原子性与干扰分析:}
在物理执行过程中,若发生中断,Runtime 必须维持观察一致性。
KOS-TL 的 Runtime 采用了 事务性 I/O (Transactional I/O) 机制。每一组由 $\mathcal{G}(\textit{ctrl})$ 生成的 $\pi$ 被包裹在一个逻辑原子块中。根据内核的“进度性”证明,该序列要么完全执行并达成 $s' \models \textsf{Refine}^{-1}(\phi)$,要么在失败时回滚并向内核提交一个 $\textsf{Invalidated}$ 证明项。在此机制下,不存在“指令执行了但未达成目标”的中间模糊状态。

\textbf{结论:} 逻辑层的语义目标 $\phi$ 能够无损地投影到物理状态空间。
\end{proof}

对“指令漂移”的防御实时可观察性(Observational Adequacy)在实际高安全场景中的意义在于:消除语义断层:在传统 C/C++ 开发中,编译器优化或驱动错误可能导致代码执行效果与设计意图不符(如指令重排导致的竞态)。在 KOS-TL 中,由于指令生成器 $\mathcal{G}$ 是经由形式化证明的,这种“意图与行为”的背离在逻辑上被消除了。可验证的物理效果:如果一个金融账户在逻辑上被封禁,实时可观察性保证了其在底层数据库中的对应记录也被执行了锁止,且该操作具有原子性保证。


我们定义系统的状态空间为 $\mathcal{S}$,并将系统状态拆分为两个视图:

逻辑视图 ($\mathcal{S}_L$):内核内存中的类型上下文 $\Gamma$ 和已规约的事实库 $\Sigma$。

物理视图 ($\mathcal{S}_P$):存储介质(磁盘或固态存储)中持久化的位流。

定义映射函数 $\textsf{Encode}: \mathcal{S}_L \to \mathcal{S}_P$,将逻辑证明项转化为物理存储格式。


\begin{Theorem}{持久化原子性与可见性(Durability Atomicity and Visibility)}

设系统在时刻 $t$ 执行状态转移 $\delta: \mathcal{S}_L \to \mathcal{S}'_L$。Runtime 层保证存在一个原子算子 $\textsf{Commit}$,满足:
\begin{enumerate}
\item \textbf{原子性}:$\mathcal{S}'_L$ 的逻辑确认(Acknowledgment)当且仅当 $\textsf{Encode}(\mathcal{S}'_L)$ 在 $\mathcal{S}_P$ 中完成持久化。
\item \textbf{可见性}:对于任何后续的读取操作 $\textsf{Recover}$,若 $\textsf{Commit}$ 已成功,则必然有 $\textsf{Recover}(\mathcal{S}_P) \equiv \mathcal{S}'_L$。
\end{enumerate}
即:不存在一个状态,使得逻辑上证明已成立,但物理重启后该证明丢失。
\end{Theorem}
\begin{proof}
我们通过构造 ``逻辑-物理同步锁'' (Logic-Physical Sync Lock) 和 幂等规约 (Idempotent Reduction) 机制来证明。
首先,我们定义系统的状态空间为 $\mathcal{S}$,并将系统状态拆分为两个视图:
\begin{itemize}
\item 逻辑视图 ($\mathcal{S}_L$):内核内存中的类型上下文 $\Gamma$ 和已规约的事实库 $\Sigma$。
\item 物理视图 ($\mathcal{S}_P$):存储介质(磁盘或固态存储)中持久化的位流。
\end{itemize}
定义映射函数 $\textsf{Encode}: \mathcal{S}_L \to \mathcal{S}_P$,将逻辑证明项转化为物理存储格式。

\textbf{步骤 1:构造证明项的持久化序列化。}

每一个逻辑变更 $\Delta \Sigma$ 在 KOS-TL 中都是一个带有 $\textsf{Id}$ 类型的证据。设 $\Delta \Sigma = (p : T)$。持久化过程被建模为一个依存对:
$ \textsf{Record} \equiv \Sigma(p : T). \textsf{Persist}(p) $
其中 $\textsf{Persist}(p)$ 是一个硬件底层的原语,只有物理写入完成才会返回见证。

\textbf{步骤 2:证明原子性。}

Runtime 维护一个写前日志 (WAL) 机制,其条目本身是 Core 层的一个项。
\begin{itemize}
\item 若系统在 $\textsf{Write}(\mathcal{S}_P)$ 过程中崩溃,由于 $\textsf{Persist}(p)$ 尚未生成有效见证,根据 Core 层的 \textbf{合流性 (Confluence)},重启后的恢复算子 $\textsf{Recover}$ 会发现该事务不满足 $\Sigma$-类型的完备性,从而自动回滚。
\item 只有当物理层返回 $p_{stored}$,逻辑层才会将 $\Gamma$ 更新为 $\Gamma \cup \{p\}$。
\end{itemize}

\textbf{步骤 3:证明可见性(一致性恢复)。}

假设系统重启。由于 KOS-TL Core 具有 \textbf{强规范化 (Strong Normalization)} 性质:
\begin{itemize}
\item 存储在 $\mathcal{S}_P$ 中的每一个证明项 $p$ 都是自包含且已化简的。
\item $\textsf{Recover}$ 算子通过重新执行类型检查 $\textsf{check}(\Gamma, p, T)$ 来重建逻辑视图。
\item 由于 Core 层是可判定的,且规约路径受合流性保护,恢复出的逻辑状态 $\mathcal{S}_L^{rec}$ 必然与崩溃前的最后一次有效 Commit 状态 $\mathcal{S}'_L$ 逻辑等价($\mathcal{S}_L^{rec} \equiv \mathcal{S}'_L$)。
\end{itemize}

\textbf{结论:}

物理存储的原子写入保证了逻辑状态的不可撤销性,而逻辑层的强规范化确保了物理数据在任何时刻重新载入后都能产生唯一的、确定的逻辑解释。
\end{proof}

\begin{Definition}{半可判定性}

一个集合 $S \subseteq \mathbb{N}$(或一个命题语言 $L$)被称为是\textbf{半可判定}的,如果存在一个图灵机(或算法)$M$,使得对于任意输入 $x$:
\begin{itemize}
\item 若 $x \in S$,则 $M(x)$ 停机并接受;
\item 若 $x \notin S$,则 $M(x)$ 可能停机并拒绝,也可能永久运行(停机问题不可知)。
\end{itemize}
\end{Definition}

\begin{Theorem}{证明搜索的半可判定性 (Semi-decidability of Proof Search)}

设 $\Gamma$ 为有限上下文,$P$ 为一命题。判定“是否存在证明项 $p$ 使得 $\Gamma \vdash p : P$”的问题是半可判定的。
\end{Theorem}
\begin{proof}
在 KOS-TL 这种包含依存类型和高阶逻辑的系统中,如果不加 $Fuel$ 限制,其证明搜索问题具有半可判定性。我们通过构造一个通用的枚举器(Enumerator)来证明该定理。
\textbf{步骤 1:证明项的可枚举性。}
KOS-TL Core 层的所有良构证明项(Proof Terms)是由一套有限的语法规则(如 $\lambda$-抽象、应用、对偶构造等)生成的。我们可以按照项的长度(或结构复杂度)对所有可能的证明项进行字典序枚举,记为序列 $\{p_1, p_2, p_3, \dots\}$。
\textbf{步骤 2:构造判定算法 $\mathcal{A}$。}
对于给定的命题 $P$ 和上下文 $\Gamma$,算法 $\mathcal{A}$ 执行以下步骤:
\begin{enumerate}
\item 开启一个循环,依次取出一个证明项 $p_i$。
\item 调用 Core 层的类型检查器(Type Checker)验证 $\textsf{check}(\Gamma, p_i, P)$。由于 Core 层具有 \textbf{强规范化性质},该步骤必然在有限时间内返回 $True$ 或 $False$。
\item 若返回 $True$,算法 $\mathcal{A}$ 停机并输出“$P$ 可证”。
\item 若返回 $False$,继续循环,检查下一个项 $p_{i+1}$。
\end{enumerate}
\textbf{步骤 3:分析停机行为。}
\begin{itemize}
\item \textbf{情形一:$P$ 确实是可证的。} 那么必然存在某个证明项 $q$ 满足条件。由于我们的枚举是完备的,在有限步内必然会遇到 $p_k = q$,此时算法停机。
\item \textbf{情形二:$P$ 是不可证的。} 算法将永远在循环中枚举并检查新的项,永远不会停机。
\end{itemize}
\textbf{结论:}
算法 $\mathcal{A}$ 能够识别所有“真”的命题(可证命题),但在面对“假”的命题(不可证命题)时无法保证停机。根据定义,该问题是半可判定的。
\end{proof}


\begin{Definition}{判定性证明的一些定义}

\begin{itemize}

    \item 命题空间 $P$:所有良构的 KOS-TL Core 命题。
    \item 证明算法 $\mathcal{A}$:Runtime 层尝试寻找命题 $p:P$ 的自动化过程。
    \item 资源向量 $\vec{\Delta} = \langle f, d, \tau \rangle$:
    \begin{itemize}
        \item $f \in \mathbb{N}$ (Fuel): 最大 $\beta$-规约步数。
        \item $d \in \mathbb{N}$ (Depth): 递归搜索的最大深度。
        \item $\tau \in \mathbb{R}^+$ (Timeout): 物理墙钟时间上限。
    \end{itemize}
\end{itemize}
\end{Definition}

\begin{Theorem}{KOS-TL Runtime 有界判定性}

设 $P$ 为 Runtime 待判定的逻辑命题。存在一个判定程序 $\mathcal{R}(P, \vec{\Delta})$,对于任意 $P$ 和有限的 $\vec{\Delta}$,$\mathcal{R}$ 在有限时间内必然停机,且其输出空间为:
$ \mathcal{O} = \{ \textsf{True}, \textsf{False}, \textsf{Unknown} \} $
其中 $\textsf{Unknown}$ 是确定性的“资源耗尽”状态。
\end{Theorem}
\begin{proof}
证明通过对执行步数 $k$ 的结构归纳法 (Structural Induction) 以及度量函数 (Measure Function) 的单调性完成。

\textbf{步骤 1:度量函数的定义。}

定义 Runtime 执行状态的度量函数 $\mu(\sigma)$,其中 $\sigma$ 是当前执行快照:

$$ \mu(\sigma) = \langle \text{fuel}, \text{depth}, \text{remaining\_time} \rangle $$

在逻辑执行的每一步(一个原子状态转移 $\sigma \to \sigma'$),该度量函数按照字典序(Lexicographical order)严格递减:

$$ \mu(\sigma') <_{lex} \mu(\sigma) $$

\textbf{步骤 2:状态转移的完备性分类。}

对于 Runtime 的单步动作,其逻辑只有三种可能:
\begin{enumerate}
\item \textbf{逻辑终结}:找到证明 $p$ 或冲突 $\pi$。此时算法直接返回 $\textsf{True}$ 或 $\textsf{False}$。
\item \textbf{继续归约}:资源尚未耗尽 ($\mu > 0$)。算法进入 $\sigma_{k+1}$,由于 $\mu$ 是良序的(Well-founded),此路径不可能无限延伸。
\item \textbf{触碰边界}:$\mu(\sigma)$ 的任一分量归零。算法立即停止并返回 $\textsf{Unknown}$。
\end{enumerate}

\textbf{步骤 3:停机性证明 (Termination)。}

由于 $\mu(\sigma)$ 的取值范围是有限的自然数集合(或受限的实数区间),根据 \textbf{良序原理 (Well-ordering Principle)},任何严格递减的序列必然在有限步内达到极小值。
在 KOS-TL Runtime 中,极小值点对应于输出集合 $\mathcal{O}$。

\textbf{步骤 4:判定性验证。}

判定性的定义是算法对所有输入均停机。由于:
\begin{itemize}
\item 每一个原子规约步是 Core 层保证可判定的;
\item 总步数受 $\vec{\Delta}$ 强制限制。
\end{itemize}
因此,Runtime 不再具有半可判定性中的“无限搜索”特征,程序变为对输入命题和资源边界的全函数 (Total Function)。
\end{proof}

\subsection{应用示例:乱序日志的因果修复}

在制造业场景中,若设备异常信号 $s_{ES}$ 因延迟晚于质检信号 $s_{QI}$  到达:

\begin{itemize}
\item \textbf{精化阻塞}:当$s_{QI}$ 到达时,精化算子发现无法构造出“已有设备异常证明”的 $p$,该事件被 Runtime 置入挂起队列。
\item \textbf{证据补齐}:$s_{ES}$ 到达后,运行层更新$\Sigma$。此时调度器检测到环境变化,重新触发 $s_{QI}$的精化。
\item \textbf{逻辑物化}:原本断裂的因果链在逻辑证据补齐后,由内核层完成原子转移,最终由运行层在物理数据库中插入溯源结论。
\end{itemize}


\begin{Example}{工业传感器触发的安全停机}

1. 配置状态 (Configuration)

当前的运行时状态 $\langle \sigma, Q, \textsf{Env} \rangle$ 如下:

$\sigma$ (逻辑快照):设备状态为 Running,温度阈值为 $80^{\circ}C$。

$Q$ (事件队列):$[ \dots ]$(当前为空)。

$\textsf{Env}$ (外部环境):连接着一个 Modbus 协议的温度传感器。

2. 外部流 (External Stream) 与 inject

传感器向系统发送了一个原始比特流:$s$ (Raw Signal): 0x4A 0x02 (代表温度读取值为 $82^{\circ}C$)。

动作:inject(s, Q) 将该十六进制信号推入待处理队列。

3. 精化过程:elaborate(s)

Runtime 层尝试将这个“无意义”的数字转换为 Kernel 层认可的“语义事件”:

精化逻辑:elaborate 查找配置规则,发现 0x4A 是温度警报。

映射结果:映射到 L1 层的事件 $e_{stop}$。

$e_{stop}.\textsf{pre}$:当前状态必须是 Running。

$e_{stop}.\textsf{act}$:将状态改为 Stopped。

$e_{stop}.\textsf{post\_prf}$:证明该操作符合“过温强制保护公理”。

4. 调度与判定 (Scheduling)

按照你给出的调度算法,系统执行如下:Pop: 从 $Q$ 中取出 $s$。Elaborate: $s$ 成功精化为 $e_{stop}$。

Kernel\_Check: Runtime 调用 Kernel 层判断式 $\Gamma \vdash e_{stop} : \textsf{Event}$。

验证通过:该事件携带了正确的 $\textsf{post\_prf}$(即使在 $82^{\circ}C$ 时停机也是符合安全定义的)。

Step: 逻辑状态更新:$\sigma_{new} = \textsf{STEP}(\sigma, e_{stop})$。

5. 持久化:commit 与 Materialize

动作:commit($\sigma_{new}$)。

 物化存储 $\mathcal{M}$:将更新后的状态写入物理数据库(如 PostgreSQL),并触发物理硬件的继电器断开电流。

\end{Example}

\section{KOS-TL系统}

将KOS-TL的内核层、核心层和运行层融合在一起便形成了KOS-TL(Knowledge Operating System - Type Logic),也称之为“知行逻辑”。知行逻辑是一个基于直觉依赖类型论并融合小步操作语义的完整逻辑系统。它通过分层架构实现了知识的静态约束、动态演化与环境精化的统一。

\subsection{总体架构}

KOS-TL 的语法由三层嵌套的表达式构成,涵盖了从抽象类型到物理配置的完整范畴。
\subsubsection{Core 层:类型定义与逻辑基座 (The Denotational Foundation)}

Core 层是系统的“大脑”,它将领域本体映射为依存类型论。

本体集成:将领域公理定义为基础类型(Base Types)和谓词。

验证机制:基于 $BHK$ 解释的类型检查器,确保每一个 $t:A$ 都是一个合法的知识构造。

职责:提供静态约束。它规定了系统“能理解什么”以及“什么是真理”。

\subsubsection{Kernel 层:动态演化与意图调度 (The Operational Engine)}

Kernel 层是系统的“心脏”,它负责状态的受控迁移。
状态模型:维护三元组 $\sigma = \langle \mathcal{K}, \mathcal{T}, \mathcal{P} \rangle$(知识、时间、意图)。
演化机制:执行小步操作语义(Small-step Semantics)。它调用 Core 层的判定能力来验证每一次状态跳转。

职责:提供动态一致性。它规定了系统“如何从当前真理演化到下一个真理”。


\subsubsection{Runtime 层:环境精化与物理执行 (The Physical Interface)}

Runtime 层是系统的“感官与肢体”,它处理与物理世界的边界交互。

精化 (Refinement):通过 $\textsf{elab}$ 算子将模糊的物理信号(Signals)“提升”为 Core 层认可的证明项。

物化 (Materialization):通过 $\mathcal{M}$ 映射将逻辑结论“降解”为持久化存储或硬件指令。

职责:提供保真性。它规定了逻辑指令如何可靠地作用于物理实体。


\subsubsection{架构全局不变式 (Global Invariant)}

KOS-TL 的 Grand Map 揭示了一个核心法则:

$\forall \text{ 物理变更 } \delta \in \mathcal{M}, \quad \exists \text{ 逻辑证明 } p \in \textsf{Core} \quad \text{s.t.} \quad \textsf{TypeCheck}(p, \textsf{Ontology}) = \textsf{Pass}$

\subsection{全局交互协议 (Global Interaction Protocol)}

该协议描述了一个物理脉冲如何穿越四层架构,最终固化为全域公认的真理。

\subsubsection{阶段 I:精化与注入 (Refinement \& Injection)}

\begin{enumerate}[label=(\arabic*)]

    \item 触发方

    Runtime Layer (External Environment)
    \item 动作
    \begin{itemize}
    \item 物理传感器产生原始信号 $s \in Q_{raw}$。
    \item Runtime 调用核心算子 $\textsf{elab}(s, \textsf{Env})$。
    \item 跨层交互:$\textsf{elab}$ 引用 Ontology 层 定义的谓词模板,并在 Core 层 构造出一个依存对证明项 $p : \textsf{Pre}(e, \Sigma)$。
    \item 结果:生成一个合法的意图项 $\langle e, p \rangle$。
    \end{itemize}
\end{enumerate}
\subsubsection{阶段 II:内核入队与排序 (Kernel Enqueue \& Sequencing)}

\begin{enumerate}[label=(\arabic*)]
    \item 触发方
        Kernel Layer

    \item 动作
    \begin{itemize}
        \item 内核接收来自 Runtime 的意图项。
        \item 调用 Kernel 算子 $\textsf{schedule}(\Sigma, e)$,将事件挂载至意图队列 $\mathcal{P}$。
        \item 此时,系统时钟 $\mathcal{T}$ 保持不变,但 $\Sigma$ 的配置已发生逻辑预占。
    \end{itemize}
\end{enumerate}
\subsubsection{阶段 III:逻辑规约与判定 (Reduction \& Judgment)}
\begin{enumerate}[label=(\arabic*)]
    \item 触发方
    Kernel Layer (Core Engine)

    \item 动作
    \begin{itemize}
        \item 内核循环调用 $\textsf{peek}(\Sigma)$ 取出队首事件。
        \item 核心校验:依据 Core 层 的类型检查规则进行判定:

  $$\Gamma, \mathcal{K}, \mathcal{T} \vdash p : \textsf{Pre}(e, \Sigma)$$

        \item 规约计算:执行 $\textsf{Op}(e)$。此时,Core 层执行 $\beta$ 与 $\iota$ 规约,计算出新状态的备选项 $\Sigma_{try}$。
        \item 后置闭环:验证 $\Sigma' \vdash p' : \textsf{Post}(e)$。
    \end{itemize}
\end{enumerate}
\subsubsection{阶段 IV:原子物化与持久化 (Materialization \& Persistence)}
\begin{enumerate}[label=(\arabic*)]
  \item 触发方

  Runtime Layer (Storage Subsystem)
  \item 动作
  \begin{itemize}
  \item 内核将校验通过的 $\Sigma'$ 下发至 Runtime。
  \item Runtime 调用物化映射 $\mathcal{M} \vdash \Sigma' \Downarrow \mathcal{M}'$。
  \item 物理确认:底层数据库返回 ACK,逻辑时钟执行 $\textsf{tick}$,正式完成状态跳转。
  \item 因果锚定:在物理存储中记录转移项 $\delta = \langle \Sigma \xrightarrow{e} \Sigma' \rangle$。
  \end{itemize}
\end{enumerate}
\begin{table}[h]
\centering
\caption{系统演化流程中的实体属性表}
\label{tab:entity-attributes}
\begin{tabular}{|c|l|l|l|l|}
\hline
步骤 & 实体 & 数据形态 & 负责层级 & 属性 \\
\hline
1 & 信号 & 原始位流 (Raw Bits) & Physical & 非确定性 \\
\hline
2 & 证明 & 依存对 $\langle e, p \rangle$ & Runtime/Core & 构造性 \\
\hline
3 & 意图 & 挂起队列 $\mathcal{P}$ & Kernel & 有序性 \\
\hline
4 & 规约 & $\lambda$-项演化 & Core/Kernel & 确定性 \\
\hline
5 & 事实 & 持久化知识 $\mathcal{K}$ & Runtime & 不可篡改性 \\
\hline
\end{tabular}
\end{table}

协议的一致性保证 (Global Invariant)

该协议强制执行一个全局不变式:

“物理存储的任何比特翻转,必须存在一条从 Ontology 延伸至 Core 的完整证明链条。”

这意味着 KOS-TL 系统不存在“未定义的行为”。任何不满足该协议路径的操作(如非法注入、证明缺失、时钟倒流)都会在各自的层级被自动拦截,并回滚至上一个良构状态 $\Sigma_{last}$。


\subsection{交互界面}

\subsubsection{Core 与 Kernel 的交互界面:类型判定接口 ($\textsf{Logic-Kernel Interface}$)}

\begin{itemize}
\item \textbf{方向}:Kernel 调用 Core。

\item \textbf{交互内容}:Kernel 将当前的意图 $e$ 及其携带的证明项 $p$ 提交给 Core。

\item \textbf{界面原语}:$\textsf{check}(\Gamma, p, \textsf{Pre}(e))$ 与 $\textsf{reduce}(\textsf{Op}(e), \sigma)$。

\item \textbf{属性}:内涵性。它是纯逻辑的,不感知物理时间或硬件状态。
\end{itemize}

\subsubsection{Kernel 与 Runtime 的交互界面:演化驱动接口 ($\textsf{Kernel-Runtime Interface}$)}

\begin{itemize}
\item \textbf{方向}:双向。

\item \textbf{交互内容}:
  \begin{itemize}
  \item \textit{Upward} (Runtime $\to$ Kernel):提供精化后的事件对 $\langle e, p \rangle$ 压入队列。
  \item \textit{Downward} (Kernel $\to$ Runtime):下发经过校验的新状态 $\sigma'$ 请求物化。
  \end{itemize}

\item \textbf{界面原语}:$\textsf{schedule}(e, p)$ 与 $\textsf{commit}(\sigma')$。

\item \textbf{属性}:原子性。确保逻辑状态的跳转与物理存储的更新同步。
\end{itemize}

\subsubsection{Core 与 Runtime 的横向依存:精化模板接口 ($\textsf{Refinement Interface}$)}

\begin{itemize}
\item \textbf{方向}:Runtime 引用 Core。

\item \textbf{交互内容}:Runtime 的 $\textsf{elab}$ 算子需要引用 Core 层定义的本体(Ontology)模板来构造合法证明。

\item \textbf{属性}:构造性。保证从物理信号提取的数据符合逻辑定义的排序(Sorts)。
\end{itemize}

\begin{figure}[h]
\centering
\begin{tikzpicture}[
    node distance=1.5cm,
    layer/.style={rectangle, rounded corners, draw=black, very thick, minimum width=8cm, minimum height=2.5cm, fill=white},
    interface/.style={fill=blue!10, draw=blue!50, dashed, thick, rounded corners, minimum width=6cm, minimum height=0.8cm},
    dataflow/.style={thick, color=orange!80!black},
    font=\sffamily
]
% --- Layers ---
\node[layer, fill=red!5] (core) {
    \begin{tabular}{c}
    \textbf{Core Layer (逻辑基座)} \\
    \footnotesize 依存类型检查 / $\beta,\iota$-规约 / 本体定义 ($\Gamma$)
    \end{tabular}
};
\node[layer, fill=green!5, below=of core] (kernel) {
    \begin{tabular}{c}
    \textbf{Kernel Layer (演化内核)} \\
    \footnotesize 状态三元组 $\sigma \langle \mathcal{K}, \mathcal{T}, \mathcal{P} \rangle$ / 意图调度 / 小步转移
    \end{tabular}
};
\node[layer, fill=blue!5, below=of kernel] (runtime) {
    \begin{tabular}{c}
    \textbf{Runtime Layer (执行环境)} \\
    \footnotesize 信号精化 (elab) / 物理存储 (M) / 硬件交互
    \end{tabular}
};
% --- Interfaces ---
\node[interface] (int1) at ($(core.south)!0.5!(kernel.north)$) {\footnotesize \textbf{Interface: Type Judgment} (判定项 $p$, 谓词 $Pre$)};
\node[interface] (int2) at ($(kernel.south)!0.5!(runtime.north)$) {\footnotesize \textbf{Interface: Commit/Schedule} (新状态 $\sigma'$, 事件 $e$)};
% --- Data Flows ---
% Runtime to Kernel
\draw[dataflow] (runtime.west) -- ++(-1,0) |- node[pos=0.25, left, align=right] {精化事件\\$\langle e, p \rangle$} (kernel.west);
% Kernel to Core
\draw[dataflow] (kernel.east) -- ++(1,0) |- node[pos=0.25, right, align=left] {逻辑校验请求\\$\Gamma \vdash p : A$} (core.east);
% Core to Kernel
\draw[dataflow] ($(core.south west)!0.3!(core.south)$) -- ($(kernel.north west)!0.3!(kernel.north)$)
    node[midway, left] {\scriptsize 规约结果};
% Kernel to Runtime
\draw[dataflow] ($(kernel.south east)!0.3!(kernel.south)$) -- ($(runtime.north east)!0.3!(runtime.north)$)
    node[midway, right] {\scriptsize 物化指令};
% External World
\node[draw, right=2cm of runtime, fill=gray!10] (env) {物理世界};
\draw[dashed, <->] (runtime) -- (env) node[midway, above] {\scriptsize 信号/IO};
\end{tikzpicture}
\caption{KOS-TL分层交互界面图}
\label{figure:KOSinterface}
\end{figure}

KOS-TL 通过$\Sigma$ -类型将“知”(静态知识与证明)与“行”(动态状态转换)耦合。在整体视角下,它是一个自洽的、可计算的逻辑实体:Core 提供语义框架,Kernel 提供演化动力,Runtime 提供环境映射。这种架构使得复杂系统不仅能够存储数据,更能通过逻辑归约实现因果追溯与合规性自我验证。

\subsection{系统性质}

\begin{Theorem}{知识单调性 - Monotonicity}

设 $\sigma$ 为内核知识库(即已接受的事实集),$ku$ 为一良构的知识对象使得 $\Gamma \vdash ku : A$。若 $\sigma$ 满足 $ku$ 的引入条件(记作 $\sigma \vdash ku$),则对于任何满足演化相容性的后续状态 $\sigma'$,若不存在针对 $ku$ 的冲突证明 $\pi$ (即 $\sigma' \nvdash \textsf{refute}(ku)$),则:

$$\sigma \subseteq \sigma' \implies (\sigma' \vdash ku)$$

即:已确立的真理在知识库的有效扩张下保持不变。
\end{Theorem}

\begin{proof}
我们通过 Kripke 语义框架 (Kripke Semantics) 和 构造性逻辑的弱化引理 (Weakening Lemma) 进行证明:

\textbf{1. 建立框架扩张模型:}
我们将内核状态演化定义为一个 Kripke 框架 $\langle W, \le, \Vdash \rangle$,其中:
\begin{itemize}
\item $W$ 是所有可能的知识库状态集合。
\item $\le$ 是定义在 $W$ 上的偏序关系,$\sigma \le \sigma'$ 表示 $\sigma'$ 是 $\sigma$ 的一个合法演化后继。
\item $\Vdash$ 是强迫关系,$\sigma \Vdash ku$ 表示在状态 $\sigma$ 下,知识对象 $ku$ 的证明项是可构造的。
\end{itemize}

\textbf{2. 证明核心层的持久性 (Persistence):}
KOS-TL 的 Core 层基于直觉主义类型论(Intuitionistic Type Theory)。在直觉逻辑中,所有算子($\Pi, \Sigma$ 等)都满足持久性。我们对 $ku$ 的证明结构进行归纳:
\begin{itemize}
\item \textbf{基础项}:若 $ku$ 是一个原子事实(如物理常数或已验证的 ID),根据 Kripke 模型定义,若 $\sigma \Vdash ku$ 且 $\sigma \le \sigma'$,由于 $\sigma \subseteq \sigma'$,则 $ku$ 及其原始证明证据在 $\sigma'$ 中依然存在。
\item \textbf{复合项}:若 $ku = \langle v, p \rangle$ 是一个依存对。根据归纳假设,$v$ 的值在扩张中保持不变。对于证明项 $p$,由于 $\sigma'$ 仅仅增加了新的事实而未引入针对 $p$ 的反证(由定理前提保证),根据类型论的 \textbf{弱化引理 (Weakening Lemma)},$\Gamma, \sigma \vdash p : P \implies \Gamma, \sigma' \vdash p : P$ 依然成立。
\end{itemize}

\textbf{3. 排除因果撤销 (Exclusion of Cancellation):}
在 KOS-TL 中,只有当内核显式构造出矛盾项 $\textsf{contra}(ku)$ 时,该项才会从“当前活跃库”移动到“历史存档区”。
若 $\sigma' \nvdash \textsf{refute}(ku)$,则说明在 $\sigma'$ 的搜索空间内没有能与 $ku$ 发生归约坍缩的反对证据。因此,逻辑演化算子 $\textsf{unify}$ 会保持 $ku$ 的可访问性。

\textbf{结论}:$\sigma' \Vdash ku$ 成立,知识具有单调性。
\end{proof}

性质讨论:对“因果溯源”的意义
知识单调性(Knowledge Monotonicity)解决了复杂系统中的“记忆不一致”问题:

证据持久性:它保证了如果银行系统在 T1 时刻证明了一笔交易是合规的,除非在 T2 时刻发现了证据造假(反证),否则该合规性结论永远不会因为数据库清理或其他交易的增加而莫名消失。

决策一致性:这使得基于 KOS-TL 构建的无人系统(如自动驾驶)能够维持长期的环境认知,避免因为处理新传感信息而“遗忘”了之前的安全边界。

\begin{Theorem}{计算反射性 - Reflexivity}

在 KOS-TL 内核中,存在一个反射算子 $\textsf{reflect}$,使得对于任何良构的项 $t \in \mathcal{D}_{Core}$ 及其在 Kernel 层发生的演化步 $\textsf{step} : t \xrightarrow{small} t'$,系统能够自动合成一个内部证明项 $\pi$,满足:

$$\Gamma \vdash \pi : \textsf{EvalPath}(t, t')$$

其中 $\textsf{EvalPath}$ 是一个依存类型,记录了从 $t$ 到 $t'$ 的所有公理化推导序列。这意味着内核的每一次状态变迁都附带一份关于其自身合法性的“元证明”。
\end{Theorem}

\begin{proof}
我们通过 Martin-Löf 类型论中的恒等类型 (Identity Types) 和 元循环映射 (Meta-circular Mapping) 进行证明:

\textbf{1. 规约步的代数映射:}
由于 KOS-TL 的 Core 层基于纯粹的、无副作用的依存类型演算,其计算语义是引用透明的 (Referential Transparent)。每一个规约步 $t \xrightarrow{small} t'$ 并非内存的随机抹写,而是应用了一个具体的归约规则(如 $\beta$-reduction 或 $\iota$-reduction)。

\textbf{2. 证明项的自动合成 (Synthesis):}
对于内核执行的每一类基本规约,我们定义映射函数 $\mathcal{R}$:
\begin{itemize}
\item \textbf{Beta 规约}:当执行 $(\lambda x. M) N \to M[N/x]$ 时,内核利用内部公理 $\textsf{beta\_axiom}$ 构造 $\pi = \textsf{refl}_{\beta}(M, N)$。
\item \textbf{Iota 规约}:当执行 $\textsf{proj}_1 \langle a, b \rangle \to a$ 时,内核利用 $\textsf{proj\_axiom}$ 构造 $\pi = \textsf{refl}_{\iota}(a, b)$。
\end{itemize}
由于所有的规约规则都在 Core 层有对应的公理定义,内核在执行计算的同时,可以同步记录所使用的公理序列。

\textbf{3. 利用 $J$-消解算子建立等价性:}
在依存类型论中,等价类型 $\textsf{Id}_A(t, t')$ 的唯一构造子是 $\textsf{refl}$。根据 $J$-算子(Identity Elimination),如果两个项在逻辑规约意义下是等价的,则它们在所有逻辑谓词下是不可区分的。
通过将 Kernel 的每一步执行动作 $t \to t'$ 映射为 $J$-算子的应用过程,内核实际上是在不断构造一个关于“我为何从 $t$ 变到 $t'$”的数学证词。

\textbf{4. 元循环自审:}
存在一个子程序 $\textsf{Audit} \subset \textsf{Kernel}$,该程序接受证明项 $\pi$ 和路径 $\textsf{EvalPath}$ 作为输入。由于 KOS-TL 具备强规范化性质,$\textsf{Audit}$ 能够在有限步内验证 $\pi$ 是否确实支撑了从 $t$ 到 $t'$ 的转换。
\end{proof}

性质讨论:对“自主系统”的意义
计算反射性(Computational Reflexivity)将 KOS-TL 提升到了**“自觉系统”**的高度:

全时自动审计:传统系统需要外部审计日志,而 KOS-TL 的日志就是它的执行路径。这意味着审计不是“事后烟”,而是“事前证明”。

决策透明化:在自动驾驶或金融交易中,当系统做出一项决策(如紧急避障或拦截转账)时,反射性保证了系统能够立即输出一份人类可读且数学有效的“合规性解释报告”。

自修复的逻辑依据:当系统检测到硬件精化映射出现偏差时,它能通过反射性对比“预期路径”与“实际路径”的逻辑差异,从而精确定位导致冲突的逻辑算子。

\begin{Theorem}{全系统安全性 - System-Wide Safety}

设 $\mathcal{S}$ 为系统的物理状态空间,$\textsf{Safe} \subseteq \mathcal{S}$ 为预定义的物理安全子集。若 KOS-TL 系统的初始化状态 $s_0 \in \textsf{Safe}$,则对于任何物理演化序列 $s_0 \xrightarrow{hw} s_1 \xrightarrow{hw} \dots \xrightarrow{hw} s_n$,始终满足:

$$\forall i \ge 0, \quad s_i \in \textsf{Safe}$$

前提条件:
\begin{enumerate}
\item Core 层满足一致性(Consistency)。
\item Kernel 层满足进度性(Progress)与演化相容性。
\item Runtime 层满足精化保真性(Refinement Fidelity)。
\end{enumerate}
\end{Theorem}

\begin{proof}
证明采用分层归纳法,将物理演化映射为逻辑证明项的规约。

\textbf{1. 基础情形 (Base Case)}
对于初始状态 $s_0$,根据 Runtime 的精化保真性:

$$\mathcal{E}(s_0) = ku_0 \quad \text{且} \quad \Gamma \vdash ku_0 : \textsf{Qualified}(s_0)$$

由于 $s_0 \in \textsf{Safe}$,在 Core 层中对应的谓词 $\textsf{is\_safe}(\textsf{proj}_1(ku_0))$ 的证明项 $p_0$ 存在。

\textbf{2. 归纳步骤 (Inductive Step)}
假设系统在第 $i$ 步处于 $s_i \in \textsf{Safe}$。考虑向第 $i+1$ 步的迁移:

\textbf{A. 物理扰动与精化:}
当物理环境发生改变 $s_i \xrightarrow{hw} s_{i+1}$(如传感器数值变化或硬件故障),Runtime 立即捕捉该变化并尝试构造新的知识对象 $ku_{i+1}$:

$$\mathcal{E}(s_{i+1}) = ku_{i+1}$$

\textbf{B. 内核逻辑判定:}
内核将 $ku_{i+1}$ 提交给 unify 算子。此时产生两种分支:
\begin{itemize}
\item \textbf{分支 1:$s_{i+1}$ 仍属于 Safe}:
内核能够基于 Core 层规则成功构造出 $p_{i+1} : \textsf{is\_safe}(s_{i+1})$。根据 Kernel 的演化相容性,状态 $\sigma$ 更新为包含 $ku_{i+1}$ 的新状态,安全性得以保持。
\item \textbf{分支 2:$s_{i+1}$ 试图越过 Safe 边界}:
此时,在 Core 层中无法构造出类型为 $\textsf{is\_safe}(s_{i+1})$ 的证明项。
根据 Core 层一致性(不能证明伪命题),内核的逻辑引擎会产生一个规约阻塞(或类型冲突)。
\end{itemize}

\textbf{C. 闭环自愈 (Self-healing Loop):}
根据 Kernel 进度性,内核不会卡死,它会转而执行 find\_root\_cause 并触发 analyze。
Runtime 接收到内核发出的安全指令 $\pi$,根据实时可观察性,该指令在物理层强制执行(如熔断、切换冗余路径),将物理状态拉回到 $s'_{i+1} \in \textsf{Safe}$。

\textbf{3. 矛盾归约 (Reductio ad Absurdum)}
假设存在某个 $s_j \notin \textsf{Safe}$:

\begin{enumerate}
\item 这意味着 Runtime 必须精化出一个逻辑项 $ku_j$,使得 $\Gamma \vdash ku_j : \textsf{is\_safe}$。
\item 而 $s_j \notin \textsf{Safe}$ 意味着 $\textsf{is\_safe}(s_j)$ 在 Core 层等价于 $\bot$(伪命题)。
\item 那么推导出 $\Gamma \vdash ku_j : \bot$。
\item 这违反了 Core 层一致性定理(系统中不存在伪命题的项)。
\item 故 $s_j \notin \textsf{Safe}$ 在逻辑上不可构造。
\end{enumerate}
\end{proof}

\begin{table}[htbp]
\centering
\caption{KOS-TL 逻辑性质全图谱 (The Logical Spectrum of KOS-TL)}
\begin{tabular}{|l|l|l|l|}
\hline
\textbf{性质名称} & \textbf{归属层级} & \textbf{核心价值} & \textbf{形式化隐喻 / 定义} \\
\hline
一致性 (Consistency) & Core & 根绝逻辑矛盾 & $\sigma \nvdash \bot$ \\
\hline
强规范化 (Normalization) & Core & 确保实时响应 & $\forall t, \exists v: \text{NormalForm}, t \twoheadrightarrow v$ \\
\hline
进度性 (Progress) & Kernel & 持续自愈运行 & $\mathcal{C} \notin \text{Final} \implies \exists \mathcal{C}': \mathcal{C} \xrightarrow{small} \mathcal{C}'$ \\
\hline
演化相容性 (Evolutionary) & Kernel & 安全状态演化 & $\sigma \xrightarrow{T} \sigma' \implies \text{TypeCheck}(\sigma') = \text{Success}$ \\
\hline
单调性 (Monotonicity) & \textbf{全系统 / Kernel} & 因果证据持久 & $\sigma \subseteq \sigma' \implies (\sigma \Vdash ku \implies \sigma' \Vdash ku)$ \\
\hline
保真性 (Fidelity) & Runtime & 物理映射不失真 & $(s, ku) \in \text{SimulationRelation}$ \\
\hline
可观察性 (Adequacy) & Runtime & 指令下达无损 & $\text{Exec}(\mathcal{G}(\textit{ctrl}), s) \models \text{Refine}^{-1}(\phi)$ \\
\hline
反射性 (Reflexivity) & \textbf{全系统} & 全路径审计追踪 & $\forall t \to t', \exists \pi: \text{Id}(t, t')$ \\
\hline
\end{tabular}
\end{table}

\subsection{特性与应用}

在跨国银行的合规审计中,处理数亿条 Swift 流水不仅是“大数据”挑战,更是“逻辑准确性”挑战。传统系统通常在“统计异常检测”(如发现大额频繁转账)和“规则硬编码”之间挣扎,容易产生海量误报。引入 KOS-TL 后,合规审计从“概率性黑盒”转变为“形式化因果系统”。

下面我们详细展开这一融合案例的逻辑架构与执行流程。

1. 逻辑抽象:定义“反洗钱公理” (AML Axioms)在 KOS-TL 中,合规性不是数据库的一个 flag 字段,而是一个证明目标(Proof Goal)。

A. 资金流向的不变性(Invariants)我们通过 Core 层的依存类型定义合规转账。一个合规转账 $T$ 必须满足:$$\textsf{ValidTx} \equiv \Sigma(t : \textsf{TxData}). \Sigma(e : \textsf{Evidence}). \textsf{CheckCompliance}(t, e)$$其中:$t$: Swift 报文数据(汇款人、收款人、金额)。$e$: 业务逻辑证明(贸易合同、报关单的 Hash 等)。$\textsf{CheckCompliance}$: 一个逻辑函数,它要求 $t$ 与 $e$ 必须在语义上对齐(例如:货品价值与转账金额在逻辑误差范围内)。

B. 拓扑公理:无环性(Acyclicity)洗钱的核心特征是“分拆与整合(Layering \& Integration)”,通常表现为资金在多个实体间兜圈子最终回到原点。公理定义:定义一个路径类型 $\textsf{Path}(A, A)$,如果能构造出该类型的项且资金属性未发生本质变化,则判定为逻辑矛盾(Contradiction)。

2. 融合执行过程:从海量数据到逻辑证据

第一阶段:大规模筛选(Database 侧 - 效率优先)底层数据库(如 ClickHouse 或图数据库 Neo4j)利用其强大的并发能力,执行初步的图算法。

任务:在数亿条记录中寻找“疑似环路”或“高风险节点关联”。结果:筛选出 10,000 条可疑链条。此时,这些链条仅具有“统计学嫌疑”,尚未定性。

第二阶段:证据请求与精化(Runtime 侧 - 保真性)KOS-TL 内核接管这 10,000 条链条。对于每一条链条,Runtime 模块向相关业务系统发起证据回填请求。

操作:请求该笔 Swift 交易对应的底层合同(Contract)和提单(Bill of Lading)。

保真性体现:证据被精化为 $ku_{evidence}$,并带上不可篡改的时间戳和来源证明。第三阶段:依存类型校验(Core/Kernel 侧 - 严谨性)这是 KOS-TL 的核心步骤。内核尝试为每一条可疑链条构造一个**“合规性证明项” $p$**。

\begin{Theorem}{交易合规性判定 - Transaction Compliance}

 对于可疑链条 $L = \{t_1, t_2, \dots, t_n\}$,若要标记为 \textsf{Verified},必须构造出一个总证明项:$$P_{total} = \langle p_1, p_2, \dots, p_n \rangle$$使得每一项 $p_i$ 都能证明该笔交易对应的证据 $e_i$ 能够消解该链条形成的“环路猜想”。

\end{Theorem}

逻辑拦截:如果某笔转账是虚假贸易(金额与合同 Hash 对不上),Core 层将无法生成证明项。后果:由于 一致性定理,内核无法将该链条状态演化为 Verified。

场景描述:冷却系统异常物理状态 $s$:冷却泵 A 的压力传感器输出电压异常波动。安全目标 $\textsf{Safe}$:压力必须维持在 $[P_L, P_H]$ 区间内,且传感器必须具有有效的校准证明。

第一步:Runtime 层的精化与保真 (Refinement \& Fidelity)当硬件产生信号时,Runtime 并不直接转发数值,而是执行精化函数 $\mathcal{E}$。 物理捕捉:传感器产生原始电平 $2.4V$。构造知识对象:Runtime 根据硬件寄存器中的单位配置,构造出:$$ku_{press} = \langle 120kPa, p_{calib} \rangle : Press$$ 性质体现:精化保真性确保了 $120kPa$ 与其校准证据 $p_{calib}$ 被强绑定。如果没有 $p_{calib}$,Runtime 无法构造出类型为 $Press$ 的项。

第二步:Kernel 层的演化与单调性 (Evolution \& Monotonicity)内核接收到 $ku_{press}$,开始更新全局状态 $\sigma$。知识合一 (Unify):内核尝试将新数据并入知识库。$$\sigma_{new} = \textsf{unify}(\sigma, ku_{press})$$性质体现:知识单调性确保了之前记录的“冷却泵 A 处于开启状态”这一事实不会消失。新旧知识在逻辑空间中共存,形成完整的因果链。

第三步:Core 层的规约与一致性 (Reduction \& Consistency)内核发起安全审计,调用 analyze 函数检查压力是否在安全阈值内。规约执行:内核执行 $\beta$-规约,将 $120kPa$ 代入谓词 $\textsf{is\_safe}(v) \equiv (P_L \le v \le P_H)$。逻辑拦截:假设 $P_H = 110kPa$。由于 $120 > 110$,Core 层判定证明项 $p_{safe} : \textsf{is\_safe}(120)$ 不可构造。性质体现:一致性定理保证了系统不能假装它是安全的。内核无法生成“安全报告”,规约步被导向了“根因分析”路径。

第四步:反射性审计与自愈 (Reflexivity \& Self-healing)内核生成自愈指令 $\textit{ctrl}$ (如:开启备用泵 B,关闭故障泵 A)。反射证明:内核生成一份证据 $\pi$。$$\pi : \textsf{Id}(\sigma_{fault}, \sigma_{recovery})$$性质体现:计算反射性让内核能向人类操作员或审计日志证明:“我关闭 A 泵是因为其压力 $120kPa$ 违反了 Core 层定义的 $110kPa$ 上限,整个决策符合逻辑宪法。”

第五步:Runtime 层的可观察执行 (Observational Adequacy)指令 $\textit{ctrl}$ 到达 Runtime。指令映射:Runtime 将逻辑指令 $\textsf{Close(Pump\_A)}$ 精化为特定的物理总线信号。性质体现:实时可观察性保证了逻辑层预期的“流量切断”物理效果能够百分之百发生。

% 本章:core层的实现
% 整合 kos-core 的完整技术文档

\chapter{Core 层的实现}

本章介绍 KOS-TL 的 Core 层在工程上的实现形式。当前实现采用 Haskell 语言,在 \texttt{kos-core/} 目录下形成独立的形式化内核,作为系统的“宪法”层,确保非法类型无法通过类型检查进入系统。

\section{设计原则与架构}

\subsection{设计原则}

kos-core 遵循以下设计原则,与 Kos.pdf 中 Core 层的角色定位一致:

\begin{itemize}
\item \textbf{Parser 即门卫}:所有 Term 仅能通过解析 Core DSL(.kos 源文件)产生,非法语法无法生成 AST。
\item \textbf{归纳构造}:代数数据类型(Algebraic Data Type)保证每个 Term 变体都有明确定义,无 $\bot$ 构造子。
\item \textbf{类型检查即合法性}:\texttt{parseAndCheck} 通过后才产出合法 Term,失败则无输出。
\item \textbf{双轴 Universe}:$\mathcal{U}_i$(计算轴)、$\textsf{Type}_i$(逻辑轴),对应 Kos.pdf 设计。
\end{itemize}

\subsection{目录结构}

\begin{verbatim}
kos-core/
├── src/
│   ├── KosCore.hs          -- 主模块,对外 API
│   └── KosCore/
│       ├── AST.hs          -- Term 代数数据类型
│       ├── Universe.hs     -- 双轴 Universe
│       ├── Context.hs       -- 类型上下文
│       ├── Substitution.hs -- 变量替换
│       ├── Reduction.hs    -- β/ι/δ/ζ/η 归约
│       ├── TypeCheck.hs    -- 双向类型检查
│       ├── Parser.hs       -- 语法分析
│       ├── Alpha.hs        -- α 等价
│       └── JSON.hs          -- JSON 序列化
├── app/Main.hs             -- CLI 入口
└── examples/               -- .kos 示例
\end{verbatim}

\section{语法与类型构造}

\subsection{类型构造(Types)}

与 Kos.pdf 的 Core 层语法对应,kos-core 实现的类型构造如表\ref{tab:core-types}所示。

\begin{table}[h]
\centering
\caption{Kos.pdf 类型构造与 kos-core 实现对照}
\label{tab:core-types}
\begin{tabular}{lll}
\toprule
\textbf{Kos.pdf 定义} & \textbf{kos-core 实现} & \textbf{状态} \\
\midrule
$\textsf{Prop}$ & \texttt{Prop Text} & 已实现 \\
$\textsf{Type}_i$ & \texttt{Universe Logical i} & 已实现 \\
$\mathcal{U}_i$ & \texttt{Universe Computational i} & 已实现 \\
$\textsf{Val}, \textsf{Time}, \textsf{ID}$ & \texttt{Val}, \texttt{Time}, \texttt{Ident} & 已实现 \\
$\Pi(x:A).B$ & \texttt{Pi Text Term Term} & 已实现 \\
$\Sigma(x:A).B$ & \texttt{Sigma Text Term Term} & 已实现 \\
$A + B$ & \texttt{Sum Term Term} & 已实现 \\
$\textsf{Id}_A(a, b)$ & \texttt{Id Term Term Term} & 已实现 \\
\bottomrule
\end{tabular}
\end{table}

\subsection{项构造(Terms)}

\begin{table}[h]
\centering
\caption{Kos.pdf 项构造与 kos-core 实现对照}
\label{tab:core-terms}
\begin{tabular}{lll}
\toprule
\textbf{Kos.pdf 定义} & \textbf{kos-core 实现} & \textbf{状态} \\
\midrule
$x$(变量) & \texttt{Var Text} & 已实现 \\
$\lambda x.t$, $t\,u$ & \texttt{Lam}, \texttt{App} & 已实现 \\
$\langle t, u \rangle$ & \texttt{Pair Term Term} & 已实现 \\
$\textsf{split}(t, x.y.u)$ & \texttt{Split Term Text Text Term} & 已实现 \\
$\textsf{inl}(t)$, $\textsf{inr}(t)$ & \texttt{InL}, \texttt{InR};\texttt{InLS}, \texttt{InRS} & 已实现 \\
$\textsf{case}(t, x.u, y.v)$ & \texttt{Case Term Text Text Text Term} & 已实现 \\
$\textsf{refl}$(Id 引入) & \texttt{Refl Term} & 已实现 \\
$\textsf{let } x=u \textsf{ in } t$ & \texttt{Let Text Term Term Term} & 已实现 \\
\bottomrule
\end{tabular}
\end{table}

\subsection{值依赖谓词(扩展)}

kos-core 在 Kos.pdf 未定义的语法基础上,增加了值依赖谓词(Value-Dependent Predicates),使类型构造可依赖项的具体数值:

\begin{itemize}
\item \texttt{gt(a, b)}:$a > b$,当 $a$、$b$ 可解析为数值时成立则类型良构。
\item \texttt{ge(a, b)}:$a \ge b$
\item \texttt{lt(a, b)}:$a < b$
\item \texttt{le(a, b)}:$a \le b$
\item \texttt{eq(a, b)}:$a = b$(支持数值或字符串相等)
\end{itemize}

例如 \texttt{gt(val "200", val "180")} 表示命题“200 > 180”,类型检查时会对两边的值进行求值比较,成立则接受。

\section{.kos 语言语法}

\subsection{模块结构}

\begin{verbatim}
module ModuleName where
  type Name : Kind
  def name : Type := term
\end{verbatim}

\texttt{type} 声明类型别名,\texttt{def} 声明项定义(支持 $\delta$ 归约)。

\subsection{原子类型与 Base Sorts}

\begin{itemize}
\item \texttt{Prop P}:命题,$\textsf{Prop} : \textsf{Type}_1$
\item \texttt{U0}, \texttt{U1}, \texttt{U}:计算轴 Universe
\item \texttt{Type0}, \texttt{Type1}, \texttt{Type}:逻辑轴 Universe
\item \texttt{val "x"}:值字面量
\item \texttt{time "2025-01-01"}:时间字面量
\item \texttt{id "BATCH1"}:标识符字面量
\end{itemize}

\subsection{Π 类型与 λ}

\begin{itemize}
\item \texttt{Pi(x:A). B} 或 \texttt{Π(x:A). B}:依赖函数类型
\item \texttt{A -> B}:箭头简写(非依赖)
\item \texttt{lam (x:A) . t}:$\lambda$ 抽象
\item \texttt{f a}:函数应用
\end{itemize}

\subsection{Σ 类型}

\begin{itemize}
\item \texttt{Sigma(x:A). B}:依赖对类型
\item \texttt{<d, p>}:对构造
\item \texttt{split (p) as x y in body}:$\Sigma$ 消除
\end{itemize}

\subsection{Sum 类型}

\begin{itemize}
\item \texttt{A + B}:不交并
\item \texttt{inl(A,B,v)}, \texttt{inr(A,B,v)}:显式类型注入
\item \texttt{inl v}, \texttt{inr v}:单参数注入(需 check 模式)
\item \texttt{case s of inl x -> t1; inr y -> t2}:Sum 消除
\end{itemize}

\subsection{Id 类型与 Let}

\begin{itemize}
\item \texttt{Id(A,a,b)}:恒等类型
\item \texttt{refl w}:自反证明
\item \texttt{let x : A := v in body}:Let 绑定($\zeta$ 归约)
\end{itemize}

\section{推导规则与归约}

\subsection{推导规则实现}

TypeCheck.hs 实现双向类型检查(infer/check),对应 Kos.pdf 的推导规则:

\begin{itemize}
\item \textbf{Π 引入}:\texttt{Lam} 分支,\texttt{check ctx' bodyTerm bodyTy}
\item \textbf{Π 消除}:\texttt{App} 分支,\texttt{Right (substitute x a body)}
\item \textbf{Σ 引入}:\texttt{Pair},\texttt{check ctx d dom},\texttt{check ctx p (substitute x d body)}
\item \textbf{Σ 消除}:\texttt{Split (Pair d p) x1 x2 body} $\to$ \texttt{substitute x2 p (substitute x1 d body)}
\item \textbf{Conversion}:\texttt{check} 结构失败时 fallback:\texttt{infer} + \texttt{definitionallyEqual}
\end{itemize}

\subsection{归约规则实现}

Reduction.hs 实现 $\beta$、$\iota$、$\delta$、$\zeta$、$\eta$:

\begin{itemize}
\item $\beta$:\texttt{App (Lam x \_ body) arg -> substitute x arg body}
\item $\iota$(Σ):\texttt{Split (Pair d p) x1 x2 body -> substitute x2 p (substitute x1 d body)}
\item $\iota$(Sum):\texttt{Case (InL ...)} 与 \texttt{Case (InR ...)} 分支
\item $\delta$:\texttt{Var x -> ctxLookupDef x ctx}
\item $\zeta$:\texttt{Let x \_ty val body -> substitute x val body}
\item $\eta$:\texttt{Lam x ty (App f (Var y))} 当 $y=x$ 且 $x \notin \text{FV}(f)$ 时归约为 \texttt{f}
\end{itemize}

定义等价 \texttt{definitionallyEqual} = \texttt{normalize} + $\alpha$ 等价。

\section{双轴 Universe 系统}

Universe.hs 实现:

\begin{itemize}
\item $\textsf{Prop} : \textsf{Type}_1$
\item $\textsf{Type}_i : \textsf{Type}_{i+1}$,$\mathcal{U}_i : \mathcal{U}_{i+1}$
\item $\mathcal{U}_i : \textsf{Type}_{i+1}$(计算轴可提升到逻辑轴)
\item $\textsf{Prop} \hookrightarrow \mathcal{U}_1$(\texttt{propEmbedToData})
\item Impredicativity:$\Pi(x:A).B : \textsf{Prop}$ 当 $B : \textsf{Prop}$
\end{itemize}

\section{与 C Runtime 的集成}

kos-core 作为独立可执行文件,通过 C Bridge(\texttt{kos\_core\_bridge.h})与 Kernel/Runtime 层集成:

\begin{itemize}
\item \texttt{kos\_core\_bridge\_check\_expr}:对 .kos 表达式进行类型检查
\item \texttt{kos\_core\_bridge\_term\_from\_kos}:解析并校验后返回 JSON 序列化
\item \texttt{kos\_core\_bridge\_infer\_from\_kos}:类型推断
\end{itemize}

C 层在添加类型定义、精化信号时,可调用上述接口,确保仅合法类型进入系统。

\section{实现状态与 Kos.pdf 对照}

表\ref{tab:core-comparison}总结 kos-core 与 Kos.pdf 的对照情况。

\begin{table}[h]
\centering
\caption{kos-core 与 Kos.pdf 实现对照}
\label{tab:core-comparison}
\begin{tabular}{lll}
\toprule
\textbf{维度} & \textbf{Kos.pdf 要求} & \textbf{kos-core 实现} \\
\midrule
类型构造 & Prop, Type$_i$, U$_i$, Val, Time, ID, Π, Σ, +, Id & 已实现 \\
项构造 & λ, app, pair, split, inl, inr, case, refl, let & 已实现 \\
Universe & 双轴、Prop↪U₁、Impredicativity & 已实现 \\
归约 & $\beta$, $\iota$, $\delta$, $\zeta$, $\eta$ & 已实现 \\
Conversion & $\Gamma \vdash t:A$, $A \equiv B$ $\Rightarrow$ $\Gamma \vdash t:B$ & 已实现 \\
元理论 & SN, Subject Reduction, Confluence & 无形式化证明 \\
本体管理 & 业务本体 $\to$ Sorts/类型 & 未实现 \\
计算自反性 & 自动合成 Id 证明 & 部分:有 Id/Refl,无自动证明合成 \\
\bottomrule
\end{tabular}
\end{table}

\subsection{已知简化}

\begin{itemize}
\item \textbf{Π 类型推断}:\texttt{infer} 对 Π 构造统一返回 \texttt{Type₁};Kos.pdf 规定 predicative 情形为 $\Pi(x:A).B : \textsf{Type}_{\max(i,j)}$。
\item \textbf{Pair 类型推断}:对依赖 $B(x)$ 的 $\Sigma$ 类型做了简化。
\end{itemize}

\subsection{进一步改进方向}

\begin{itemize}
\item J 消除子:Id 类型的依赖消除(J rule),当前仅支持 refl 引入
\item 元理论形式化:SN、Subject Reduction、Confluence 的机器可验证证明
\item 本体管理器:业务本体 $\to$ Sorts/类型 的 TypeOntology 模块
\item 计算自反性自动合成:归约时自动合成 $\textsf{Id}(t,t')$ 证明
\end{itemize}


\chapter{KOS-TL 应用}

\section{KOS-TL的应用示例}

设想一家大型离散制造企业,其核心问题为:
\begin{quote}
\emph{当某批次产品出现严重质量问题时,系统是否能够自动追溯其生产过程,
识别潜在的设备、人员或原材料异常,并给出可执行、可解释的因果链?}
\end{quote}

该问题具有以下典型特征:
\begin{itemize}
  \item 数据来源多样(工单、设备日志、人员排班、质检记录);
  \item 强时间顺序与因果约束;
  \item 推理结果需直接支持生产决策与责任界定。
\end{itemize}

系统中涉及如下核心表(来自不同系统):
\begin{enumerate}[label=(\arabic*)]
  \item 生产记录:Product(ProductID, Model)
  \item 批次记录:Batch(BatchID, ProductID, ProduceDate)
  \item 生产线:ProductionLine(LineID, Factory)
  \item 工艺路径表:ProcessRoute(Model, StepName, StepOrder, TargetLineType)
  \item 工艺阈值表:ProcessThreshold(Model, StepName, ParamName, MinValue, MaxValue)
  \item 步骤执行详表:StepExecution(WOID, StepName, StartTime, EndTime, EquipID)
  \item 传感器序列表:SensorTimeSeries(EquipID, ParamName, Value, Timestamp, DeviceStatus)
  \item 工单;WorkOrder(WOID, BatchID, LineID)
  \item 操作员:Operator(OperatorID, Name, Role)
  \item 操作日志:OperationLog(LogID, WOID, OperatorID, Time)
  \item 装配:Equipment(EquipID, LineID)
  \item 装配状态:EquipmentStatus(EventID, EquipID, Status, Time)
  \item 过程参数:ProcessParam(LogID, ParamName, Value)
  \item 质量审核:QualityInspection(InspectID, BatchID, Result, Time)
  \item 缺陷:DefectReport(ReportID, BatchID, DefectType)
  \item 供应链:SupplierPart(PartID, SupplierID, BatchID)
\end{enumerate}

为了举例说明,我们跟踪一个轴承生产质量追溯中的因果推理流程(如表\ref{tab:causal-reasoning}所示)。
\begin{table}[h]
\centering
\caption{轴承生产质量追溯因果推理}
\label{tab:causal-reasoning}
\begin{tabular}{>{\centering\arraybackslash}p{2cm} >{\centering\arraybackslash}p{3cm} >{\raggedright\arraybackslash}p{9cm}}
\toprule
\textbf{步骤} & \textbf{系统动作} & \textbf{具体数据实例} \\
\midrule
输入 & 质检系统上报异常 &Batch\_202310-01 在 2023-10-10 10:00 检测到“硬度不均”。 \\
类型实例化 & 构造$f_{fail}$ & $f_{fail} : \mathsf{FailureEvent} = \langle \text{"B2310"}, \text{"HARD\_ERR"}, \text{10:00} \rangle$ \\
内核推理 & 搜索因果证据 & 检索到该批次在 08:00 经过 HeatTreatment\_03 炉,且该炉在 07:55 有个温控波动 $a_{temp}$。 \\
逻辑合成 & 构造因果链 & $r = \langle f_{fail}, a_{temp}, \text{prf}_{causal} \rangle$。此时证明项 $\text{prf}_{causal}$ 自动校验了 $07:55 < 10:00$。 \\
\bottomrule
\end{tabular}
\end{table}

通过使用KOS-TL的推理,最终输出给用户的不是一条SQL查询结果,而是一个逻辑证明包。用户点击报告时,系统可以展开 $\text{prf}_{causal}$,直接定位到该温度波动的原始PLC原始日志,因为该日志是构造报告 $r$ 的组成部分。


\subsection{KOS-TL的应用过程}

KOS-TL在应用过程中主要涉及到如下环节:
\begin{enumerate}
    \item 定义初始的原子类型、谓词类型和事件与约束。这部分是属于内核层(Core)。这种定义本质上是规定我们所刻画的逻辑系统的合法性边界。
    \item 运行时层(runtime)是系统与外部的界面,通过运行时,KOS-TL系统获取到数据,并将其精化为类型对象(逻辑可操作对象)。
    \item 核心层(Kernel)负责具体的知识操作。
\end{enumerate}


\subsubsection{内核层:规则定义与逻辑约束}


    针对“轴承生产质量追溯因果推理”问题,内核层定义相应的类型和约束。

(1)基础原子类型

基础原子类型如表\ref{tab:domain-types-refinement}所示。包括:$\mathsf{BatchID}, \mathsf{Machine}, \mathsf{Time}$ 。

\begin{table}[h]
\centering
\caption{领域概念与类型精化关系表格}
\label{tab:domain-types-refinement}
\begin{tabular}{>{\centering\arraybackslash}p{2cm} >{\raggedright\arraybackslash}p{3.5cm} >{\raggedright\arraybackslash}p{3.5cm} >{\raggedright\arraybackslash}p{5cm}}
\toprule
\textbf{领域概念} & \textbf{逻辑层类型 (Core层定义)} & \textbf{内核底层原子类型 (Kernel实现)} & \textbf{精化逻辑 (Refinement)} \\
\midrule
Time & Time & Float / UInt64 & 直接映射,表示 Unix 时间戳或逻辑时钟。 \\
BatchID & BatchID & Val / String & $\Sigma(s:Val).\text{Proof}(isIDFormat(s))$ \\
Machine & Machine & Val / Enum & $\Sigma(v:Val).\text{Proof}(v\in EquipRegistry)$ \\
\bottomrule
\end{tabular}
\end{table}

(2)谓词类型

谓词类型包括:

\begin{itemize}

    \item $\mathsf{InRoute}(b, m)$

    定义了“批次 $b$ 是否允许在机器 $m$ 上加工”。

    \item $\mathsf{Overlap}(t, dur)$

    定义时间点 $t$ 是否落在区间 $dur$ 之内。
\end{itemize}

(3)事件与约束

\begin{enumerate}[label=(\roman*)]

    \item 失效事件类型 ($\mathsf{FailEvt}$)

    $$\mathsf{FailEvt} \equiv \Sigma(b: \mathsf{BatchID}). \Sigma(err: \mathsf{ErrorCode}). \Sigma(t: \mathsf{Time}). \mathsf{Proof}(t \in \text{Shift}_{QA})$$

    该类型不仅记录了哪个批次坏了,还强制要求携带一个“检测时间必须在质检班次内”的证明。

    \item 生产过程类型 ($\mathsf{ProcStep}$)

    $$\mathsf{ProcStep} \equiv \Sigma(b: \mathsf{BatchID}). \Sigma(m: \mathsf{Machine}). \Sigma(dur: \mathsf{Time} \times \mathsf{Time}). \mathsf{Proof}(\mathsf{InRoute}(b, m))$$

    这里通过 $\mathsf{InRoute}$ 约束,确保了该批次轴承在 $m$ 机器上的记录是符合工艺路径定义的。

    \item 环境异常类型 ($\mathsf{Anomaly}$)

    $$\mathsf{Anomaly} \equiv \Sigma(m: \mathsf{Machine}). \Sigma(p: \mathsf{Param}). \Sigma(v: \mathsf{Val}). \Sigma(t: \mathsf{Time})$$

    \item 因果有效性约束($\mathsf{CausalProof}(a, f)$)

    $$\mathsf{isBefore}(t(a), t(f)) \land \mathsf{isSameResource}(\text{location}(a), \text{process}(f))$$

    其中工艺一致性约束 $\mathsf{IsValidRoute}$定义一个谓词,强制要求生产记录 $e$ 中的 MachineID 必须属于该产品定义的 StandardRoute。

    追溯被定义为一个证明搜索问题:
    $$\forall f : \mathsf{Failure}, \exists (a, \pi) : \Sigma(a:\mathsf{Anomaly}) . \mathsf{CausalProof}(a, f)$$

    对于每一个失效,必须能构造出一个对应的异常及其因果证明。

    \item 因果证明($\mathsf{CausalProof}$)

    因果证明被定义为一个依存乘积类型($\Sigma$-Type),它要求同时满足时序、位置和工艺逻辑的一致性。

    $$\mathsf{CausalProof}(a, f) \equiv \Sigma(e : \mathsf{ProcStep}). \text{Prop}_{causal}(a, e, f)$$

    其中,$\text{Prop}_{causal}$ 是一个复合谓词,要求:
    \begin{itemize}
        \item 时序逻辑:异常 $a$ 发生在过程 $e$ 之内,且过程 $e$ 完成于失效 $f$ 之前。$a.t \in e.dur \land e.dur.end < f.t$

        \item 空间逻辑:异常设备 $a.m$ 正是生产该批次的过程设备 $e.m$。$a.m = e.m$

        \item 批次一致性:过程 $e$ 所处理的批次正是失效的批次 $f.b$。$e.b = f.b$
    \end{itemize}

    因果证明的构造函数mkCausalProof:

    $$\text{mkCausalProof} : \Pi(a:\mathsf{Anomaly})(f:\mathsf{FailEvt})(e:\mathsf{ProcStep}). \dots \to \mathsf{CausalProof}(a, f)$$

    它要求调用者提供以上所有逻辑条件的证明项。

    \item 因果报告($\mathsf{RootCauseReport}$ )

    $$\mathsf{RootCauseReport} \equiv \Sigma(f : \mathsf{FailEvt}) . \Sigma(a : \mathsf{Anomaly}) . \mathsf{CausalProof}(a, f)$$

    这个定义在逻辑语义上表达了三层含义:
    \begin{itemize}

        \item 失效存在性 ($f$):必须指明一个已经发生的质量失效(如:轴承硬度不均)。

        \item 异常存在性 ($a$):必须指明一个在生产过程中发生的物理异常(如:电压跌落)。

        \item 因果证明 ($\mathsf{CausalProof}$):最关键的部分。它不是一个布尔值,而是一个证明项。只有当 Kernel 层能成功构造出满足时空约束的证据链时,这个类型才被视为“非空”(Inhabited)。
    \end{itemize}
\end{enumerate}

\subsubsection{运行时层:数据抓取与对象精化(Elaboration)}


    运行时层从外部数据库提取数据,并将其转化为 KOS-TL 内核可理解的对象项,从而实现“数据逻辑化”。

    运行时层依赖的原始数据表包括:
\begin{itemize}

    \item Product\_Master: 维护工艺路径(Batch\_202310-01 $\to$ 轴承A型 $\to$ 需经过“热处理”工序)。

    \item Execution\_Log: 记录工单执行(Batch\_202310-01 在 M\_03 热处理炉上生产,时间 08:00-09:30)。

    \item IoT\_Sensor\_Stream: 记录 M\_03 的传感器流(08:15 电压波动 $15\%$)。

    \item Quality\_Report: 质检记录(10:00 检测到“硬度不均”)。
\end{itemize}

    运行时层将上述数据“精化”为带证明的项:

    \begin{enumerate}[label=(\roman*)]

        \item 失效事件$f_0$:

         $$f_0 = \text{mkFailure}(\text{Batch}\_{202310-01}, \text{Hardness\_Issue}, 10:00, \pi_{QA\_Sign})$$

        精化逻辑:从 Quality\_Report 读取,$\pi_{QA\_Sign}$ 是对该记录真实性的逻辑背书。

        \item 生产记录项$e_{proc}$:

        $$e_{proc} = \text{mkStep}(\underbrace{\text{Batch}\_{202310-01}}_{b}, \underbrace{\text{M\_03}}_{m}, \underbrace{\langle 08:00, 09:30 \rangle}_{dur}, \underbrace{\pi_{route}}_{\mathsf{Proof}(\mathsf{InRoute}(b,m))})$$

        精化逻辑:关联工单与设备日志,确保时间段闭合。

        \item 设备异常项$a_{volt}$:

        $a_{volt}$:$$a_{volt} = \text{mkAnomaly}(\text{M\_03}, \text{Voltage\_Drop}, 08:15, \pi_{iot\_hash})$$

    \end{enumerate}

\subsubsection{核心层:证明构造与小步演化}


核心层是执行层,它接收运行时层的对象,按照内核层的规则进行计算。其证明构造和演化如算法\ref{alg:step-semntaics1}所示。
\begin{algorithm}
\caption{KOS-TL 内核小步操作逻辑}
\label{alg:step-semntaics1}
\begin{algorithmic}[1]
    \REQUIRE $\mathcal{C}_0 \leftarrow \langle \Gamma_0, \sigma_0, f_0 \rangle$ \COMMENT{$\text{上下文}\Gamma_0,\text{知识库} \sigma_0,\text{失效事件} f_0$}
    \ENSURE $r_{final}$\COMMENT{因果报告}
    \STATE \textbf{步骤1:精化与投射}
        \STATE \hspace*{\algorithmicindent}$f_0 = \langle \text{Batch\_202310-01}, \text{Hardness\_Low}, 10:00, \text{prf}_{qa} \rangle$
        \STATE \hspace*{\algorithmicindent}$\langle \Gamma_0, \sigma_0, \mathsf{analyze}(f_0) \rangle \xrightarrow{small} \langle \Gamma_0[b \mapsto f_0.b], \sigma_0, \mathsf{getProductionContext}(b) \rangle$ \COMMENT{提取批次 ID: Batch\_202310-01}
    \STATE \textbf{步骤2:上下文绑定(生产环境物化)}
        \STATE \hspace*{\algorithmicindent}$\langle \Gamma_1, \sigma_0, \mathsf{getProductionContext}(b) \rangle \xrightarrow{small} \langle \Gamma_1[e \mapsto e_{proc}], \sigma_0, \mathsf{matchAnomaly}(e) \rangle$ \COMMENT{从 $\sigma_0$ 检索到 $e_{proc} = \langle \text{Batch}\_{202310-01},\text{M\_03}, \langle 08:00, 09:30 \rangle, \pi_{route} \rangle$}
    \STATE \textbf{步骤3: 候选匹配}
        \STATE \hspace*{\algorithmicindent} \text{$\mathsf{matchAnomaly}$($e_{proc}$)}匹配到两个异常项$a_{volt}$和$a_{temp}$
        \STATE \hspace*{\algorithmicindent} $a_{volt} = \langle \text{M\_03}, \text{Voltage\_Drop}, 08:15, \pi_{iot\_v} \rangle : \mathsf{Anomaly}$ \COMMENT{电压变动}
        \STATE \hspace*{\algorithmicindent} $a_{temp} = \langle \text{M\_03}, \text{High\_Temp}, 09:45, \pi_{iot\_t} \rangle : \mathsf{Anomaly}$ \COMMENT{环境温度升高}
    \STATE \textbf{步骤 4: 证明合成与过滤 (Synthesis \& Filtering)}
    \FORALL{异常 $a \in \mathcal{A}$}
        \STATE 尝试构造证明项 $\pi_{causal}(a) \leftarrow \mathsf{mkCausalProof}(a, f_0, e_{proc})$:
        \STATE \hspace*{\algorithmicindent}\textbf{步骤 4.1: 时序嵌套检查 (Temporal Inclusion)}
        \STATE \hspace*{\algorithmicindent} $\pi_{in} \leftarrow \mathsf{check}(a.t \in [e_{proc}.start, e_{proc}.end])$
        \STATE \hspace*{\algorithmicindent} \COMMENT{对于 $a_{volt}$: $08:15 \in \to \text{Success}$}
        \STATE \hspace*{\algorithmicindent} \COMMENT{对于 $a_{temp}$: $09:45 \in \to \text{Fail}$}
        \STATE \hspace*{\algorithmicindent}\textbf{步骤 4.2: 时序先后检查 (Temporal Sequencing)}
        \STATE \hspace*{\algorithmicindent} $\pi_{seq} \leftarrow \mathsf{check}(e_{proc}.end < f_0.t)$
        \STATE \hspace*{\algorithmicindent} \COMMENT{验证结论发生的逻辑顺序: $09:30 < 10:00 \to \text{Success}$}
        \STATE \hspace*{\algorithmicindent}\textbf{步骤 4.3: 资源合一检查 (Resource Unification)}
        \STATE \hspace*{\algorithmicindent} $\pi_{uni} \leftarrow \mathsf{unify}(a.m, e_{proc}.m)$
        \STATE \hspace*{\algorithmicindent} \COMMENT{验证设备一致性: $a.m \equiv \text{M\_03} \to \text{Success}$}
        \STATE \hspace*{\algorithmicindent} \textbf{步骤 4.4: 项实例化判定 (Term Inhabitation)}
        \IF {所有证明项 $\{\pi_{in}, \pi_{seq}, \pi_{uni}\}$ 均非空}
            \STATE $\pi_{causal} \leftarrow \langle a, e_{proc}, \langle \pi_{in}, \pi_{seq}, \pi_{uni} \rangle \rangle$
            \STATE \textbf{Success:} 将 $\langle a, \pi_{causal} \rangle$ 加入有效证据链
        \ELSE
            \STATE \textbf{Reject:} 丢弃 $a$ (由于类型不匹配,无法构造证明项)
        \ENDIF
    \ENDFOR
    \STATE \textbf{步骤5:证明生成与物化}
        \STATE \hspace*{\algorithmicindent}$r_{final} = \langle f_0, a_{volt},\pi_{causal} \rangle$
        \STATE \hspace*{\algorithmicindent}$\langle \Gamma_2, \sigma_0, r_{final} \rangle \xrightarrow{small} \langle \Gamma_2, \sigma_0 \cup \{ r_{final} \},{unit} \rangle$
    \STATE \textbf{return} $r_{final}$
\end{algorithmic}
\end{algorithm}

小步操作的输入是初始配置$\mathcal{C}_0 \leftarrow \langle \Gamma_0, \sigma_0, f_0 \rangle$ ,其中$\Gamma_0$是小步操作过程中的上下文,包含了所有预定义的类型、公理和函数签名。$\sigma_0$是一个事实集合,包含了此时此刻系统已知的所有已物化的碎片(Fragments)。$f_0$是本次演算的初始项,通常是一个新发生的、需要解释的观测事实。

小步操作的过程如下:
\begin{enumerate}[label=(\arabic*)]
    \item 步骤1:输入精化与投射

    小步操作的第一步是在运行时层捕获质检记录,是在$text{Batch\_202310-01}$批次下,于$10:00$检测出的一个硬度过低的故障事件,精化为项 $f_0$。

\centerline{$f_0 = \langle \text{Batch\_202310-01}, \text{Hardness\_Low}, 10:00, \text{prf}_{qa} \rangle$}

    在逻辑演算中,$f_0$ 被视为一个需要寻求解释的“结果”。$f_0$ 实际上发出了一个查询请求:“我是一个已证实的失效事实,请在 $\sigma$ 中寻找能通过 mkCausalProof 与我达成合一(Unification)的异常项 $a$ 和过程项 $e$。”这就像是一块带有特定卡槽的积木,它的形状(Hardness\_Low)和位置($text{Batch\_202310-01}$,$10:00$)决定了只有特定的因果积木(如 Voltage\_Drop)才能与它拼合。

    $f_0$ 的构造完成了从“信息”到“知识碎片”的跃迁。它确保了后续的追溯不是在杂乱的数据堆里翻找,而是在一个受类型系统保护的逻辑空间内进行精确的轨迹还原。

    内核通过解构(Deconstruction)失效事实 $f_0$,从中提取关键索引,并转换当前的计算焦点(Focus of Computation)。

    $$\langle \Gamma_0, \sigma_0, \mathsf{analyze}(f_0) \rangle \xrightarrow{small} \langle \Gamma_0[b \mapsto f_0.b], \sigma_0, \mathsf{getProductionContext}(b) \rangle$$

    左侧状态:系统接收到一个待分析的失效项 $f_0$,目前的任务是执行宏函数 analyze。这是一个高阶目标,相当于“查出这个失效的原因”。右侧状态:内核将大任务分解,第一步转化为“查找该批次的生产上下文”。这是一种降阶(Reduction)。上下文扩展中进行了符号绑定(Symbolic Binding)动作,内核从 $f_0$ 对象中提取出 .b 属性(即$ text{Batch\_202310-01}$)。它在当前的局部推理环境 $\Gamma_0$ 中创建了一个新的映射。这意味着在接下来的所有推导步中,符号 $b$ 都被严格绑定到了具体的批次ID上。

    \item 步骤2:上下文绑定

    在推演中,如果不知道生产发生的具体时间和地点,搜索异常就如同大海捞针。这一步演算通过执行 getProductionContext(b),将一个抽象的批次 ID 转换为了一个具体的、带有物理属性的生产步骤项 $e_{proc}$。

    $$\langle \Gamma_1, \sigma_0, \mathsf{getProductionContext}(b) \rangle \xrightarrow{small} \langle \Gamma_1[e \mapsto e_{proc}], \sigma_0, \mathsf{matchAnomaly}(e) \rangle$$

    左侧状态:内核已知批次 ID $b$,正在尝试还原该批次的物理加工路径。右侧状态:内核已经成功在知识库 $\sigma_0$ 中找到了证明项 $e_{proc}$,并准备开始针对该工序进行异常匹配。

    $$e_{proc} = \langle \text{Batch}\_{202310-01},\text{M\_03}, \langle 08:00, 09:30 \rangle, \pi_{route} \rangle$$


    $\Gamma_1 \to \Gamma_1[e \mapsto e_{proc}]$符号绑定:内核将新找到的复杂对象 $e_{proc}$ 绑定到局部变量 $e$ 上。此时的推理环境 $\Gamma$ 不仅知道“我们在聊哪个批次”,还知道了“这个批次在什么时候、哪台机器上”。所有的后续推导(如 $a.t \in e.dur$)都将依存于这个 $e$ 的属性。

     $e_{proc}$ 是一个典型的受控事实项:$$\langle \text{Batch}\_{202310-01}, \text{M\_03}, \langle 08:00, 09:30 \rangle, \pi_{route} \rangle$$设备锚点 (M\_03):限定了异常搜索的空间范围。只有来自 M\_03 炉子的传感器数据才是相关的。时间锚点 (08:00-09:30):限定了异常搜索的时间窗口。工艺证明 ($\pi_{route}$):这是由运行时层(如 MES 系统)提供的证明,确证了该批次轴承确实在那个时段出现在了那台机器上,而非逻辑伪造。

    \item 步骤3: 候选匹配

    这里的关键在于,系统并不预判谁是真因,而是将所有在空间(设备)上相关的异常全部“打捞”上来,交给下一步的严苛逻辑筛选。

    $\mathsf{matchAnomaly}$ 实际上执行了一个广义匹配。它的第一优先级通常是 资源合一(Resource Unification),搜索准则是a.EquipID == e.EquipID,由于 $a_{volt}$ 和 $a_{temp}$ 的设备字段均为 M\_03,它们均满足初步的空间相关性,因此被作为“候选者(Candidates)”提取到当前演算环境 $\Gamma$ 中。

    1)$a_{volt}$(电压变动项构造内容):

    $$\langle \text{M\_03}, \text{Voltage\_Drop}, 08:15, \pi_{iot\_v} \rangle$$

    这是一个发生在加工过程内部的异常。时间 (08:15)落在 $e_{proc}$ 的区间 $[08:00, 09:30]$ 之内。 证据 ($\pi_{iot\_v}$)是物理层的确定性证明(如电网频率记录)。$a_{volt}$是一个极其强力的因果候选,因为它在物理上直接干预了热处理的能量输入。

    2)$a_{temp}$(环境温度升高项构造内容):

    $$\langle \text{M\_03}, \text{High\_Temp}, 09:45, \pi_{iot\_t} \rangle$$

   这是一个发生在加工过程结束之后的异常。虽然设备匹配,但时间戳(09:45)晚于工序结束时间(09:30)。证据 ($\pi_{iot\_t}$):由厂房环境传感器提供。在 KOS-TL 的严谨逻辑下,$a_{temp}$是一个“伪因果”。虽然它发生在同一台机器,但因为不满足时序嵌套约束,它将在下一步被剔除。

    \item 步骤 4: 证明合成与过滤

    由于候选集 $\mathcal{A} = \{a_{volt}, a_{temp}\}$,内核会并行或顺序地开启两个推导分支。

    分支 1:针对 $a_{volt}$ 的推导(成功路径)

    初始配置 $\mathcal{C}_{4.0}$:$$\langle \Gamma_1[a \mapsto a_{volt}], \sigma_0, \mathsf{mkCausalProof}(a, f_0, e) \rangle$$

    环境绑定了电压异常。

    子项归约 $\mathcal{C}_{4.1}$(时序嵌套检查):$$\langle \Gamma_{curr}, \sigma_0, \mathsf{check}(08:15 \in) \rangle \xrightarrow{small} \langle \Gamma_{curr}, \sigma_0, \pi_{in} \rangle$$

    由于数值比较成立,类型 $\mathsf{Proof}(08:15 \in)$ 被实例化,得到了证明项 $\pi_{in}$。

    终态实例化 $\mathcal{C}_{4.4}$:$$\langle \Gamma_{curr}, \sigma_0, \langle a_{volt}, e, \langle \pi_{in}, \dots \rangle \rangle \rangle$$

    由于所有子证明项(Inclusion, Sequencing, Unification)都成功归约为非空实例,复合项 $\pi_{causal}$ 构造成功。

    分支 2:针对 $a_{temp}$ 的推导(失败/修剪路径)

    初始配置 $\mathcal{C}'_{4.0}$:$$\langle \Gamma_1[a \mapsto a_{temp}], \sigma_0, \mathsf{mkCausalProof}(a, f_0, e) \rangle$$

    关键归约失败 $\mathcal{C}'_{4.1}$:$$\langle \Gamma_{curr}, \sigma_0, \mathsf{check}(09:45 \in) \rangle \xrightarrow{small} \langle \Gamma_{curr}, \sigma_0, \bot \rangle$$

    在 KOS-TL 中,逻辑谓词为假意味着该项归约为 $\bot$或空集合。此时,$\pi_{in}$ 无法被实例化(Inhabitation Failure)。

    停机与拒绝 $\mathcal{C}'_{4.4}$:由于 $\pi_{causal}$ 的构造函数要求所有参数必须是有效实例,而 $\pi_{in}$ 为空,导致整个 $\mathsf{mkCausalProof}$ 函数的归约陷入停滞(Stuck)或抛出类型错误。推演动作:该分支被内核丢弃,不进入下一步的 $\sigma$ 并集操作。

    \item 步骤5:证明生成与物化

    系统状态的原子跳变:
    $$\langle \Gamma_2, \sigma_0, r_{final} \rangle \xrightarrow{small} \langle \Gamma_2, \sigma_0 \cup \{ r_{final} \}, unit \rangle$$

    左侧:不稳定的演算态组件:此时 $r_{final}$ 位于求值项(Term)的位置。内涵:这意味着报告已经生成,但它仍然只存在于当前推导线程的临时缓存中。如果系统此时断电,这个因果关系将会丢失。

    右侧:稳定的物化态$\sigma_0 \cup \{ r_{final} \}$:这是最核心的动作。$r_{final}$ 被并入全局知识库。物化意义:一旦并入,该报告就成为了全系统共享的事实。其他并发运行的程序(如你最初提到的 analyzeAudit 财务程序)现在可以扫描到这个 $r_{final}$,并根据其中的 $\pi_{causal}$ 自动触发扣款或挂账。

    $unit$:演算的湮灭内涵。$unit$ 表示计算已经完全归约。所有的中间变量和推导逻辑已经消耗完毕,系统回到了一个“静止且一致”的平衡状态,准备处理下一个失效信号。

    这个过程不仅合成了因果链,更重要的是通过“物化”操作,让原本孤立的质量失效($f_0$)与物理异常($a_{volt}$)在逻辑空间中永远地耦合在了一起。

    $r_{final}$ 对象构造:

    $$r_{final} = \langle f_0, a_{volt}, \pi_{causal} \rangle$$

    在这一步,内核将零散的证据缝合成一个逻辑完备的根因报告对象。其字段具有严格的对应关系:

    $f_0$ (现象层):定义了“发生了什么”(低硬度缺陷)。

    $a_{volt}$ (物理层):定义了“为什么发生”(电压跌落)。

    $\pi_{causal}$ (逻辑层):它是整个对象的“灵魂”,包含了我们在步骤4中合成的全部证明(时序嵌套、设备一致性等)。

    在 KOS-TL 中,没有 $\pi_{causal}$ 的报告是不被允许物化的。这保证了 $\sigma$(知识库)中存在的每一条结论都是经过数学验证的“真理”。
\end{enumerate}

上述算法中,涉及到analyze、getProductionContext和matchAnomaly三个函数。

(1)analyze 函数

    analyze 的函数签名(Type Signature)定义如下:

$$\mathsf{analyze} : \Pi (f : \mathsf{FailEvt}) \to \text{Option}(\mathsf{RootCauseReport})$$

    这个签名意味着:输入一个失效事件 $f$,该函数试图返回一个包含“异常、生产过程、以及因果证明”的复合报告。如果找不到合法的证明构造,则返回“空”。

\begin{algorithm}
\caption{KOS-TL 核心追溯函数:$\mathsf{analyze}(f)$}
\begin{algorithmic}[1]
\REQUIRE 待分析的失效事实项 $f = \langle b, d, t, \pi_{qa} \rangle : \mathsf{FailEvt}$
\ENSURE 根因审计报告集合 $\mathcal{R} \subseteq \mathsf{RootCauseReport}$ 或 $\emptyset$

\STATE \textbf{Step A: 环境投影 (Projection \& Binding)}
\STATE 从失效项 $f$ 中解构批次索引:$b \leftarrow f.b$
\STATE \COMMENT{利用 $\Sigma$-类型约束确保 $b \in \mathsf{BatchID}$}
\STATE 更新推理上下文:$\Gamma \leftarrow \Gamma[b \mapsto f.b]$

\STATE \textbf{Step B: 依存搜索 (Dependent Search)}
\STATE 在状态库 $\sigma$ 中检索与 $b$ 绑定的生产过程项集合:
\STATE $\mathcal{E} \leftarrow \{ e \in \sigma \mid e : \mathsf{ProcStep} \wedge \mathsf{Unify}(e.b, b) \}$
\STATE \textbf{if} $\mathcal{E} = \emptyset$ \textbf{then} \RETURN $\emptyset$ \COMMENT{若无生产轨迹,推理路径中断}
\STATE 锁定关键步骤:$e_{proc} \leftarrow \text{Select}(\mathcal{E})$ \COMMENT{如锁定 M\_03 炉的热处理记录}

\STATE \textbf{Step C: 证明合成尝试 (Proof Synthesis)}
\STATE 初始化根因集合:$\mathcal{R} \leftarrow \emptyset$
\STATE 提取潜在物理异常候选:$\mathcal{A} \leftarrow \{ a \in \sigma \mid a : \mathsf{Anomaly} \wedge \mathsf{SpaceMatch}(a, e_{proc}) \}$

\FORALL{异常项 $a \in \mathcal{A}$}
    \STATE \COMMENT{尝试调用 Core 层定义的因果构造函数}
    \STATE $\pi_{try} \leftarrow \mathsf{mkCausalProof}(a, f, e_{proc})$

    \IF{$\pi_{try} \not\equiv \bot$}
        \STATE \COMMENT{证明项实例化成功,合成终态报告}
        \STATE $r_{final} \leftarrow \langle f, a, e_{proc}, \pi_{try} \rangle$
        \STATE $\mathcal{R} \leftarrow \mathcal{R} \cup \{ r_{final} \}$
    \ELSE
        \STATE \textbf{continue} \COMMENT{证明坍缩,排除伪相关干扰项}
    \ENDIF
\ENDFOR

\STATE \textbf{Step D: 终态物化 (Materialization)}
\STATE $\sigma \leftarrow \sigma \cup \mathcal{R}$ \COMMENT{将派生事实并入全局知识库}
\RETURN $\mathcal{R}$
\end{algorithmic}
\end{algorithm}

analyze 函数内部由三个子逻辑块构成,每个块都对应 KOS-TL 的一个核心能力:

\begin{enumerate}[label=(\arabic*)]
    \item 环境投影 (Projection  Binding)

    首先,函数需要从 $f$ 中提取关键索引。

    let b = f.b in ...

    依赖于 $\mathsf{FailEvt}$ 的 $\Sigma$-类型定义,确保 $b$ 一定是合法的 BatchID。

    \item 依存搜索 (Dependent Search)

    利用提取的 $b$,在 Runtime 精化后的状态空间 $\sigma$ 中寻找与之绑定的生产步骤。

    find e : ProcStep where e.b = b

    依赖于 $\mathsf{ProcStep}$ 类型中的 BatchID 字段。

    在此步骤,Kernel 锁定了 Batch\_202310-01 对应的热处理记录 $e_{proc}$ (M\_03 炉)。

    \item 证明合成尝试 (Proof Synthesis)

    它会启动一个模式匹配过程,寻找满足 $\mathsf{CausalProof}$ 约束的异常 $a$。

    $$\text{foreach } a \in \sigma(\mathsf{Anomaly}) \text{ do:}$$$$\text{try } \pi = \mathsf{mkCausalProof}(a, f, e)$$

    依赖于我们在 Core 层定义的 mkCausalProof 构造函数。
\end{enumerate}

(2)getProductionContext 函数

getProductionContext 函数不是简单的数据库查询,而是一个从“结果”回溯到“过程”的依存搜索程序。

在 Kernel 层,getProductionContext 的类型定义体现了强烈的依赖关系:

$$\mathsf{getProductionContext} : \Pi(b : \mathsf{BatchID}) \to \Sigma(e : \mathsf{ProcStep}). \mathsf{Proof}(e.b = b)$$

输入:一个特定的批次标识 $b$(如:Batch\_202310-01)。

输出:一个 $\Sigma$-类型对象。这个对象不仅包含生产步骤记录 $e$,还强制携带一个逻辑证明,确保该记录中的批次字段与输入的 $b$ 在逻辑上完全一致(Unification)。

该函数在 Kernel 层 执行,但其有效性依赖于 Runtime 层 注入的环境。其内部逻辑包含三个核心步骤:
\begin{enumerate}[label=(\arabic*)]

    \item 语义解构 (Destruction)

    函数接收到批次 ID 后,首先在当前系统状态 $\sigma$ 的索引空间中进行寻址。实例数据:输入 Batch\_202310-01。函数会检索状态库中所有类型为 $\mathsf{ProcStep}$ 的项。

    \item 谓词过滤与合一 (Filtering  Unification)

    它不仅查找 ID 匹配,还要验证过程的完备性。它会检查:

    ID 匹配:$e.b \equiv \text{Batch\_202310-01}$。

    工艺完整性:利用 $e$ 中携带的 $\pi_{route}$ 证明,确认这个生产过程是合法的。

    时序闭环:确认该过程已经完成(即 $dur.end < \text{currentTime}$)。

    \item 上下文物化 (Context Materialization)

    一旦找到匹配项,函数会将该步骤的环境属性(如使用的机器 M\_03、具体的加工时段 08:00-09:30)提取出来,构建出一个“计算上下文”。
\end{enumerate}

(3)$\mathsf{matchAnomaly}(e)$函数

$\mathsf{matchAnomaly}(e)$函数它不是简单的数据库 JOIN,而是一个受限的存在性搜索函数。其核心任务是在当前知识状态 $\sigma$ 中,寻找那些在时间和空间上与生产步骤 $e$ 重叠的异常事件 $a$。

定义在 Core 层,该函数的类型签名定义了严密的输入输出约束:

$$\mathsf{matchAnomaly} : \Pi(e : \mathsf{ProcStep}) \to \text{List} \left( \Sigma(a : \mathsf{Anomaly}). \mathsf{Proof}(\text{IsRelated}(a, e)) \right)$$

输入参数 ($e$):一个已经物化的生产步骤(例如:Batch\_10-01 在 M\_03 炉的热处理过程)。

输出结果:一个 $\Sigma$-类型的列表。每个元素不仅包含异常 $a$,还必须包含一个证明项,证明 $a$ 确实发生在 $e$ 的时空范围内。

当 Kernel 执行 $\mathsf{matchAnomaly}(e)$ 时,它会启动一个基于类型导向(Type-Directed)的搜索过程,其 Small-step 演化包含以下三个子步骤:
\begin{enumerate}[label=(\arabic*)]

    \item 空间过滤 (Spatial Filtering)

    Kernel 首先提取 $e.\mathsf{EquipID}$(如 M\_03)。它只会在状态库 $\sigma$ 中查询那些与该设备关联的异常候选集。$$\text{unify}(a.\mathsf{EquipID}, e.\mathsf{EquipID})$$


    \item 时间窗口对齐 (Temporal Alignment)

    提取 $e$ 的生命周期 $[t_{start}, t_{end}]$。Kernel 校验异常发生的时间 $a.t$ 是否落在该闭区间内。逻辑判定:$e.t_{start} \le a.t \le e.t_{end}$。

    \item 语义精化 (Semantic Refinement)

    如果 $a$ 是电压波动,而 $e$ 是热处理,Kernel 会调用 Core 层的知识公理(Knowledge Axiom),确认“电压”是否是“热处理”工序的一个敏感关联维度。
\end{enumerate}

假设 $e_{proc}$ 是热处理工序,其时间为 [08:00, 09:30],设备为 M\_03。输入:$\mathsf{matchAnomaly}(e_{proc})$。

搜索 $\sigma$:

发现 $a_1$:环境温度异常,时间 10:30,设备 M\_03。$\rightarrow$ 被排除(时间超出范围)。

发现 $a_2$:电压跌落异常,时间 08:15,设备 M\_01。$\rightarrow$ 被排除(设备不匹配)。

发现 $a_3$:电压跌落异常,时间 08:15,设备 M\_03。$\rightarrow$ 命中。

构造证明:Kernel 计算数值包含证明 $\pi_{in}$,并合成返回项:$$\langle a_3, \pi_{in} \rangle$$

上述流程可以总结为表\ref{tab:stage-actions}。

\begin{table}[h]
\centering
\caption{KOS-TL应用总结}
\label{tab:stage-actions}
\begin{tabular}{>{\centering\arraybackslash}p{2cm} >{\raggedright\arraybackslash}p{6cm} >{\raggedright\arraybackslash}p{8cm}}
\toprule
\textbf{阶段} & \textbf{动作内容} & \textbf{数据/逻辑产出} \\
\midrule
Runtime & 从数据库读取 10:00 硬度异常和 08:15 电压异常。 & 实例化:$f_0$ 和 $a_{volt}$ 对象被创建。 \\
Kernel & 发现两者存在关联,启动 CausalSearch 指令。 & 演化:状态从“发现问题”向“寻找证据”迁移。 \\
Core & 检查 (08:15$<$10:00) 以及该批次是否在 M\_03 生产。 & 验证:通过类型检查,准许构造因果链。 \\
终态 & 合成并物化 RootCauseReport。 & 结论:输出一份包含原始数据指针和逻辑路径的硬因果报告。 \\
\bottomrule
\end{tabular}
\end{table}


\subsection{KOS-TL应对业务规则的变化}

我们继续以上一节“轴承热处理”为例,假设业务规则发生如下变化:

\begin{quote}
\emph{为了防止回火脆性,热处理过程中若发生电压异常,必须同时检查该时段的“冷却水循环压力”是否正常。只有两项都异常,才判定为严重质量缺陷。}
\end{quote}

在上一节的KOS-TL中只需要修改 Core 层中关于 CausalProof 的类型定义。

\begin{itemize}

    \item 修改前的类型定义

$$\mathsf{CausalProof}(a, f) \equiv \Sigma(e : \mathsf{ProcStep}). \text{Prop}_{time}(a, e, f)$$

    \item 修改后的类型定义(注入新标准)

我们将 CausalProof 重定义为必须包含“双重异常证明”的依存类型:
$$\mathsf{CausalProof}(a, f) \equiv \Sigma(e : \mathsf{ProcStep}). \Sigma(w : \mathsf{WaterPressureAnomaly}). \text{Prop}_{joint}(a, w, e, f)$$

新的约束不仅要求有电压异常 $a$,还强制要求在同一生产过程 $e$ 中存在水压异常 $w$。
\end{itemize}

一旦 Core 层的类型变了,在核心层(Kernel )的小步归约逻辑会自动发生链式反应:
\begin{enumerate}[label=(\arabic*)]

    \item 构造函数失效

    原先的 mkCausalProof 构造函数由于缺少参数(少了 $w$ 和相应的证明),在 Kernel 进行类型检查时会立即失败。

    \item 自动触发新搜索

    Kernel 发现当前目标类型需要 WaterPressureAnomaly,它会自动在环境 $\sigma$ 中启动对水压数据的检索。
\end{enumerate}

证明合成路径重组:
\begin{itemize}
    \item Case 1 (只有电压异常)

    Kernel 搜不到匹配的水压异常 $w$,无法构造出完整的 $\pi_{causal}$ 项。推理结果自动判定为“不成立”。

    \item Case 2 (双重异常)

    Kernel 自动将电压、水压、生产过程三者缝合,生成新的 $r_{final}$。
\end{itemize}
这种能力的本质是类型驱动开发 (Type-Directed Development) 的极端应用,如表\ref{tab:kos-tl-mechanisms}所示

\begin{table}[htbp]  % 浮动表格,位置建议:h=here, t=top, b=bottom, p=page
\centering
\caption{KOS-TL 处理机制特性表}
\label{tab:kos-tl-mechanisms}  % 用于交叉引用
\begin{tabular}{p{2cm}p{8cm}p{5cm}}  % 三列:特性3cm,处理机制6cm,意义6cm;| 为竖线分隔
\toprule
\textbf{特性} & \textbf{处理机制} & \textbf{意义} \\
\midrule
自愈性 (Self-Healing) & 如果数据源(Runtime)没有提供水压数据,Kernel 会报“类型缺失错误”而非给出错误结论。 & 强制保证结论的安全性,杜绝盲目追溯。 \\
\hline
\addlinespace[0.5ex]  % 行间小间距(可选)
逻辑下推 (Push-down) & 新规则通过 mkCausalProof 签名向下传递,Kernel 的搜索算法会自动感知到新的参数需求。 & 开发者不需要重新编写搜索算法,算法随类型自适应。 \\
\hline
\addlinespace[0.5ex]
零冗余 & 旧的追溯代码不需要删除,只要它们引用的类型被更新,它们的行为就自动改变。 & 实现了真正的“配置即逻辑”。 \\
\bottomrule
\end{tabular}
\end{table}

在这个例子中,我们并没有修改 analyze 函数,也没有修改 getProductionContext 函数。你只是通过修改 Core 层的类型签名,重新定义了“什么是真相”的物理边界。

Kernel 层就像一个全自动的拼图机器,你改变了拼图模板(Core),它就会自动改变寻找拼图碎片(Data)的策略和最终拼出的图案(Result)。这正是 KOS-TL 在应对复杂、多变的工业环境时的核心价值,即让系统的逻辑演进与业务规则的变更保持原子级的一致。

\subsection{KOS-TL实现跨领域的逻辑一致性}

KOS-TL还可以通过“共享逻辑内核”和“类型交叉引用”(Cross-domain Type Reference)来实现跨领域的逻辑一致性。

在 KOS-TL 中,无论是财务还是质量,它们并不直接去读取原始数据库,而是共同订阅 Kernel 层 物化出来的同一个 知识项(Knowledge Object)。

仍然延续上面的“轴承热处理示例”,设想如下场景:

\begin{quote}
\emph{如果热处理电压异常(质量风险),则该批次关联的原材料供应商货款必须自动进入“待审计”状态,且对应的财务凭证需附带质量异常证明(财务风险控制)。}
\end{quote}

KOS-TL 通过在 Core 层定义交叉依存类型来实现这一目标。

\begin{enumerate}[label=(\arabic*)]

    \item 质量领域的逻辑扩展

在质量 Core 层,我们已经定义了 $r_{quality} : \mathsf{RootCauseReport}$。它证明了电压波动导致了质量问题。

    \item 财务领域的类型定义(新增)

我们在财务 Core 层定义一个新的类型 $\mathsf{AuditLock}$:

$$\mathsf{AuditLock} \equiv \Sigma(inv : \mathsf{Invoice}). \Sigma(r : \mathsf{RootCauseReport}). \mathsf{Proof}(inv.batch = r.f.b)$$

这个财务对象 $AuditLock$ 的构造函数强制要求输入一个 $r : \mathsf{RootCauseReport}$。如果没有生成质量报告 $r$,财务层就无法构造出合法的“审计锁”对象。

\end{enumerate}

当质量追溯完成,Kernel 状态 $\sigma$ 中新增了项 $r_{final}$。此时,跨领域的逻辑引擎会自动触发下一步归约:
\begin{itemize}

    \item 财务观察者触发

    财务系统的 analyzeAudit 程序监测到状态 $\sigma$ 中出现了一个与发票 $inv_{01}$ 批次相同的质量异常报告 $r_{final}$。

    \item 跨领域项合成

    Kernel 尝试通过 Unification(合一运算),将质量领域的证据 $r_{final}$ 直接填充到财务领域的 $\mathsf{AuditLock}$ 对象中。

    \item 原子化更新

    $$\langle \sigma, \text{FinanceGoal} \rangle \xrightarrow{small} \langle \sigma \cup \{ \text{lock\_01} \}, \text{unit} \rangle$$其中 $\text{lock\_01} = \langle inv_{01}, r_{final}, \pi_{match} \rangle$。
\end{itemize}

在 KOS-TL 的跨领域架构中,财务系统的 analyzeAudit 程序并不是一个独立的软件模块,而是一个挂载在共享逻辑内核(Kernel)上的“谓词监听器”。它的核心任务是:监听状态 $\sigma$ 的演化,并在满足特定“财务-质量”交叉约束时,自动派生出审计动作。

(1) analyzeAudit函数

在 Core 层中,analyzeAudit 被定义为一个高阶函数,其目的是构造审计锁定项 $\mathsf{AuditLock}$:$$\mathsf{analyzeAudit} : \Pi(inv : \mathsf{Invoice}) \to \text{Option}(\mathsf{AuditLock})$$

其内部的派生逻辑遵循以下规则:$$\frac{inv \in \sigma \quad \exists r : \mathsf{RootCauseReport} \text{ s.t. } \text{Unify}(inv.batch, r.f.b)}{\mathsf{derive}(\mathsf{AuditLock}(inv, r))}$$

\begin{itemize}
    \item 输入:待处理的财务发票 $inv$。

    \item 触发条件:在内核状态空间中发现一个质量报告 $r$,其关联的批次与发票批次合一(Unify)。

    \item 输出:如果条件满足,则产生一个包含质量证据的审计对象;否则返回空。
\end{itemize}

当生产线发生电压波动并生成质量报告时,analyzeAudit 在 Kernel 层经历以下演化:
\begin{enumerate}[label=(\arabic*)]
    \item Step 1: 跨域探测 (Cross-Domain Detection)

    Kernel 扫描到新物化的 $r_{final}$(质量报告)。由于 analyzeAudit 订阅了 BatchID 类型的变动,它立即被激活。

    \item Step 2: 依存项提取 (Dependency Extraction)

    程序通过 inv.batch 索引,定位到关联的财务发票。例如:$r_{final}$ 对应批次:Batch\_10-01找到发票:Inv\_2023\_009 (供应商:SteelCo)

    \item Step 3: 证明透传 (Proof Transparency)

    这是最关键的一步。analyzeAudit 并不只是简单地标记“有问题”,而是直接引用(Reference)质量领域的证明项 $\pi_{causal}$。$$lock_{item} = \langle Inv_{009}, r_{final}, \pi_{match} \rangle$$这意味着财务系统现在持有了质量异常的物理证据(电压曲线、计算记录)。

    \item Step 4: 动作触发 (Action Materialization)

    一旦 $lock_{item}$ 在内核中实例化,Runtime 层的财务插件会捕捉到这一状态变化,立即在 ERP 系统中执行:挂起付款:冻结 Inv\_2023\_009 的结算。审计背书:自动将 $r_{final}$ 的逻辑轨迹附加到发票附件中,供财务人员审核。
\end{enumerate}

\begin{algorithm}
\caption{KOS-TL 跨域审计程序:$\mathsf{analyzeAudit}(r_{final})$}
\begin{algorithmic}[1]
\REQUIRE 新物化的质量根因报告 $r_{final} = \langle f, a, e, \pi_{causal} \rangle : \mathsf{RootCauseReport}$
\ENSURE 审计锁定项 $\mathsf{AuditLock}$ 或 $\emptyset$
\STATE \textbf{Step 1: 跨域探测 (Cross-Domain Detection)}
\STATE 监听内核状态 $\sigma$。当新事实 $r_{final}$ 被物化时,触发器激活。
\STATE 提取关联批次索引:$b \leftarrow r_{final}.f.b$
\STATE \COMMENT{利用 $BatchID$ 作为跨域合一(Cross-domain Unification)的唯一键}
\STATE \textbf{Step 2: 依存项提取 (Dependency Extraction)}
\STATE 在财务域状态空间 $\sigma_{fin}$ 中检索关联发票:
\STATE $inv \leftarrow \{ i \in \sigma_{fin} \mid i : \mathsf{Invoice} \wedge \mathsf{Unify}(i.batch, b) \}$
\IF{$inv = \emptyset$}
    \STATE \RETURN $\emptyset$ \COMMENT{当前批次无挂单发票,无需审计动作}
\ENDIF
\STATE \textbf{Step 3: 证明透传 (Proof Transparency)}
\STATE 构造证明匹配谓词 $\pi_{match}$,证明发票 $inv$ 与质量失效 $r_{final}$ 的因果关联。
\STATE 引用质量域证明:$\pi_{ref} \leftarrow r_{final}.\pi_{causal}$
\STATE 实例化审计锁对象:$lock_{item} \leftarrow \langle inv, r_{final}, \pi_{match}, \pi_{ref} \rangle$
\STATE \COMMENT{此时财务对象持有物理层的电压曲线、计算记录等原始证据证明}
\STATE \textbf{Step 4: 动作触发 (Action Materialization)}
\STATE \textbf{原子物化}:$\sigma \leftarrow \sigma \cup \{ lock_{item} \}$
\STATE \COMMENT{Runtime 层财务插件监听此状态变更并执行以下物理动作:}
\STATE \textbf{Action A:} 执行 $\mathsf{FreezePayment}(inv)$ \COMMENT{冻结发票结算流程}
\STATE \textbf{Action B:} 执行 $\mathsf{AttachEvidence}(inv, r_{final}.trace)$ \COMMENT{自动附加逻辑轨迹至 ERP 附件}
\RETURN $lock_{item}$
\end{algorithmic}
\end{algorithm}

假设财务规则增加了一条:“只有当质量损失金额超过货款 $20\%$ 时才触发锁定。”

在 KOS-TL 中,你只需要在 analyzeAudit 的定义中增加一个逻辑算子:
$$\mathsf{Proof}(r.loss\_estimate > inv.total \times 0.2)$$

即使产生了质量报告,如果损失评估不够高,analyzeAudit 的合一运算就会失败,财务系统会自动保持“正常结算”状态。这种逻辑下推确保了财务部门的业务决策始终与生产一线的物理现实保持数学上的一致。

\subsection{KOS-TL的反事实推理}


在 KOS-TL 中,反事实推理不是通过“猜测”实现的,而是通过内核对平行状态空间(Parallel State Spaces)的模拟来完成的。

1. KOS-TL 如何定义反事实

在传统 AI 中,反事实通常是 $P(y | do(x))$ 的概率计算;而在 KOS-TL 中,反事实被定义为一个虚拟配置的求值过程:“如果在 $\sigma_0$ 中,事实 $a$ (电压波动)没有发生,那么 $r_{final}$(失效证明)是否还能在内核中实例化?”

2. 实现机制:虚拟上下文 (Virtual Context)

KOS-TL 通过 $\Gamma$ 的分支(Branching) 来实现这一功能。

\begin{enumerate}[label=(\arabic*)]

    \item 构造影子状态 ($\sigma'$):内核复制当前的知识库 $\sigma_0$,但有针对性地移除或修改某个事实项(例如移除 $a_{volt}$)。

    \item 假设性求值:在新的配置 $\langle \Gamma, \sigma', \mathsf{analyze}(f_0) \rangle$ 下重新启动 Small-step 演化。

    \item 引理对比:如果演化结果归约为 $\bot$(空类型),说明 $a_{volt}$ 是 $f_0$ 的必要条件(必要因果)。如果演化依然能生成 $r'_{final}$,说明存在冗余因果或 $a_{volt}$ 只是干扰噪声。
\end{enumerate}


3. 反事实推理的三种应用场景

\begin{enumerate}[label=(\arabic*)]

    \item 根因敏感度分析 (Sensitivity Analysis)提问:如果电压波动仅为 $2\%$ 而不是 $10\%$,硬度还会不达标吗?内核动作:修改 $\sigma$ 中的 $a_{volt}$ 数值,观察 $\mathsf{mkCausalProof}$ 在 Core 层定义的物理阈值检查中是否还能通过。

    \item 责任归属判定 (Liability Attribution)提问:如果没有供应商 A 提供的这批次原材料,单凭 M\_03 炉的波动会造成失效吗?内核动作:在虚拟环境中移除原材料事实,观察证明链是否断裂。这直接支撑了上层 analyzeAudit 的追偿逻辑。

    \item 预防性模拟 (Preemptive Simulation)提问:如果明天把 M\_03 炉的功率提升 $5\%$,会诱发类似的 $f_0$ 吗?内核动作:这种“前瞻性反事实”允许系统在物理世界发生事故前,先在逻辑空间内完成“虚拟事故”的推演。
\end{enumerate}


4. 逻辑表达形式

在 KOS-TL 的公式化表达中,反事实推理通常记作:

$$\mathcal{C} \vdash \neg a \implies \neg (\exists \pi : \mathsf{Proof}(f))$$

即在当前配置下,证明“若 $a$ 不存在,则 $f$ 的证明项不可被实例化”。

在KOS-TL中,如何利用反事实推理(Counterfactual Reasoning)来确定故障的根本原因(Root Cause)。
\begin{enumerate}[label=(\arabic*)]
    \item 构造与验证的解耦(Construction vs. Validation)

    在 analyze 函数中,$r_{final}$ 确实是一个包含了 $a_{volt}$ 的候选构造。但反事实推理询问的是:“如果从公理系统中抽离掉 $a_{volt}$ 这个原子项,是否还能构造出另一个同样指向 $f_0$ 的合法证明 $\pi'$?”如果能:说明 $a_{volt}$ 虽然在 $r_{final}$ 里,但它只是“充分非必要条件”(或者是多重因果)。如果不能:说明 $a_{volt}$ 是“必要条件”。

    \item 内核的“平行空间”机制:$\sigma$ 与 $\sigma \setminus \{a\}$

    在执行反事实推理时,Kernel 并不是在处理同一个三元组,而是产生了一个环境切片。现实轨迹:$\langle \Gamma, \sigma, f_0 \rangle \implies r_{final}$ (包含 $a_{volt}$)反事实轨迹:$\langle \Gamma, \sigma \setminus \{a_{volt}\}, f_0 \rangle \implies \bot$ (推导坍缩,得到空集)当反事实轨迹归约为 $\bot$ (底类型/空集)时,它在逻辑上倒逼并加强了现实轨迹的权威性。这在逻辑学上称为“通过否定来肯定”。

    \item 类型论中的“消除规则”(Elimination Rule)

    反事实推理实际上是在计算贡献度(Contribution):$$\text{Contrib}(a, f) \iff (\sigma \vdash f) \wedge (\sigma \setminus \{a\} \nvdash f)$$ 这里的 $\nvdash$ 表示在没有 $a$ 的情况下,无法从剩下的知识库中推导出失效 $f$ 的证明。如果这个等式成立,就排除了“即便是电压不波动,硬度也会因为其他原因降低”的可能性。
\end{enumerate}

\subsection{逻辑作为系统内核}

通过上述推理过程,KOS-TL 展示了其作为逻辑系统的核心优势:

\begin{enumerate}

\item \textbf{构造即证据}

    生成的$\mathsf{RootCauseReport}$不仅仅是文本,它包含了指向原始设备日志和质检记录的指针及逻辑证明链,实现了“可解释性”的硬保证。
\item \textbf{状态保持性}

    在 Small-step 每一跳中,内核均验证$Post$ 约束。如果追溯结果违反了物理逻辑(如原因发生在结果之后),该步转移将失败,确保系统状态  始终处于逻辑一致的空间。
\item \textbf{计算完备性}

    与描述逻辑的 Open World Assumption 不同,KOS-TL 在内核层利用强规范化性质,保证溯源程序在有限步内必然给出确定的因果结论或报告证据不足。
\end{enumerate}

在该实例中,KOS-TL 将质量追溯从“事后审计”转变为“实时逻辑演算”。系统不是在询问“为什么坏了”,而是在尝试构造一个关于“坏了”的完整逻辑证明。这种从 \textbf{Truth-judgement}(真假判断)向 \textbf{Proof-construction}(证明构造)的范式转移,使得制造业的复杂知识管理具备了操作系统级的严谨性。

本节通过大型制造业的质量异常分析场景,展示 KOS-TL 如何将异构业务表映射为依存类型,并通过 Small-step 操作语义实现知识的自动派生与因果追溯。

\begin{table}[htbp]
\centering
\caption{描述逻辑 (DL) 与 KOS-TL 的根本差异}
\begin{tabular}{l|l|l}
\hline
\textbf{维度} & \textbf{描述逻辑 (DL)} & \textbf{KOS-TL} \\ \hline
核心范式 & 静态本体一致性 & 动态操作语义 \\
逻辑性质 & 真值判定 (Truth) & 证明构造 (Proof) \\
时间处理 & 外部扩展 (Temporal DL) & 内生时间排序约束 \\
应用目标 & 知识描述与查询 & 知识操作系统内核 \\ \hline
\end{tabular}
\end{table}

\begin{quote}
\textit{“KOS-TL 并非用于简单地描述世界,而是用于运行一个以知识为内核的、逻辑闭环的操作系统。”}
\end{quote}

在 KOS-TL 中,知识的派生(Derivation)本质上就是一种高阶函数的构造与应用过程。

传统系统是“用硬编码的函数去处理数据”,而 KOS-TL 体现的是“数据驱动了逻辑项的合成,进而形成了具备推理能力的派生函数体”。我们可以从以下三个维度来理解这种“数据派生函数”的机制:
\begin{enumerate}

    \item 从“数据元组”到“证明项”的升维
    
    在 Runtime 层,原始数据(Batch\_202310-01, 10:00)只是被动的信息。但在精化(Elaboration)过程中,它被封装成了带有逻辑签名的 $\Sigma$-项。实质:这些项在 Kernel 层看来,不再是单纯的“值”,而是一个个小型的、待组合的函数片段。例如,$e_{proc}$ 携带的 $\pi_{route}$ 实际上是一个能够证明“路径合法性”的逻辑函数。

    \item 动态生成的“因果链”,即函数合成

    analyze 函数或 RootCauseReport 的构造过程,实际上是 Kernel 根据实时数据,动态地合成了(Compose)一个新的逻辑函数。

    当 Kernel 发现 $a_{volt}$ 和 $f_0$ 的数据匹配时,它并不是简单地把它们连起来,而是构造了一个新的 Lambda 项:
    
    $$\lambda (t) . \text{proof\_of\_overlap}(t, e_{proc}.dur)$$

    这个新生成的逻辑项 $\pi_{final}$ 实际上就是一个特化(Specialized)的函数。它专门用于处理“在这个特定时间、这台特定机器上发生的这一类硬度问题”。

    \item 三层架构中的“派生”分工

    轴承案例展示了函数是如何在不同层级间“流动”并被“物化”的(如表\ref{tab:layer-derived-behavior}所示。)。

\begin{table}[h]
\centering
\caption{层次与派生行为本质在轴承案例中的表现}
\label{tab:layer-derived-behavior}
\begin{tabular}{>{\centering\arraybackslash}p{1.5cm} >{\raggedright\arraybackslash}p{4cm} >{\raggedright\arraybackslash}p{9cm}}
\toprule
\textbf{层次} & \textbf{派生行为的本质} & \textbf{轴承案例中的表现} \\
\midrule
Core & 定义函数的“范畴” & 定义了 mkCausalProof 这个高阶函数模板。它规定了函数必须满足的输入输出类型。 \\
Runtime & 提供函数的“算子” & 从 SQL 表中提取具体数值,并将其精化为具有逻辑含义的原子项(如 $f_0, a_{volt}$)。 \\
Kernel & 实现函数的“演算” & 通过 Small-step 语义,将碎片化的数据项填充进 Core 层的模板中,物化出一个具体的因果证明函数 $R$。 \\
\bottomrule
\end{tabular}
\end{table}

    在传统软件中,如果你想增加一种“电压导致硬度不均”的追溯逻辑,你需要手动写一个 checkVoltage() 函数。但在 KOS-TL 中:你只需在 Core 层定义一般的因果原理(即 $t(a) < t(f)$ 且物理关联)。当 Batch\_202310-01 的特定电压数据进入系统时,Kernel 会利用这些数据**派生(Derive)**出一个针对该批次的、特化的证明函数实例。

    函数不是预先写死的,而是根据当前的系统状态 $\sigma$ 和捕获的事实 $f$ 动态推导出来的逻辑结果。总结这个例子完美诠释了 "Proof as Program" (证明即程序) 的哲学:推导出一个质量根因的证明,等同于派生出了一个能够解释该失效的逻辑函数。数据(电压、时间、批次)不再仅仅是被处理的对象,它们成为了构造这个“逻辑函数”的零部件。
\end{enumerate}

\section{本章小结}

\subsection{哲学范式:从“真理理论”转向“可执行规范”}
自弗雷格以来,传统逻辑(一阶逻辑、描述逻辑)的核心是静态真值($\mathcal{M} \models \varphi$),其目标是刻画世界“是什么样”。KOS-TL 明确拒绝了这种“单一模型中心论”,实现了逻辑角色的根本转变:

\begin{itemize}
\item \textbf{状态语义(State-based Semantics)}

语义的基本单位不再是永恒不变的模型,而是动态演化的状态序列($\sigma_0 \to \sigma_1 \to \dots$)。逻辑判断的中心从“命题是否为真”转向了“状态转移是否合法”。

\item \textbf{构造主义立场(Constructivism)}

继承 Martin-Löf 类型论的精髓,将“命题即类型,证明即程序”推广为“知识即类型,操作即程序,事件即构造子”。在 KOS-TL 中,知识的存在性由其是否可被构造决定,这使得系统在哲学上是自洽且透明的。

\item \textbf{知行合一}

逻辑不再仅仅用于描述,而是成为一种可执行的规范系统。它不仅刻画真理,更刻画了知识如何被操作、更新与执行,弥合了逻辑推理与物理操作之间的断层。
\end{itemize}

\subsection{逻辑特性:事件驱动与操作语义的融合}
KOS-TL 的本质创新在于将操作语义引入逻辑核心,使其具备了“计算即推理”的能力:

\begin{itemize}
\item \textbf{事件作为一等公民(Events as First-class Constructors)}

不同于描述逻辑中被动的事实记录,事件在 KOS-TL 中是连接理解与操作的枢纽。它是类型论中的构造子,规定了对象生成的合法机制,为系统提供了处理时间敏感性、因果关系和状态迁移的对象基础。

\item \textbf{操作性派生规则(Operational Derivation)}

派生规则不再是纯粹的真值保持,而是一种操作性规范。它定义了系统在特定上下文和时间约束下,如何“合法地”产生新语义标注(如风险预警)。这种规范性立场允许系统在不同场景下表现出不同的推理策略,具备极高的灵活性与可解释性。

\item \textbf{小步操作语义(Small-step Operational Semantics)}

逻辑判断的形式演变为 $\langle \Sigma, c \rangle \to \Sigma'$。这种微观层面的演化刻画,使得 KOS-TL 能够精准管理知识库的单调性与资源消耗,使其更接近“因果推理计算机”。
\end{itemize}

\subsection{系统能力:计算反射性驱动的自治与审计}
反射性(Reflexivity)在 KOS-TL 中从编程技巧提升到了形式化自省的高度:

\begin{itemize}
\item \textbf{计算即证明的“自证”}

基于柯里-霍华德同构,内核在执行每一步计算(规约)时,都会同步合成等价性证明项($\text{Id}$ 证明)。这意味着系统每走一步都会留下不可篡改的逻辑脚印,证明其行为符合预设公理。

\item \textbf{全路径逻辑审计(Auditability)}

由于存在反射机制,审计不再依赖外部文本日志,而是变成了一种实时数学校验。证明链的完备性直接决定了系统状态的合法性,强制实现了“透明治理”。

\item \textbf{逻辑自愈(Self-Inspection \& Healing)}

反射性允许内核“回看”决策路径,在遇到逻辑矛盾时能够定位公理冲突并触发自愈算子,为系统的自主运行提供了决策依据。
\end{itemize}

\subsection{工程范式:类型程序设计(TDD)的边界拓宽}
KOS-TL 将类型系统从“内存安全工具”升华为“系统自治公理”,为现代系统设计提供了核心启示:

\begin{itemize}
\item \textbf{从“类型安全”到“逻辑确定性”}

通过物理-逻辑双向精化类型,将物理定律和业务规则内生为类型的属性。这种“带有证据的类型”确保了非法状态在系统设计层面就是不可表示的。

\item \textbf{持久化依存类型存储}

打破了数据库作为原始字节堆填的现状,使存储成为类型系统的运行时延伸。利用类型的单调性约束管理数据生命周期,确保了知识演化的因果一致性。

\item \textbf{跨层级逻辑透镜(Refinement Lenses)}

通过严密的双向精化映射,解决了高层业务实体(如“合规转账”)与底层物理存储(如 SQL 条目)之间的断层,保证了每一步物理动作都忠实于高层逻辑意图。
\end{itemize}

\subsection{KOS-TL 的本质}
KOS-TL 是一个将直觉主义类型论的构造语义、操作语义与知识工程实践完美融合的逻辑系统。它不仅仅是一套算法,更是一套\textbf{“系统宪法”}。
它证明了:通过 依存类型($\Pi$ \& $\Sigma$)+ 事件构造子 + 反射性,可以构建出一种逻辑自洽、因果可追溯、且能与物理世界严密对齐的智能自治系统。它将证明从单纯的程序正确性校验,进化到了驱动复杂现实世界决策的核心动力。
