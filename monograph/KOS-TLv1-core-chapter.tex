% 本章:core层的实现
% 整合 kos-core 的完整技术文档

\chapter{Core 层的实现}

本章介绍 KOS-TL 的 Core 层在工程上的实现形式。当前实现采用 Haskell 语言,在 \texttt{kos-core/} 目录下形成独立的形式化内核,作为系统的“宪法”层,确保非法类型无法通过类型检查进入系统。

\section{设计原则与架构}

\subsection{设计原则}

kos-core 遵循以下设计原则,与 Kos.pdf 中 Core 层的角色定位一致:

\begin{itemize}
\item \textbf{Parser 即门卫}:所有 Term 仅能通过解析 Core DSL(.kos 源文件)产生,非法语法无法生成 AST。
\item \textbf{归纳构造}:代数数据类型(Algebraic Data Type)保证每个 Term 变体都有明确定义,无 $\bot$ 构造子。
\item \textbf{类型检查即合法性}:\texttt{parseAndCheck} 通过后才产出合法 Term,失败则无输出。
\item \textbf{双轴 Universe}:$\mathcal{U}_i$(计算轴)、$\textsf{Type}_i$(逻辑轴),对应 Kos.pdf 设计。
\end{itemize}

\subsection{目录结构}

\begin{verbatim}
kos-core/
├── src/
│   ├── KosCore.hs          -- 主模块,对外 API
│   └── KosCore/
│       ├── AST.hs          -- Term 代数数据类型
│       ├── Universe.hs     -- 双轴 Universe
│       ├── Context.hs       -- 类型上下文
│       ├── Substitution.hs -- 变量替换
│       ├── Reduction.hs    -- β/ι/δ/ζ/η 归约
│       ├── TypeCheck.hs    -- 双向类型检查
│       ├── Parser.hs       -- 语法分析
│       ├── Alpha.hs        -- α 等价
│       └── JSON.hs          -- JSON 序列化
├── app/Main.hs             -- CLI 入口
└── examples/               -- .kos 示例
\end{verbatim}

\section{语法与类型构造}

\subsection{类型构造(Types)}

与 Kos.pdf 的 Core 层语法对应,kos-core 实现的类型构造如表\ref{tab:core-types}所示。

\begin{table}[h]
\centering
\caption{Kos.pdf 类型构造与 kos-core 实现对照}
\label{tab:core-types}
\begin{tabular}{lll}
\toprule
\textbf{Kos.pdf 定义} & \textbf{kos-core 实现} & \textbf{状态} \\
\midrule
$\textsf{Prop}$ & \texttt{Prop Text} & 已实现 \\
$\textsf{Type}_i$ & \texttt{Universe Logical i} & 已实现 \\
$\mathcal{U}_i$ & \texttt{Universe Computational i} & 已实现 \\
$\textsf{Val}, \textsf{Time}, \textsf{ID}$ & \texttt{Val}, \texttt{Time}, \texttt{Ident} & 已实现 \\
$\Pi(x:A).B$ & \texttt{Pi Text Term Term} & 已实现 \\
$\Sigma(x:A).B$ & \texttt{Sigma Text Term Term} & 已实现 \\
$A + B$ & \texttt{Sum Term Term} & 已实现 \\
$\textsf{Id}_A(a, b)$ & \texttt{Id Term Term Term} & 已实现 \\
\bottomrule
\end{tabular}
\end{table}

\subsection{项构造(Terms)}

\begin{table}[h]
\centering
\caption{Kos.pdf 项构造与 kos-core 实现对照}
\label{tab:core-terms}
\begin{tabular}{lll}
\toprule
\textbf{Kos.pdf 定义} & \textbf{kos-core 实现} & \textbf{状态} \\
\midrule
$x$(变量) & \texttt{Var Text} & 已实现 \\
$\lambda x.t$, $t\,u$ & \texttt{Lam}, \texttt{App} & 已实现 \\
$\langle t, u \rangle$ & \texttt{Pair Term Term} & 已实现 \\
$\textsf{split}(t, x.y.u)$ & \texttt{Split Term Text Text Term} & 已实现 \\
$\textsf{inl}(t)$, $\textsf{inr}(t)$ & \texttt{InL}, \texttt{InR};\texttt{InLS}, \texttt{InRS} & 已实现 \\
$\textsf{case}(t, x.u, y.v)$ & \texttt{Case Term Text Text Text Term} & 已实现 \\
$\textsf{refl}$(Id 引入) & \texttt{Refl Term} & 已实现 \\
$\textsf{let } x=u \textsf{ in } t$ & \texttt{Let Text Term Term Term} & 已实现 \\
\bottomrule
\end{tabular}
\end{table}

\subsection{值依赖谓词(扩展)}

kos-core 在 Kos.pdf 未定义的语法基础上,增加了值依赖谓词(Value-Dependent Predicates),使类型构造可依赖项的具体数值:

\begin{itemize}
\item \texttt{gt(a, b)}:$a > b$,当 $a$、$b$ 可解析为数值时成立则类型良构。
\item \texttt{ge(a, b)}:$a \ge b$
\item \texttt{lt(a, b)}:$a < b$
\item \texttt{le(a, b)}:$a \le b$
\item \texttt{eq(a, b)}:$a = b$(支持数值或字符串相等)
\end{itemize}

例如 \texttt{gt(val "200", val "180")} 表示命题“200 > 180”,类型检查时会对两边的值进行求值比较,成立则接受。

\section{.kos 语言语法}

\subsection{模块结构}

\begin{verbatim}
module ModuleName where
  type Name : Kind
  def name : Type := term
\end{verbatim}

\texttt{type} 声明类型别名,\texttt{def} 声明项定义(支持 $\delta$ 归约)。

\subsection{原子类型与 Base Sorts}

\begin{itemize}
\item \texttt{Prop P}:命题,$\textsf{Prop} : \textsf{Type}_1$
\item \texttt{U0}, \texttt{U1}, \texttt{U}:计算轴 Universe
\item \texttt{Type0}, \texttt{Type1}, \texttt{Type}:逻辑轴 Universe
\item \texttt{val "x"}:值字面量
\item \texttt{time "2025-01-01"}:时间字面量
\item \texttt{id "BATCH1"}:标识符字面量
\end{itemize}

\subsection{Π 类型与 λ}

\begin{itemize}
\item \texttt{Pi(x:A). B} 或 \texttt{Π(x:A). B}:依赖函数类型
\item \texttt{A -> B}:箭头简写(非依赖)
\item \texttt{lam (x:A) . t}:$\lambda$ 抽象
\item \texttt{f a}:函数应用
\end{itemize}

\subsection{Σ 类型}

\begin{itemize}
\item \texttt{Sigma(x:A). B}:依赖对类型
\item \texttt{<d, p>}:对构造
\item \texttt{split (p) as x y in body}:$\Sigma$ 消除
\end{itemize}

\subsection{Sum 类型}

\begin{itemize}
\item \texttt{A + B}:不交并
\item \texttt{inl(A,B,v)}, \texttt{inr(A,B,v)}:显式类型注入
\item \texttt{inl v}, \texttt{inr v}:单参数注入(需 check 模式)
\item \texttt{case s of inl x -> t1; inr y -> t2}:Sum 消除
\end{itemize}

\subsection{Id 类型与 Let}

\begin{itemize}
\item \texttt{Id(A,a,b)}:恒等类型
\item \texttt{refl w}:自反证明
\item \texttt{let x : A := v in body}:Let 绑定($\zeta$ 归约)
\end{itemize}

\section{推导规则与归约}

\subsection{推导规则实现}

TypeCheck.hs 实现双向类型检查(infer/check),对应 Kos.pdf 的推导规则:

\begin{itemize}
\item \textbf{Π 引入}:\texttt{Lam} 分支,\texttt{check ctx' bodyTerm bodyTy}
\item \textbf{Π 消除}:\texttt{App} 分支,\texttt{Right (substitute x a body)}
\item \textbf{Σ 引入}:\texttt{Pair},\texttt{check ctx d dom},\texttt{check ctx p (substitute x d body)}
\item \textbf{Σ 消除}:\texttt{Split (Pair d p) x1 x2 body} $\to$ \texttt{substitute x2 p (substitute x1 d body)}
\item \textbf{Conversion}:\texttt{check} 结构失败时 fallback:\texttt{infer} + \texttt{definitionallyEqual}
\end{itemize}

\subsection{归约规则实现}

Reduction.hs 实现 $\beta$、$\iota$、$\delta$、$\zeta$、$\eta$:

\begin{itemize}
\item $\beta$:\texttt{App (Lam x \_ body) arg -> substitute x arg body}
\item $\iota$(Σ):\texttt{Split (Pair d p) x1 x2 body -> substitute x2 p (substitute x1 d body)}
\item $\iota$(Sum):\texttt{Case (InL ...)} 与 \texttt{Case (InR ...)} 分支
\item $\delta$:\texttt{Var x -> ctxLookupDef x ctx}
\item $\zeta$:\texttt{Let x \_ty val body -> substitute x val body}
\item $\eta$:\texttt{Lam x ty (App f (Var y))} 当 $y=x$ 且 $x \notin \text{FV}(f)$ 时归约为 \texttt{f}
\end{itemize}

定义等价 \texttt{definitionallyEqual} = \texttt{normalize} + $\alpha$ 等价。

\section{双轴 Universe 系统}

Universe.hs 实现:

\begin{itemize}
\item $\textsf{Prop} : \textsf{Type}_1$
\item $\textsf{Type}_i : \textsf{Type}_{i+1}$,$\mathcal{U}_i : \mathcal{U}_{i+1}$
\item $\mathcal{U}_i : \textsf{Type}_{i+1}$(计算轴可提升到逻辑轴)
\item $\textsf{Prop} \hookrightarrow \mathcal{U}_1$(\texttt{propEmbedToData})
\item Impredicativity:$\Pi(x:A).B : \textsf{Prop}$ 当 $B : \textsf{Prop}$
\end{itemize}

\section{与 C Runtime 的集成}

kos-core 作为独立可执行文件,通过 C Bridge(\texttt{kos\_core\_bridge.h})与 Kernel/Runtime 层集成:

\begin{itemize}
\item \texttt{kos\_core\_bridge\_check\_expr}:对 .kos 表达式进行类型检查
\item \texttt{kos\_core\_bridge\_term\_from\_kos}:解析并校验后返回 JSON 序列化
\item \texttt{kos\_core\_bridge\_infer\_from\_kos}:类型推断
\end{itemize}

C 层在添加类型定义、精化信号时,可调用上述接口,确保仅合法类型进入系统。

\section{实现状态与 Kos.pdf 对照}

表\ref{tab:core-comparison}总结 kos-core 与 Kos.pdf 的对照情况。

\begin{table}[h]
\centering
\caption{kos-core 与 Kos.pdf 实现对照}
\label{tab:core-comparison}
\begin{tabular}{lll}
\toprule
\textbf{维度} & \textbf{Kos.pdf 要求} & \textbf{kos-core 实现} \\
\midrule
类型构造 & Prop, Type$_i$, U$_i$, Val, Time, ID, Π, Σ, +, Id & 已实现 \\
项构造 & λ, app, pair, split, inl, inr, case, refl, let & 已实现 \\
Universe & 双轴、Prop↪U₁、Impredicativity & 已实现 \\
归约 & $\beta$, $\iota$, $\delta$, $\zeta$, $\eta$ & 已实现 \\
Conversion & $\Gamma \vdash t:A$, $A \equiv B$ $\Rightarrow$ $\Gamma \vdash t:B$ & 已实现 \\
元理论 & SN, Subject Reduction, Confluence & 无形式化证明 \\
本体管理 & 业务本体 $\to$ Sorts/类型 & 未实现 \\
计算自反性 & 自动合成 Id 证明 & 部分:有 Id/Refl,无自动证明合成 \\
\bottomrule
\end{tabular}
\end{table}

\subsection{已知简化}

\begin{itemize}
\item \textbf{Π 类型推断}:\texttt{infer} 对 Π 构造统一返回 \texttt{Type₁};Kos.pdf 规定 predicative 情形为 $\Pi(x:A).B : \textsf{Type}_{\max(i,j)}$。
\item \textbf{Pair 类型推断}:对依赖 $B(x)$ 的 $\Sigma$ 类型做了简化。
\end{itemize}

\subsection{进一步改进方向}

\begin{itemize}
\item J 消除子:Id 类型的依赖消除(J rule),当前仅支持 refl 引入
\item 元理论形式化:SN、Subject Reduction、Confluence 的机器可验证证明
\item 本体管理器:业务本体 $\to$ Sorts/类型 的 TypeOntology 模块
\item 计算自反性自动合成:归约时自动合成 $\textsf{Id}(t,t')$ 证明
\end{itemize}
