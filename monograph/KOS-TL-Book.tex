% KOS-TL 知行逻辑:知识操作系统的类型论基础
% 参考 Coc.tex 的中文图书样式,使用 pdflatex 编译
%
% 编译:仅用 pdflatex(不运行 BibTeX)。推荐双击 monograph/build.bat,或:
%   pdflatex KOS-TL-Book && pdflatex KOS-TL-Book
% 若出现 "invalid character" 或 "Buffer size exceeded",请先删除 KOS-TL-Book.aux 再编译。
%
% 全文审查:已修正导言区重复宏包与编号深度、脚注表述、标点与正文中 ** 为 \textbf。

\documentclass[9pt]{book}

% 编码与 CJK 支持(使用 pdfLaTeX 编译)
\usepackage[UTF8]{ctex}
\usepackage[T1]{fontenc}
\usepackage{textcomp}

\usepackage{tikz}
\usepackage{tikz-qtree}
\usetikzlibrary{shapes,arrows,backgrounds,fit,positioning,calc}
\usepackage{booktabs}  % KOS-TLv1 表格使用 \toprule, \midrule, \bottomrule
\usepackage{algorithm}
\usepackage{algorithmic}
\usepackage{stmaryrd}
\usepackage{alltt}
\usepackage{tabularx}
\usepackage{multirow} % 用于表格中的多行合并
\usepackage{float}    % 提供 [H] 固定浮动体位置,避免插入列表导致 missing \item

% Springer 图书格式设置(紧凑版)
\usepackage{geometry}
\geometry{
  paperwidth=155mm,
  paperheight=235mm,
  textwidth=125mm,
  textheight=200mm,
  left=15mm,
  right=15mm,
  top=15mm,
  bottom=20mm,
  headsep=8pt,
  footskip=10pt
}

% 其他包
\usepackage{listings}
\usepackage{xcolor}
\usepackage{amsmath}
\usepackage{mathtools}  % 提供 \xRightarrow
\usepackage{amssymb}
\usepackage{amsthm}
\usepackage{array}  % KOS-TLv1 表格使用 >{\raggedright\arraybackslash}
\usepackage[colorlinks=true,
            linkcolor=blue,
            citecolor=blue,
            urlcolor=blue,
            bookmarks=true,
            bookmarksopen=true,
            bookmarksdepth=2,
            pdfstartview=FitH,
            unicode=true]{hyperref}
% 本稿使用 \begin{thebibliography},不运行 BibTeX;若用 texify 请改为仅 pdflatex 或使用 monograph/build.bat
\usepackage{fancyhdr}
\usepackage{titlesec}
\usepackage{titletoc}

% 页眉页脚
\pagestyle{fancy}
\fancyhf{}
\fancyhead[LE]{\small\leftmark}
\fancyhead[RO]{\small\rightmark}
\fancyfoot[C]{\small\thepage}
\renewcommand{\headrulewidth}{0.3pt}
\renewcommand{\footrulewidth}{0pt}
\setlength{\headheight}{12pt}

% chapter (第一级)
\titleformat{\chapter}[display]
  {\normalfont\normalsize\bfseries\centering}
  {\chaptertitlename\ \thechapter}{10pt}{\normalsize}
\titlespacing*{\chapter}{0pt}{10pt}{8pt}

% section (第二级)
\titleformat{\section}
  {\normalfont\normalsize\bfseries}
  {\thesection}{0.8em}{}
\titlespacing*{\section}{0pt}{8pt}{4pt}

% subsection (第三级)  ← 你想用的第三小节
\titleformat{\subsection}
  {\normalfont\normalsize\bfseries}
  {\thesubsection}{0.8em}{}
\titlespacing*{\subsection}{0pt}{6pt}{3pt}

% subsubsection (第四级)  ← 必须补上,否则第三级下面的小节会出问题
\titleformat{\subsubsection}
  {\normalfont\normalsize\bfseries\itshape}
  {\thesubsubsection}{0.8em}{}
\titlespacing*{\subsubsection}{0pt}{5pt}{2.5pt}

% 章节与目录编号深度(3:章/节/小节)
\renewcommand{\thechapter}{\arabic{chapter}}
\renewcommand{\thesection}{\thechapter.\arabic{section}}
\renewcommand{\thesubsection}{\thesection.\arabic{subsection}}
\setcounter{secnumdepth}{3}
\setcounter{tocdepth}{3}

% 行距与段落
\usepackage{setspace}
\singlespacing
\setlength{\parindent}{0pt}
\setlength{\parskip}{3pt plus 1pt minus 0.5pt}
\setlength{\baselineskip}{11pt}

% 图表
\usepackage{caption}
\captionsetup{font=scriptsize, labelfont=bf, textfont=it, skip=4pt}

% 列表(参考 Coc.tex)
\usepackage{enumitem}
\setlist{nosep, leftmargin=*}
\setlist[itemize]{topsep=2pt, itemsep=2pt, parsep=1pt, partopsep=1pt}
\setlist[enumerate]{topsep=2pt, itemsep=2pt, parsep=1pt, partopsep=1pt}
\setlist[description]{topsep=2pt, itemsep=2pt, parsep=1pt, partopsep=1pt}

% 公式间距
\setlength{\abovedisplayskip}{4pt plus 1pt minus 1pt}
\setlength{\belowdisplayskip}{4pt plus 1pt minus 1pt}
\setlength{\abovedisplayshortskip}{2pt plus 1pt}
\setlength{\belowdisplayshortskip}{2pt plus 1pt}

% 全局设置
\usepackage{fancyvrb}
\fvset{fontsize=\tiny, frame=none, baselinestretch=0.9}
\raggedbottom
\sloppy
\emergencystretch=3em
\hbadness=10000
\hfuzz=0.1pt
\vfuzz=0.1pt
\urlstyle{tt}
\renewcommand{\UrlFont}{\ttfamily\footnotesize}

\setlength{\floatsep}{6pt plus 2pt minus 1pt}
\setlength{\textfloatsep}{6pt plus 2pt minus 1pt}
\setlength{\intextsep}{4pt plus 2pt minus 1pt}

% 定理环境(KOS-TLv1 使用 Kos.tex 风格的大写环境名)
\theoremstyle{definition}
\newtheorem{definition}{定义}[chapter]
\newtheorem{Definition}{定义}[chapter]  % KOS-TLv1: \begin{Definition}{...}
\newtheorem{example}{例}[chapter]
\newtheorem{Example}{例}[chapter]       % KOS-TLv1: \begin{Example}
\theoremstyle{plain}
\newtheorem{theorem}{定理}[chapter]
\newtheorem{Theorem}{定理}[chapter]     % KOS-TLv1: \begin{Theorem}
\newtheorem{lemma}[theorem]{引理}
\newtheorem{Lemma}[theorem]{引理}       % KOS-TLv1: \begin{Lemma}
\newtheorem{proposition}[theorem]{命题}
\newtheorem{corollary}[theorem]{推论}
\theoremstyle{remark}
\newtheorem{remark}{注}[chapter]

\title{知行逻辑:知识操作系统的类型论基础\\[0.5em]\large KOS-TL (Knowledge Operation System Type Logic)}
\author{陈鹏}

\begin{document}

\frontmatter
\maketitle


% ---------------- 致谢扉页 ----------------
\begin{titlepage}
  \centering
  \vspace*{3cm}   % 垂直居中调整
  {\large To Know is To Act.
  \par}
  \vspace{2cm}
  {\large 知道即行动。\par}
  \vfill
%  {\large 陈鹏\par}
%  {\large 2025年9月 \par}
\end{titlepage}

% 目录
\tableofcontents
\cleardoublepage

\mainmatter
\pagenumbering{arabic}



\chapter{绪论}

\section{面向知识操作的逻辑新需求}

知识表征与推理是逻辑的一个重要的应用领域,然而伴随着大规模数据整合与复杂决策系统的普及,知识表征与推理的研究对象正从静态知识库逐步转向持续演化的知识操作体系中。在面向持续演化的知识操作中概念建模或本体一致性验证不再是核心的关注点,其核心挑战转换为如何在事件驱动、状态演化与强工程约束的环境中,对知识进行\textbf{可执行}、\textbf{可追溯}且\textbf{可验证}的操作。

面对知识操作领域的新需求,现有主流逻辑框架(尤其是以描述逻辑(Description Logic, DL)及其语义网实现(如 OWL)为代表的形式体系)在理论假设与语义结构上,与上述需求之间存在根本性不匹配。具体体现在如下几个方面。

\begin{enumerate}[label=(\arabic*)]

    \item 静态语义与动态操作之间的张力

    描述逻辑以静态模型论语义为基础,其核心推理问题围绕概念可满足性、概念包含与实例判定展开。这一范式默认知识是对“世界可能状态”的描述,而非对“系统运行状态”的刻画。相比之下,知识操作中的核心对象是事件、操作与状态转移,其基本问题不再是某一断言是否在某个模型中为真,而是某一操作是否可被合法执行,以及其执行后系统状态如何演化。

    描述逻辑的模型论语义以静态解释结构为核心,其基本目标是刻画“世界在逻辑上可能是什么样”。概念被解释为个体域上的集合,角色被解释为二元关系,推理问题主要围绕可满足性、概念包含与实例判定展开。这种语义结构天然适合于分类学、本体工程和术语推理。

    然而,在知识操作领域所面对的应用场景中,知识并非一个静态集合,而是一个随时间持续演化的状态系统。核心问题不再是“某一断言是否在某个模型中为真”,而是\textbf{“某个事件是否已经发生、某个状态是否已经被更新,以及这些变化将触发哪些新的事实”}。描述逻辑并未将事件与状态转移视为一等逻辑对象,其对动态过程的支持只能通过外部机制或重述化(reification)方式间接实现,这在工程上代价高昂且语义不透明。

    \item 开放世界假设与操作语义的根本冲突

    描述逻辑所坚持的开放世界假设(Open World Assumption)与知识操作中普遍采用的封闭或半封闭世界语义存在根本冲突。在工程实践中,缺失信息往往被视为异常状态或操作失败的依据,而非逻辑上的“未知”。

    描述逻辑坚持开放世界假设(Open World Assumption, OWA),即未知并不等价于假。这一假设在语义网与开放知识环境中具有合理性,但在知识操作中却往往成为障碍。在企业治理、风险控制、合规审计等场景中,“缺失记录”本身即构成一种负面信息或异常状态。

    知识操作系统语义更接近封闭世界假设,未出现的事实被视为未发生的事件,未满足的约束被视为系统错误。这种以封闭世界与可执行约束为核心的语义取向,使得描述逻辑的模型存在性推理难以直接服务于实际系统运行。

    \item 概念语义与名义类型语义(Nominal Type Semantics)的差异

    描述逻辑中的概念语义是外延性的,其成员资格可通过推理动态决定;而在实际知识操作中,类型更多承担的是名义性与约束性的角色。对象是否属于某一类型,并非通过逻辑蕴涵推导,而是系统在数据摄取与操作阶段必须满足的前提条件,其直接决定了操作是否合法、流程是否可继续,而类型错误表现为系统不可执行状态,而非单纯的推理失败。这种类型语义更接近于程序语言与操作系统中的类型系统,而非传统本体逻辑。

    \item 推理目标的根本转移

    最后,描述逻辑的推理目标主要是逻辑蕴涵的证明,而知识操作中的“推理”则更接近于规则驱动的事实物化(materialization),即在给定当前数据状态下,哪些新事实应当被立即生成、存储并参与后续计算\footnote{此处“推理”与规则驱动的方式(如 Datalog、触发器及操作语义)更为接近,而非以证明论或模型论为中心的经典逻辑推理。}。推理的结果并非仅用于回答查询,而是直接改变系统的可观察状态,并对后续操作产生约束。这种推理模式在经典逻辑框架中缺乏直接的形式化刻画。

\end{enumerate}


在知识操作的各类应用中,核心问题主要围绕以下几个方面展开:

\begin{itemize}
    \item \textbf{决策合法性与追溯性:} 如何确保每一个决策背后都可以追溯到其合法性基础,并且可以回放执行路径?
    \item \textbf{反事实推理与根因追溯:} 如何推理出某一决策如果不发生,系统将如何变化,尤其是在复杂系统中,如何找出根本原因?
    \item \textbf{合规性与责任归属:} 如何确保每个行动都符合法律、规范或行业标准,且能明确责任归属?
    \item \textbf{执行轨迹与审计:} 如何提供一个可验证的执行轨迹,以审计操作历史并进行问责?
\end{itemize}

在知识操作领域,问题并不在于逻辑系统是否足够强大,而在于其是否能够原生支持事件、时间、状态变化与可执行规则。传统描述逻辑在静态知识表示方面依然具有不可替代的价值,但其逻辑假设与语义结构并不适合直接作为知识操作系统的内核。

上述差异揭示了一个关键的理论空白,传统的逻辑体系尚未为“知识作为可操作对象”的系统提供一个统一的形式基础。这直接催生了对一种新逻辑系统的需求,该系统应当以事件与状态转移为核心对象,采用类型化与操作语义驱动的推理方式,内建类型化约束以刻画操作合法性,支持规则驱动的状态演化,并在可靠性与可终止性之间取得工程可行的平衡。

针对这一理论缺口,我们研究并提出一种新的形式逻辑体系———知行逻辑(Knowledge Operation System Type Logic,简称KOS-TL)。KOS-TL 旨在以直觉类型论为基础,引入事件化与操作语义,使逻辑系统能够原生刻画知识的理解、操作与状态更新过程,从而为知识操作系统提供一个可验证、可执行且可扩展的逻辑内核。

\section{知识与知识操作逻辑综述}


\subsection{知识逻辑研究的历史脉络}

知识逻辑与知识表示的形式化研究可追溯至二十世纪六七十年代,并随人工智能与逻辑学的发展几经范式更迭。

二十世纪六十年代末至七十年代,语义网络(Semantic Networks)与框架(Frames)等结构被提出,用于表示概念、属性与关系。这类工作多依赖图结构或槽-值表示,缺乏统一的逻辑语义,可判定性与表达能力难以兼顾。同一时期,哲学与语言哲学对“知识”与“信念”的逻辑刻画催生了认知逻辑(Epistemic Logic)与信念修正理论,其重点在于主体信念的静态关系与更新公理,而非面向系统执行的“知识操作”。

二十世纪八十年代专家系统与规则引擎兴起,知识以产生式规则或逻辑子句形式存储,推理以前向/后向链为主。这一时期明确了“知识库 + 推理机”的分离架构,但规则与事实的更新、冲突消解与可追溯性多依赖工程实现,缺乏统一的逻辑语义;大规模规则库的可维护性与一致性成为长期难点。

二十世纪八十年代末至九十年代,描述逻辑(Description Logic)从语义网络与框架中抽象出可判定的概念语言与 TBox/ABox 语义,在表达力与可满足性判定之间取得形式化平衡\cite{Baader2003}。本世纪初,语义网(Semantic Web)与 OWL 将描述逻辑推向 Web 与工业本体,RDF 三元组与 SPARQL 查询成为事实标准。描述逻辑与 OWL~2 逐渐成为语义网与本体工程的主流形式基础,广泛应用于医学、金融与政务本体。知识表示的重心集中在“概念层次、实例分类与一致性维护”,知识图谱在工业界侧重实体关系抽取、链接与问答,逻辑层多基于 RDF/OWL 或图数据库。模型论与开放世界假设成为主流语义假定。

在“世界如何因行动而改变”这一问题上,麦卡锡(McCarthy)提出的情境演算(Situation Calculus)\cite{McCarthy1963} 与后继的事件演算、流演算等,将状态、动作与结果公理化,并用于规划与推理\cite{Reiter2001}。这类工作为动态世界提供了逻辑刻画,侧重公理化的状态转移与查询应答,但推理目标多为“是否存在满足目标的动作序列”或“某公式在某情境是否成立”,与“每一状态转移均携带构造性证明、轨迹可审计”的知识操作需求仍有距离。哲学上,戴维森(Davidson)对行动句逻辑形式的分析\cite{Davidson1967} 将“事件”确立为一等实体,对后来的事件语义与形式语义学影响深远。

马丁-洛夫(Martin-Löf)的直觉主义类型论(ITT)\cite{MartinLof1984} 与“命题即类型、证明即程序”的柯里-霍华德同构对应,在程序验证与证明助手中得到广泛应用\cite{Girard1989}。Coq、Agda、Lean 等证明助手主要服务于程序正确性验证与数学形式化,并未直接针对“知识对象”的存储、演化与合规性审计。类型论在知识表示中的系统应用(如用 $\Sigma$ 类型绑定数据与证明、用 $\Pi$ 类型刻画依赖约束)直至近年才在部分工作中被显式提出;将类型论与操作语义、事件驱动演化及轨迹审计统一在一个分层框架内,仍是开放方向。

Datalog 及各类规则引擎采用前向/后向推理与物化(materialization)维护衍生事实,在数据集成与策略引擎中应用广泛。其推理目标多为“在固定规则下推导出哪些事实成立”。

\subsection{知识逻辑研究中的难点}

知识表示与知识操作逻辑在理论与工程实践中长期面临若干核心困难,部分难点直接驱动了 KOS-TL 的设计取舍。

\subsubsection{可判定性与表达力的张力}

描述逻辑通过限制构造子(如仅允许受限存在量词)在表达力与可满足性可判定性之间取得平衡;一旦引入全称量词、不动点或高阶约束,可判定性往往丧失或复杂度急剧上升。知识操作场景既需要表达丰富的领域约束(如“该批次必须经过某工序且时间在某一区间内”),又需要在有限时间内完成合法性判定。KOS-TL 采用依存类型论作为核心层,在保持构造性可判定的前提下,通过 $\Sigma/\Pi$ 与命题层表达依赖约束,并将“判定”与“证明构造”统一为类型检查,从而在表达力与可判定性之间寻求新的平衡点。

\subsubsection{开放世界与封闭世界的冲突}
描述逻辑的开放世界假设(未知不等于假)适合语义网与开放数据融合,但在企业合规、审计与操作系统中,“未记录即未发生”的封闭世界语义更为自然。知识操作逻辑若完全采用开放世界,则难以将“缺失证据”视为操作失败或异常;若完全采用封闭世界,又与现有本体与链接数据的语义假设不一致。KOS-TL 在内核层与运行层采用封闭世界下的合规判定(未通过类型检查即拒绝),而在核心层的类型定义上仍保持逻辑上的清晰边界,从而在分层中区分“逻辑边界”与“操作假定”。

\subsubsection{动态更新与一致性的代价}
知识库的动态更新(增、删、改)在传统 KR 中常引发一致性维护与推理物化的高昂代价;大规模 TBox/ABox 的增量推理与冲突检测至今仍是难点。KOS-TL 将知识演化建模为“事件驱动的小步状态转移”,每步转移均经核心层验证并产生证明项,状态单调增长(或显式回滚),从而将“一致性”内置于每步的类型检查与轨迹记录中,避免事后全局一致性检查的负担。

\subsubsection{证据缺失与追溯薄弱}
传统知识库与规则引擎中,事实与规则多与“为何成立”的证明脱钩;审计与合规往往依赖外部日志或元数据,易被篡改或与逻辑结论不同步。知识操作逻辑若无法将“合规性依据”与“知识项”原子化绑定,则难以实现“每一结论均可追溯到形式化证明”。KOS-TL 通过 $\Sigma(d:D).P(d)$ 将数据与证明项耦合,并在内核层将轨迹显式记录为演化序列,使“轨迹即证明”成为系统内生的可审计性保证。

\subsubsection{规模、复杂度与工程可行性}
形式化系统往往面临“形式越严、规模越难扩展”的困境。完整依赖类型检查与证明构造在大规模、高并发场景下的性能与可用性,是知识操作逻辑落地必须面对的工程难点。KOS-TL 通过证明无关性、类型擦除与分层隔离(核心层仅负责边界定义,运行层可并行精化)在严谨性与可实施性之间做权衡,其具体实现与优化仍是持续研究课题。

综合历史脉络、发展现状与难点可知,在知识表示与推理方面,描述逻辑与知识图谱侧重静态结构与查询;在行动与变化方面,情境/事件演算等侧重公理化状态转移;在类型论与证明论方面,ITT 与证明助手侧重数学与程序验证。面向“知识操作”的系统化逻辑——即同时涵盖\textbf{知识的类型化表示}、\textbf{事件驱动的小步演化}与\textbf{轨迹即证明的可追溯性}——仍缺乏统一的形式化底座,且长期受制于可判定性与表达力、开放与封闭世界、动态一致性、证据与追溯以及规模与复杂度等困难。KOS-TL 旨在针对这些困难,以分层形式化定义(Core / Kernel / Runtime)将静态逻辑论域、操作语义论域与系统演化论域有机衔接,为可执行、可审计的知识操作系统提供类型论基础。

\chapter{文献综述}

本章对与 KOS-TL 相关的类型论、逻辑框架、程序验证与知识表示等领域的发展脉络做学术性综述,并在各节中\textbf{展开相关理论的核心架构}(如构造演算的语法与推理规则、MLTT 的 $\Pi/\Sigma/\mathsf{Id}$ 形成与消去、LF 的“判断即类型”与编码、HoTT 的恒等类型与 univalence、Hoare 类型论的规范类型与分离逻辑等),最后明确 KOS-TL 作为\textbf{知识操作理论}创新在现有谱系中的定位与差异。

\section{引言:类型论与知识操作}

类型论自二十世纪七十年代 Martin-Löf 直觉主义类型论与八十年代 Coquand 的构造演算(Calculus of Constructions)以来,已形成从“命题即类型、证明即程序”到依赖类型、高阶归纳类型与同伦类型论等丰富谱系。这些工作主要回答两类问题:一是\textbf{数学对象的构造性与可判定性}(何为合法类型、何为合法证明);二是\textbf{程序与规范的关系}(类型安全、规范验证、效应封装)。然而,面向“知识”的\textbf{操作}——即系统级状态的唯一演化、事件轨迹的确定性可重放、以及知识项与证明的原子化绑定与可追溯性——在主流类型论中并未作为核心对象被内建。

知识表示与推理(KR)传统上依赖描述逻辑、本体与规则引擎,侧重静态结构、可满足性与查询;情境演算、事件演算等则侧重状态转移的公理化,但多与类型论和证明论脱钩。KOS-TL 的提出正在于填补“类型化知识表示”与“可执行、可审计的状态演化”之间的空白,其本质是\textbf{知识操作理论}的创新。后续各节将依次展开:纯类型系统与 $\lambda$ 立方体(\ref{sec:pts})、依赖类型论谱系(\ref{sec:dt})、逻辑框架(\ref{sec:lf})、类型论与操作语义的统一(\ref{sec:pfpl})、同伦类型论(\ref{sec:hott})、带效应类型论与 Hoare 类型论(\ref{sec:htt})、线性逻辑(\ref{sec:linear})、动态逻辑(\ref{sec:dyn})、事件溯源与知识表示(\ref{sec:es}--\ref{sec:kr}),以及体系对比与 KOS-TL 的定位(\ref{sec:matrix}--\ref{sec:kos-position})。

\subsection{类型论发展的时间线与文献脉络}

类型论的系统发展可追溯至二十世纪七十年代:Martin-Löf 于 1971--1972 年提出直觉主义类型论的早期版本\cite{MartinLof1984},后经多次修订形成现代 MLTT;Curry-Howard 对应将逻辑与 $\lambda$ 演算的联系明确化\cite{Girard1989}。八十年代,Coquand 与 Huet 提出构造演算(CoC)\cite{Coquand1988},Barendregt 等人给出纯类型系统与 $\lambda$ 立方体的统一框架\cite{Barendregt1992};Harper、Honsell 与 Plotkin 的 LF(LICS 1987)奠定了逻辑框架的传统\cite{Harper1993}。九十年代至本世纪初,ECC\cite{Luo1994}、CIC 与 Coq/Agda 等证明助手的成熟将依赖类型论推向工程应用;同伦类型论与 univalence 则在 2009 年前后由 Voevodsky 等人系统化\cite{Voevodsky2014},并催生 HoTT Book 与 Univalent Foundations 计划\cite{HoTTBook2013}。带效应的依赖类型论(如 Hoare 类型论\cite{Nanevski2008}、Ynot、F$^\star$)与分离逻辑\cite{Reynolds2002}在 Coq 中的实现(如 Iris)则在程序验证与资源推理方面拓展了类型论的边界。上述脉络共同构成了“静态可构造性”与“程序—规范”关系的理论底座;而“状态演化、事件轨迹、确定性重放”作为系统级对象的形式化,则仍在类型论与事件溯源、动态逻辑的交叉处留有空白,KOS-TL 即针对此空白提出知识操作理论的分层形式化。

\section{纯类型系统与 $\lambda$ 立方体}
\label{sec:pts}

\subsection{纯类型系统的统一框架}

Barendregt 等人提出的纯类型系统(Pure Type Systems, PTS)\cite{Barendregt1992}将多种类型化 $\lambda$ 演算统一在一个元框架之下。一个 PTS 由三元组 $(S, A, R)$ 定义:$S$ 为 sort 的集合(如 $\{\ast, \Box\}$ 分别表示“类型”与“种类”),$A \subseteq S \times S$ 为公理集,$R \subseteq S \times S \times S$ 为规则集。\textbf{核心推理规则}形式化表述如下。

\textbf{公理}:$(s_1, s_2) \in A \Rightarrow \vdash s_1 : s_2$。\textbf{起始}:$\Gamma \vdash A : s \Rightarrow \Gamma, x:A \vdash x:A$。\textbf{弱化}:$\Gamma \vdash M:A$,$\Gamma \vdash B:s$ $\Rightarrow \Gamma, y:B \vdash M:A$。\textbf{依赖积形成}:$(s_1,s_2,s_3)\in R$,$\Gamma \vdash A:s_1$,$\Gamma, x:A \vdash B:s_2$ $\Rightarrow \Gamma \vdash (\Pi x:A.\, B) : s_3$。\textbf{抽象}:$\Gamma, x:A \vdash M:B$,$\Gamma \vdash (\Pi x:A.\, B):s$ $\Rightarrow \Gamma \vdash (\lambda x:A.\, M) : (\Pi x:A.\, B)$。\textbf{应用}:$\Gamma \vdash M:\Pi x:A.\, B$,$\Gamma \vdash N:A$ $\Rightarrow \Gamma \vdash M\,N : B[N/x]$。

$\beta$-归约定义为 $(\lambda x:A.\, M)\,N \to_\beta M[N/x]$(及在子项中的兼容闭包)。PTS 满足\textbf{主题约化}:若 $\Gamma \vdash M:A$ 且 $M \to_\beta M'$,则 $\Gamma \vdash M':A$。对适当的 $(S,A,R)$(如 $\lambda P\omega$ 对应 CoC)可证强正规化,进而可得一致性。KOS-TL 的 Core 层在概念上对应某一 PTS 实例;Kernel 与 Runtime 则在此合法性约束之上引入状态与事件,超出经典 PTS 的论域。

\subsection{$\lambda$ 立方体与依赖性的层次}

$\lambda$ 立方体\cite{Barendregt1992}在 PTS 框架下将“类型依赖于类型”“类型依赖于项”“项依赖于类型”三个维度组合,得到八种系统(如 $\lambda\to$、$\lambda 2$、$\lambda P$、$\lambda P\omega$ 等),其中 $\lambda P\omega$ 与构造演算同构。立方体揭示了依赖性的层次及与 CoC 的对应关系。

\section{依赖类型论谱系:从 CoC 到 MLTT}
\label{sec:dt}

\subsection{构造演算的核心理论架构}

Coquand 与 Huet 的构造演算(Calculus of Constructions, CoC)\cite{Coquand1988}是 impredicative 的依赖类型系统,其\textbf{核心理论架构}由语法、层与 sort、推理规则与元理论四部分构成。

\textbf{(1)语法与层。} 项 $M,N,A,B$ 由变量 $x$、sort $\mathsf{Prop}$ 与 $\mathsf{Type}$、依赖积 $\Pi x:A.\, B$、抽象 $\lambda x:A.\, M$、应用 $M\,N$ 组成。CoC 仅有两个 sort:$\mathsf{Prop}$(命题/证明的居所)与 $\mathsf{Type}$(有时记作 $\mathsf{Set}$)。\textbf{公理}唯一:$\vdash \mathsf{Prop} : \mathsf{Type}$,即“命题本身是类型”。

\textbf{(2)推理规则的形式化表述。} 除公理与起始、弱化外,依赖积的形成分三种情形(对应 PTS 规则 $(\mathsf{Prop},\mathsf{Prop},\mathsf{Prop})$、$(\mathsf{Type},\mathsf{Type},\mathsf{Type})$ 与 $(\mathsf{Type},\mathsf{Prop},\mathsf{Prop})$):
\begin{itemize}
  \item \textbf{$\Pi$-形成(Prop-Prop)}:若 $\Gamma \vdash A : \mathsf{Prop}$ 且 $\Gamma, x:A \vdash B : \mathsf{Prop}$,则 $\Gamma \vdash \Pi x:A.\, B : \mathsf{Prop}$。
  \item \textbf{$\Pi$-形成(Type-Type)}:若 $\Gamma \vdash A : \mathsf{Type}$ 且 $\Gamma, x:A \vdash B : \mathsf{Type}$,则 $\Gamma \vdash \Pi x:A.\, B : \mathsf{Type}$。
  \item \textbf{$\Pi$-形成(Type-Prop,impredicativity)}:若 $\Gamma \vdash A : \mathsf{Type}$ 且 $\Gamma, x:A \vdash B : \mathsf{Prop}$,则 $\Gamma \vdash \Pi x:A.\, B : \mathsf{Prop}$。后者使“$\forall A:\mathsf{Type}.\, P(A)$”成为合法命题,从而可编码高阶逻辑。
\end{itemize}
\textbf{抽象规则}:若 $\Gamma, x:A \vdash M : B$ 且 $\Gamma \vdash \Pi x:A.\, B : s$($s \in \{\mathsf{Prop},\mathsf{Type}\}$),则 $\Gamma \vdash \lambda x:A.\, M : \Pi x:A.\, B$。\textbf{应用规则}:若 $\Gamma \vdash M : \Pi x:A.\, B$ 且 $\Gamma \vdash N : A$,则 $\Gamma \vdash M\,N : B[N/x]$。\textbf{转换规则}:若 $\Gamma \vdash M : A$ 且 $\Gamma \vdash B : s$ 且 $A =_\beta B$,则 $\Gamma \vdash M : B$。

\textbf{(3)项与证明。} 命题 $P$ 的证明即项 $p$ 满足 $\Gamma \vdash p : P$(Curry-Howard 对应);全称命题 $\Pi x:A.\, P(x)$ 的证明即依赖函数,应用即实例化。

\textbf{(4)$\beta$-归约与主题约化。} 归约定义为 $(\lambda x:A.\, M)\,N \to_\beta M[N/x]$(及在子项中的兼容闭包)。CoC 满足\textbf{主题约化}:若 $\Gamma \vdash M : A$ 且 $M \to_\beta M'$,则 $\Gamma \vdash M' : A$。一致性($\mathsf{Prop}$ 上无非平凡闭证明)依赖强正规化与典范形式:闭证明项可归约为 $\lambda$-抽象,不产生“命题的证明”的典范矛盾。

\textbf{(5)Girard 悖论与 ECC。} 若允许公理 $\mathsf{Type}:\mathsf{Type}$,则可在系统内编码 Girard 悖论导致不一致。扩展构造演算(ECC)\cite{Luo1994}引入\textbf{层级宇宙}:$\mathsf{Type}_0 : \mathsf{Type}_1 : \mathsf{Type}_2 : \cdots$,并将 $\mathsf{Prop}$ 与各层 $\mathsf{Type}_i$ 区分;$\Pi$-形成规则在相应层级上封闭,从而在表达力与一致性之间取得平衡,成为 Coq(CIC)的前身。

CoC/ECC 的工程属性可概括为:静态、纯函数、无原生状态、无内建操作语义;不回答“系统状态是否唯一”“日志是否可重放”。KOS-TL 的 Core 层建立在依赖类型论($\Sigma/\Pi$ 与命题层)之上,而“如何改变”与“轨迹即证明”交由 Kernel 与 Runtime。

\subsection{Martin-Löf 类型论的核心架构}

Martin-Löf 直觉主义类型论(MLTT)\cite{MartinLof1984}采用 predicative 宇宙与归纳族,其\textbf{核心理论架构}由判断形式、类型构造与宇宙层次三部分构成。

\textbf{(1)判断形式。} MLTT 使用“$\Gamma \vdash A\,\mathsf{type}$”($A$ 为类型)与“$\Gamma \vdash a : A$”($a$ 为 $A$ 的项);类型相等与项相等可写为 $\Gamma \vdash A \equiv B$ 与 $\Gamma \vdash a \equiv b : A$,满足自反、对称、传递及与构造的兼容性。

\textbf{(2)依赖函数类型 $\Pi x:A.\, B$。} 形成:$\Gamma \vdash A\,\mathsf{type}$,$\Gamma, x:A \vdash B\,\mathsf{type}$ $\Rightarrow$ $\Gamma \vdash \Pi x:A.\, B\,\mathsf{type}$。引入:$\Gamma, x:A \vdash b : B$ $\Rightarrow$ $\Gamma \vdash \lambda x.\, b : \Pi x:A.\, B$。消去:$\Gamma \vdash f : \Pi x:A.\, B$,$\Gamma \vdash a : A$ $\Rightarrow$ $\Gamma \vdash f\,a : B[a/x]$。$\beta$-规则:$(\lambda x.\, b)\,a \equiv b[a/x]$。

\textbf{(3)依赖和类型 $\Sigma x:A.\, B$。} 形成:$\Gamma \vdash A\,\mathsf{type}$,$\Gamma, x:A \vdash B\,\mathsf{type}$ $\Rightarrow$ $\Gamma \vdash \Sigma x:A.\, B\,\mathsf{type}$。引入:$\Gamma \vdash a:A$,$\Gamma \vdash b:B[a/x]$ $\Rightarrow$ $\Gamma \vdash (a,b) : \Sigma x:A.\, B$。消去:$\Gamma \vdash p : \Sigma x:A.\, B$ $\Rightarrow$ $\Gamma \vdash \mathsf{fst}\,p : A$,$\Gamma \vdash \mathsf{snd}\,p : B[\mathsf{fst}\,p/x]$;$\eta$-规则 $\mathsf{fst}(a,b)\equiv a$,$\mathsf{snd}(a,b)\equiv b$。$\Sigma$ 将“数据与证明”耦合,与 KOS-TL 中“知识项与证明项原子化绑定”一致。

\textbf{(4)恒等类型 $\mathsf{Id}_A(a,b)$。} 形成:$\Gamma \vdash A\,\mathsf{type}$,$\Gamma \vdash a:A$,$\Gamma \vdash b:A$ $\Rightarrow$ $\Gamma \vdash \mathsf{Id}_A(a,b)\,\mathsf{type}$。引入:$\Gamma \vdash a:A$ $\Rightarrow$ $\Gamma \vdash \mathsf{refl}_a : \mathsf{Id}_A(a,a)$。消去(J 规则):若 $\Gamma, x:A, y:A, p:\mathsf{Id}_A(x,y) \vdash C\,\mathsf{type}$ 且 $\Gamma, x:A \vdash d : C(x,x,\mathsf{refl}_x)$,则 $\Gamma \vdash J_{A,C}(d;a,b,p) : C(a,b,p)$。恒等类型是同伦类型论中“道路”的前身。

\textbf{(5)宇宙与归纳类型。} MLTT 采用 predicative 宇宙 $\mathcal{U}_0 : \mathcal{U}_1 : \cdots$($\mathcal{U}_i : \mathcal{U}_{i+1}$);归纳类型(如 $\mathbb{N}$、列表、归纳族)通过形成/引入/消去与归纳原理定义,消去规则给出依赖消去子与归纳原理。与 CoC 相比,MLTT 更偏“构造”;不内建状态或动态执行模型。

\subsection{对知识操作的含义}

依赖类型论为“知识”的静态结构提供精确语言;KOS-TL 的 Core 层据此定义合法类型与谓词,“如何改变”与“轨迹即证明”则由 Kernel 与 Runtime 承担。

\section{逻辑框架:从 LF 到 Twelf}
\label{sec:lf}

\subsection{LF 的核心理论架构}

Edinburgh 逻辑框架(LF)\cite{Harper1993}基于 $\lambda\Pi$(依赖类型 $\lambda$ 演算),其\textbf{核心思想}是\textbf{用依赖类型编码推理规则};核心理论架构由语法层次与类型规则、判断即类型编码与 adequacy 定义三部分构成。

\textbf{(1)语法层次与类型规则。} LF 有三层:\textbf{kind}(如 $\mathsf{Type}$)、\textbf{type}($\Gamma \vdash A : \mathsf{Type}$)、\textbf{object}($\Gamma \vdash t : A$)。类型可依赖对象:形成 $\Gamma \vdash A : \mathsf{Type}$,$\Gamma, x:A \vdash B : \mathsf{Type}$ $\Rightarrow \Gamma \vdash \Pi x:A.\, B : \mathsf{Type}$;抽象 $\Gamma, x:A \vdash M : B$ $\Rightarrow \Gamma \vdash \lambda x:A.\, M : \Pi x:A.\, B$;应用 $\Gamma \vdash M : \Pi x:A.\, B$,$\Gamma \vdash N : A$ $\Rightarrow \Gamma \vdash M\,N : B[N/x]$。LF 无“类型:类型”,故仅为 $\lambda\Pi$ 片段。

\textbf{(2)判断即类型(judgements as types)。} 对象逻辑的“判断”被编码为 LF 的\textbf{类型};该判断的“证明”被编码为 LF 的\textbf{项}。例如自然演绎“$A \supset B$ 可证”编码为类型 $\mathsf{nd}\,A\,B$($\mathsf{nd} : \mathsf{tm} \to \mathsf{tm} \to \mathsf{Type}$);$\supset$-引入编码为 $\mathsf{implIntro} : \Pi A,B:\mathsf{tm}.\, (\Pi \_.\,\mathsf{nd}\,A\,B \to \mathsf{nd}\,A\,B)$;消去编码为 $\mathsf{implElim} : \Pi A,B.\, \mathsf{nd}\,A\,B \to \mathsf{nd}\,A \to \mathsf{nd}\,B$。对象逻辑的推导与 LF 的规范形式一一对应。

\textbf{(3)Adequacy 的形式化。} 编码 \emph{adequate} 指:对象逻辑的推导 $\mathcal{D}$ 推导判断 $J$ 当且仅当存在 LF 规范形式 $M$ 满足 $\vdash M : \llbracket J \rrbracket$,且该对应由 $J$ 与 $\mathcal{D}$ 的结构组合地给出。Adequacy 保证元定理可在 LF 上机械验证;Twelf\cite{Pfenning1999}基于此进行元理论证明。

\textbf{(4)与 KOS-TL 的关系。} Core 层的类型与谓词可视为“被编码的对象逻辑”;Kernel 的状态演化与轨迹是系统级操作语义的一阶对象,而非在 LF 中编码的推理规则。

\subsection{扩展与局限}

子结构 LF、模态 LF、线性 LF 等扩展用于编码资源敏感逻辑;LF 不回答“运行时状态是否唯一”“事件序列是否可确定性重放”。

\section{类型论与操作语义的统一:Harper 的 PFPL}
\label{sec:pfpl}

\subsection{PFPL 的核心框架}

Robert Harper 的《Programming Languages: Practical Foundations》(PFPL)\cite{Harper2016}将\textbf{类型系统}与\textbf{操作语义}置于同一框架;核心架构由判断式相等、结构规则与类型安全定理三部分构成。

\textbf{(1)静态与动态。} \textbf{静态语义}(typing)由形如 $\Gamma \vdash e : \tau$ 的判断给出;\textbf{动态语义}由小步关系 $e \mapsto e'$ 或大步关系 $e \Downarrow v$ 给出。\textbf{判断式相等}(judgemental equality)$\Gamma \vdash e \equiv e' : \tau$ 表示在 $\Gamma$ 下 $e$ 与 $e'$ 同类型且语义等价,满足自反、对称、传递及与类型构造的兼容(如 $\Gamma \vdash e_1 \equiv e_1' : \tau_1$,$\Gamma \vdash e_2 \equiv e_2' : \tau_2$ $\Rightarrow$ $\Gamma \vdash (e_1,e_2) \equiv (e_1',e_2') : \tau_1 \times \tau_2$)。

\textbf{(2)结构规则与类型安全。} 结构规则包括弱化、交换、收缩等(依具体系统而定)。\textbf{Progress}:若 $\vdash e : \tau$ 则 $e$ 为值或存在 $e'$ 使得 $e \mapsto e'$。\textbf{Preservation}:若 $\Gamma \vdash e : \tau$ 且 $e \mapsto e'$,则 $\Gamma \vdash e' : \tau$。二者合称类型安全,保证良型程序不会“卡住”且归约保持类型。

\textbf{(3)与 KOS-TL 的对应。} Core 层对应“静态可构造性”;Kernel 层对应“操作语义与状态变化”;“归约”在 KOS-TL 中为“事件驱动的小步状态转移”,“保持类型”对应每步经 Core 约束并产生可记录证明项。目标为知识状态演化的确定性、可回放性与可审计性。

\section{同伦类型论与等价概念}
\label{sec:hott}

\subsection{HoTT 的核心理论架构}

同伦类型论(HoTT)\cite{HoTTBook2013,Voevodsky2014}将类型论与同伦论结合;核心理论架构由恒等类型语义、道路运算与 univalence 公理三部分构成。

\textbf{(1)恒等类型解释为道路。} 类型 $A$ 解释为空间($\infty$-群胚);$\mathsf{Id}_A(a,b)$ 解释为从 $a$ 到 $b$ 的\textbf{道路空间}。项 $p : \mathsf{Id}_A(a,b)$ 即“从 $a$ 到 $b$ 的一条道路”;$\mathsf{refl}_a : \mathsf{Id}_A(a,a)$ 为恒等道路。\textbf{道路连接}:$p : \mathsf{Id}(a,b)$ 与 $q : \mathsf{Id}(b,c)$ 可连接为 $p \cdot q : \mathsf{Id}(a,c)$。\textbf{传输(transport)}:给定 $P : A \to \mathcal{U}$ 与 $p : \mathsf{Id}_A(a,b)$,有 $\mathsf{transport}^P(p) : P(a) \to P(b)$,满足 $\mathsf{transport}^P(\mathsf{refl}_a) = \mathsf{id}_{P(a)}$。

\textbf{(2)道路归纳与高阶结构。} 道路之间可有 2-道路($\mathsf{Id}_{\mathsf{Id}_A(a,b)}(p,q)$)、3-道路等,形成 $\infty$-群胚结构;类型成为“$\infty$-群胚”的语法表示。同伦层次对应类型的更高恒等。

\textbf{(3)等价与 Univalence 公理。} 类型等价 $A \simeq B$ 定义为存在 $f:A \to B$ 与 $g:B \to A$ 及道路证明 $f \circ g \sim \mathsf{id}$、$g \circ f \sim \mathsf{id}$。\textbf{Univalence}:$(A \simeq B) \simeq (A =_{\mathcal{U}} B)$,即“等价”与“在宇宙中的相等”可识别。这改变了“相等”与“等价”的处理方式,但不引入状态、效应或确定性重放。

\textbf{(4)与知识操作的关系。} HoTT 关注数学对象与等价的结构;KOS-TL 关注系统历史与操作的唯一性与可重构性;二者论域不同。

\section{带效应的类型论与 Hoare 类型论}
\label{sec:htt}

\subsection{HTT 的核心理论架构}

Hoare 类型论(HTT)\cite{Nanevski2008}将程序与规范统一在依赖类型系统中;核心架构由规范类型、分离逻辑与能力边界三部分构成。

\textbf{(1)规范类型。} 程序类型形如 $\{P\}\,x:A\,\{Q\}$(或 $\mathsf{ST}\,A\,P\,Q$),表示:执行前状态满足前置条件 $P$,执行后得到 $A$ 且后置条件 $Q$ 成立。类型规则通常为:$\Gamma \vdash e : \{P\}\,x:A\,\{Q\}$ 需在“假设当前状态满足 $P$”下证明“$e$ 终止且结果 $x:A$ 满足 $Q$”。实现上通过 \textbf{indexed monad} 或 \textbf{Dijkstra monad} 将前置/后置条件编码为类型索引。

\textbf{(2)分离逻辑与小足迹。} 在堆语义下,$P$、$Q$ 常为分离逻辑公式\cite{Reynolds2002}(含 $\ast$、$\mathrel{-\mkern-6mu\ast}$ 等);仅描述程序实际访问的堆片段(小足迹)。\textbf{帧规则}:若 $\{P\}\,e\,\{Q\}$ 且 $R$ 与 $P$ 的堆不相交,则 $\{P \ast R\}\,e\,\{Q \ast R\}$,其余堆自动保持不变量。

\textbf{(3)能力与边界。} HTT 可表达可变状态、堆操作、非终止与异常;但不以“全局确定性重放”“事件溯源”“轨迹优先”为核心。HTT 的世界观以\textbf{程序}为中心,KOS-TL 以\textbf{状态演化与事件轨迹}为中心;前者是 program verification logic,后者是 knowledge state evolution logic。KOS-TL 可借鉴 state-indexed typing,但不降格为 HTT 变体。

\section{线性逻辑与资源语义}
\label{sec:linear}

\subsection{线性逻辑的核心架构}

Girard 线性逻辑(Linear Logic)\cite{Girard1987}的核心是\textbf{资源不可复制};核心架构由联结词、sequent 规则与资源语义三部分构成。

\textbf{(1)联结词。} \textbf{线性蕴含} $A \multimap B$:消耗一个 $A$ 得到 $B$。\textbf{张量} $A \otimes B$:同时拥有 $A$ 与 $B$。\textbf{选择} $A \oplus B$、\textbf{与} $A \binampersand B$ 等。\textbf{指数} $!A$:$A$ 可重复使用(复制);$?A$ 为其对偶。在单侧 sequent 形式下,例如 $\otimes$ 右规则:$\Gamma \vdash A$,$\Delta \vdash B$ $\Rightarrow \Gamma, \Delta \vdash A \otimes B$(上下文分裂);$\multimap$ 右规则:$\Gamma, A \vdash B$ $\Rightarrow \Gamma \vdash A \multimap B$($A$ 被消耗);$!$ 规则允许将 $!A$ 弱化/收缩。

\textbf{(2)与经典/直觉逻辑的差异。} 经典逻辑中“前提可重复使用”;线性逻辑中线性假设仅能使用一次,对应“资源消耗”。Session types、Rust 借用系统受此影响。

\textbf{(3)与 KOS-TL 的关联。} Kernel 层事件从队列消费具有“一次性”直觉;物化项可从资源角度理解。KOS-TL 不直接采用线性证明论,资源主要体现在事件队列与状态 $\sigma$ 的受控演化。

\section{动态逻辑与程序验证}
\label{sec:dyn}

\subsection{动态逻辑的核心架构}

Harel 等人的动态逻辑(Dynamic Logic)\cite{Harel2000}将程序与逻辑结合;核心架构由模态公式、公理与 Kripke 语义三部分构成。

\textbf{(1)模态公式与程序构造。} $[\alpha]\phi$:执行程序 $\alpha$ 后(若终止)$\phi$ 成立。$\langle\alpha\rangle\phi$:存在一次执行 $\alpha$ 终止且 $\phi$ 成立。程序 $\alpha$ 为一阶对象,可由赋值 $x:=t$、顺序 $\alpha;\beta$、选择 $\alpha \cup \beta$、循环 $\alpha^*$ 等组合。满足 $[\alpha \cup \beta]\phi \leftrightarrow [\alpha]\phi \wedge [\beta]\phi$,$[\alpha;\beta]\phi \leftrightarrow [\alpha][\beta]\phi$,以及归纳公理 $[\alpha^*]\phi \leftrightarrow \phi \wedge [\alpha][\alpha^*]\phi$ 等。

\textbf{(2)与 Hoare 逻辑的关系。} 部分正确性 $\{P\}\alpha\{Q\}$ 可表为 $P \to [\alpha]Q$;完全正确性涉及终止(如 $\langle\alpha\rangle\top$)。动态逻辑统一了多种程序验证方法;语义为 Kripke 结构,状态与程序解释为转移关系。

\textbf{(3)与 KOS-TL 的类比。} “$\sigma \xrightarrow{\langle e,p \rangle} \sigma'$”可与“执行某程序后状态满足某性质”类比;但 KOS-TL 强调轨迹的确定性重放、日志不可篡改与证明绑定,这些在经典动态逻辑中并非核心。

\section{事件溯源与确定性重放}
\label{sec:es}

\textbf{事件溯源}(Event Sourcing)\cite{Fowler2005}将状态变化记录为不可变事件序列,以追加式日志为唯一真实来源;当前状态由从初始状态按序重放全部事件得到。\textbf{确定性重放}要求:相同事件序列在相同初始状态下产生相同结果,从而保证可重复性与审计可信度。多数工程实现不将“事件类型”与“合法性证明”作为类型论对象。KOS-TL 将事件溯源与类型论结合:事件对 $\langle e, p \rangle$ 经 Core 层约束,轨迹既是状态演化序列也是可追溯的证明结构,即“Event Sourcing + Type Enforcement + Kernel Determinism”。

\section{知识表示与描述逻辑}
\label{sec:kr}

\textbf{描述逻辑}(DL)\cite{Baader2003}以概念、角色、TBox/ABox 为基础,通过受限量词在表达力与可满足性可判定性之间取得平衡;OWL 建于此上。\textbf{情境演算}与\textbf{事件演算}公理化动作与状态变化(fluents、后继状态公理等),但与类型论和证明项无直接对应。\textbf{Datalog} 以规则与不动点进行推理,无“类型—证明”的 Curry-Howard 对应。它们与类型论的区别在于:类型论强调构造性证明与 Curry-Howard 对应,DL 侧重可满足性与推理算法。KOS-TL 的 Core 层采用类型论以将“合法性”与“证明项”紧密绑定,再通过 Kernel 与 Runtime 实现带证明的操作演化。

\section{体系对比矩阵与结构差异}
\label{sec:matrix}

表\ref{tab:type-theory-comparison} 从“依赖类型”“状态内建”“操作语义”“资源意识”“可重放/轨迹”等维度对前述体系做简要对比。主流类型论(CoC、MLTT、LF、HoTT)在状态与操作语义上多为“否”;HTT 与线性逻辑为“部分”;动态逻辑与事件溯源在状态与执行上较强。KOS-TL 在依赖类型(Core)基础上内建状态 $\sigma$、事件队列 $P$、小步操作语义与确定性重放。

\begin{table}[htbp]
\centering
\caption{类型论及相关体系对比(简要)}
\label{tab:type-theory-comparison}
\small
\begin{tabular}{lccccc}
\toprule
\textbf{体系} & \textbf{依赖类型} & \textbf{状态内建} & \textbf{操作语义} & \textbf{资源意识} & \textbf{可重放/轨迹} \\
\midrule
CoC / ECC / MLTT & $\checkmark$ & $\times$ & $\times$ & $\times$ & $\times$ \\
LF / Twelf & $\checkmark$ & $\times$ & $\times$ & $\times$ & $\times$ \\
HoTT & $\checkmark$ & $\times$ & $\times$ & $\times$ & $\times$ \\
DTT + effects / HTT & $\checkmark$ & 部分 & 部分 & 部分 & $\times$ \\
线性逻辑 & 本体不同 & 部分 & $\times$ & $\checkmark$ & $\times$ \\
动态逻辑 & $\times$ & $\checkmark$ & $\checkmark$ & $\times$ & $\times$ \\
事件溯源(工程) & $\times$ & $\checkmark$ & $\checkmark$ & $\times$ & $\checkmark$ \\
\textbf{KOS-TL} & $\checkmark$(Core) & $\checkmark$(Kernel $\sigma$) & $\checkmark$(小步) & 部分 & $\checkmark$(轨迹即证明) \\
\bottomrule
\end{tabular}
\end{table}

现有主流类型论共同刻画“静态可构造性”;它们不直接回答全局状态是否唯一、日志是否可重放、知识是否可回溯。KOS-TL 将“知识的类型化表示”“事件驱动的小步演化”与“轨迹即证明的可追溯性”统一在一个分层形式化系统中,其本质是\textbf{知识操作理论}的创新。

\section{KOS-TL 的定位:知识操作理论的创新}
\label{sec:kos-position}

KOS-TL 不属于 CoC/MLTT 的纯逻辑分支、不属于 HoTT 的等价与同伦分支、不属于 LF 的元逻辑编码分支。它在概念上最接近“Dynamic Logic + Event Sourcing + Type Enforcement + Kernel Determinism”的交叉区域。Hoare 类型论回答“程序执行是否满足规范”;KOS-TL 回答“系统历史是否唯一且可重构”。若将 KOS-TL 设计成 $\Gamma \vdash e : \mathsf{ST}\,A\,(\mathsf{requires}\,P)\,(\mathsf{ensures}\,Q)$ 的形式,则会落入 HTT 的范畴;KOS-TL 选择将轨迹与状态转移作为一等对象,在 Kernel 层实现“每步可验证、整体可回放”。理论升级时可借鉴 PFPL 的框架、线性逻辑的资源观与动态逻辑的模态结构;实现上可将 Core 层委托给证明助手或自建类型检查器,Kernel 与 Runtime 为独立的状态机与事件处理层。

\section{小结}

本章对与 KOS-TL 相关的类型论、逻辑框架、程序验证与知识表示做了综述,并\textbf{展开了各理论的核心架构}(内容均直接置于本书正文,无单独 \texttt{.tex} 文件):PTS 的 $(S,A,R)$ 与公理/起始/弱化/依赖积/抽象/应用及主题约化;CoC 的语法与层、$\Pi$-形成三种情形、抽象/应用/转换规则、$\beta$-归约与主题约化、Girard 悖论与 ECC 宇宙;MLTT 的判断形式、$\Pi/\Sigma/\mathsf{Id}$ 的形成/引入/消去与 J 规则、宇宙与归纳类型;LF 的语法层次与类型规则、判断即类型编码、adequacy 的形式化;PFPL 的静态/动态与判断式相等、结构规则与 progress-preservation;HoTT 的恒等类型语义、道路连接与 transport、univalence;HTT 的规范类型与分离逻辑帧规则;线性逻辑的联结词与 sequent 规则;动态逻辑的模态公式与公理。在此基础上明确了 KOS-TL 的定位:以类型论为静态合法性底座、以事件驱动的小步语义为动态论域、以轨迹即证明为可追溯性保证,与程序验证逻辑和纯数学基础的既有谱系形成互补。后续章节将展开 KOS-TL 的分层形式化定义与各层技术细节。

\chapter{KOS-TL 的分层形式化定义}


\section{L0: Core(核心层)—— 静态逻辑论域}

为了将逻辑严谨性与操作表达能力结合,知行逻辑(KOS-TL)被定义为一个\emph{分层形式化系统}。分别包括三层形式化定义:\emph{Core (核心)}、\emph{Kernel (内核)}与\emph{Runtime(运行时)},每一层在逻辑角色与语义承诺上均有所区分(如表\ref{tab:kos-layers} 所示)。

\begin{table}[htbp]
\centering
\caption{KOS-TL 层次结构概述}
\label{tab:kos-layers}
\begin{tabular}{llp{4cm}p{3cm}l}
\toprule
\textbf{层次} & \textbf{形式化名称} & \textbf{核心职责} & \textbf{逻辑对象} & \textbf{判定性} \\
\midrule
L0  Core & 静态真理层 (Logic) & 定义“什么是合法的”。基于依存类型论(ITT)确立类型构造与约束。 & 类型 $T$、证明项 $p$、命题 $P$ & 强可判定 \\
L1  Kernel & 动态转换层 (Dynamics) & 定义“如何改变”。引入小步操作语义,处理事件对状态的转换。 & 状态 $\sigma$、事件 $e$、转移 $\to$ & 局部可判定 \\
L2  Runtime & 环境演化层 (System) & 定义“如何运行”。处理外部 I/O、时间线挂载、非确定性输入。 & 队列 $Q$、外部源 $\mathcal{E}_{\mathrm{env}}$ & 半判定 \\
\bottomrule
\end{tabular}
\end{table}

知行逻辑的分层原则如下:
\begin{itemize}
    \item 逻辑合法性仅由核心层决定;
    \item 操作正确性由内核层强制;
    \item 系统演化在运行时层实现。
\end{itemize}

这种分层设计保证即便在开放、非确定性的系统演化中,核心逻辑的正确性仍被保持。

\section{L0: Core(核心层)—— 静态逻辑论域}

从知识操作系统的全局视野审视,KOS-TL 的核心层超越了传统意义上的数据库模式(Schema)范畴。它作为整个系统的逻辑宪法,承担着将模糊的现实世界业务规则精准转化为严密数学约束的核心使命。其设计目标在于:在处理大规模动态数据流的同时,通过元物理规范确保每一比特的操作均具备逻辑完备性与可溯源性。

\subsection{核心层需求分析与逻辑构造}

在构造性表达需求方面,传统系统往往仅侧重于数据的静态存储,而在高安全性与高可信场景下,系统必须过渡到对“知识”的存储。基于直觉主义类型论(ITT),每一条知识项被定义为一个依存对 $\Sigma(d:D).P(d)$,其中数据 $d$ 与证明其满足业务本体的凭证 $P(d)$ 强耦合。这种设计从源头上消除了“无根数据”,确保所有进入内核的知识均经过构造性验证。

物理与逻辑的深度对齐是另一项关键需求。通过依存类型(Dependent Types)的表达力,系统能够将工业或金融领域的物理定律与合规性约束内生化,而非依赖外部的临时逻辑判断。这种“非法状态不可表示”(Make Illegal States Unrepresentable)的范式,确保任何违反公理的操作尝试在类型检查阶段即告失败,从而维持系统的稳态。

此外,为应对高频合规要求,系统引入了计算反射性(Computational Reflexivity)。通过要求核心层描述自身的规约规则,系统在执行每一小步逻辑演化($\beta, \iota, \delta$ 规约)时均能自动合成等价性证明 $\text{Id}(t, t')$。这种全路径自动化审计将传统的“事后追查”提升为“运行时的实时形式化验证”。最后,通过定义精化算子($\textsf{elab}$)模板,核心层为运行时(Runtime)信号提供了语义提升的基准,建立了从物理比特(Bits)到逻辑真理(Truth)的唯一映射路径。

\subsection{总体架构描述}

核心层被架构为一个基于依存类型论的强校验微内核,主要由以下三个功能模块驱动:其一,类型构造器与本体管理器作为系统的“立法者”,负责将业务领域本体转化为类型系统中的目(Sorts)与依存类型结构,明确合法对象的边界。其二,规约引擎作为系统的“推理机”,负责处理逻辑项的细粒度演化,通过执行函数应用与结构拆解计算知识演化后的逻辑稳态。其三,类型检查器作为系统的“守门人”,执行双向类型检查(Bidirectional Type Checking),确保所有进入内核层(Kernel Layer)的操作均满足预先验证的正确性(Correct-by-construction)。

\subsection{关键设计决策}

核心层的技术选型与架构决策反映了对逻辑严密性与工程可行性的权衡。

\begin{enumerate}[label=(\arabic*)]

    \item 依存类型论(MLTT)对一阶逻辑(FOL)的替代

    传统知识库基于 FOL 或描述逻辑(DL),导致逻辑断言与具体数据脱节。核心层选择依存类型论,利用 $\Sigma$ 类型实现数据与约束的原子化封装。这一决策解决了知识操作中“证据缺失”的顽疾,从架构级别强制执行物理约束。

    \item 计算反射性的引入与内生审计

    针对传统审计滞后且易被篡改的风险,核心层将规约规则建模为逻辑处理对象。每当状态变更,系统自动合成同一性证明(Identity Proof)。这种设计使审计行为转化为自动化的类型检查过程,只要证明链完整,系统行为即具绝对合规性。

    \item 双轴世界体系与命题收缩(Prop-Shrinking)

    为解决形式化证明带来的计算开销问题,核心层设计了多级世界($\mathcal{U}_i$)用于复杂建模,并引入专门的证明空间(Prop)。基于证明无关性(Proof Irrelevance)原则,系统在完成严苛校验后通过类型擦除(Erasure)技术收缩证明细节。这一决策实现了逻辑深度与工程效能的平衡,支持大规模实时知识流的高效处理。

    \item 从指称语义向操作语义的范式转移

    传统逻辑侧重静态真值,而 KOS-TL 核心层通过定义逻辑项的小步规约(Small-step reduction)规则,将知识操作定义为规约过程。这一决策确保了从核心层到内核层的转换是无损且确定的,逻辑推演步与物理计算步实现了高度同构。
\end{enumerate}

核心层的决策与传统的知识库架构的对比如表\ref{tab:KosCorecomparison}所示。

\begin{table}[h]
\caption{传统架构与KOS-TL Core 层决策比较}
\label{tab:KosCorecomparison}
\centering
\begin{tabular}{|l|l|l|}
\hline
\textbf{设计维度} & \textbf{传统架构决策} & \textbf{KOS-TL Core 层决策} \\
\hline
知识载体 & 数据库记录 + 外部校验 & 代码带有依存证明的逻辑项 (Terms) \\
\hline
约束触发 & 运行时拦截 / 业务代码 if-else & 类型检查 (Type Checking) \\
\hline
演化动力 & 数据库事务 (Transactions) & 逻辑规约 (Reduction) \\
\hline
信任根源 & 系统管理员权限 / 日志记录 & 不可篡改的数学证明链 \\
\hline
\end{tabular}
\end{table}

\section{L1: Kernel(内核层)—— 操作语义论域}

在 KOS-TL 系统的分层架构中,内核层承担着将核心层定义的静态真理转化为动态演化动力的中枢职能。若将核心层比作“宪法”,则内核层即为“行政中枢”与“动力引擎”。其核心目标在于通过形式化规约机制,确保知识系统在处理高频业务流与时间流时,维持逻辑上的确定性与演化上的连续性。

\subsection{内核层需求建模:动态演化与状态确定性}

针对复杂知识操作系统,内核层的设计首要解决知识状态的“坍缩”与“生成”问题。

首先是因果追溯需求,系统要求每一项知识状态的变更必须具备显式的因果关联,即所有变更必须溯源至特定的精化事件(Event)。其次是状态一致性需求,尤其在分布式或并发环境下,必须保证状态迁移的原子性,从逻辑层面根除冲突状态的产生。最后是实时性与进度保证(Progress Guarantee),系统必须确保在接收到合法输入后,能够逻辑确定地演化至下一个稳步,避免因非确定性导致的未定义行为或逻辑死循环。

\subsection{设计方法论:小步操作语义与状态机模型}

内核层拒绝采用传统的黑盒式批处理,而是建立在形式化小步操作语义(Small-step Operational Semantics)的基础之上。

其方法论核心在于状态三元组模型。内核将系统抽象为 $\sigma = \langle \mathcal{K}, \mathcal{TS}, \mathcal{P} \rangle$,其中 $\mathcal{K}$ 代表当前经校验的知识真理集,$\mathcal{TS}$ 为逻辑时钟与物理锚点的耦合,$\mathcal{P}$ 则为已精化但尚未执行的事件队列。内核层本质上是以“事件”为核心,基于事件的本体论地位\cite{Davidson1967},事件被定义为状态转移算子(Events as Transitions):每一类事件均对应一个显式的转换算子,该算子通过同步调用核心层的判定结果,驱动旧状态 $\sigma$ 向新状态 $\sigma'$ 的单调演化。

\subsection{总体架构描述}

作为介于逻辑基座(Core)与物理环境(Runtime)之间的“逻辑路由器”,内核层由三个核心模块协同工作:

\begin{itemize}
\item 事件队列管理器(Event Queue Manager):负责接收来自运行层精化后的事件包 $\langle e, p \rangle$,并执行严格的定序(Sequencing)与依赖冲突检测。

\item 演化调度器(Evolution Scheduler):驱动系统的核心演化循环。该模块采用“Peek-Verify-Reduce-Confirm”流程,即在消费事件前调用核心层验证前置条件,执行规约操作,并校验后置证明。其操作对象始终为逻辑项(Terms)而非底层物理比特。

\item 状态镜像库(State Mirror):维护逻辑上的最新“真值视图”,为运行层的状态查询及核心层的语境验证提供一致性上下文。
\end{itemize}

\subsection{关键设计决策}

\begin{enumerate}[label=(\arabic*)]

    \item 强顺序提交与异步精化的解耦

    针对物理信号高频产生与逻辑校验(深层证明)耗时之间的矛盾,内核层采取“异步精化、顺序提交”的决策。运行层可并行执行信号精化,但内核层坚持顺序提交(Sequential Commit)。这一决策确保了因果链条的唯一性,使知识演化轨迹呈现为确定性的线性路径,规避了复杂的逻辑分支回溯风险。

    \item 闭环演化:前置验证与后置证据合成

    为确保操作后的系统状态持续符合本体定义,内核层确立了“前置验证、后置合成”的闭环机制。在执行操作前,内核强制验证核心层提供的 $Pre(e)$;操作完成后,自动触发核心层合成 $Post(e)$ 的新证明。该决策保证了系统始终在“已证明为真的状态”之间迁移,消除了逻辑真空期。

    \item 逻辑时间与物理时间的松耦合机制

    针对分布式网络中物理时间先后与因果逻辑不一致的痛点,内核层维护独立的逻辑序。物理时间戳仅作为证据锚点,系统仅在物化阶段(Runtime)进行同步。此决策利用逻辑证明中的时间证据进行重排序,从根源上解决了“幻觉时序”问题,保障了\textbf{因果序(Causality Order)}的绝对保真。

    \item 确定性规约(Deterministic Reduction)

    内核层严禁非确定性(Non-deterministic)选择。若某一事件在规约过程中指向多个可能的演化分支,该事件将在核心层被判定为类型错误。这一决策极大增强了系统的可预测性:在相同的初始状态与事件序列下,内核层将产生数学上唯一的、可复现的知识视图。

\end{enumerate}

\section{L2: Runtime(运行时层)—— 系统演化论域}

在 KOS-TL(知行逻辑)的体系结构中,运行层被定义为逻辑世界与非确定性物理世界之间的关键“锚点”。若将核心层视为系统的“宪法”,内核层视为“行政中枢”,则运行层即为系统的“感官、肢体与物理载体”。其核心使命在于解决形式逻辑在复杂工程环境落地过程中的“最后一公里”问题,实现逻辑语义与底层物理信号的无缝衔接。

\subsection{需求建模:物理保真与环境精化}

运行层的设计旨在应对物理世界与逻辑抽象之间的根本张力。

首先是信号提升需求,物理世界产生的原始位流(Raw Bits)缺乏内在语义,运行层需将其“提升”为具备逻辑内涵的事件对象。其次是资源映射需求,逻辑层输出的抽象状态更新最终必须精准落实于磁盘比特、内存条目或硬件控制器的电平状态。最后是实时性与并发需求,在处理高频传感器信号时,系统必须在确保低延迟摄入的同时,不破坏内核层维持的因果顺序一致性。

\subsection{设计方法论:双向映射中的精化与物化}

运行层的方法论核心在于建立逻辑语义与物理资源之间的双向精化关系。

\begin{enumerate}[label=(\arabic*)]
    \item 精化(Refinement/Elaboration)方法论

    系统引入 $\textsf{elab}$ 算子实现信号的提升。与传统的语法解析(Parsing)不同,$\textsf{elab}$ 是一个证明构造过程:它对照核心层的本体模板,为原始信号 $s$ 寻找逻辑证据 $p$,从而将其转换为符合内核接口的带证明事件 $\langle e, p \rangle$。这一过程确保了进入系统的所有输入均具备形式化的合法性依据。

    \item 物化(Materialization)方法论

    通过 $\mathcal{M}$ 算子,运行层负责将内核层中抽象的知识状态 $\mathcal{K}$ “降解”为具体的物理形态。例如,将一个逻辑上的“转账成功”断言物化为数据库中的 ACID 事务或区块链上的交易条目。物化机制确保了逻辑结论在物理世界中的保真执行。
\end{enumerate}

\subsection{总体架构描述}

作为负责“跨境交互”的枢纽,运行层由以下关键组件构成:

\begin{itemize}
\item 精化引擎 (Elaborator):连接物理 I/O 设备,负责监听外部中断及传感器数据流,并执行双向精化,将非确定性信号锚定为确定的逻辑事件。

\item 物理存储管理器 (Physical Storage Manager):屏蔽底层介质差异,管理数据库及内存映射,确保物化操作的原子性与持久性。

\item 调度中继 (Scheduler Relay):作为内核顺序演变的缓冲器,负责多线程信号的并发摄入与保序入队。
\end{itemize}

\subsection{关键设计决策}

\begin{enumerate}[label=(\arabic*)]
    \item 基于“精化”的输入过滤决策

    针对传统系统直接读取变量易受环境干扰(脏数据)的问题,运行层规定所有输入必须通过 $\textsf{elab}$ 算子。若信号无法在核心层构造出合法证明 $p$,则判定为无效输入并立即丢弃。该决策建立了逻辑防火墙,保证了内核层不被非预期信号污染。

    \item 原子化提交栅栏 (Atomic Commit Fence)

    为解决“知行不一”(即逻辑更新成功但物理写入失败)的问题,运行层引入了类似两阶段提交的栅栏机制。只有当物理层返回写入确认(ACK)后,逻辑时钟 $\mathcal{TS}$ 才正式向前推进。这一决策实现了物理存储与逻辑真理的精确同步。

    \item 资源抽象与多后端插拔支持

    为适应从嵌入式设备到云端集群的差异化环境,运行层将物化算子 $\mathcal{M}$ 设计为可插拔的后端。该决策实现了逻辑便携性:核心层与内核层的逻辑公理保持不变,通过更换运行层后端即可实现跨平台的语义一致性运行。

    \item 确定性轨迹重放与灾后自愈

    基于物理环境的不可靠性,运行层完整记录所有精化事件的原始轨迹。利用内核层确定性规约的特性,系统在故障后可通过重放(Replay)事件流,在逻辑上重新推演并修复物理配置。这一决策为系统提供了极强的逻辑健壮性与自愈能力。
\end{enumerate}

整个KOS-TL系统的层次结构及稳定性保障如表\ref{tab:KOSHiera}所示。

\begin{table}[htbp]
\centering
\caption{系统层次结构与稳定性保证}
\label{tab:KOSHiera}
\begin{tabular}{llll}  % l: 左对齐;p: 固定宽度段落
\toprule  % 上粗线(需booktabs包)
层次 & 核心关注点 & 核心算子 & 稳定性保证 \\
\midrule  % 中线
L0: Core & 知识的静态结构 & $\Pi, \Sigma, \textsf{Prop}$ & 类型检查 (Type Checking) \\
L1: Kernel & 基于事件的状态转移 & $\textsf{STEP}, \textsf{Ev}$ & 逻辑证明 (Proof Verification) \\
L2: Runtime & 现实与逻辑的映射 & $\textsf{elaborate}, \mathcal{M} $ & 事务一致性 (Transactional Consistency) \\
\bottomrule  % 下粗线
\end{tabular}
\end{table}

\chapter{KOS-TL 核心层 (Core Layer):静态逻辑基础}

核心层(Core Layer)是 KOS-TL 系统的“形式宪法”,基于直觉依赖类型论(Intuitive Dependent Type Theory, IDTT)。其核心任务是定义知识的静态结构、逻辑约束及其合法性证明,为上层执行提供不可篡改的逻辑根基。它不随时间改变,只负责定义什么是“合法的构造”。

\section{语法 (Syntax)}


\subsection{1. 论域($\mathcal{D}_{Core}$)}

核心层的论域($\mathcal{D}_{Core}$)世界由双轴构成,一个是数据轴,另一个是逻辑轴。

\begin{itemize}
    \item \textbf{双轴世界 (Universes)}:
    \begin{itemize}
        \item \textbf{计算轴 ($\mathcal{U}_i$)}:遵循谓述性。$\mathcal{U}_0$ 包含基础目(Sorts),$\mathcal{U}_{i+1}$ 包含 $\mathcal{U}_i$ 作为元素。用于建模具有物理效应的数据。
        \item \textbf{逻辑轴 ($\textsf{Type}_i$)}:遵循谓述性,但其底层 $\textsf{Prop} : \textsf{Type}_1$ 具有非谓述性。$\textsf{Type}_i$ 用于建模逻辑谓词空间及元逻辑规则。
        \item \textbf{层级关系}:
        \begin{itemize}
            \item $\textsf{Prop} : \textsf{Type}_1, \quad \textsf{Type}_i : \textsf{Type}_{i+1}$
            \item $\mathcal{U}_i : \mathcal{U}_{i+1}, \quad \mathcal{U}_i : \textsf{Type}_{i+1}$ (计算宇宙可作为逻辑讨论的对象)
            \item $\textsf{Prop} \hookrightarrow \mathcal{U}_1$ (命题可以嵌入数据轴)
        \end{itemize}
    \end{itemize}
    \item \textbf{基础目 (Sorts)}:$\textsf{Val}$(原子值)、$\textsf{Time}$(时间点标量)、$\textsf{ID}$(唯一标识符)。
    \item \textbf{知识对象 (Objects)}:所有的依存记录类型(Dependent Record Types)的实例。
\end{itemize}

\subsection{2. 语法}

\begin{itemize}

\item \textbf{类型构造 (Types)}:

$$\begin{aligned}
A, B ::= & \ \textsf{Prop} \mid \textsf{Type}_i \mid \mathcal{U}_i & (\text{Universes}) \\
& \mid \textsf{Val} \mid \textsf{Time} \mid \textsf{ID} & (\text{Base Sorts}) \\
& \mid \Pi(x:A).B \mid \Sigma(x:A).B \mid A + B \mid \textsf{Id}_A(a, b) & (\text{Constructors})
\end{aligned}$$

\item \textbf{项构造 (Terms)}:

$$
\begin{aligned}
t, u ::= & \ x \mid \lambda x.t \mid t\,u & (\Pi\text{ Intro/Elim}) \\
& \mid \langle t, u \rangle \mid \textsf{split}(t, x.y.u) & (\Sigma\text{ Intro/Elim}) \\
& \mid \textsf{inl}(t) \mid \textsf{inr}(t) \mid \textsf{case}(t, x.u, y.v) & (+\text{ Intro/Elim}) \\
& \mid \textsf{refl} & (\text{Id Intro})
\end{aligned}
$$

\item \textbf{判断式 (Judgments)}:
\begin{itemize}
    \item $\Gamma \vdash A : \mathcal{S}$ :表示 $A$ 是一个合法类型,其中 $\mathcal{S} \in \{ \textsf{Type}_i, \mathcal{U}_i \}$。
    \item $\Gamma \vdash t : A$ :表示 $t$ 是类型 $A$ 的合法实例(对于数据轴)或合法证明(对于逻辑轴)。
    \item $\Gamma \vdash A \equiv B$ 及 $\Gamma \vdash t \equiv u$ :表示类型或项在计算上等价(转换规则)。
\end{itemize}
\end{itemize}

为了保持核心层的简洁性,$\textsf{Prop}$ 中的逻辑运算通过 $\mathcal{U}$ 中的类型构造算子实现。具体的语义映射关系如表\ref{tab:correspondenceofpro2type}所示。

注:关于 $\textsf{Prop}$ 的特殊性。KOS-TL Core 遵循 \textbf{证言忽略 (Proof Irrelevance)} 原则:对于任何 $P : \textsf{Prop}$,若 $p, q : P$,则在语义层面上 $\llbracket p \rrbracket = \llbracket q \rrbracket$。这意味着在制造业看板中,我们只关心“安全属性是否被证明”,而不关心证明的具体推导路径。这确保了逻辑层不占用额外的运行内存。

\begin{table}[h]
\centering
\caption{逻辑命题与类型的同构}
\label{tab:correspondenceofpro2type}
\begin{tabular}{|l|l|l|}
\hline
\textbf{逻辑命题 ($\textsf{Prop}$)} & \textbf{类型构造 ($\mathcal{T}$)} & \textbf{项构造 (Terms)} \\
\hline
全称量词 $\forall x:A.\, P(x)$ & 依存乘积 $\Pi(x:A).P$ & $\lambda x.\, p$  \\
\hline
存在量词 $\exists x:A.\, P(x)$ & 依存求和 $\Sigma(x:A).P$ & $\langle a, p \rangle$ \\
\hline
逻辑蕴含 $P \to Q$ & 函数空间 $P \to Q$ & $\lambda p.\, q$ \\
\hline
逻辑合取 $P \wedge Q$ & 积类型 $P \times Q$ & $\langle p, q \rangle$ \\
\hline
逻辑析取 $P \vee Q$ & 和类型 $P + Q$ & $\textsf{inl}(p) / \textsf{inr}(q)$ \\
\hline
\end{tabular}
\end{table}

根据项与类型的构造规则,KOS-TL 核心层的类型集合 $\mathcal{T}$ 由以下规则归纳定义。

\begin{enumerate}[label=(\arabic*)]

    \item 基础规则 (Base Rules)

    $$\begin{aligned}
        & \textsf{Prop} : \textsf{Type}_1 \quad (\text{逻辑轴起点}) \\
        & \textsf{Type}_i : \textsf{Type}_{i+1} \quad (\text{逻辑宇宙累积}) \\
        & \mathcal{U}_i : \mathcal{U}_{i+1} \quad (\text{数据宇宙累积}) \\
        & \textsf{Prop} \hookrightarrow \mathcal{U}_1 \quad (\text{提升规则:命题可以作为数据处理,这是一个隐式强制类型转换(Coercion)})
    \end{aligned}$$

    $\textsf{Prop} \hookrightarrow \mathcal{U}_1$ 是一种单向嵌入,允许我们将证明作为对象嵌入到数据记录(如 $\Sigma$ 类型)中,但 $\mathcal{U}_i$ 中的普通数据不能直接作为命题进行逻辑推导。

    原子类型: $\textsf{Val} \in \mathcal{T}, \textsf{Time} \in \mathcal{T}, \textsf{ID} \in \mathcal{T}$。

    \item 依存乘积构造 (Pi-types / Dependent Products)

    \begin{itemize}
        \item \textbf{逻辑/计算混合规则}:

        $$\frac{\Gamma \vdash A : \textsf{Type}_i/\mathcal{U}_i \quad \Gamma, x:A \vdash B : \textsf{Prop}}{\Gamma \vdash \Pi(x:A).B : \textsf{Prop}} (\text{非谓述性})$$

        \item \textbf{纯 Universe 规则}:

        $$\frac{\Gamma \vdash A : \textsf{Type}_i \quad \Gamma, x:A \vdash B : \textsf{Type}_j}{\Gamma \vdash \Pi(x:A).B : \textsf{Type}_{\max(i, j)}} (\text{谓述性})$$
    \end{itemize}
    $\textsf{Prop}$ 具有一种特殊的性质,称为非谓述性 (Impredicativity)。无论 $A$ 的层级多高,只要 $B$ 属于 $\textsf{Prop}$,那么 $\Pi(x:A).B$ 通常仍然属于 $\textsf{Prop}$。意义:这允许我们对“无限的对象”进行逻辑判定。例如,“对于所有 $\mathcal{U}_1$ 中的类型,它们都满足安全属性 $P$”,这个判定本身依然只是一个简单的 $\textsf{Prop}$ (真或假),而不会爆炸成一个超级复杂的类型。

    \item 依存求和构造 (Sigma-types / Dependent Sums)

    为了防止逻辑悖论(类似 Girard's Paradox),$\Sigma$ 类型在 KOS-TL 中必须是谓述性的。若 $A$ 是一个类型,且在变量 $x:A$ 的假设下 $B$ 是一个类型,则:

    $$\frac{\Gamma \vdash A : \textsf{Type}_i \quad \Gamma, x:A \vdash B : \textsf{Type}_j}{\Gamma \vdash \Sigma(x:A).B : \textsf{Type}_{\max(i, j)}}$$

    注:若 $A, B \in \mathcal{U}$,则结果在 $\mathcal{U}$ 中;若涉及证明提取,其最高层级受逻辑轴 Universe 约束。

    依存求和构造建模知识对象。$\Sigma$ 类型是 KOS-TL 的核心,它强制要求数据 $x$ 必须关联一个证明项 $p:B(x)$。

    \item 和类型构造 (Sum Types / Disjoint Union)

    若 $A$ 和 $B$ 分别是合法类型,则它们的析取和(不交并)也是一个类型:

$$\frac{\Gamma \vdash A : \mathcal{U} \quad \Gamma \vdash B : \mathcal{U}}{\Gamma \vdash A + B : \mathcal{U}}$$

  和类型建模逻辑中的“析取($\vee$)”关系。在制造业场景中,它用于建模“互斥状态”或“备选路径”。例如,一个任务的状态要么是 $\textsf{Success}$,要么是 $\textsf{Failure}$。

    \item 等价类型构造 (Identity Types)

    若 $A$ 是一个类型,且 $u, v$ 是类型 $A$ 的两个项,则:

    $$\frac{\Gamma \vdash A : \mathcal{U} \quad \Gamma \vdash u:A \quad \Gamma \vdash v:A}{\Gamma \vdash \textsf{Id}_A(u, v) : \textsf{Prop}}$$

    等价类型构造建模知识的等价性,它执行“因果追溯”和“状态回滚”时判断两个事实是否一致的逻辑基础。
\end{enumerate}


在 KOS-TL Core 中,$\textsf{Prop}$ 是一个特殊的论域,专门用于处理逻辑断言。与普通的 $\mathcal{U}_i$ 不同,它在 $\Pi$ 构造下表现出非谓述性。

\begin{Definition}{非谓述 $\Pi$ 构造规则}

对于任意层级的类型 $A : \mathcal{U}_i$,若在假设 $x:A$ 下 $B$ 是一个命题,则其全称量词(或函数空间)依然映射回最小的命题世界:

$$\frac{\Gamma \vdash A : \mathcal{U}_i \quad \Gamma, x:A \vdash B : \textsf{Prop}}{\Gamma \vdash \Pi(x:A).B : \textsf{Prop}}$$

\textbf{逻辑闭环要点}:这意味着命题的复杂度不随其量词覆盖范围的扩大而提升等级。这一性质允许我们在不触发 Universe 爆炸的前提下,对全量数据(即使是 $\mathcal{U}_k$ 层级的对象)进行一致性断言。
\end{Definition}


\begin{Definition}{ Universe 提升与包含规则 }

为了支撑双轴语义的闭环,系统引入以下隐式转换:
\begin{itemize}
    \item \textbf{计算到逻辑的观测}:

    $$\frac{\Gamma \vdash A : \mathcal{U}_i}{\Gamma \vdash A : \textsf{Type}_{i+1}}$$

    这意味着任何计算类型都可以被当作逻辑命题的讨论对象(例如在 $\textsf{Type}$ 层级讨论 SensorData 的代数性质)。

    \item \textbf{命题的计算嵌入}:

    $$\frac{\Gamma \vdash P : \textsf{Prop}}{\Gamma \vdash P : \mathcal{U}_1}$$

    这允许逻辑证明项 $p:P$ 被包装进 $\Sigma$ 记录中,作为实时计算系统的输入(即“带有证明的数据包”)。
\end{itemize}

\end{Definition}

\subsection{3. 判定规则 (Judgmental Rules)}

为了确保上述构造在逻辑上是良构的,KOS-TL Core 遵循以下推导规则。

\begin{enumerate}[label=(\arabic*)]

    \item 依存乘积 ($\Pi$-Types)
    \begin{itemize}
        \item 引入规则 (Introduction Rules)

        对于 $\Pi$类型,其构造(引入)规则为:

        $$\frac{\Gamma, x:A \vdash t : B}{\Gamma \vdash \lambda x:A. t : \Pi(x:A).B}$$

        在当前的上下文 $\Gamma$ 中,如果我们假设有一个类型为 $A$ 的变量 $x$,并且能够构造出一个类型为 $B$ 的项 $t$。那么我们可以构造一个 $\lambda$ 抽象(即函数),它的类型就是 $\Pi(x:A).B$。

        \item 消解规则 (Elimination Rules)

        对于 $\Pi$类型,有一个通用规则 $f$(类型为 $\Pi(x:A).B$)和有一个具体的对象 $a$(类型为 $A$)。将 $f$ 应用于 $a$ (记作 $f\,a$),得到的结果类型是 $B[a/x]$。结果的类型中,所有的 $x$ 都被替换成了具体的值 $a$。

        $$\frac{\Gamma \vdash f : \Pi(x:A).B \quad \Gamma \vdash a : A}{\Gamma \vdash f\,a : B[a/x]}$$
    \end{itemize}

    \item 依存求和 ($\Sigma$-Types)

    \begin{itemize}
        \item 引入规则 (Introduction Rules)

        对于 $\Sigma$ 类型,其构造(引入)规则为:

        $$\frac{\Gamma \vdash a : A \quad \Gamma \vdash b : B[a/x]}{\Gamma \vdash \langle a, b \rangle : \Sigma(x:A).B}$$

        引入规则体现了 “构造即证明”,只有当你能提供满足 $B(a)$ 的证据 $b$ 时,知识对象才能被创建。

        \item 消解规则 (Elimination Rules)

        对于 $\Sigma$ 类型,我们定义通用的依存消解算子 $\textsf{split}$。它允许通过匹配对结构来构造依赖于该对整体的目标项。

        $$\frac{\Gamma \vdash p : \Sigma(x:A).B \quad \Gamma, x:A, y:B \vdash t : C[\langle x, y \rangle / z]}{\Gamma \vdash \textsf{split}(p, x.y.t) : C[p/z]}$$

        传统的投影算子可以作为 $\textsf{split}$ 的特例被定义:
        \begin{align*}
            \textsf{proj}_1(p)\text{(左投影)} & \equiv \textsf{split}(p, x.y.x) \\
            \textsf{proj}_2(p)\text{(右投影)} &\equiv \textsf{split}(p, x.y.y)
        \end{align*}
    \end{itemize}

    \item 和类型 ($A + B$)
    \begin{itemize}
        \item 引入规则 (Introduction Rules)

        引入规则定义了如何制造一个类型为 $A+B$ 的对象。它有两个分支,分别对应“左选择”和“右选择”。

        $$\frac{\Gamma \vdash a : A}{\Gamma \vdash \textsf{inl}(a) : A + B} \quad \frac{\Gamma \vdash b : B}{\Gamma \vdash \textsf{inr}(b) : A + B}$$

        如果有一个 $A$ 类型的证据 $a$,可以通过标签 $\textsf{inl}$(In-Left)把它包装成 $A+B$ 类型。同理,如果有一个 $B$ 类型的证据 $b$,通过 $\textsf{inr}$(In-Right)也能包装成 $A+B$。

        \item 消解规则 (Elimination Rules)

        消解规则定义了当拿到一个 $A+B$ 类型的对象时,如何安全地使用它。由于不知道它内部到底是 $A$ 还是 $B$,必须准备好两套方案。

        $$\frac{\Gamma \vdash s : A + B \quad \Gamma, x:A \vdash t:C \quad \Gamma, y:B \vdash u:C}{\Gamma \vdash \textsf{case}(s, x.t, y.u) : C}$$

        $s : A + B$是输入。$\Gamma, x:A \vdash t:C$是方案一。如果 $s$ 最终证明是 $A$ 类型的,把里面的数据取出来给 $x$,然后按逻辑 $t$ 算出一个结果,结果类型是 $C$。$\Gamma, y:B \vdash u:C$是方案二。如果 $s$ 是 $B$ 类型的,把数据给 $y$,按逻辑 $u$ 同样算出一个 $C$ 类型的结果。$\textsf{case}(s, x.t, y.u) : C$表明无论 $s$ 走哪条路,最终都能得到一个类型为 $C$ 的确定的结果。
    \end{itemize}

    \item 转换规则 (Conversion Rule)

    $$\frac{\Gamma \vdash t : A \quad \Gamma \vdash A \equiv B \quad \Gamma \vdash B : \mathcal{S}}{\Gamma \vdash t : B}$$

    该规则确保了如果项 $t$ 在 $\mathcal{U}_1$ 中合法,且 $\mathcal{U}_1 \hookrightarrow \textsf{Type}_2$ 成立,则 $t$ 自动具备在更高层级被观测的合法性。
\end{enumerate}

\subsection{4. 规约规则}

我们使用 $\to$ 表示单步规约(One-step reduction),使用 $\twoheadrightarrow$ 表示多步规约(计算的闭包)。

\begin{enumerate}[label=(\arabic*)]

\item \textbf{函数规约 ($\beta$-reduction)}

针对 $\Pi$ 类型构造($\lambda$ 抽象):
$$\frac{}{\Gamma \vdash (\lambda x:A.t)\,u \to t[u/x]}$$

\item \textbf{依存记录规约 ($\iota$-reduction)}

针对 $\Sigma$ 类型的结构化消解。这是核心修正点,通过 \textsf{split} 算子直接解构配对:
$$\frac{}{\Gamma \vdash \textsf{split}(\langle u, v \rangle, x.y.t) \to t[u/x, v/y]}$$

在这种定义下,传统的投影规约可以作为特例自然导出:
\begin{itemize}
    \item \textbf{左投影}:$\textsf{proj}_1(\langle u, v \rangle) \equiv \textsf{split}(\langle u, v \rangle, x.y.x) \to u$
    \item \textbf{右投影}:$\textsf{proj}_2(\langle u, v \rangle) \equiv \textsf{split}(\langle u, v \rangle, x.y.y) \to v$
\end{itemize}

\item \textbf{和类型规约 ($\iota$-reduction)}

针对 $+$ 类型的分支判定。通过 $\textsf{case}$ 算子对标签进行匹配:
$$\frac{}{\Gamma \vdash \textsf{case}(\textsf{inl}(u), x.t, y.v) \to t[u/x]} \quad \frac{}{\Gamma \vdash \textsf{case}(\textsf{inr}(w), x.t, y.v) \to v[w/y]}$$

\item \textbf{等价项规约 ($\iota$-reduction)}

当判定项已归约为 $\textsf{refl}$ 时,等价判定自动消解。
\end{enumerate}

为了支持工程实践中的模块化定义与局部变量,系统定义了以下辅助转换规则。与核心规约($\beta, \iota$)不同,这些转换通常在类型检查器的\textbf{等价性判定(Conversion Check)}阶段按需触发,不计入核心逻辑步。

\begin{itemize}
\item \textbf{全局展开 ($\delta$-conversion)}:\\
  若项 $c$ 在上下文 $\Gamma$ 中被定义为 $c := t:A$,则在判定等价性时允许展开:\\
  \[
  \frac{(c := t:A) \in \Gamma}{\Gamma \vdash c \equiv_\delta t}
  \]\\
  意义:允许系统识别别名(Alias)与其原始定义在逻辑上是同一对象。

\item \textbf{局部绑定展开 ($\zeta$-conversion)}:\\
  针对局部定义的 \texttt{let} 结构,其语义等价于立即代换:\\
  \[
  \Gamma \vdash (\textsf{let } x = u \textsf{ in } t) \equiv_\zeta t[u/x]
  \]\\
  意义:在不增加函数调用($\beta$)开销的前提下,支持项的局部复用。

\item \textbf{外延等价性 ($\eta$-conversion)}:\\
  为了保证函数的一致性,系统支持函数外延性判定:\\
  \[
  \Gamma \vdash \lambda x:A.(f\,x) \equiv_\eta f \quad (x \notin \text{FV}(f))
  \]\\
  意义:确保函数抽象与直接引用的行为一致,支持函数式编程范式。
\end{itemize}


“$\delta$ 与 $\zeta$ 确保了系统具有定义透明性 (Definitional Transparency),即名称的引用不改变逻辑项的语义本质。”

\begin{Definition}{等价判定 (Definitional Equality)}
KOS-TL 的判断等价关系 $\equiv$ 是由上述所有规约($\beta, \iota$)和转换($\delta, \zeta, \eta$)生成的最小等价关系(满足对称性、传递性与同余性)。
\end{Definition}



\section{核心层逻辑性质}

\begin{Definition}{范型(Normal Form)}

一个项 $t$ 被称为处于 \textbf{范型}(记作 $t \in \textsf{NF}$),当且仅当它不包含任何 \textbf{Redex}(可归约式)。即对于 KOS-TL Core 中定义的归约关系 $\to$,不存在项 $t'$ 使得:
\[
t \to t'
\]
\end{Definition}


\begin{Definition}{强范型(Strong Normalization,$\textsf{SN}$)}

一个项 $t$ 被称为是\textbf{强范型}的,当且仅当不存在从 $t$ 出发的无限规约序列。即所有可能的规约路径 $t \to t_1 \to t_2 \dots$ 均在有限步内终止于某个正规形式(Normal Form)。

\end{Definition}


\begin{Definition}{可还原性集合$\textsf{Red}_A$}

对于任意类型 $A$ 和项 $t : A$,可还原性集合 $\textsf{Red}_A \subseteq \textsf{Val}_A$ 通过对 $A$ 的结构归纳定义如下:

\begin{enumerate}[label=(\arabic*)]

    \item \textbf{世界类型情况 (Universe Rules)}

    \begin{itemize}
        \item $A \equiv \mathcal{U}_i$
        \[ t \in \textsf{Red}_{\mathcal{U}_i} \iff t \in \mathcal{RC} \land \textsf{level}(t) < i \]
        其中 $\mathcal{RC}$ 是所有满足 CR 属性(SN, 稳定性, 中性项构造)的项集集合。

     \item $A \equiv \textsf{Type}_i$
        \[ t \in \textsf{Red}_{\textsf{Type}_i} \iff t \in \mathcal{RC} \land \textsf{level}(t) < i \]

        \item \textbf{跨轴约束}:
        由于 $\textsf{Prop} : \textsf{Type}_1$,则 $\textsf{Prop} \in \textsf{Red}_{\textsf{Type}_1}$。
        由于 $\textsf{Val} : \mathcal{U}_0$,则 $\textsf{Val} \in \textsf{Red}_{\mathcal{U}_1}$。
    \end{itemize}

    \item 基类型情况
    \begin{itemize}

        \item $A \equiv \textsf{Val}$

    $t \in \textsf{Red}_A \iff t \in \textsf{SN} \land \exists c \in \textsf{Const}_\mathbb{N}.\ t \to^* c$

        \item $A \equiv \textsf{Time}$

    $t \in \textsf{Red}_A \iff t \in \textsf{SN} \land t \text{ 表征合法时间戳或时长(即 } t \to^* \mathsf{timestamp}(n) \text{ 或 } t \to^* \mathsf{duration}(n) \text{ 对于 } n \in \mathbb{N}\text{)}$

    \end{itemize}
    \item 构造类型情况
    \begin{itemize}
        \item $A \equiv \Pi(x : B).C$

     $t \in \textsf{Red}_A \iff \forall u \in \textsf{Red}_B.\ (t \  u) \in \textsf{Red}_{C[u/x]}$

        \item $A \equiv \Sigma(x : B).C$

    $t \in \textsf{Red}_{\Sigma(x:B).C} \iff t \twoheadrightarrow \langle u, v \rangle \land u \in \textsf{Red}_B \land v \in \textsf{Red}_{C[u/x]}$

     $t \in \textsf{Red}_A \iff t \in \textsf{SN} \land \left( t \to^* \mathsf{inl}(u) \implies u \in \textsf{Red}_B \right) \land \left( t \to^* \mathsf{inr}(v) \implies v \in \textsf{Red}_D \right)$

        \item $A \equiv \textsf{Id}_B(a, b)$

     $t \in \textsf{Red}_A \iff t \in \textsf{SN} \land \left( t \to^* \mathsf{refl}(w) \implies w \in \textsf{Red}_B \land a \equiv_B w \land b \equiv_B w \right)$
    \end{itemize}
\end{enumerate}

其中:
\begin{itemize}
\item $\textsf{Const}_\mathbb{N}$ 是自然数常量集合(数值常量)。
\item $\to^*$ 表示 $\beta\eta$-归约的多步闭包(multi-step $\beta\eta$-reduction)。
\item $\equiv_B$ 表示在类型 $B$ 下的可还原性等价:$p \equiv_B q \iff \exists v \in \textsf{Red}_B.\ (p \to^* v \land q \to^* v)$。
\item 对于 $\textsf{Time}$,需预定义“合法时间戳或时长”的精确语义(如 $\mathsf{timestamp}(n)$ 表示 Unix 时间戳 $n$,$\mathsf{duration}(n)$ 表示 $n$ 毫秒时长),以确保形式化。
\end{itemize}

 需要指出的是$\textsf{Prop}$ 的解释不依赖于其所在的世界等级。
\end{Definition}


\begin{Definition}{中性项(Neutral Term)}

在 KOS-TL Core 层中,一个项 $t$ 被称为 \textbf{中性项},当且仅当它满足以下两个条件之一:
\begin{enumerate}[label=(\arabic*)]
\item $t$ 是一个变量 $x$。
\item $t$ 的头部(Head)是一个变量,且该项正处于被消去(Elimination)的过程中,但无法进行进一步的归约。
\end{enumerate}
\end{Definition}

类型$A$ 的可还原集 $\textsf{Red}_A$必须满足三个关键的饱和属性(Saturation Properties),通常称为 $CR1,CR2,CR3$。
\begin{Definition}{可还原集 $\textsf{Red}_A$的饱和属性}
\begin{enumerate}[label=(\arabic*)]
    \item \textbf{CR 1 (包含性)}

    若 $t \in \textsf{Red}_A$,则 $t \in \textsf{SN}$(即 $t$ 必须先是强规范化的)。
    \item \textbf{CR 2 (稳定性)}

    若 $t \in \textsf{Red}_A$ 且 $t \to t'$,则 $t' \in \textsf{Red}_A$。
    \item \textbf{CR 3 (中性项构造)}

    若 $t$ 是一个中性项,且 $t$ 的所有单步归约项 $t'$ 都在 $\textsf{Red}_A$ 中,则 $t \in \textsf{Red}_A$。所有变量都是中性项。
\end{enumerate}
\end{Definition}

\begin{Lemma}{替换引理(Substitution Lemma)}
\label{lemma:KOSsubstitution}

若 $\Gamma, x:B, \Delta \vdash t : A$ 且 $\Gamma \vdash u : B$,则 $\Gamma, \Delta[u/x] \vdash t[u/x] : A[u/x]$。

其中, $\Delta$通用上下文,用以处理 $x$ 之后定义的依赖于 $x$ 的变量。在简单情况下,$\Delta$ 为空。
\end{Lemma}

\begin{proof}

依据类型推导树的结构进行归纳证明。根据 $t$ 的构造规则,分以下几种核心情况讨论:
\begin{enumerate}[label=(\arabic*)]

\item \textbf{变量情况 (Variable)}\\
假设 $t$ 是一个变量 $y$。

\begin{itemize}
\item \textbf{子情况 1:}$y = x$\\
  根据推导规则,此时 $A = B$。\\
  我们需要证明 $\Gamma, \Delta[u/x] \vdash x[u/x] : B[u/x]$。\\
  由于 $x[u/x] = u$,且前提已知 $\Gamma \vdash u : B$。根据上下文弱化规则(Weakening),可在 $\Gamma$ 后增加 $\Delta[u/x]$,故结论成立。

\item \textbf{子情况 2:}$y \neq x$\\
  此时 $y$ 必须在 $\Gamma$ 或 $\Delta$ 中定义。\\
  若 $y \in \Gamma$,则 $y[u/x] = y$ 且 $A[u/x] = A$(因为 $\Gamma$ 中的类型不依赖 $x$),结论显然。\\
  若 $y \in \Delta$,则 $y[u/x] = y$,其类型为 $A[u/x]$,这正是 $\Delta[u/x]$ 中对应的声明。
\end{itemize}

\item \textbf{$\Pi$-类型引入 ($\lambda$-抽象)}\\
假设 $t = \lambda y:C. M$,且 $A = \Pi(y:C). D$。\\
推导最后一步为
\[
\frac{\Gamma, x:B, \Delta, y:C \vdash M : D}{\Gamma, x:B, \Delta \vdash \lambda y:C. M : \Pi(y:C). D}
\]

\begin{itemize}
\item 应用归纳假设:对 $M$ 使用归纳假设(此时上下文为 $\Delta, y:C$):
  \[
  \Gamma, \Delta[u/x], y:C[u/x] \vdash M[u/x] : D[u/x]
  \]

\item 构造结论:应用 $\Pi$-引入规则:
  \[
  \Gamma, \Delta[u/x] \vdash \lambda y:C[u/x]. M[u/x] : \Pi(y:C[u/x]). D[u/x]
  \]
  这等价于 $(\lambda y:C. M)[u/x] : (\Pi(y:C). D)[u/x]$。
\end{itemize}

\item \textbf{$\Pi$-类型消去 (应用)}\\
假设 $t = (f \, v)$,其中 $\Gamma, x:B, \Delta \vdash f : \Pi(y:C). D$ 且 $\Gamma, x:B, \Delta \vdash v : C$。

\begin{itemize}
\item 归纳假设 1:$\Gamma, \Delta[u/x] \vdash f[u/x] : (\Pi(y:C). D)[u/x]$。
\item 归纳假设 2:$\Gamma, \Delta[u/x] \vdash v[u/x] : C[u/x]$。
\item 组合:应用消去规则:
  \[
  (f[u/x] \, v[u/x]) : D[u/x][v[u/x]/y]
  \]
  根据代换的可交换性,上述类型等价于 $D[v/y][u/x]$,即 $A[u/x]$。
\end{itemize}

\item \textbf{$\Sigma$-类型构造 (配对)}

假设 $t = \langle t_1, t_2 \rangle$,且 $A = \Sigma(y:C). D$。

\begin{itemize}
\item 由归纳假设,$t_1[u/x] : C[u/x]$。
\item 由归纳假设,$t_2[u/x] : D[t_1/y][u/x]$。
\item 根据规则构造出 $\langle t_1[u/x], t_2[u/x] \rangle : (\Sigma(y:C). D)[u/x]$。
\end{itemize}

\item \textbf{$\Sigma$-类型消去 (结构化消解)}

假设 $t = \textsf{split}(p, x.y.u)$,且最后推导步为:

$$\frac{\Gamma, z:B, \Delta \vdash p : \Sigma(x:A).B \quad \Gamma, z:B, \Delta, x:A, y:B \vdash u : C(\langle x, y \rangle)}{\Gamma, z:B, \Delta \vdash \textsf{split}(p, x.y.u) : C(p)}$$

(此处 $z:B$ 是我们要替换的变量)
\begin{itemize}
    \item \textbf{归纳假设 1}:对 $p$ 应用归纳假设,得到 $\Gamma, \Delta[v/z] \vdash p[v/z] : (\Sigma(x:A).B)[v/z]$。
    \item \textbf{归纳假设 2}:对消解体 $u$ 应用归纳假设(此时上下文多了 $x, y$),得到:$$ \Gamma, \Delta[v/z], x:A[v/z], y:B[v/z] \vdash u[v/z] : C(\langle x, y \rangle)[v/z] $$
    \item \textbf{构造结论}:重新应用 $\Sigma$-消去规则($\textsf{split}$ 规则):

    $$ \Gamma, \Delta[v/z] \vdash \textsf{split}(p[v/z], x.y.u[v/z]) : C(p)[v/z] $$

    由于 $\textsf{split}(p, x.y.u)[v/z] = \textsf{split}(p[v/z], x.y.u[v/z])$,结论成立。
\end{itemize}


\item \textbf{$+$-类型引入 (注入)} \\
假设 $t = \textsf{inl}_D(s)$,且 $A = C + D$($\textsf{inr}$ 情况对称)。
\begin{itemize}
    \item \textbf{前提}:已知 $\Gamma, x:B, \Delta \vdash s : C$ 且 $\Gamma, x:B, \Delta \vdash D : \mathcal{U}$。
    \item \textbf{归纳假设}:对 $s$ 有 $s[u/x] : C[u/x]$;对类型 $D$ 有 $D[u/x] : \mathcal{U}$。\item \textbf{构造结论}:应用 $+$-引入规则:$$ \Gamma, \Delta[u/x] \vdash \textsf{inl}_{D[u/x]}(s[u/x]) : C[u/x] + D[u/x] $$即 $(\textsf{inl}_D(s))[u/x] : (C + D)[u/x]$。
\end{itemize}


\item \textbf{$+$-类型消去 (分支判定)}\\
假设 $t = \textsf{case}(s, y.t_1, z.t_2)$,且最后一步推导为:$$\frac{\Gamma', s:C+D \quad \Gamma', y:C \vdash t_1 : A \quad \Gamma', z:D \vdash t_2 : A}{\Gamma' \vdash \textsf{case}(s, y.t_1, z.t_2) : A}$$(其中 $\Gamma'$ 简记为 $\Gamma, x:B, \Delta$)
\begin{itemize}
    \item \textbf{归纳假设 1}:对判定项 $s$,有 $\Gamma, \Delta[u/x] \vdash s[u/x] : C[u/x] + D[u/x]$。
    \item \textbf{归纳假设 2}:对左分支 $t_1$(此时上下文增加 $y:C$),有 $\Gamma, \Delta[u/x], y:C[u/x] \vdash t_1[u/x] : A[u/x]$。
    \item \textbf{归纳假设 3}:对右分支 $t_2$ (此时上下文增加 $z:D$),有 $\Gamma, \Delta[u/x], z:D[u/x] \vdash t_2[u/x] : A[u/x]$。
    \item \textbf{组合}:应用 $+$-消去规则($\textsf{case}$ 规则):$$ \Gamma, \Delta[u/x] \vdash \textsf{case}(s[u/x], y.t_1[u/x], z.t_2[u/x]) : A[u/x] $$结论成立。
\end{itemize}

\item \textbf{恒等类型 (Identity)}\\
若 $t = \textsf{refl}_a$,则 $A = \textsf{Id}_C(a, a)$。

\begin{itemize}
\item 由归纳假设,$a[u/x] : C[u/x]$。
\item 直接应用构造规则得到 $\textsf{refl}_{a[u/x]} : \textsf{Id}_{C[u/x]}(a[u/x], a[u/x])$,即 $A[u/x]$。
\end{itemize}
\end{enumerate}
\end{proof}


\begin{Lemma}{可还原基本引理}
\label{lemma:KOSreduction}

设 $\Gamma = \{x_1:A_1, \dots, x_n:A_n\}$ 是一个良构上下文。若 $\Gamma \vdash t : C$,且存在一个可还原替换 $\gamma = [u_1/x_1, \dots, u_n/x_n]$ 使得对于所有 $i$,均有 $u_i \in \textsf{Red}_{A_i}$。
则替换后的项 $t[\gamma]$ 必定满足:
\[
t[\gamma] \in \textsf{Red}_C
\]
\end{Lemma}

\begin{proof}

依据类型推导树的结构进行归纳证明。
\begin{enumerate}[label=(\arabic*)]

    \item \textbf{变量规则 (Variables)}

    若推导为 $\Gamma \vdash x_i : A_i$。

    根据替换定义,$x_i[\gamma] = u_i$。由前提 $u_i \in \textsf{Red}_{A_i}$,命题显然成立。

    \item \textbf{$\Pi$-类型引入 ($\lambda$-抽象)}

    若 $\Gamma \vdash \lambda x:A.M : \Pi(x:A).B$。\\
    需证:对于任意 $u \in \textsf{Red}_A$,$((\lambda x:A.M)[\gamma] \, u) \in \textsf{Red}_{B[u/x]}$。

    \begin{enumerate}[label=(\alph*)]
        \item 该项 $\beta$-归约为 $M[\gamma, u/x]$。
        \item 由于 $u \in \textsf{Red}_A$ 且 $\gamma \in \textsf{Red}_\Gamma$,则 $(\gamma, u/x)$ 是语境 $(\Gamma, x:A)$ 下的一个可还原替换。
        \item 根据归纳假设,$M[\gamma, u/x] \in \textsf{Red}_B$。
        \item 由于可还原集合对逆向归约封闭,故 $(\lambda x:A.M)[\gamma] \in \textsf{Red}_{\Pi(x:A).B}$。
    \end{enumerate}

    \item \textbf{$\Pi$-类型消去 (应用)}

    若 $\Gamma \vdash (f \, v) : C$,其中 $C = B[v/x]$ 且 $\Gamma \vdash f : \Pi(x:A).B, \Gamma \vdash v : A$。
    \begin{enumerate}[label=(\alph*)]
        \item 由归纳假设,$f[\gamma] \in \textsf{Red}_{\Pi(x:A).B}$。
        \item 由归纳假设,$v[\gamma] \in \textsf{Red}_A$。
        \item 根据 $\textsf{Red}_{\Pi}$ 的定义:若一个项属于 $\Pi$ 类型的可还原集,则它作用于任何属于参数类型可还原集的项,其结果必属于结果类型的可还原集。
        \item 因此,$(f[\gamma] \, v[\gamma]) \in \textsf{Red}_{B[v[\gamma]/x]}$。
        \item 由于 $(f[\gamma] \, v[\gamma]) = (f \, v)[\gamma]$ 且 $B[v[\gamma]/x] = C[\gamma]$,结论成立。
    \end{enumerate}


    \item \textbf{$\Sigma$-类型引入 (配对)}

    若 $\Gamma \vdash \langle a, b \rangle : \Sigma(x:A).B$。

    \begin{enumerate}[label=(\alph*)]
        \item 由归纳假设,$a[\gamma] \in \textsf{Red}_A$。
        \item 由归纳假设,$b[\gamma] \in \textsf{Red}_{B[a[\gamma]/x]}$。
        \item 根据 $\textsf{Red}_{\Sigma}$ 定义,分量均可还原,则 $\langle a, b \rangle[\gamma] \in \textsf{Red}_{\Sigma(x:A).B}$。
    \end{enumerate}

    \item \textbf{$\Sigma$-类型消去 (split 算子)}

    若 $\Gamma \vdash \textsf{split}(p, x.y.t) : C$,其中 $\Gamma \vdash p : \Sigma(x:A).B$。

    \begin{enumerate}[label=(\alph*)]
        \item \textbf{归纳假设}:由归纳假设知,$p[\gamma] \in \textsf{Red}_{\Sigma(x:A).B}$。这意味着 $p[\gamma]$ 强规范化且最终归约为某个配对 $\langle u, v \rangle$,其中 $u \in \textsf{Red}_A, v \in \textsf{Red}_{B[u/x]}$。
        \item \textbf{规约分析}:根据 $\iota$-规约,$\textsf{split}(p[\gamma], x.y.t[\gamma]) \twoheadrightarrow t[\gamma, u/x, v/y]$。
        \item \textbf{应用归纳假设}:由于 $(\gamma, u/x, v/y)$ 是语境 $(\Gamma, x:A, y:B)$ 下的合法可还原替换,对 $t$ 应用归纳假设得:$$ t[\gamma, u/x, v/y] \in \textsf{Red}_C $$
        \item \textbf{封闭性}:利用可还原集合对逆向 $\iota$-规约的封闭性(CR 3 属性的推论),原始项 $\textsf{split}(p, x.y.t)[\gamma]$ 亦属于 $\textsf{Red}_C$。
    \end{enumerate}

    \item \textbf{$+$-类型引入 (注入)}

    若 $\Gamma \vdash \textsf{inl}(a) : A + B$($\textsf{inr}$ 同理)。
    \begin{enumerate}[label=(\alph*)]
        \item 由归纳假设,$a[\gamma] \in \textsf{Red}_A$。
        \item 根据 $\textsf{Red}_{A+B}$ 的定义(通常由中性项和注入项的性质定义):由于 $a[\gamma]$ 是可还原的,则其注入项 $\textsf{inl}(a[\gamma])$ 在 $+$-类型的可还原体系下也是可还原的(利用逆向归约封闭性)。
        \item 故 $\textsf{inl}(a)[\gamma] \in \textsf{Red}_{A+B}$。
    \end{enumerate}

    \item \textbf{和类型消去 (Case Analysis)}

    若 $\Gamma \vdash \textsf{case}(t, x.M, y.N) : C$。
    \begin{enumerate}[label=(\alph*)]
        \item 由归纳假设,$t[\gamma] \in \textsf{Red}_{A+B}$。
        \item $t[\gamma]$ 将归约为 $\textsf{inl}(u)$ 或 $\textsf{inr}(v)$。假设为 $\textsf{inl}(u)$,则 $u \in \textsf{Red}_A$。
        \item 此时 $\textsf{case}$ 项归约为 $M[\gamma, u/x]$。
        \item 根据归纳假设,$M[\gamma, u/x] \in \textsf{Red}_C$。同理可证 $\textsf{inr}$ 情况。
    \end{enumerate}

    \item \textbf{恒等类型引入 ($\textsf{refl}$)}

    若 $\Gamma \vdash \textsf{refl}_a : \textsf{Id}_A(a, a)$。

    \begin{enumerate}
        \item 由归纳假设,$a[\gamma] \in \textsf{Red}_A$。
        \item 显然 $a[\gamma] \cong a[\gamma]$ 且 $\textsf{refl} \in \textsf{SN}$。
        \item 故 $\textsf{refl}_{a[\gamma]} \in \textsf{Red}_{\textsf{Id}_A(a[\gamma], a[\gamma])}$。
    \end{enumerate}

    \item \textbf{$\delta$-规约与局部定义 ($\textsf{let}$)}
    若 $\Gamma \vdash \textsf{let } x=u \textsf{ in } t : C$。
    \begin{enumerate}[label=(\alph*)]
        \item 由归纳假设,$u[\gamma] \in \textsf{Red}_A$。
        \item 构造扩展替换 $\gamma' = [\gamma, u[\gamma]/x]$。由于 $u[\gamma]$ 可还原,$\gamma'$ 是良构语境下的可还原替换。
        \item 由归纳假设,$t[\gamma'] \in \textsf{Red}_C$。
        \item 因为 $(\textsf{let } x=u \textsf{ in } t)[\gamma] \to t[\gamma, u[\gamma]/x]$,根据可还原集合对逆向 $\zeta$- 规约的封闭性,结论成立。
    \end{enumerate}
\end{enumerate}
\end{proof}

\begin{Theorem}{KOS-TL内核的强范型(Strong Normalization for KOS-TL Core)}

设 $\Gamma$ 是一个良构语境。若项 $t$ 满足 $\Gamma \vdash t : A$,则 $t$ 是强范型(即 $t \in \textsf{SN}$)。

这意味着从 $t$ 出发的任何归约序列 $t \to t_1 \to t_2 \dots$ 都是有限的。
\end{Theorem}

\begin{proof}

证明参考Tait-Girard 方法\cite{Girard1972}\cite{Girard1989}。证明的核心逻辑是利用基本引理将“类型合法性”转化为“可还原性”,再利用可还原项必强规范化的性质。

\textbf{第一步:引入恒等替换 (Identity Substitution)}

对于语境 $\Gamma = \{x_1:A_1, \dots, x_n:A_n\}$,我们构造一个特殊的替换 $\gamma_{id}$:
\[
\gamma_{id} = [x_1/x_1, \dots, x_n/x_n]
\]
为了应用基本引理,我们需要证明 $\gamma_{id}$ 是一个可还原替换。这意味着对于每一个变量 $x_i$,必须证明其自身属于可还原集合,即 $x_i \in \textsf{Red}_{A_i}$。

\textbf{第二步:变量的可还原性 (Reducibility of Variables)}

根据可还原性集合(Girard's $\mathcal{RC}$)的性质,所有 $\textsf{Red}_A$ 集合都满足以下两个关键属性:
\begin{enumerate}
\item CR 1 (SN 包含性):若 $t \in \textsf{Red}_A$,则 $t \in \textsf{SN}$。
\item CR 3 (中性项属性):若 $t$ 是中性项(即变量或变量的应用,且无法进一步归约)且其所有单步归约项都在 $\textsf{Red}_A$ 中,则 $t \in \textsf{Red}_A$。
\end{enumerate}
由于变量 $x_i$ 是基础的中性项且没有归约式,根据 CR 3,推导出 $x_i \in \textsf{Red}_{A_i}$。\\
因此,$\gamma_{id}$ 满足基本引理的前提条件。

\textbf{第三步:应用基本引理 (Application of Fundamental Lemma)}

由于前提 $\Gamma \vdash t : A$ 成立,且 $\gamma_{id}$ 是可还原替换,根据基本引理\ref{lemma:KOSreduction}:
\[
t[\gamma_{id}] \in \textsf{Red}_A
\]
由于 $t[\gamma_{id}]$ 在语法上等同于项 $t$ 本身,我们得到:
\[
t \in \textsf{Red}_A
\]

\textbf{第四步:结论导出}

根据可还原集合的属性 CR 1(所有属于可还原集合的项都是强规范化的):
\[
t \in \textsf{Red}_A \implies t \in \textsf{SN}
\]
至此,证明完毕。
\end{proof}

\begin{Theorem}{类型保持性(Subject Reduction)}

在 KOS-TL Core 中,归约操作不改变项的类型。若 $\Gamma \vdash t : A$ 且 $t \to t'$,则 $\Gamma \vdash t' : A$。
\end{Theorem}

\begin{proof}

基于替换引理\ref{lemma:KOSsubstitution},结构归纳证明如下。

\begin{enumerate}[label=(\arabic*)]
    \item \textbf{主要归约情形:$\beta$-归约}

    考虑最基本的归约步 $(\lambda x:B.t)u \to t[u/x]$:
    \begin{itemize}
        \item 由前提 $\Gamma \vdash (\lambda x:B.t)u : A$ 可知,必存在类型 $B$,使得 $\Gamma \vdash \lambda x:B.t : \Pi(x:B).A'$ 且 $\Gamma \vdash u : B$。
        \item 根据 $\Pi$-引入规则的反转,可知 $\Gamma, x:B \vdash t : A'$。
        \item 应用代换引理,直接得出 $\Gamma \vdash t[u/x] : A'[u/x]$。
    \end{itemize}

    \item \textbf{$\Sigma$-类型规约:$\iota$-规约}

    考虑情形 $\textsf{split}(\langle u, v \rangle, x.y.t) \to t[u/x, v/y]$:
    \begin{itemize}
        \item \textbf{前提}:由 $\Gamma \vdash \textsf{split}(\langle u, v \rangle, x.y.t) : C$ 可知:
        \begin{enumerate}[label=(\alph*)]
            \item $\Gamma \vdash \langle u, v \rangle : \Sigma(x:A).B$
            \item $\Gamma, x:A, y:B \vdash t : C'$ (其中 $C$ 实际上是 $C'[\langle u, v \rangle / z]$)
        \end{enumerate}
        \item \textbf{推导}:从 (a) 依据引入规则反转,得 $\Gamma \vdash u : A$ 且 $\Gamma \vdash v : B[u/x]$。
        \item \textbf{应用}:对 $t$ 连续应用两次代换引理(先换 $x$ 后换 $y$),直接得到:$$ \Gamma \vdash t[u/x, v/y] : C'[\langle u, v \rangle / z] $$ 类型保持一致,结论成立。
    \end{itemize}

\item \textbf{和类型归约:$\iota$-归约}

考虑情形 $\textsf{case}(\textsf{inl}(u), x.M, y.N) \to M[u/x]$:
\begin{itemize}
    \item 由前提 $\Gamma \vdash \textsf{case}(\textsf{inl}(u), x.M, y.N) : A$ 可知:
    \begin{enumerate}[label=(\alph*)]
        \item $\Gamma \vdash \textsf{inl}(u) : B + C$
        \item $\Gamma, x:B \vdash M : A$ 且 $\Gamma, y:C \vdash N : A$。
    \end{enumerate}
    \item 从 (a) 根据 $+$-引入规则的反转,可知 $\Gamma \vdash u : B$。
    \item 应用代换引理(引理\ref{lemma:KOSsubstitution}),由 $\Gamma, x:B \vdash M : A$ 和 $\Gamma \vdash u : B$ 可得 $\Gamma \vdash M[u/x] : A[u/x]$。
    \item 若 $A$ 不依赖于判定项,则 $A[u/x] = A$,结论成立。(对于依赖类型情形,代换同样保持类型的一致性)。
\end{itemize}

\item \textbf{局部定义归约:$\zeta$-归约}

考虑情形 $\textsf{let } x = u \textsf{ in } t \to t[u/x]$:
\begin{itemize}
    \item 由前提 $\Gamma \vdash \textsf{let } x = u \textsf{ in } t : A$ 可知,必存在类型 $B$ 使得 $\Gamma \vdash u : B$ 且 $\Gamma, x:B \vdash t : A$。
    \item 这是一个标准的代换引理应用场景。根据引理\ref{lemma:KOSsubstitution},直接推导得 $\Gamma \vdash t[u/x] : A[u/x]$。
\end{itemize}

\item \textbf{定义展开归约:$\delta$-归约}

考虑情形 $c \to t$,其中 $(c := t:A) \in \Gamma$:
\begin{itemize}
    \item 由前提 $\Gamma \vdash c : A$ 可知,$c$ 是在上下文中声明的常量。
    \item 根据 $\delta$-规约的定义,标识符 $c$ 的类型与其定义体 $t$ 的类型在 $\Gamma$ 中是完全一致的。
    \item 故由 $\Gamma$ 的良构性直接得 $\Gamma \vdash t : A$。
\end{itemize}

\item \textbf{同余情形 (Congruence Cases)}

若归约发生在子项中,例如 $t = f\,u \to f'\,u$(其中 $f \to f'$):
\begin{itemize}
    \item 根据归纳假设,$f'$ 保持了 $f$ 的类型 $\Pi(x:B).A$。
    \item 重新应用 $\Pi$-消去规则,整体项的类型依然为 $A[u/x]$。
    \item 同理可证所有其他构造(配对、投影、注入等)在同余规约下的类型保持性。
\end{itemize}

\end{enumerate}

    由于所有基本的计算归约($\beta, \iota$ 等)均满足类型保持,且归约关系 $\to$ 在上下文构造下是封闭的,通过对归约关系的结构归纳,证明定理对所有归约步成立。
\end{proof}


我们将类型 $A$ 解释为一个由项组成的集合 $\llbracket A \rrbracket$。这些集合必须满足前文提到的可还原性候选者(CR)性质。

\begin{Definition}{类型语义}

\begin{itemize}
\item $\llbracket \textsf{Val} \rrbracket = \{ t \mid t \in \textsf{SN} \land t \text{ 最终归约为数值常量} \}$
\item $\llbracket \textsf{Time} \rrbracket = \{ t \mid t \in \textsf{SN} \land t \text{ 最终归约为合法时间戳} \}$
\item $\llbracket \Pi(x:A).B \rrbracket = \{ f \mid \forall u \in \llbracket A \rrbracket, (f \, u) \in \llbracket B \rrbracket [u/x] \}$
\item $\llbracket \Sigma(x:A).B \rrbracket = \{ p \mid p \twoheadrightarrow \langle u, v \rangle \land u \in \llbracket A \rrbracket \land v \in \llbracket B \rrbracket[u/x] \}$
\item $\llbracket A + B \rrbracket = \{ t \mid t \in \textsf{SN} \land (t \twoheadrightarrow \textsf{inl}(u) \Rightarrow u \in \llbracket A \rrbracket) \land (t \twoheadrightarrow \textsf{inr}(v) \Rightarrow v \in \llbracket B \rrbracket) \}$
\item $\llbracket \textsf{Id}_A(a, b) \rrbracket = \{ \textsf{refl} \mid a, b \in \llbracket A \rrbracket \land a \simeq_{Red} b \}$。 其中 $\simeq_{Red}$ 表示它们规约到相同的范式。
\end{itemize}
\end{Definition}



解释函数 $\llbracket t \rrbracket_\rho$ 负责将带变量的语法项转换为其对应的语义值。在强规范化证明中,这种“解释”通常就是代换(Substitution)操作。

\begin{Definition}{项的语义解释}

设 $\rho$ 是一个从变量到语义值的映射(赋值)。
\begin{itemize}
\item $\llbracket x \rrbracket_\rho = \rho(x)$ (直接从环境中读取赋值)
\item $\llbracket \lambda x:A. M \rrbracket_\rho = \text{一个函数 } v \mapsto \llbracket M \rrbracket_{\rho[x \mapsto v]}$
\item $\llbracket f \, a \rrbracket_\rho = \llbracket f \rrbracket_\rho (\llbracket a \rrbracket_\rho)$ (函数作用)
\item $\llbracket \langle a, b \rangle \rrbracket_\rho = (\llbracket a \rrbracket_\rho, \llbracket b \rrbracket_\rho)$ (语义配对)
\item $\llbracket \textsf{refl} \rrbracket_\rho = \textsf{refl}$ (常量解释为其自身)
\end{itemize}
\end{Definition}

为了使 $\llbracket t \rrbracket_\rho \in \llbracket A \rrbracket$ 成立,赋值 $\rho$ 必须是“合法的”。

\begin{Definition}{命题语义的逻辑闭环(Logical Closure of Prop)}

在语义模型 $\mathcal{M}$ 中,$\textsf{Prop}$ 的解释 $\llbracket \textsf{Prop} \rrbracket$ 被定义为所有满足以下性质的项集 $S$ 的集合:
\begin{enumerate}[label=(\arabic*)]
    \item \textbf{SN 性}:$S \subseteq \textsf{SN}$。\item \textbf{CR 属性}:$S$ 对归约和逆向归约封闭,且包含所有中性项。

    \item \textbf{非谓述语义算子}:对于任意集合 $X$ 和函数 $F: X \to \llbracket \textsf{Prop} \rrbracket$,交集运算定义为:

    $$ \llbracket \Pi(x:A).B \rrbracket_\rho = \bigcap_{u \in \llbracket A \rrbracket_\rho} \{ f \mid (f \, u) \in \llbracket B \rrbracket_{\rho[x \mapsto u]} \} $$
\end{enumerate}

\textbf{闭环修正}:由于 $\llbracket \textsf{Prop} \rrbracket$ 包含了它自身构造出的所有 $\Pi$ 类型解释(通过对可还原候选者 $\mathcal{RC}$ 的交集运算),这保证了即使 $A$ 是无限大的 Universe,其映射后的结果依然留在 $\llbracket \textsf{Prop} \rrbracket$ 预定义的集合内。
\end{Definition}

\begin{Definition}{合法赋值}

称赋值 $\rho$ 满足语境 $\Gamma$(记作 $\rho \models \Gamma$),当且仅当对于 $\Gamma$ 中的每一个绑定 $(x:A)$,都有:
\[
\rho(x) \in \llbracket A \rrbracket_\rho
\]
\end{Definition}


\begin{Theorem}{语义可靠性}
\label{Theorem:KosSoundness}

若 $\Gamma \vdash t : A$ 且 $\rho \models \Gamma$,则 $\llbracket t \rrbracket_\rho \in \llbracket A \rrbracket_\rho$。
\end{Theorem}

\begin{proof}

对推导树 $\Gamma \vdash t : A$ 的结构进行归纳。由于 $\textsf{Prop}$ 的非谓述性,其可还原候选者($\mathcal{RC}$)的构造基于 Girard 的分层候选集方法,而非简单的 Tarski 风格语义。

\textbf{A. 变量规则 (Variable)}

假设推导为
\[
\frac{(x:A) \in \Gamma}{\Gamma \vdash x : A}
\]

\begin{itemize}
\item 证明:由前提 $\rho \models \Gamma$ 可知,环境中每个绑定的变量其赋值必属于该类型的解释。因此 $\rho(x) \in \llbracket A \rrbracket_\rho$。
\item 由于 $\llbracket x \rrbracket_\rho = \rho(x)$,结论 $\llbracket x \rrbracket_\rho \in \llbracket A \rrbracket_\rho$ 成立。
\end{itemize}

\textbf{B. $\Pi$-类型引入 ($\lambda$-抽象)}

假设推导最后一步为
\[
\frac{\Gamma, x:A \vdash M : B}{\Gamma \vdash \lambda x:A. M : \Pi(x:A). B}
\]

\begin{itemize}
\item 目标:证 $\llbracket \lambda x:A. M \rrbracket_\rho \in \llbracket \Pi(x:A). B \rrbracket_\rho$。
\item 根据 $\Pi$ 的语义定义,需证:对任意 $u \in \llbracket A \rrbracket_\rho$,有 $(\llbracket \lambda x:A. M \rrbracket_\rho \cdot u) \in \llbracket B \rrbracket_{\rho[x \mapsto u]}$。
\item 由项的解释定义,$(\llbracket \lambda x:A. M \rrbracket_\rho \cdot u) \to_\beta \llbracket M \rrbracket_{\rho[x \mapsto u]}$。
\item 由于 $u \in \llbracket A \rrbracket_\rho$ 且 $\rho \models \Gamma$,则新赋值 $\rho' = \rho[x \mapsto u]$ 满足 $\rho' \models (\Gamma, x:A)$。
\item 应用归纳假设:$\llbracket M \rrbracket_{\rho'} \in \llbracket B \rrbracket_{\rho'}$。
\item 由于可还原集合对逆向归约封闭(CR 属性),故原应用项亦属于该集合。
\end{itemize}

\textbf{C. $\Pi$-类型消去 (应用)}

假设推导最后一步为应用规则:
\[
\frac{\Gamma \vdash f : \Pi(x:A).B \quad \Gamma \vdash u : A}{\Gamma \vdash f \, u : B[u/x]}
\]

\begin{itemize}
\item 证明:由归纳假设,$\llbracket f \rrbracket_\rho \in \llbracket \Pi(x:A).B \rrbracket_\rho$ 且 $\llbracket u \rrbracket_\rho \in \llbracket A \rrbracket_\rho$。
\item 根据 $\Pi$ 类型语义的定义,函数 $\llbracket f \rrbracket_\rho$ 作用于任何属于 $\llbracket A \rrbracket_\rho$ 的元素,其结果必属于 $B$ 的解释。
\item 因此 $\llbracket f \rrbracket_\rho(\llbracket u \rrbracket_\rho) \in \llbracket B \rrbracket_{\rho[x \mapsto \llbracket u \rrbracket_\rho]}$。
\item 根据替换引理,该集合即为 $\llbracket B[u/x] \rrbracket_\rho$。
\end{itemize}

\textbf{D. $\Sigma$-类型引入 (配对)}

假设推导为:
\[
\frac{\Gamma \vdash a : A \quad \Gamma \vdash b : B[a/x]}{\Gamma \vdash \langle a, b \rangle : \Sigma(x:A). B}
\]

\begin{enumerate}
\item 由归纳假设,$\llbracket a \rrbracket_\rho \in \llbracket A \rrbracket_\rho$。
\item 由归纳假设,$\llbracket b \rrbracket_\rho \in \llbracket B[a/x] \rrbracket_\rho$。
\item 根据 $\Sigma$ 的语义定义:$\llbracket \Sigma(x:A). B \rrbracket_\rho = \{ (u, v) \mid u \in \llbracket A \rrbracket_\rho, v \in \llbracket B \rrbracket_{\rho[x \mapsto u]} \}$。
\item 结合替换引理 $\llbracket B[a/x] \rrbracket_\rho = \llbracket B \rrbracket_{\rho[x \mapsto \llbracket a \rrbracket_\rho]}$,可知 $\llbracket \langle a, b \rangle \rrbracket_\rho$ 的两个分量完全符合定义。
\end{enumerate}

\textbf{E. 依存求和消解 ($\Sigma$-消解 / split)}

假设推导的最后一步应用了 $\Sigma$ 消解规则:
\[
\frac{\Gamma \vdash p : \Sigma(x:A).B \quad \Gamma, x:A, y:B \vdash t : C(\langle x, y \rangle)}{\Gamma \vdash \textsf{split}(p, x.y.t) : C(p)}
\]
我们需要证明:若 $\rho \models \Gamma$,则 $\llbracket \textsf{split}(p, x.y.t) \rrbracket_\rho \in \llbracket C(p) \rrbracket_\rho$。

\begin{enumerate}
\item \textbf{语义前提推导} \\
根据归纳假设(Inductive Hypothesis):\\
\begin{itemize}
\item 对于项 $p$,有 $\llbracket p \rrbracket_\rho \in \llbracket \Sigma(x:A).B \rrbracket_\rho$。
\item 根据 $\Sigma$ 类型的语义定义,存在 $u \in \llbracket A \rrbracket_\rho$ 且 $v \in \llbracket B \rrbracket_{\rho[x \mapsto u]}$,使得 $\llbracket p \rrbracket_\rho$ 在语义上等价于对 $(u, v)$。
\end{itemize}

\item \textbf{构造合法赋值} \\
定义一个新的赋值 $\rho' = \rho[x \mapsto u, y \mapsto v]$。\\
\begin{itemize}
\item 由于 $u \in \llbracket A \rrbracket_\rho$,则 $\rho[x \mapsto u] \models (\Gamma, x:A)$。
\item 由于 $v \in \llbracket B \rrbracket_{\rho[x \mapsto u]}$,且 $\rho[x \mapsto u]$ 满足前置上下文,则 $\rho' \models (\Gamma, x:A, y:B)$。
\end{itemize}

\item \textbf{应用归纳假设} \\
对消解体 $t$ 应用归纳假设:\\
\[
\llbracket t \rrbracket_{\rho'} \in \llbracket C(\langle x, y \rangle) \rrbracket_{\rho'}
\] \\
根据项的语义解释定义:\\
\[
\llbracket \textsf{split}(p, x.y.t) \rrbracket_\rho = \llbracket t \rrbracket_{\rho[x \mapsto \pi_1(\llbracket p \rrbracket_\rho), y \mapsto \pi_2(\llbracket p \rrbracket_\rho)]}
\] \\
代入 $\llbracket p \rrbracket_\rho$ 的分量 $u$ 和 $v$,得:\\
\[
\llbracket \textsf{split}(p, x.y.t) \rrbracket_\rho = \llbracket t \rrbracket_{\rho'}
\]

\item \textbf{类型的一致性 (Conversion)} \\
为了完成证明,必须确保结果所属的集合一致。\\
根据依存类型的代换性质:\\
\[
\llbracket C(\langle x, y \rangle) \rrbracket_{\rho'} = \llbracket C \rrbracket_{\rho[p' \mapsto (u, v)]}
\] \\
由于 $\llbracket p \rrbracket_\rho \simeq (u, v)$(在可还原性等价意义下),且 $\textsf{Red}$ 集合对计算等价性封闭:\\
\[
\llbracket C \rrbracket_{\rho[p' \mapsto \llbracket p \rrbracket_\rho]} = \llbracket C(p) \rrbracket_\rho
\] \\
因此,$\llbracket t \rrbracket_{\rho'} \in \llbracket C(p) \rrbracket_\rho$,结论成立。
\end{enumerate}

\textbf{F. 关于 $\iota$-规约的语义保值性 (Reduction Invariance)}

为了支撑上述步骤,我们需要证明 $\iota$-规约步不改变语义属性。对于 $\Sigma$ 类型:
\[
\textsf{split}(\langle u, v \rangle, x.y.t) \to_\iota t[u/x, v/y]
\]
\begin{enumerate}
\item \textbf{语义一致性}:根据定义,左侧的解释 $\llbracket \textsf{split}(\langle u, v \rangle, x.y.t) \rrbracket_\rho$ 会展开为 $\llbracket t \rrbracket_{\rho[x \mapsto \llbracket u \rrbracket_\rho, y \mapsto \llbracket v \rrbracket_\rho]}$。

\item \textbf{替换引理}:根据替换引理(Lemma \ref{lemma:KOSsubstitution}),右侧项的解释 $\llbracket t[u/x, v/y] \rrbracket_\rho$ 与上述展开形式在语义上完全一致。

\item \textbf{结论}:由于两者的语义解释在集合论意义下是同一个元素,且 $\textsf{Red}_C$ 满足 CR 2(稳定性),规约后的项依然保持在对应的可还原集合中。
\end{enumerate}

\textbf{F. $+$ 类型引入 (注入)}

假设推导最后一步为左注入规则(右注入 $\textsf{inr}$ 同理):

$$\frac{\Gamma \vdash a : A \quad \Gamma \vdash B : \mathcal{U}}{\Gamma \vdash \textsf{inl}_B(a) : A + B}$$

\begin{itemize}
    \item \textbf{证明}:由归纳假设,$\llbracket a \rrbracket_\rho \in \llbracket A \rrbracket_\rho$。
    \item 根据和类型的语义定义:$\llbracket A + B \rrbracket_\rho = \{ \textsf{inl}(u) \mid u \in \llbracket A \rrbracket_\rho \} \cup \{ \textsf{inr}(v) \mid v \in \llbracket B \rrbracket_\rho \} \cup \textsf{Neutral}$。
    \item 由于 $\llbracket \textsf{inl}_B(a) \rrbracket_\rho = \textsf{inl}(\llbracket a \rrbracket_\rho)$,且已知 $\llbracket a \rrbracket_\rho \in \llbracket A \rrbracket_\rho$。
    \item 按照集合构造,$\textsf{inl}(\llbracket a \rrbracket_\rho)$ 显然属于 $\llbracket A + B \rrbracket_\rho$ 的左分支定义部分。
\end{itemize}

latex\textbf{G. 和类型消去 ($+$-消解 / case)}

假设推导的最后一步应用了 $+$ 消解规则:
\[
\frac{\Gamma \vdash s : A + B \quad \Gamma, x:A \vdash t : C \quad \Gamma, y:B \vdash u : C}{\Gamma \vdash \textsf{case}(s, x.t, y.u) : C}
\]
我们需要证明:若 $\rho \models \Gamma$,则 $\llbracket \textsf{case}(s, x.t, y.u) \rrbracket_\rho \in \llbracket C \rrbracket_\rho$。

\begin{enumerate}
\item \textbf{分支前提分析} \\
根据归纳假设(IH):\\
\begin{itemize}
\item 对于判定项 $s$,有 $\llbracket s \rrbracket_\rho \in \llbracket A + B \rrbracket_\rho$。
\item 根据 $A+B$ 的语义定义,$\llbracket s \rrbracket_\rho$ 必须强规范化(SN),且最终归约为 $\textsf{inl}(a)$ 或 $\textsf{inr}(b)$ 形式。
\end{itemize}

\item \textbf{分支讨论 (Case Analysis)} \\
我们需要分两种语义路径讨论:\\
\begin{itemize}
\item \textbf{路径一:左注入 ($\textsf{inl}$)} \\
  \begin{enumerate}
  \item 假设 $\llbracket s \rrbracket_\rho \twoheadrightarrow \textsf{inl}(a)$,由定义知 $a \in \llbracket A \rrbracket_\rho$。
  \item 构造赋值 $\rho_x = \rho[x \mapsto a]$。由于 $a \in \llbracket A \rrbracket_\rho$,则 $\rho_x \models (\Gamma, x:A)$。
  \item 对左分支项 $t$ 应用归纳假设:$\llbracket t \rrbracket_{\rho_x} \in \llbracket C \rrbracket_{\rho_x}$。
  \item 由于 $C$ 在此规则中不依赖于 $s$ 的具体值(简单消解情形),$\llbracket C \rrbracket_{\rho_x} = \llbracket C \rrbracket_\rho$。
  \end{enumerate}

\item \textbf{路径二:右注入 ($\textsf{inr}$)} \\
  \begin{enumerate}
  \item 假设 $\llbracket s \rrbracket_\rho \twoheadrightarrow \textsf{inr}(b)$,由定义知 $b \in \llbracket B \rrbracket_\rho$。
  \item 构造赋值 $\rho_y = \rho[y \mapsto b]$。则 $\rho_y \models (\Gamma, y:B)$。
  \item 对右分支项 $u$ 应用归纳假设:$\llbracket u \rrbracket_{\rho_y} \in \llbracket C \rrbracket_{\rho_y}$。
  \item 同理,$\llbracket u \rrbracket_{\rho_y} \in \llbracket C \rrbracket_\rho$。
  \end{enumerate}
\end{itemize}

\item \textbf{语义解释的统一} \\
根据 $\textsf{case}$ 算子的语义定义:
\[
\llbracket \textsf{case}(s, x.t, y.u) \rrbracket_\rho =
\begin{cases}
\llbracket t \rrbracket_{\rho[x \mapsto a]} & \text{if } \llbracket s \rrbracket_\rho \twoheadrightarrow \textsf{inl}(a) \\
\llbracket u \rrbracket_{\rho[y \mapsto b]} & \text{if } \llbracket s \rrbracket_\rho \twoheadrightarrow \textsf{inr}(b)
\end{cases}
\]
无论 $\llbracket s \rrbracket_\rho$ 塌陷到哪个分支,结果都属于 $\llbracket C \rrbracket_\rho$。

\item \textbf{逆向归约封闭性 (CR 3 应用)} \\
由于 $\textsf{case}(s, x.t, y.u)$ 是通过规约($\iota$-reduction)到达 $\llbracket t \rrbracket_{\rho_x}$ 或 $\llbracket u \rrbracket_{\rho_y}$ 的,根据可还原集合的 CR 3 属性(以及对逆向规约的封闭性),原始的 $\textsf{case}$ 项本身也必然属于可还原集合 $\llbracket C \rrbracket_\rho$。结论成立。
\end{enumerate}

\end{proof}

可靠性意味着“凡是能证明的,都是真的”。在定理\ref{Theorem:KosSoundness}中,“能证明的”是类型判断 $\Gamma \vdash t : A$,“真的”是项在语义模型中的归属 $\llbracket t \rrbracket \in \llbracket A \rrbracket$。它保证了 KOS-TL Core 的语法构造没有脱离其逻辑语义。

\begin{Theorem}{一致性定理(Consistency Theorem)}

不存在项 $t$ 使得 $\emptyset \vdash t : \bot$。则称系统为一致的。
\end{Theorem}

\begin{proof}

该证明的核心思想是:语法推导不能逃脱语义边界。我们分三个阶段展开论证。

\textbf{第一阶段:利用强范型(Syntactic Normalization)}

根据 KOS-TL Core 的强规范化定理,如果存在一个项 $t$ 满足 $\emptyset \vdash t : \bot$,则 $t$ 必然可以规约到一个范式(Normal Form) $t_{nf}$,且类型保持不变:
\[
\emptyset \vdash t_{nf} : \bot
\]
在空上下文中,范式只能是构造项(Constructor)。然而,根据 $\bot$ 类型的定义,它没有引入规则(Introduction Rules),即不存在任何构造算子能产生 $\bot$。 这意味着在语法层面上,$t_{nf}$ 是不存在的。

\textbf{第二阶段:语义可靠性映射(Semantic Soundness)}

为了使论证在数学上无懈可击,我们使用解释模型 $\mathcal{M}$。

\textbf{步骤 1:建立映射}\\
解释函数 $\llbracket \cdot \rrbracket$ 将语法世界(Types/Terms)映射到语义世界(Sets/Elements)。

\begin{itemize}
\item 对于任何类型 $A$,其解释 $\llbracket A \rrbracket$ 是一个集合。
\item 对于任何项 $t : A$,其解释 $\llbracket t \rrbracket$ 必须是集合 $\llbracket A \rrbracket$ 中的一个元素。
\end{itemize}

\textbf{步骤 2:空类型的特殊性}

在定义 $\bot$ 的语义时,我们将其映射为数学上的绝对空集:
\[
\llbracket \bot \rrbracket = \emptyset
\]
这是合理的,因为逻辑上的“伪”在模型论中对应的就是没有任何见证者(Witness)的状态。

\textbf{步骤 3:应用语义可靠性}

根据语义可靠性定理\ref{Theorem:KosSoundness}:
\[
\text{若 } \Gamma \vdash t : A, \text{ 则对所有合法赋值 } \rho, \llbracket t \rrbracket_\rho \in \llbracket A \rrbracket
\]
在空上下文 $\emptyset$ 下,不需要任何赋值 $\rho$,直接得到:
\[
\llbracket t \rrbracket \in \llbracket \bot \rrbracket
\]

\textbf{第三阶段:归谬与矛盾(Reduction to Absurdity)}

\begin{enumerate}[label=(\arabic*)]
\item 由上述推导得:$\llbracket t \rrbracket \in \emptyset$。
\item 根据集合论的外延公理与空集公理:$\forall x, x \notin \emptyset$。
\item 判定:$\llbracket t \rrbracket \in \emptyset$ 与 $\forall x, x \notin \emptyset$ 构成了直接的逻辑矛盾。
\item 回溯:由于语义解释和集合论公理是预设正确的,矛盾的源头只能是假设“存在项 $t$”。
\end{enumerate}

结论:假设不成立,$\neg \exists t, \emptyset \vdash t : \bot$。一致性得证。
\end{proof}

在 KOS-TL Core 这种基于柯里-霍华德同构的系统中,逻辑一致性等价于证明空类型(Empty Type)是不可居住的(Uninhabited)。

推论:逻辑一致性 (Consistency)

由于 KOS-TL 核心层满足强规范化(SN)且具有类型保持性(Subject Reduction),且系统中不存在空类型 $\bot$ 的构造子,因此不存在项 $t$ 使得 $\vdash t : \bot$。这证明了核心层作为“形式宪法”在逻辑上是无矛盾的。

在类型论的严谨定义下,KOS-TL Core 层是绝对可判定的(Decidable)。Core 层作为整个系统的数学基石,必须在理论上保证所有基本操作(类型检查、等价判定)在任何情况下都能在有限步内停机。

\begin{Definition}{合流性(Confluence)}

对于任意的 Core 层项 $M, N, P$,如果存在规约路径使得 $M \twoheadrightarrow N$ 且 $M \twoheadrightarrow P$,则必然存在一个项 $Q$,使得 $N \twoheadrightarrow Q$ 且 $P \twoheadrightarrow Q$。

\end{Definition}

设 $\to$ 为单步规约关系(包含 $\beta, \delta, \zeta, \eta$ 规约),$\twoheadrightarrow$ 为其传递闭包(多步规约)。合流性在逻辑上意味着所有的分叉最终都会汇合。

\begin{Theorem}{KOS-TL Core 合流性定理}

KOS-TL Core 层的所有良构项均满足合流性,且具有唯一的范式(Unique Normal Form)。
\end{Theorem}
\begin{proof}

我们采用 Tait-Martin-L\"{o}f 平行规约法 (Parallel Reduction) 结合强规范化性质进行证明。

\textbf{步骤 A:定义平行规约 ($\Rightarrow$)}

为了处理单步规约无法覆盖的“分叉同时规约”问题,我们定义一种平行规约关系 $\Rightarrow$,它是单步规约 $\rightarrow$ 的推广,允许同时规约项的多个子部分。具体地,$\Rightarrow$ 是最小关系,满足以下规则:
\begin{enumerate}
\item 若 $M \rightarrow M'$(单步 $\beta$-规约),则 $M \Rightarrow M'$。
\item 对于任意规约上下文 $C[\cdot]$,若 $M \Rightarrow M'$,则 $C[M] \Rightarrow C[M']$。
\item 并行应用规约:
$$\frac{M \Rightarrow M' \quad N \Rightarrow N'}{(\lambda x. M) N \Rightarrow M'[x := N']}$$
其中 $[x := N']$ 表示捕获避免的替换。
\item 对于构造子(如 $\Pi x : A. M$),允许同时规约域和体:
$$\frac{A \Rightarrow A' \quad M \Rightarrow M'}{\Pi x : A. M \Rightarrow \Pi x : A'. M'}$$
($\Sigma$ 和 $\mathsf{Id}$ 类型类似)。
\end{enumerate}

\textbf{步骤 B:证明钻石性质 (Diamond Property)}

我们证明平行规约 $\Rightarrow$ 满足钻石性质:若 $M \Rightarrow N$ 且 $M \Rightarrow P$,则存在 $Q$ 使得 $N \Rightarrow Q$ 且 $P \Rightarrow Q$。
证明通过对 $M$ 的结构进行归纳:
\begin{itemize}
\item \textbf{基础情形}:$M$ 是变量或原子,则 $N = P = M$,取 $Q = M$。
\item \textbf{归纳情形}:假设对所有真子项成立。

\textbf{情形:$M = (\lambda x. M_1) M_2$}:由 $\Rightarrow$ 定义,$N$ 和 $P$ 源于不同子规约分叉。归纳假设应用于 $M_1$ 和 $M_2$,存在 $Q_1, Q_2$ 使得子项汇合,则 $Q = Q_1[x := Q_2]$。
\textbf{情形:构造子(如 $\Pi x : A. M$)}:KOS-TL Core 的构造子正交(无重叠规约规则),归纳应用于 $A$ 和 $M$,汇合为 $\Pi x : A''. M''$。$\mathsf{Id}$ 类型类似,无内部冲突。
\end{itemize}
由正交性和类型守恒,规约无批判对,确保钻石性质。

\textbf{步骤 C:从平行规约推导到多步规约}

多步规约 $\twoheadrightarrow$ 是 $\Rightarrow$ 的反射-传递闭包:$M \twoheadrightarrow N$ 当且仅当存在链 $M = M_0 \Rightarrow M_1 \Rightarrow \cdots \Rightarrow M_k = N$。
由 Newman 引理,若 $\Rightarrow$ 满足钻石性质且系统强规范化(SN,无无限链),则 $\twoheadrightarrow$ 满足合流性:对于 $M \twoheadrightarrow N_1$ 和 $M \twoheadrightarrow N_2$,存在 $Q$ 使得 $N_1 \twoheadrightarrow Q$ 和 $N_2 \twoheadrightarrow Q$。SN 确保弱规范化,钻石蕴涵局部合流。
\textbf{步骤 D:唯一范式证明}
假设 $M$ 有两个范式 $N_1$ 和 $N_2$(不可再规约)。由合流性,存在 $Q$ 使得 $N_1 \twoheadrightarrow Q$ 和 $N_2 \twoheadrightarrow Q$。 但 $N_1, N_2$ 是范式,故 $N_1 = Q = N_2$。由 SN,每项有范式,故唯一。
\end{proof}

KOS-TL Core 的判定性主要由以下两个性质支撑:
\begin{enumerate}
\item 类型检查的可判定性:给定 $\Gamma, t, A$,判定 $\Gamma \vdash t : A$ 是否成立。
\item 等价判定的可判定性:给定 $\Gamma, t, u$,判定 $t \equiv u$(在当前上下文下两项是否逻辑等价)是否成立。
\end{enumerate}

Core 层作为整个系统的数学基石,必须在理论上保证所有基本操作(类型检查、等价判定)在任何情况下都能在有限步内停机。

\begin{Theorem}{Core 层可判定性定理}

KOS-TL Core 语言的类型检查问题和项的等价判定问题是可判定的。
\end{Theorem}

\begin{proof}

该证明建立在强规范化(Strong Normalization, SN)和合流性(Confluence)的基础之上。

\textbf{1. 归约序列的有限性}

根据 Tait-Girard 方法证明的强规范化定理,KOS-TL Core 中的任何良构项 $t$ 不存在无限长的规约序列。这意味着从任何项开始,通过 $\beta, \delta, \zeta$ 等规约步,必然在有限步内达到唯一的范式 $\textsf{nf}(t)$。

\textbf{2. 等价判定的算法化}

判定 $t \equiv u$ 的过程可以转化为:
\begin{enumerate}
\item 将 $t$ 规约至范式 $t^*$。
\item 将 $u$ 规约至范式 $u^*$。
\item 比较 $t^*$ 和 $u^*$ 是否在字面上(Syntactically)完全一致。
\end{enumerate}
由于步骤 1 和 2 保证在有限时间内完成,且步骤 3 是简单的符号匹配,因此 $t \equiv u$ 是可判定的。

\textbf{3. 类型检查算法的递归停机}

类型检查器在处理项 $t$ 时,按照其语法结构进行递归:
\begin{itemize}
\item 对于应用项 $(f \, a)$,检查 $f$ 的类型是否为 $\Pi$-类型,并判定 $a$ 的类型是否匹配。
\item 类型匹配过程中调用上述的等价判定算法。
\end{itemize}

由于项的构造是有限的,且等价判定是可判定的,整个递归过程必然停机。
\end{proof}



\section{应用示例:质量异常知识建模}

在制造业场景中,一个“合格批次”不仅仅是一个数据记录,它必须包含质检通过的证据。

\begin{itemize}
\item \textbf{类型定义}:

$$\textsf{QualifiedBatch} \equiv \Sigma(b : \textsf{BatchID}). \Sigma(res : \textsf{Result}). \textsf{Id}_{\textsf{Result}}(res, \textsf{Pass})$$

\item \textbf{逻辑释义}:
该类型要求任何实例必须包含一个批次 ID、一个质检结果,以及一个证明该结果等于  的恒等项。
\item \textbf{构造尝试}:
若批次  的质检结果为 ,则由于类型  是空的(无构造子),系统在核心层将拒绝实例化该对象,从而在逻辑层阻止了不合格品进入后续流程。
\end{itemize}


\begin{Example}{构造一个在安全范围内的温度读数}

(1)声明原子谓词(作为公理或基本判定)

在 Core 层的初始化上下文中,我们需要引入量纲判定谓词(is\_unit):

$$\Gamma \vdash \textsf{is\_unit\_Celsius} : \Pi(v: \textsf{Val}). \textsf{Prop}$$

此外,引入比较谓词构造 $\textsf{is\_safe}$

$$\Gamma \vdash L, H : \textsf{Val}$$$$\Gamma \vdash \textsf{is\_safe} \equiv \lambda v. \textsf{And}(L \le v, v \le H) : \textsf{Val} \to \textsf{Prop}$$

(2)构造 $Temp$ 类型

利用 $\Sigma$-类型构造规则:

$$\frac{\Gamma \vdash \textsf{Val} : \mathcal{U} \quad \Gamma, v:\textsf{Val} \vdash \textsf{is\_unit\_Celsius}(v) : \textsf{Prop}}{\Gamma \vdash \Sigma(v: \textsf{Val}). \textsf{is\_unit\_Celsius}(v) : \mathcal{U}}$$

此时,$Temp \equiv \Sigma(v: \textsf{Val}). \textsf{is\_unit\_Celsius}(v)$ 正式成为一个合法的类型。

(3)在此基础上构造 $QualifiedTemp$

 现在,我们再叠加逻辑安全谓词 $\textsf{is\_safe}$:

 $$QualifiedTemp \equiv \Sigma(t: Temp). \textsf{is\_safe}(\textsf{proj}_1(t))$$

(4)构造对象实例

  假设 $v = 25$:$p_{unit} : \textsf{is\_unit\_Celsius}(25)$$p_{range} : \textsf{is\_safe}(25)$$obj = \langle \langle 25, p_{unit} \rangle, p_{range} \rangle$

\end{Example}

\chapter{KOS-TL 内核层 (Kernel Layer):状态演化与操作语义}

内核层(Kernel Layer)是 KOS-TL 的动力学核心。它基于核心层的静态类型系统,引入了时间维度与状态转移机制,通过受控的事件驱动实现知识库从状态  向  的原子演化。

\section{语法 (Syntax)}

KOS-TL内核层引入了“动态”的概念,论域从静态对象扩展到了状态转移轨迹。它是连接逻辑与执行的桥梁。

KOS-TL内核层可以表征为一个三元组:

$$\langle \Sigma, \textsf{Ev}, \Delta \rangle$$

\begin{Definition}{状态 ($\Sigma$)}

    内核层状态($\Sigma$)定义为一个配置三元组:

    $$\Sigma \equiv \langle \mathcal{K}, \mathcal{TS}, \mathcal{P} \rangle$$

\begin{enumerate}[label=(\arabic*)]
    \item 知识库 ($\mathcal{K}$ - Knowledge Base)

    $$\mathcal{K} = \{ (id_i, t_i, A_i) \mid \Gamma_{Core} \vdash t_i : A_i \}$$

    它存储了系统当前所有已证实的真理(Facts)
    \item 逻辑时钟 ($\mathcal{TS}$ - Logical Clock)

    基于 Core 层基础目 $\textsf{Time}$。它不仅仅是一个计数器,而是状态全序关系的度量。

    $$\mathcal{TS} : \textsf{Time} \quad \text{满足单调性规则:} \Sigma \to \Sigma' \implies \mathcal{T}' > \mathcal{T}$$

    \item  挂起队列 ($\mathcal{P}$ - Pending Events)

    一个由受限的事件(Events)组成的有序序列。

\end{enumerate}
\end{Definition}


\begin{Definition}{事件 $\textsf{Ev}$ }

事件 $\textsf{Ev}$ 是一个在全局上下文 $\Gamma$ 下良构的五元组:

$$\textsf{Ev} \equiv \langle \textsf{Args}, \textsf{Pre}, \textsf{Op}, \textsf{Post}, \textsf{Prf} \rangle$$

其中各组件的类型限定与逻辑语义为:
\begin{enumerate}[label=(\arabic*)]
    \item $\textsf{Args}$ (参数集):

        $$\textsf{Args} : A$$(其中 $A \in \mathcal{U}_{Core}$)。

        从外部世界(Runtime 层)摄入的实例化数据,如 sensor\_value 或 transaction\_amount。
    \item $\textsf{Pre}$ (前置条件谓词):

        $$\textsf{Pre} : \textsf{Args} \to \Sigma \to \textsf{Prop}$$

        一个依存命题,定义了该事件在当前状态 $\Sigma$ 下必须满足的逻辑前提。它可以引用当前知识库 $\mathcal{K}$ 或逻辑时间 $\mathcal{TS}$。
    \item $\textsf{Op}$ (操作算子):

        $$\textsf{Op} : \textsf{Args} \to \Sigma \to \Sigma$$

        核心的演化函数。它描述了如何生成新状态 $\Sigma'$。
        \begin{itemize}
            \item $\mathcal{K} \to \mathcal{K}'$:知识项的增减。
            \item $\mathcal{TS} \to \mathcal{TS} + \Delta t$:逻辑时钟的步进。
            \item $\mathcal{P} \to \mathcal{P}'$:挂起意图队列的更新(消耗自身,或派生新意图)。
        \end{itemize}

    \item $\textsf{Post}$ (后置约束/不变式):

    $$\textsf{Post} : \Sigma' \to \textsf{Prop}$$

定义转换后必须满足的安全准则(Safety Properties),如“能量守恒”、“账户非负”或“时钟单调性”。

    \item $\textsf{Prf}$ (证明项):

$$\textsf{Prf} : \textsf{Pre}(\textsf{Args}, \Sigma)$$

这是事件的“通行证”。在 Runtime 层将信号精化为事件时,必须构造出前置条件成立的构造性证明。如果没有有效的 $\textsf{Prf}$,内核将拒绝执行该事件。

\end{enumerate}
\end{Definition}


\begin{Definition}{转移记录 ($\Delta$) }

为了支撑因果追溯,$\Delta$ 需要记录时钟的跳变:

$$\Delta \subseteq \Sigma \times \textsf{Ev} \times \Sigma$$

一次典型的转移记录项:

$$\delta = \langle \langle \mathcal{K}, \mathcal{TS}, \mathcal{P} \rangle \xrightarrow{e} \langle \mathcal{K}', \mathcal{TS}', \mathcal{P}' \rangle \rangle$$

$\forall \langle \Sigma \xrightarrow{e} \Sigma' \rangle \in \Delta, \quad \Sigma'.\mathcal{T} > \Sigma.\mathcal{T}$。
这在理论上锁定了“时间箭头”,保证了知识演化的不可逆性。

每一个新注入 $\mathcal{K}'$ 的知识项 $ku_{new}$ 都会隐式携带当前的 $\mathcal{T}'$。

转移发生时,$e$ 从 $\mathcal{P}$ 中出队,执行 $\textsf{Op}$ 后,其结果合并入 $\mathcal{K}'$。
\end{Definition}

\begin{Definition}{演化确定性 (Evolutionary Determinism)}

给定状态 $\Sigma$ 与事件 $e$,若存在满足操作语义的 $\Sigma'$,则其范式(Normal Form)在内涵等价意义下是唯一的。
\end{Definition}

\section{操作语义}

内核层的演化遵循“Small-step 操作语义”。令$\Sigma$为系统的配置(Configuration),其状态转移规则定义为:

\begin{equation} \frac{ e = \textsf{head}(\Sigma.\mathcal{P}) \quad \Gamma, \Sigma.\mathcal{K}, \Sigma.\mathcal{TS} \vdash p : \textsf{Pre}(\textsf{Args}_e, \Sigma) \quad \Sigma' = \textsf{Op}(\textsf{Args}_e, \Sigma) \quad \Sigma' \vdash p' : \textsf{Post}(\textsf{Args}_e, \Sigma') }{ \langle \Sigma, e \rangle \longrightarrow_{KOS} \Sigma' }
\end{equation}

其中:
\begin{itemize}
    \item 意图触发条件 ($e = \textsf{head}(\Sigma.\mathcal{P})$)

    明确了事件 $e$ 的来源。转移不是随机发生的,而是由挂起队列 $\mathcal{P}$ 的头部事件驱动。保证了演化的序属性,即内核按照意图队列的逻辑顺序进行调度。

    \item 环境感知的证明判定 ($\Gamma, \Sigma.\mathcal{K}, \Sigma.\mathcal{T} \vdash p$)

    显式列出了证明项 $p$ 所依赖的上下文。前置条件的证明不仅依赖全局上下文 $\Gamma$,还必须与当前知识库 $\mathcal{K}$ 中的事实以及逻辑时间 $\mathcal{TS}$ 保持一致。这落实了“只有在正确的时间和事实基础上,事件才能发生”的逻辑。

    \item 参数化算子应用 ($\Sigma' = \textsf{Op}(\textsf{Args}_e, \Sigma)$)

    引入了 $\textsf{Args}_e$。强调了转移是基于事件携带的具体参数(来自 Runtime 层的精化)对当前三元组配置的整体变换。

    \item 后置约束的自洽性 ($\Sigma' \vdash p' : \textsf{Post}(\textsf{Args}_e, \Sigma')$)

    明确了 $\textsf{Post}$ 是在新状态 $\Sigma'$ 下进行的判定。这定义了逻辑提交(Commit)的硬门槛。如果演化后的状态无法满足其安全不变式(例如:时钟没有步进,或者知识库出现了不一致),则该判定式不成立,转移规则失效。
\end{itemize}

该语义规定了一次有效的知识转移必须同时满足“前提可证”与“结果合规”。若任何一个条件无法在核心层得到证明,状态将保持不变(即执行回滚)。

内核层不负责处理物理失败的重试策略,它仅定义了“逻辑上合法的演化轨迹”。任何未能通过 Post 校验的物理尝试,在 Kernel 层表现为“未发生的转移”,从而在逻辑层强制实现了事务的原子性。

\begin{Example}{规约算例展示}
\label{exam:KOSKernel-reduction}

 假定如下的场景。传感器数据融合假设系统中有两个独立的传感器 $ku_1$(温度)和 $ku_2$(湿度)。我们需要一个合并函数 combine,将它们封装成一个“环境状态”对象。

\begin{enumerate}

    \item 基础类型与谓词定义

    目标类型$Env \equiv \Sigma(t:Temp).(Humi)$,环境状态是一个包含温湿度的依存对。其中:

\centerline{$Temp \equiv \Sigma(v:\textsf{Val}). \textsf{is\_T}(v)$}

\centerline{$Humi \equiv \Sigma(v:\textsf{Val}). \textsf{is\_H}(v)$}

    \item 具体的知识项(实例)
[[
$ku_1 = \langle 25, p_T \rangle : Temp$

$ku_2 = \langle 60, p_H \rangle : Humi$
]]

    \item 合并函数($\Pi$-类型)

    定义一个接收温度和湿度,并返回环境对象的函数:

$$\textsf{combine} \equiv \lambda t:Temp. \lambda h:Humi. \langle t, h \rangle$$

    其类型为$\Pi(t:Temp). \Pi(h:Humi). Env$。
\end{enumerate}

现在我们展示将 $\textsf{combine}$ 应用于 $ku_1$ 和 $ku_2$ 的规约过程。这通常发生在 Kernel 接收到两个信号并试图更新全局状态时。

步骤 1:构造初始应用项在 Kernel 的控制流中,产生了一个待规约的项:

$$(\textsf{combine} \,\, ku_1) \,\, ku_2$$

步骤 2:第一次 $\beta$-规约(替换温度)

根据 $\beta$-规约规则 $(\lambda x. M) N \to M[N/x]$:

$$(\lambda t. \lambda h. \langle t, h \rangle) \,\, ku_1 \to \lambda h. \langle ku_1, h \rangle$$

函数“吃掉”了温度数据,变成了一个“等待湿度输入”的特化函数。

步骤 3:第二次 $\beta$-规约(替换湿度)

$$(\lambda h. \langle ku_1, h \rangle) \,\, ku_2 \to \langle ku_1, ku_2 \rangle$$

湿度数据被填入,生成了一个完整的配对。

步骤 4:展开与结构归约($\iota$-规约)

如果系统需要进一步提取其中的原始数值(例如为了执行 analyze),则会发生 $\iota$- 规约:

$$\textsf{proj}_1(\langle ku_1, ku_2 \rangle) \to ku_1 = \langle 25, p_T \rangle$$

$$\textsf{proj}_1(\textsf{proj}_1(\langle ku_1, ku_2 \rangle)) \to 25$$

上述规约过程都伴随类型判定过程,根据类型保持性 (Subject Reduction),整个规约过程中的每一项都必须是良构的:

起始项:$(\textsf{combine} \,\, ku_1) \,\, ku_2$ 具有类型 $Env$。

中间项:$\lambda h. \langle ku_1, h \rangle$ 具有类型 $\Pi(h:Humi). Env$。

最终项:$\langle ku_1, ku_2 \rangle$ 具有类型 $Env$。

在这个算例中,规约操作完成了从 “逻辑意图”(如何合并数据)到 “逻辑事实”(已合并的数据对象)的转化。在 Core 层看来,$(\textsf{combine} \,\, ku_1) \,\, ku_2$ 与 $\langle ku_1, ku_2 \rangle$ 是判断等价(Judgmentally Equal)的。它们是同一个真理的不同表达形式。在 Kernel 层看来,规约是一次计算求值。它消耗了 CPU 周期,将两个分散的内存指针合并到了一个新的 $\Sigma$ 结构体中。
\end{Example}

KOS-TL 通过静态类型语义在逻辑执行之前构建“防火墙”,类型不匹配拦截 (Type Mismatch Interception)发生在规约之前。根据核心层的判定规则,如果项(Term)无法通过类型检查,它就永远不会被推入 Kernel 的规约引擎。

假设我们有合并函数 combine,它期待一个湿度对象 $Humi$:$$\textsf{combine} : \Pi(t:Temp). \Pi(h:Humi). Env$$现在,Runtime 层错误地捕获了一个压力信号 $p : Press$,并试图执行合并:

$$(\textsf{combine} \,\, ku_1) \,\, p$$

Kernel 调用 $\Pi$-消解规则(Application Rule):

$$\frac{\Gamma \vdash f : \Pi(h:Humi).Env \quad \Gamma \vdash p : A}{\Gamma \vdash f\,p : Env [p/h] \quad (\text{要求 } A \equiv Humi)}$$

内核尝试判定 $Press \equiv Humi$。

$Humi \equiv \Sigma(v:\textsf{Val}). \textsf{is\_H}(v)$

$Press \equiv \Sigma(v:\textsf{Val}). \textsf{is\_P}(v)$

由于谓词 $\textsf{is\_H} \neq \textsf{is\_P}$,类型合一(Unification)失败。

因此,项 $(\textsf{combine} \,\, ku_1) \,\, p$ 被判定为非良构(Ill-typed)。规约引擎拒绝执行 $\beta$-规约,系统状态 $\sigma$ 保持不变,同时触发一个类型错误异常。

\begin{Example}{因果回溯分析 (Causal Backtracking Analysis)}

    基于例\ref{exam:KOSKernel-reduction},当系统发现合并后的结果虽然“类型正确”,但“逻辑异常”(例如 $Env$ 的值超出了安全范围)时,就需要利用 Id 类型(等价类型) 进行因果回溯。

    假设我们已经得到了合并对象 $obj = \langle ku_1, ku_2 \rangle$,但 analyze 谓词判定其非法。

    回溯过程遵循以下逻辑归约:

\begin{enumerate}[label=(\arabic*)]

    \item 解构对象 (Deconstruction)

    通过 Core 层的投影算子 $\textsf{proj}_i$,将复合对象拆解回原始证据:

$$t = \textsf{proj}_1(obj) \quad h = \textsf{proj}_2(obj)$$

    \item 等价性验证 (Identity Verification)

    内核构造一个等价性声明,要求证明当前数据与输入源是一致的:

$$\textsf{Id}_{Temp}(t, ku_1) \wedge \textsf{Id}_{Humi}(h, ku_2)$$

    如果 refl(自反性证明)无法在此处构造,说明合并过程中发生了计算错误或内存污染。

    \item 定位根因 (Root Cause Localization)

    回溯分析函数 analyze 会沿着规约步逆向搜索。在 KOS-TL 中,这表现为对证明项的检查:检查 $ku_1$ 的右投影 $\textsf{proj}_2(ku_1)$,即温度安全证明 $p_T$。 若 $p_T$ 校验失败,则判定:根因在于传感器 1 的输入数据。若 $p_T$ 校验通过,则判定:根因在于合并函数 combine 的逻辑演算。
\end{enumerate}
\end{Example}

\section{状态 $\sigma$ 的全面描述与形式化框架}

在 KOS-TL 中,状态($\sigma$)不仅是系统在某一时刻的知识快照,更是决定系统行为边界与可执行路径的逻辑实体。它承载了从认识论到操作语义的完整映射。

\subsection{状态 $\sigma$ 的本质与特性}

\subsubsection{逻辑快照与轨迹依赖}
$\sigma$ 代表了系统在某一逻辑时刻的知识、配置与约束。在 KOS-TL 中,状态并非独立存在的全局变量,而是\textbf{轨迹(Trace)的函数}:
\begin{itemize}
    \item \textbf{非全局唯一性:} $\sigma$ 总是与特定的轨迹 $\pi$ 相关联。不同的演化路径、规则或事件序列,都会导致不同的 $\sigma$。
    \item \textbf{动态演化:} 状态的迁徙必然伴随着一个合法的执行步骤(Step),每一步演化都是对前一状态的逻辑接续与精化。
\end{itemize}

\subsection{状态 $\sigma$ 的形式化定义}

\subsubsection{状态的最小记录类型(Record Type)}
参考直觉主义类型论,状态 $\sigma$ 可以形式化地定义为一个包含知识、规则、配置与依赖的记录类型:
\begin{equation}
\mathsf{State} :=
\begin{cases}
\mathcal{K} : \mathsf{Set \ Knowledge} & \text{当前生效的知识集合} \\
\mathcal{R} : \mathsf{Set \ Rule} & \text{当前生效的推理与演化规则} \\
\mathcal{C} : \mathsf{Config} & \text{系统物理/逻辑配置参数} \\
\mathcal{D} : \mathsf{DependencyGraph} & \text{知识与规则间的因果依赖图}
\end{cases}
\end{equation}

\subsubsection{状态的演化算子}
状态的迁移遵循小步操作语义。给定初始状态 $\sigma_0$ 与事件序列 $e_k$,演化过程表示为:
$$\sigma_0 \xrightarrow{e_1, s_1} \sigma_1 \xrightarrow{e_2, s_2} \cdots \xrightarrow{e_k, s_k} \sigma_k$$
其中每一个 $s_i : \mathsf{Step}(\sigma_{i-1}, e_i, \sigma_i)$ 提供了状态转移的合法性证明。

\subsection{状态与轨迹的交互逻辑}

\subsubsection{投影关系:$\mathsf{StateOf}$}
状态是轨迹的“现时投影”。定义函数 $\mathsf{StateOf} : \mathsf{Trace} \to \mathsf{State}$,它将一条完整的演化路径映射为其最终达成的逻辑状态。

\subsubsection{轨迹关系对状态的影响}
\begin{itemize}
    \item \textbf{轨迹等价(Trace Equivalence):} 关注结果一致性。
    $$\mathsf{TraceEq}(\pi_1, \pi_2) \equiv \forall \sigma_1 \sigma_2, (\mathsf{StateOf}(\pi_1) = \sigma_1 \land \mathsf{StateOf}(\pi_2) = \sigma_2) \to (\sigma_1 = \sigma_2)$$
    \item \textbf{轨迹包含(Trace Inclusion):} 关注因果继承性。
    $$\mathsf{TraceIn}(\pi_1, \pi_2) \equiv \exists \pi_3, \pi_2 = \mathsf{extend}(\pi_1, e_3, s_3)$$
\end{itemize}

\subsection{非单调环境下的回滚与路径重构}

KOS-TL 的状态管理允许系统在面对冲突或失效时,通过操作因果轨迹实现灵活的回滚:

\begin{enumerate}
    \item \textbf{“走回去”式回滚(Strict Rollback):}
    回归到历史状态 $\sigma$ 并确保后续路径与原轨迹 $\pi$ 保持一致:
    $$\mathsf{RollBack}(\pi, \sigma) \equiv \exists \pi', \mathsf{StateOf}(\pi') = \sigma \land (\pi' \sqsubseteq \pi)$$
    \item \textbf{“换一条路”式回滚(Path Swapping):}
    回归到历史状态 $\sigma$ 但注入新的事件 $e$,从而在逻辑空间中重塑完全不同的演化分支:
    $$\mathsf{SwitchPath}(\pi, \sigma, e, \pi') \equiv \mathsf{StateOf}(\pi) = \sigma \land \mathsf{Step}(\sigma, e, \sigma') \land \pi' = \mathsf{extend}(\pi, e, s)$$
\end{enumerate}

\subsection{量化范式:KOS-TL vs. 时序逻辑}

与 LTL(线性时序逻辑)和 CTL(计算树逻辑)关注全局状态空间的可能路径不同,KOS-TL 的量化是\textbf{局部合规驱动}的:
\begin{itemize}
    \item \textbf{LTL/CTL:} 对全局顺序或全路径属性进行逻辑量化。
    \item \textbf{KOS-TL:} 专注于验证\textbf{当前执行轨迹}的合法性。它确保每一条被选择的路径都具有构造性证明:
    $$\forall \pi \in \mathsf{Trace}, \forall e \in \mathsf{Event}, \mathsf{StateOf}(\pi) = \sigma \to \mathsf{Step}(\sigma, e, \sigma') \to \exists \pi' : \mathsf{Trace}$$
\end{itemize}

\subsection{结论:$\sigma$ 的本体论地位}
在 KOS-TL 中,状态 $\sigma$ 不再是静态的存储单元,而是轨迹的动态函数。其变化不仅反映了数据的更新,更体现了因果链条的延伸与剪枝。通过回滚与路径切换机制,$\sigma$ 为系统提供了应对非单调环境的逻辑韧性。


\section{轨迹与路径在 KOS-TL 中的形式化描述}

本节将 KOS-TL 中的“因果演化”收紧为严格的形式系统。轨迹(Trace)在 KOS-TL 中并非简单的状态序列,而是带有标签的小步执行证明链。

\subsection{轨迹的形式化定义}

\subsubsection{基本定义:作为依赖和类型的轨迹}
轨迹不是状态的集合,而是由事件驱动的、步步经过验证的迁移序列:
$$\mathsf{Trace} \;=\; \sigma_0 \;\xrightarrow{e_1}\; \sigma_1 \;\xrightarrow{e_2}\; \cdots \;\xrightarrow{e_n}\; \sigma_n$$

在直觉主义类型论(ITT)的框架下,轨迹被定义为一个高阶的依赖积类型($\Sigma + \Pi$):
$$\mathsf{Trace}(\sigma_0, \sigma_n) \;:=\; \Sigma (n : \mathbb{N}), \Sigma (e_1,\dots,e_n), \Sigma (\sigma_1,\dots,\sigma_n), \prod_{i=1}^n \mathsf{Step}(\sigma_{i-1}, e_i, \sigma_i)$$

\textbf{关键洞察:} 轨迹本身是一个构造性对象,它不仅记录了“发生了什么”,还包含了“为什么每一步都是合法”的证明。

\subsubsection{Step 算子的构造语义}
$\mathsf{Step}(\sigma, e, \sigma')$ 不仅仅是一个真值判断,它是一个包含以下要素的可构造对象:
\begin{itemize}
    \item \textbf{前置条件证明:} 证明事件 $e$ 在状态 $\sigma$ 下是满足触发约束的。
    \item \textbf{规则匹配证明:} 证明该迁移符合本体定义的演化逻辑。
    \item \textbf{后态生成函数:} 定义从 $\sigma$ 到 $\sigma'$ 的确定性归约路径。
\end{itemize}

\subsubsection{知识在轨迹中的地位}
知识项 $k$ 在 KOS-TL 中不具备“全局永真性”,其存在性依赖于特定的轨迹前缀和状态:
$$k : \mathsf{Derivable}(\pi_{\le i}, \sigma_i)$$
其中 $\pi_{\le i}$ 代表轨迹的因果历史。这种定义自然导出了非单调性:
$$\mathsf{Derivable}(\pi_1, \sigma_1, k) \;\not\Rightarrow\; \mathsf{Derivable}(\pi_1 \cdot \pi_2, \sigma_2, k)$$
即:旧知识在历史 $\pi_1$ 中成立,但在延伸后的轨迹 $\pi_1 \cdot \pi_2$ 中可能由于前提不可构造而失效。

\subsection{轨迹的代数属性:等价与包含}

\subsubsection{轨迹等价(Trace Equivalence)}
轨迹等价关注的是结果状态的路径无关性:
$$\mathsf{TraceEq}(\pi_1, \pi_2) := \forall \sigma_1 \sigma_2, (\mathsf{StateOf}(\pi_1) = \sigma_1) \to (\mathsf{StateOf}(\pi_2) = \sigma_2) \to (\sigma_1 = \sigma_2)$$
若两条轨迹最终达成相同的逻辑快照,则称其在结果上等价。

\subsubsection{轨迹包含(Trace Inclusion)}
轨迹包含描述了知识历史的继承关系:
$$\mathsf{TraceIn}(\pi_1, \pi_2) := \exists \pi_3, \pi_2 = \mathsf{extend}(\pi_1, e_3, s_3)$$
这意味着 $\pi_1$ 是 $\pi_2$ 的逻辑前缀,$\pi_2$ 完整保留了 $\pi_1$ 的因果证据。

\subsection{回滚机制:逻辑回归与路径切换}

在 KOS-TL 中,回滚被重新定义为对因果链的操作,而非简单的数据恢复。

\begin{enumerate}
    \item \textbf{“走回去”式回滚(Backtracking):} 回滚到历史轨迹中的某一驻点 $\sigma$,并尝试恢复原有路径:
    $$\mathsf{RollBack}(\pi, \sigma) := \exists \pi', \mathsf{StateOf}(\pi') = \sigma \land (\pi' \sqsubseteq \pi)$$
    \item \textbf{“换一条路”式回滚(Path Switching):} 回滚到 $\sigma$ 状态,但选择一个新的事件 $e$ 开启不同的演化分支:
    $$\mathsf{SwitchPath}(\pi, \sigma, e, \pi') := \mathsf{StateOf}(\pi) = \sigma \land \mathsf{Step}(\sigma, e, \sigma') \land \pi' = \mathsf{extend}(\pi, e, s)$$
\end{enumerate}

\subsection{轨迹量化:KOS-TL 与时序逻辑(LTL/CTL)的对比}

传统时序逻辑关注全局路径的属性量化,而 KOS-TL 专注于\textbf{局部执行路径的合法性验证}。

\begin{itemize}
    \item \textbf{LTL/CTL:} 对所有可能路径的全局约束进行量化(如“未来必然发生某事”)。
    \item \textbf{KOS-TL:} 对具体执行路径进行验证。它不问“所有可能的轨迹是什么”,而问“当前这条轨迹是否能合法地延伸”:
    $$\forall \pi, \sigma, e, (\mathsf{StateOf}(\pi) = \sigma \to \mathsf{Step}(\sigma, e, \sigma') \to \exists \pi' : \mathsf{Trace}, \pi' = \mathsf{extend}(\pi, e, s))$$
\end{itemize}

\textbf{总结:} CTL/LTL 是对全局可能路径的\textbf{时序量化};KOS-TL 是对局部执行路径的\textbf{构造验证}。KOS-TL 通过回滚与路径切换,为动态环境下的自律系统提供了比传统时序逻辑更灵活的逻辑支撑。

\section{通用算子}

\subsection{1. 状态投影算子 (State Projection Operators)}

投影算子定义了从复杂依存对(Dependent Pairs)中提取组件的逻辑。

\begin{itemize}

\item \textbf{知识提取算子 ($\textsf{get\_K}$)}
\begin{itemize}
\item \textbf{Core 类型}: $\textsf{get\_K} : \Pi(\sigma : \Sigma). \textsf{Set}(\textsf{Facts})$
\item \textbf{算子定义}: $\textsf{get\_K} \equiv \lambda \sigma. \textsf{proj}_1(\sigma)$
\item \textbf{说明}: 该算子利用第一投影提取知识库 $\mathcal{K}$。在 Core 层中,它确保了返回的集合项皆为良构的类型实例。
\end{itemize}
\item \textbf{时钟读取算子 ($\textsf{now}$)}
\begin{itemize}
    \item \textbf{Core 类型}: $\textsf{now} : \Pi(\sigma : \Sigma). \textsf{Time}$
    \item \textbf{算子定义}: $\textsf{now} \equiv \lambda \sigma. \textsf{proj}_1(\textsf{proj}_2(\sigma))$
    \item \textbf{说明}: 提取三元组中间项 $\mathcal{T}$。该算子是所有时序逻辑判断(如“合同是否过期”)的基础。
\end{itemize}
\end{itemize}

\subsection{2. 意图调度算子 (Intention Scheduling Operators)}

调度算子通过递归列表操作来管理意图队列 $\mathcal{P}$。

\begin{itemize}
\item \textbf{意图压入算子 ($\textsf{schedule}$)}
\begin{itemize}
\item \textbf{Core 类型}: $\textsf{schedule} : \Pi(\sigma : \Sigma). \Pi(e : \textsf{Ev}). \Sigma$
\item \textbf{算子定义}: $\textsf{schedule} \equiv \lambda \sigma. \lambda e. \langle \textsf{get\_K}(\sigma), \textsf{now}(\sigma), \textsf{append}(\textsf{proj}_2(\textsf{proj}_2(\sigma)), e) \rangle$
    \item \textbf{说明}: 该算子构造一个新的 $\Sigma$ 实例。其核心是在队列末尾追加一个符合 $\textsf{Ev}$ 类型限制的五元组。
\end{itemize}\end{itemize}

\subsection{3. 演化控制算子 (Evolution Control Operators)}

这是驱动系统向前演化的核心,涉及计算与一致性判定的融合。
\begin{itemize}
\item \textbf{时钟步进算子 ($\textsf{tick}$)}
\begin{itemize}
\item \textbf{Core 类型}: $\textsf{tick} : \Pi(\sigma : \Sigma). \Sigma$
\item \textbf{算子定义}: $\textsf{tick} \equiv \lambda \sigma. \langle \textsf{get\_K}(\sigma), \textsf{now}(\sigma) + 1, \textsf{consume}(\sigma) \rangle$
    \item \textbf{说明}: 它不仅增加时间计数,通常还伴随着当前事件的消耗,代表逻辑步进的完成。\end{itemize}
    \item \textbf{知识合一算子 ($\textsf{unify}$)}
\begin{itemize}
    \item \textbf{Core 类型}: $\textsf{unify} : \Pi(\sigma : \Sigma). \Pi(f : \textsf{Fact}). \Sigma$
    \item \textbf{算子定义}: $\textsf{unify} \equiv \lambda \sigma. \lambda f. \textsf{if } \textsf{is\_consistent}(\textsf{get\_K}(\sigma), f) \text{ then } \langle \textsf{get\_K}(\sigma) \cup \{f\}, \textsf{now}(\sigma), \dots \rangle \text{ else } \sigma$
    \item \textbf{说明}: 这是最复杂的算子。它在合并新事实前,利用 Core 层的判定规则验证 $f$ 与现有 $\mathcal{K}$ 的逻辑相容性(Consistency)。
\end{itemize}
\end{itemize}

\subsection{4. 因果追溯算子 (Causal \& Trace Operators)}

利用等价类型(Identity Type)进行深层审计。

\begin{itemize}
\item \textbf{等价校验算子 ($\textsf{verify\_id}$)}
\begin{itemize}
\item \textbf{Core 类型}: $\textsf{verify\_id} : \Pi(\sigma_1 : \Sigma). \Pi(\sigma_2 : \Sigma). \textsf{Type}$
\item \textbf{算子定义}: $\textsf{verify\_id} \equiv \lambda \sigma_1. \lambda \sigma_2. \textsf{Id}_{\Sigma}(\sigma_1, \sigma_2)$
\item \textbf{说明}: 返回一个命题类型(Prop)。若要在 Kernel 中执行,必须提供构造性证明(如 $\textsf{refl}$),用以验证两个配置在逻辑上是否为同一个真理。
\end{itemize}
\end{itemize}

当上述 Core 层算子被 Kernel 层调用时,其执行遵循以下规约路径:
\begin{enumerate}
\item \textbf{参数替换 ($\beta$- 规约)}: 将 Kernel 层的当前状态实例(如 $\sigma_{current}$)代入算子的 $\lambda$-项。
\item \textbf{结构展开 ($\iota$-规约)}: 投影算子 $\textsf{proj}$ 提取三元组中的具体组件。
\item \textbf{状态物化}: 规约得到的最终项(如新的 $\Sigma'$)被存入内核存储,成为下一个循环的输入。
\end{enumerate}

\begin{Definition}{算子的终止性 (Termination)}

由于 Core 层基于强规范化(Strong Normalization)的计算模型,所有内核通用算子在有限步内必将终止并给出结果。这从理论上避免了内核在处理状态转移时发生“死循环”。
\end{Definition}

\section{证明搜索机(Proof-Search Machine, PSM)}

\subsection{基本类型与谓词}

我们假设如下基本类型:
$$
\mathsf{Anomaly},;\mathsf{FailEvt},;\mathsf{ProcStep}.
$$

依赖谓词(作为类型)定义为:
$$
\begin{aligned}
\mathsf{TimeOK}(a,e,f) &\triangleq (a.t \in e.dur) \land (e.dur.end < f.t), \
\mathsf{SpaceOK}(a,e)  &\triangleq (a.m = e.m), \
\mathsf{BatchOK}(e,f)  &\triangleq (e.b = f.b).
\end{aligned}
$$

因果证明类型被定义为一个依赖类型族:
$$
\mathsf{CausalProof} : \mathsf{Anomaly} \to \mathsf{FailEvt} \to \mathsf{Type}.
$$

其唯一构造子由如下推理规则给出:
$$
\frac{
\begin{array}{c}
e : \mathsf{ProcStep} \
p_t : \mathsf{TimeOK}(a,e,f) \
p_s : \mathsf{SpaceOK}(a,e) \
p_b : \mathsf{BatchOK}(e,f)
\end{array}
}{
\mathsf{mkCausalProof}(a,f,e,p_t,p_s,p_b)
: \mathsf{CausalProof}(a,f)
}
$$

\subsection{证明搜索目标}

KOS-TL 的证明搜索目标语言定义如下:
$$
G ::= \mathsf{CausalProof}(a,f)
\mid \mathsf{TimeOK}(a,e,f)
\mid \mathsf{SpaceOK}(a,e)
\mid \mathsf{BatchOK}(e,f)
\mid \Sigma x:T.,G
\mid \top \mid \bot.
$$

其中:$\Sigma x:T.,G$ 表示存在性目标,$\top$ 与 $\bot$ 分别表示可平凡完成与失败目标。

\subsection{机器配置}

KOS-TL 证明搜索机的一个配置(configuration)定义为四元组:
$$
\langle \Gamma ; \sigma ; G ; \Delta \rangle,
$$
其中:
\begin{itemize}
\item $\Gamma$ 为类型上下文与已选取的见证;
\item $\sigma$ 为知识库(异常、工序、事件及其关系);
\item $G$ 为当前待证明的目标;
\item $\Delta$ 为尚未完成的子目标栈。
\end{itemize}

\subsection{小步证明搜索语义}

定义小步转移关系:
$$
\longrightarrow_{\mathrm{ps}} \subseteq \mathrm{Config} \times \mathrm{Config},
$$
表示一次证明搜索推进。

\paragraph{($\Sigma$-展开规则:存在性搜索)}
$$
\frac{
G = \Sigma x:T.,G' \qquad v \in \mathsf{Candidates}(T,\sigma)
}{
\langle \Gamma;\sigma;G;\Delta \rangle
\longrightarrow_{\mathrm{ps}}
\langle \Gamma[x \mapsto v];\sigma;G'[v/x];\Delta \rangle
}
$$

\paragraph{(因果证明引入规则)}
$$
\frac{ G = \mathsf{CausalProof}(a,f) }{
\langle \Gamma;\sigma;G;\Delta \rangle
\longrightarrow_{\mathrm{ps}}
\langle \Gamma;\sigma;
\Sigma e:\mathsf{ProcStep}.
(\mathsf{TimeOK}(a,e,f) \land
\mathsf{SpaceOK}(a,e) \land
\mathsf{BatchOK}(e,f));
\Delta \rangle
}
$$

\paragraph{(合取分解规则)}
$$
\frac{ G = G_1 \land G_2 }{
\langle \Gamma;\sigma;G;\Delta \rangle
\longrightarrow_{\mathrm{ps}}
\langle \Gamma;\sigma;G_1;G_2 :: \Delta \rangle
}
$$

\paragraph{(原子谓词判定)}
以时间一致性为例(其余谓词类似):
$$
\frac{ \mathsf{checkTime}(a,e,f) = \mathsf{true} }{
\langle \Gamma;\sigma;\mathsf{TimeOK}(a,e,f);\Delta \rangle
\longrightarrow_{\mathrm{ps}}
\langle \Gamma;\sigma;\top;\Delta \rangle
}
$$

$$
\frac{ \mathsf{checkTime}(a,e,f) = \mathsf{false} }{
\langle \Gamma;\sigma;\mathsf{TimeOK}(a,e,f);\Delta \rangle
\longrightarrow_{\mathrm{ps}}
\langle \Gamma;\sigma;\bot;\Delta \rangle
}
$$

\paragraph{(目标完成)}
$$
\frac{}{ \langle \Gamma;\sigma;\top;G' :: \Delta \rangle
\longrightarrow_{\mathrm{ps}}
\langle \Gamma;\sigma;G';\Delta \rangle }
$$

\paragraph{(失败规则)}
$$
\frac{}{ \langle \Gamma;\sigma;\bot;\Delta \rangle \longrightarrow_{\mathrm{ps}} \mathsf{Fail} }
$$

\subsection{接受条件}

当证明搜索达到如下配置时,认为搜索成功:
$$
\langle \Gamma;\sigma;\top;[,] \rangle.
$$

此时,可由上下文 $\Gamma$ 中记录的见证与证明项,构造一个类型为
$$
\Sigma a:\mathsf{Anomaly}.,\mathsf{CausalProof}(a,f)
$$
的因果分析结果。

\subsection{语义解释}

KOS-TL 证明搜索机将因果分析刻画为一个\emph{目标驱动的依赖类型证明构造过程}。系统执行并非命令式算法的运行,而是对目标类型
$$
\Sigma a:\mathsf{Anomaly}.,\mathsf{CausalProof}(a,f)
$$
进行小步证明搜索;任何成功的搜索轨迹均对应一条可验证、可审计的因果证明。


\section{内核层逻辑性质}

在 KOS-TL 内核架构中,状态保持性(Preservation),也常被称为类型保持性(Subject Reduction)在状态机维度的扩展。它确保了系统在执行环境演化或事务提交时,逻辑上的“良构性”(Well-formedness)不会因为数据的流入而坍塌。

\begin{Theorem}{状态保持性 - Preservation}

设 $\Gamma$ 为系统全局上下文。若内核状态 $\Sigma$ 是良构的(记作 $\Gamma \vdash \Sigma \,\, \textsf{ok}$),且存在一个由事件 $e$ 触发的转移步骤 $\Sigma \xrightarrow{e} \Sigma'$,则转移后的新状态 $\Sigma'$ 依然是良构的:
\[
\Gamma \vdash \Sigma \,\, \textsf{ok} \quad \wedge \quad \Sigma \xrightarrow{e} \Sigma' \implies \Gamma \vdash \Sigma' \,\, \textsf{ok}
\]
其中,良构性 $\Sigma \,\, \textsf{ok}$ 定义为:对于状态中包含的所有事实项 $ku_i \in \Sigma$,均存在类型 $A_i$ 使得 $\Gamma \vdash ku_i : A_i$,且 $\Sigma$ 满足一致性 $\Sigma \not\vdash \bot$。
\end{Theorem}

\begin{proof}
我们根据事件 $e$ 的性质,通过对转移算子的结构归纳法进行证明。

\textbf{1. 内部演算步 (Internal Computation):}

若 $e$ 对应内核内部的规约(如表达式化简 $\beta$-reduction),则 $\Sigma' = \Sigma$ 且仅涉及控制项 $ctrl$ 的变化。
\begin{itemize}
\item 根据 Core 层的 \textbf{Subject Reduction} 定理:若 $\Gamma, \Sigma \vdash ctrl : A$ 且 $ctrl \to ctrl'$,则 $\Gamma, \Sigma \vdash ctrl' : A$。
\item 由于 $\Sigma$ 本身未发生改变,其良构性 $\Gamma \vdash \Sigma \,\, \textsf{ok}$ 自动保持。
\end{itemize}

\textbf{2. 外部事实注入 (Fact Injection):}

若 $e$ 对应向知识库注入新事实 $ku_{new}$,则转移由 $\textsf{unify}(\Sigma, ku_{new})$ 定义。
\begin{itemize}
\item \textbf{类型预检}:转移的前提是 $ku_{new}$ 必须通过类型检查,即 $\Gamma, \Sigma \vdash ku_{new} : A_{new}$。
\item \textbf{一致性冲突处理}:
\begin{itemize}
\item \textit{分支 A (相容)}:若 $\Sigma \cup \{ku_{new}\} \not\vdash \bot$,则 $\Sigma' = \Sigma \cup \{ku_{new}\}$。根据弱化引理 (Weakening Lemma),原有的事实项在更大的语境下依然保持良型。
\item \textit{分支 B (冲突)}:若 $\Sigma \cup \{ku_{new}\} \vdash \pi : \bot$,内核不会直接合并,而是构造 $\Sigma' = \Sigma \cup \{ \textsf{Invalidated}(ku_{new}, \pi) \}$。
\end{itemize}
\item 在两种分支下,$\Sigma'$ 都不包含可直接推导出的 $\bot$,且所有元素都拥有对应的构造证明。因此,$\Gamma \vdash \Sigma' \,\, \textsf{ok}$。
\end{itemize}

\textbf{3. 环境消解步 (Elimination Step):}

若 $e$ 对应消去规则的应用(例如从 $\Sigma$-类型事实中提取投影项)。
\begin{itemize}
\item 设 $\Sigma$ 中存在 $\langle a, p \rangle : \Sigma(x:A).B$。转移步产生 $a : A$。
\item 根据 $\Sigma$-消去规则的语义可靠性(Soundness),投影出的项 $a$ 的类型 $A$ 是预定义的且合法的。
\item 这种操作只是对既有良构知识的展开,不会引入不一致性,故 $\Sigma'$ 保持良构。
\end{itemize}

\textbf{结论:} 综上所述,无论何种转移事件 $e$,新状态 $\Sigma'$ 均能保持逻辑上的良构性与一致性。
\end{proof}

在 KOS-TL Kernel 层中,确定性 (Determinism) 是确保分布式共识(Consensus)和逻辑可追溯性的基石。在依存类型系统下,确定性不仅意味着计算结果的一致性,还意味着规约路径的合流性(Confluence)。

\begin{Theorem}{内核演算确定性(Determinism of Kernel Evolution)}

设 $\textsf{Op}$ 为内核状态转移函数,$\Sigma$ 为当前良构的内核状态,$e$ 为触发事件。若转移规则定义为 $\Sigma' = \textsf{Op}(\Sigma, e)$,则对于相同的输入对 $(\Sigma, e)$,产出的新状态 $\Sigma'$ 在逻辑等价意义下是唯一的:
\[
\forall \Sigma, e, \Sigma'_1, \Sigma'_2: (\Sigma \xrightarrow{e} \Sigma'_1 \wedge \Sigma \xrightarrow{e} \Sigma'_2) \implies \Sigma'_1 \equiv \Sigma'_2
\]
其中 $\equiv$ 表示内涵相等(Intensional Equality),即两者的范式(Normal Form)完全一致。
\end{Theorem}

\begin{proof}
证明基于 KOS-TL Core 的纯函数性质与强规范化演算的合流性,分为以下三个维度展开:

\textbf{1. 算子的纯函数性 (Purity of Operators):}

内核中的所有转移算子(如 $\textsf{unify}$, $\textsf{subst}$, $\textsf{eval}$)均定义为核心层(Core Layer)中的项。
\begin{itemize}
\item 在核心层理论中,所有构造子均满足 \textbf{计算一致性}。给定相同的输入 $\rho$(赋值环境),解释函数 $\llbracket \textsf{Op} \rrbracket_\rho$ 是一个数学意义上的单值函数。
\item 既然 $\textsf{Op}$ 不依赖于任何外部隐式状态或随机源,其映射关系 $\Sigma \times e \to \Sigma'$ 在函数式语义下是确定的。
\end{itemize}

\textbf{2. 强规范化与合流性 (Confluence):}

由于 KOS-TL 具有强规范化(Strong Normalization)性质,根据 \textbf{Church-Rosser 定理},该演算系统具有合流性。
\begin{itemize}
\item 即使在规约 $\Sigma \xrightarrow{e} \Sigma'$ 的过程中存在多个可选的规约红式(Redex),合流性保证了无论采取何种规约顺序(Reduction Strategy),最终得到的范式 $\textsf{nf}(\Sigma')$ 是唯一的。
\item 因此,虽然物理内存中的中间步可能略有差异,但逻辑层面的状态(即能够参与后续推导的事实集合)是唯一的。
\end{itemize}

\textbf{3. 冲突解决的确定性 (Deterministic Conflict Resolution):}

在处理 $e$ 导致的一致性冲突时,内核的分支判定逻辑:
\begin{itemize}
\item \textsf{unify} 算子按照类型推导规则的优先级(Typing Rule Priority)进行穷举搜索。
\item 矛盾证明项 $\pi$ 的构造遵循标准的搜索算法(如统一化算法 Unification Algorithm)。在给定的搜索空间内,第一个被找到的最小证明项是确定的。
\item 分支 A 或分支 B 的选择完全由“是否存在证明项 $\pi$”这一逻辑真值决定,不具备非确定性选择(Non-deterministic Choice)。
\end{itemize}

\textbf{结论:} 综上所述,由于算子的纯函数定义及底层演算系统的合流性,KOS-TL 内核的状态转移具有严格的确定性。

\end{proof}


进度性 (Progress) 是确保内核实时响应能力和鲁棒性的核心数学基石。它保证了逻辑自愈引擎在任何时刻都不会陷入“计算死胡同”。

\begin{Theorem}{内核进度性 (Kernel Progress)}

设 $\mathcal{C} = \langle \Sigma, \textsf{Ev}, \Delta \rangle$ 为一个良构的 KOS-TL 内核配置,其中状态 $\Sigma = \langle \mathcal{K}, \mathcal{TS}, \mathcal{P} \rangle$ 且转移记录 $\Delta$ 满足时间单调性。若 $\mathcal{C}$ 满足全局类型分配且当前活跃事件 $\textsf{Ev}$ 在上下文 $\Sigma$ 下是良构的,则必有以下之一成立:
\begin{enumerate}
\item[\textbf{PP1.}] \textbf{逻辑稳态 (Logical Quiescence)}:

$\textsf{Ev} = \textsf{null}$ 且挂起队列 $\mathcal{P} = \emptyset$。此时系统所有因果链均已在 $\Delta$ 中物化,计算暂时终止。
\item[\textbf{PP2.}] \textbf{因果推进 (Causal Advancement)}:

存在一个新的配置 $\mathcal{C}' = \langle \Sigma', \textsf{Ev}', \Delta \cup \{ \delta \} \rangle$,使得系统通过以下规约步之一向前推进:
\begin{itemize}
\item \textbf{执行步 (Execution)}:若 $\textsf{Ev} = e \neq \textsf{null}$,则执行 $\textsf{Op}$ 产生新状态 $\Sigma'$,并生成转移记录 $\delta = \langle \Sigma \xrightarrow{e} \Sigma' \rangle$。
\item \textbf{激活步 (Activation)}:若 $\textsf{Ev} = \textsf{null}$ 且 $\mathcal{P} = e_0 :: \mathcal{P}_{rest}$,则通过提取算子将队列首位事件激活。
\end{itemize}
\end{enumerate}
\end{Theorem}
\begin{proof}
对当前活跃项 $\textsf{Ev}$ 的构造及队列 $\mathcal{P}$ 的状态进行分类讨论:

\textbf{1. 活跃事件的演算 (Calculus of Active Events)}

当 $\textsf{Ev} = e \equiv \langle \textsf{Args}, \textsf{Pre}, \textsf{Op}, \textsf{Post}, \textsf{Prf} \rangle$ 时,由于配置良构,存在证明项 $\textsf{Prf}$ 满足 $\textsf{Pre}(\textsf{Args}, \Sigma)$。根据核心层(Core Layer)的强规范化性质:

\begin{itemize}
\item $\textsf{Op}$ 作为一个全函数(Total Function),对于输入对 $(\textsf{Args}, \Sigma)$ 必有定义的输出值 $\Sigma'$。
\item 根据消解规则的完备性,后置谓词 $\textsf{Post}(\Sigma')$ 在 $\Sigma'$ 构建完成后是可判定的。
\end{itemize}

因此,执行算子必能产生一个新的转移项 $\delta$,从而使系统向 PP2 转换。

\textbf{2. 队列动力学 (Queue Dynamics)}

若 $\textsf{Ev} = \textsf{null}$,系统检查挂起队列 $\mathcal{P}$:
\begin{itemize}
\item 若 $\mathcal{P} = e_0 :: \mathcal{P}_{rest}$,根据内核的操作语义,存在一个良定义的“入队- 出队”转换,将 $\textsf{Ev}$ 更新为 $e_0$。此步骤不改变 $\mathcal{K}$,但改变了系统的动能分配。
\item 若 $\mathcal{P} = \emptyset$,则系统满足 PP1 描述的稳态条件。
\end{itemize}

\textbf{3. 时间箭头约束 (Temporal Arrow Constraint)}

在所有转移 $\Sigma \xrightarrow{e} \Sigma'$ 中,单调性规则 $\Sigma'.\mathcal{TS} > \Sigma.\mathcal{TS}$ 确保了转移项 $\delta$ 的唯一性。由于转移记录集 $\Delta$ 是单调递增的合集,系统排除了逻辑循环(Cycles)的可能性。根据依存类型论的规范形式引理(Canonical Forms Lemma),在良构的 $\Sigma$ 下,没有任何 $\textsf{Op}$ 能够产生类型不匹配的挂起。综上所述,良构的 KOS-TL 内核配置始终具备向下一步演化的能力,直到所有事件被清空。

\end{proof}

\begin{Theorem}{演化相容性 (Evolutionary Consistency)}

设良构的内核配置为 $\mathcal{C} = \langle \Sigma, \textsf{Ev}, \Delta \rangle$,其中状态 $\Sigma = \langle \mathcal{K}, \mathcal{TS}, \mathcal{P} \rangle$。若其满足:
\begin{enumerate}[label=(\arabic*)]
\item \textbf{状态合法性}:对于所有 $(id_i, t_i, A_i) \in \mathcal{K}$,均有 $\Gamma_{Core} \vdash t_i : A_i$ 且 $\mathcal{K} \not\vdash \bot$。
\item \textbf{因果完备性}:活跃事件 $\textsf{Ev}$ 携带有效的证明项 $\textsf{Prf} : \textsf{Pre}(\textsf{Args}, \Sigma)$。
\end{enumerate}

若内核执行小步演化 $\mathcal{C} \xrightarrow{step} \mathcal{C}'$,则新配置 $\mathcal{C}' = \langle \Sigma', \textsf{Ev}', \Delta \cup \{ \delta \} \rangle$ 依然保持状态合法性与全局逻辑一致性。
\end{Theorem}

\begin{proof}

我们通过对配置转换的原子性质进行分类讨论来展开证明。

\begin{enumerate}[label=(\arabic*)]
\item \textbf{计算规约步(原子项演化)}

若规约仅涉及 $\textsf{Ev} \to \textsf{Ev}'$(如证明项的内涵简化或参数代换),而知识库 $\mathcal{K}$ 保持不变:
\begin{itemize}
    \item \textbf{一致性继承}:由于 $\mathcal{K}$ 未发生变化且已知 $\mathcal{K} \not\vdash \bot$,一致性自然保持。
    \item \textbf{主项规约 (Subject Reduction)}:根据 KOS-TL Core 的元理论,依存类型项的规约保持其类型。由于 $\textsf{Ev}$ 在 $\Sigma$ 下良构,规约后的 $\textsf{Ev}'$ 依然满足原有的类型签名。
\end{itemize}

\item \textbf{状态转换步(核心演化算子)}

当执行 $\textsf{Op}$ 导致 $\mathcal{K} \to \mathcal{K}'$ 时,系统执行算子 $\textsf{commit}(\mathcal{K}, ku_{new})$。我们对新知识项 $ku_{new}$ 及其证明进行一致性分析:

\textbf{情况 A:单调扩张 (Monotonic Expansion)}

\begin{itemize}
    \item \textbf{弱化引理应用}:若 $ku_{new}$ 与现存知识无冲突,根据构造性逻辑的弱化引理 (Weakening Lemma),原有 $\mathcal{K}$ 中的所有证明在 $\mathcal{K}' = \mathcal{K} \cup \{ku_{new}\}$ 中依然有效。
    \item \textbf{安全闭环}:由于 $\textsf{Ev}$ 包含 $\textsf{Post} : \Sigma' \to \textsf{Prop}$,内核在物化 $\mathcal{K}'$ 前会强制检查后置约束。若检查通过,则 $\mathcal{K}' \not\vdash \bot$ 得到形式化保证。
\end{itemize}

\textbf{情况 B:潜在冲突检测 (Conflict Mitigation)}

\begin{itemize}
    \item \textbf{逻辑隔离}:若 $ku_{new}$ 引入了逻辑矛盾(即存在 $\pi : \mathcal{K}, ku_{new} \vdash \bot$),内核的防护机制将阻止直接合并。
    \item \textbf{否定引入构造}:内核转而构造 $\textsf{absurd}(ku_{new}, \pi)$ 并存入知识库。在语义模型中,这相当于将冲突转化为对输入信号的证伪结论。由于矛盾被包裹在否定构造子中,它无法作为消除规则的前提,从而保护了全局一致性。
\end{itemize}

\item \textbf{时间箭头与转移记录的约束}

为了证明演化的轨迹是合法的,内核利用 $\Delta$ 构造因果证据 $\delta$:
\begin{itemize}
    \item \textbf{时钟单调性证明}:每一次演化必伴随 $\mathcal{TS}' > \mathcal{TS}$。这证明了 $\Sigma'$ 不是 $\Sigma$ 的简单循环,而是逻辑上的单调后继。
    \item \textbf{物化归纳}:系统状态 $\Sigma_n$ 的合法性可以通过归纳法溯源至初始空状态 $\Sigma_0$:
    \[
    \Sigma_n = \textsf{Apply}(\delta_n, \textsf{Apply}(\delta_{n-1}, \dots \Sigma_0))
    \]
    每一项 $\delta_i = \langle \Sigma_{i-1} \xrightarrow{e_i} \Sigma_i \rangle$ 都包含了对 $\textsf{Pre}$ 的验证和 $\textsf{Post}$ 的满足,确保了演化链条的每一环都对齐了逻辑基座。
\end{itemize}
\end{enumerate}
\end{proof}

这个证明解释了 KOS-TL 如何处理现实世界中的“脏数据”(如银行的错误流水、传感器的误报):逻辑防火墙:演化相容性确保了任何试图破坏系统一致性的数据,都会在 $\textsf{unify}$ 阶段被转换成“关于矛盾的证据”,而不是让系统本身变得矛盾。双向同步的安全性:在您之前提到的“全球供应链”例子中,即使底层数据库被非法篡改(产生冲突数据),演化相容性也会强制内核生成一个 Refute 项,从而在本体视图层保持逻辑的纯净。

这两大性质共同界定了 KOS-TL Kernel 的运行边界:

进度性 (Progress) 保证了内核的活锁自由(Livelock Freedom)。只要逻辑是良构的,内核分析程序就一定能跑下去,给出分析结果。

演化相容性 (Evolutionary Consistency) 保证了内核的运行安全性(Runtime Safety)。它确保了内核知识库在动态运行过程中,永远不会退化为一个自相矛盾的废弃系统。

\begin{Theorem}{局部可判定性定理(Local Decidability Theorem)}

给定内核状态 $\Sigma$、新事实 $ku_{new}$ 以及搜索边界 $\Delta = \{depth, fuel\}$,内核算子 $\textsf{unify}(\Sigma, ku_{new}, \Delta)$ 的执行过程是可判定的。
\end{Theorem}

\begin{proof}
证明通过对搜索空间和规约步数的双重归纳完成:

\textbf{1. 搜索空间的有限性 (Finite Search Space):}

由于内核限制了 $depth$(递归深度),证明搜索树被强制修剪为有限高度。在每一层级,合一(Unification)候选者的数量由上下文 $\Gamma$ 中的变量数决定,这同样是有限的。

\textbf{2. 规约步数的强制停机 (Forced Termination via Fuel):}

引入 $fuel$ 参数(计算能量)。每执行一次 $\beta$-规约或 $\delta$-展开,消耗一个单位的 $fuel$。
\begin{itemize}
\item 算法在每次操作前检查 $fuel > 0$。
\item 既然 $fuel$ 是自然数且随步数严格递减,根据良序原理,计算必然在有限步内要么得到范式,要么耗尽 $fuel$。
\end{itemize}

\textbf{3. 结果集的完备性:}

当计算停止时:
\begin{itemize}
\item 若范式匹配,返回 \textbf{True}。
\item 若发现结构冲突(如构造子不匹配),返回 \textbf{False}。
\item 若因 $depth$ 或 $fuel$ 耗尽而停止,返回 \textbf{Unknown}。
\end{itemize}
由于算法在所有路径上都保证停机,因此该过程是可判定的。
\end{proof}

\section{应用示例:质量异常溯源派生}

在制造业场景中,当检测到批次缺陷时,内核层自动触发溯源逻辑。

\begin{itemize}
\item \textbf{事件定义}:设$e_{trace}$ 为溯源事件。
\begin{itemize}
\item $\textsf{Pre}$:状态中存在某批次的缺陷报告项 $r : \textsf{DefectReport}$。
\item $\textsf{Op}$:根据生产日志,寻找与该批次关联的设备异常记录 $s : \textsf{EquipmentStatus}$。
\item  $\textsf{Post}$:生成并物化一个因果链知识对象 $cc : \textsf{CausalChain}$。
\end{itemize}
\item \textbf{演化过程}:
一旦 Runtime 将缺陷报告精化并存入$\Sigma$,内核层通过上述方程式发现 $p: \textsf{Pre}$成立,自动执行$\textsf{Op}$。系统从“已知有缺陷”的状态小步迁移到“已知缺陷原因”的更高熵状态。
\end{itemize}


\begin{Example}{跨国合规转账事件 ($e_{transfer}$)}

假设当前系统状态为 $\sigma$,包含账户 $A$ 和 $B$ 的余额。我们要定义一个从 $A$ 向 $B$ 转账金额 $v$ 的事件。

1. 状态定义 ($\sigma \in \Sigma$)

状态 $\sigma$ 是一个知识快照,包含:

$Balance(A, \sigma) = 1000$

$Balance(B, \sigma) = 500$

2.事件具体构造 ($e_{transfer} : \textsf{Ev}$)

 根据你的 $\textsf{Ev}$ 定义,该事件由三部分组成:

 前提条件 ($\textsf{pre}$):

 $$\textsf{pre} \equiv (Balance(A, \sigma) \ge v) \land \textsf{IsVerified}(A)$$

 (解释:$A$ 的余额必须足够,且 $A$ 必须通过了实名认证。)

 动作算子 ($\textsf{act}$):

 $$\textsf{act}(\sigma) \equiv \sigma [Balance(A) \gets Balance(A) - v, Balance(B) \gets Balance(B) + v]$$

 (解释:这是一个函数,描述了状态如何改变:$A$ 减钱,$B$ 加钱。)

 后置变换自证 ($\textsf{post\_prf}$):这是一个证明项,它保证了:对于任何满足 $\textsf{pre}$ 的状态 $\sigma$,执行 $\textsf{act}$ 后的新状态一定满足守恒定律($Sum_{after} = Sum_{before}$)。

 $$\textsf{post\_prf} : \Pi(\sigma:\Sigma). \textsf{pre} \to \textsf{Correct}(\textsf{act}(\sigma))$$

 3. 执行过程:小步转移 ($\Delta$)

 当内核尝试执行这个转账时,会发生以下判定过程:

 类型检查:内核首先验证 $\Gamma \vdash e_{transfer} : \textsf{Event}$。

 如果开发者写的 $\textsf{act}$ 逻辑有误(比如只减钱不加钱),那么 $\textsf{post\_prf}$ 将无法构造,该事件在编译阶段就会被拒绝。

 触发转移:输入当前快照 $\sigma$ 和事件 $e$。

 $$\textsf{STEP}(\sigma, e_{transfer}) \to \sigma'$$

 生成转移记录 ($\Delta$):产生一个三元组 $(\sigma \xrightarrow{e_{transfer}} \sigma')$。这条记录被永久存入演化轨迹$\Delta$ 中。
\end{Example}


\chapter{KOS-TL 运行层 (Runtime Layer):环境交互与信号精化}

运行层(Runtime Layer)是 KOS-TL 逻辑系统与物理世界之间的边界。它负责处理非确定性的外部信号、管理计算资源、调度事件队列,并将内核层生成的逻辑状态持久化为物理存储。

\section{语法 (Syntax)}

运行层状态由配置 $Cfg$ 描述,它将逻辑内核嵌入到物理宿主中:

\begin{equation}
Cfg \equiv \langle \Sigma, Q_{raw}, \textsf{Env}, \mathcal{M} \rangle
\end{equation}

其中:
\begin{itemize}
    \item $\Sigma$ —— 逻辑内核状态 (Logical Kernel State)

    这是系统的“大脑”在逻辑层面的当前形态。它包含知识库 $\mathcal{K}$、逻辑时钟 $\mathcal{T}$ 和挂起意图队列 $\mathcal{P}$。 它代表了系统当前认为“真实”且“经过证明”的所有逻辑事实。运行层通过观察 $\Sigma$,决定下一步应该对外执行什么动作,或者如何响应外部信号。

    \item $Q_{raw}$ —— 物理原始信号队列 (Raw Signal Queue)

    这是系统的“感知输入缓冲区”。存放的是来自外部物理世界(如传感器、网络包、用户点击)但尚未被逻辑化的原始二进制或文本数据。信号的到达是随机的。这里的信号还没有对应的逻辑证明(Proof),只是“脏数据”。它是 $\textsf{elab}$(精化算子)的原料。系统会从 $Q_{raw}$ 取出信号,尝试将其提升(Promote)为内核可理解的事件。

    \item $\textsf{Env}$ —— 物理运行时环境 (Physical Environment)

    这是系统所在的物理宿主上下文。包含逻辑层无法直接感知、但运行层必须掌握的物理资源:(1)物理时钟 ($T_{wall}$):类似现实世界的墙上时间,用于处理超时判定。(2)I/O 句柄:数据库连接池、网络套接字、硬件寄存器地址。(3)计算资源:内存状态、线程池负载等。在执行 $\textsf{elab}$ 时,$\textsf{Env}$ 提供了构造逻辑证明所需的物理证据(例如:传感器自检成功的状态位)。

    \item $\mathcal{M}$ —— 物化映射与存储 (Materialization \& Storage)

    $$\mathcal{M} : \mathcal{K} \to \textsf{PhysicalStorage}$$

    这是系统的“记忆体”和“物理投影”。$\mathcal{M}$ 既代表物理存储(如磁盘上的数据库),也代表逻辑项到物理表示的映射规则。它的作用包括:(1)投影:将内核层抽象的“依存对知识”映射为数据库中的“行、列、索引”。(2)持久化:确保 $\Sigma$ 中的逻辑演化结果被安全地写入非易失性存储。(3)外部一致性:保证物理世界看到的状态(如屏幕上显示的余额)与逻辑内核 $\Sigma$ 保持同步。

\end{itemize}
%
%调度主循环(Scheduler)的逻辑应体现在驱动内核算子上:
%\begin{alltt}
%while $Q_{raw}$ is not empty:
%    s = pop($Q_{raw}$)
%    match elab(s, Env):
%        case (e):
%        // 调用内核层定义的转移规则
%            if $\langle \Sigma, e \rangle \longrightarrow_{KOS} \Sigma'$:
%                $\Sigma = \Sigma'$
%            Commit_to_Storage($\mathcal{M}$, $\Sigma$)
%        case None:
%            Log_Refinement_Failure(s)
%\end{alltt}


运行层引入了精化算子(Elaborator),其语法功能是将“脏数据”转化为核心层可理解的“构造项”:

$$\textsf{elab} : \textsf{RawSignal} \to \textsf{Env} \to \textsf{Option} \left( \sum_{e : \textsf{Ev}} \textsf{Pre}(e, \Sigma) \right)$$

精化算子是实现“信号逻辑化”的网关。其核心任务是为外部数据补充证明项:
\begin{equation}
\textsf{elab}(s, \textsf{Env}) =
\begin{cases}
\textsf{Some}(\langle e, \pi \rangle) & \text{若能构造 } \pi : \textsf{Pre}(e, \Sigma) \\ \textsf{None} & \text{否则}
\end{cases}
\end{equation}


精化过程包括:
\begin{itemize}
\item \textbf{信号解析}:将外部 JSON/二进制流解析为基础排序值($\textsf{Val}, \textsf{ID}$)。
\item \textbf{证明构造}:根据当前$\textsf{Env}$ 自动尝试构造前置条件的逻辑证明项$p$。
\item \textbf{时间锚定}:将物理接收时间映射为核心层的$\textsf{Time}$ 类型。
\end{itemize}

\textbf{事件与观测的区分。} 形式上,$\textsf{elab}$ 的 codomain 仅为“事件 + 证明”对 $\langle e, \pi \rangle$($e : \textsf{Ev}$),因此\textbf{只有能精化为某事件类型}且能构造前置证明的信号才会进入内核事件队列 $\mathcal{P}$;无法匹配任何事件类型或证明构造失败时 $\textsf{elab}(s) = \textsf{None}$,该信号不会进入队列(可记录为精化失败或挂起)。\textbf{一般传感器信号}在设计中并非“一律不予理会”:若在本体中配置了对应的事件类型(例如“观测事件”$\mathsf{Obs}(\mathsf{SensorData})$ 或领域事件如 $\mathsf{FailEvt}$、$\mathsf{Anomaly}$),则传感器数据可被精化为该事件,经小步执行后其携带的类型实例会按转移规则写入 $\sigma$(知识库 $\mathcal{K}$);若未配置任何可匹配的事件类型,则当前形式化下该信号不会进入 $\mathcal{P}$,也不会自动写入 $\sigma$。工程上可通过在 Core 层定义“观测/数据摄入”类事件类型,使传感器读数以“带证明的类型实例”形式经事件路径进入内核并更新 $\sigma$,从而与“仅事件类型才进入系统”的严格解释保持一致。

\section{运行语义 (Runtime Semantics)}

运行层的演化呈现为一种“异步驱动的小步转移”,其核心规则为“精化-提交”循环:

运行层的语义规则必须包含外部环境的更新和存储的持久化:
\begin{equation}
 \frac{ s = \textsf{head}(Q_{raw}) \quad \textsf{elab}(s, \textsf{Env}) = \textsf{Some}(\langle e, \pi \rangle) \quad \langle \Sigma, e, \pi \rangle \longrightarrow_{KOS} \Sigma' \quad \textsf{Persist}(\Sigma', \mathcal{M}) = \textsf{Success} }{ \langle \Sigma, s :: Q, \textsf{Env}, \mathcal{M} \rangle \xRightarrow{\textsf{commit}} \langle \Sigma', Q, \textsf{Env}', \mathcal{M}' \rangle }
\end{equation}
这里,$\mathcal{M} \vdash \Sigma' \Downarrow \mathcal{M}'$ 表示新状态 $\Sigma'$ 被成功“降解”(Down-cast)并物化到物理介质 $\mathcal{M}'$ 中。

\section{逻辑性质 (Logical Properties)}

在 Runtime 层,我们将每一个执行动作(Action)建模为一对 $e = (t, p)$,其中 $t$ 是目标命题(任务),$p$ 是其对应的证明项。

\begin{Definition}{因果依赖序}

设 $\mathcal{E}$ 为系统中所有可能的执行项集合。定义因果依赖关系 $\prec_{L} \subseteq \mathcal{E} \times \mathcal{E}$:若在 Core 层中,命题 $t_2$ 的构造项中包含对 $t_1$ 的引用(即 $t_1$ 是 $t_2$ 的前提),则称 $e_1 \prec_{L} e_2$。

\end{Definition}

\begin{Definition}{Runtime 执行序列}

执行序列 $S = [e_1, e_2, \dots, e_n]$ 是一个全序集,代表了 Runtime 层实际处理数据的物理时间顺序。
\end{Definition}


\begin{Theorem}{因果序一致性 (Causal Ordering Consistency)}

对于任意 Runtime 执行序列 $S$,若该序列被内核接受(Accepted),则对于 $S$ 中任意两个执行项 $e_i$ 和 $e_j$,必须满足:
$ \text{若 } e_i \prec_{L} e_j, \text{ 则在序列 } S \text{ 中 } e_i \text{ 必须先于 } e_j \text{ 被完成规约。} $
若物理网络导致 $e_j$ 先于 $e_i$ 到达,Runtime 必须阻塞 $e_j$ 的执行直到 $e_i$ 的证明项补齐。
\end{Theorem}
\begin{proof}

我们通过反证法(Proof by Contradiction)结合 Core 层的类型检查机制进行证明。

\textbf{步骤 1:假设存在逆序执行。}

假设 Runtime 接受了一个违反因果序的序列 $S'$,其中存在 $e_j$ 在 $e_i$ 之前完成执行,且已知 $e_i \prec_{L} e_j$。

\textbf{步骤 2:核心层约束映射。}

根据 $\prec_{L}$ 的定义,在 Core 层中,$e_j$ 的证明项校验依赖于 $e_i$ 的存在。其类型检查规则如下:

$$ \frac{\Gamma \vdash p_i : T_i \quad \Gamma, x:T_i \vdash p_j : T_j}{\Gamma \vdash \langle p_i, p_j \rangle : \Sigma(x:T_i).T_j} $$

这意味着,要判定 $e_j$ 合法,内核必须在上下文 $\Gamma$ 中已经包含 $e_i$ 的证明项。

\textbf{步骤 3:Runtime 状态演化。}

Runtime 的状态由上下文序列 $\Gamma_t$ 表示。在执行 $e_j$ 时,其算子为 $\textsf{check}(\Gamma_{current}, e_j)$。
\begin{itemize}
\item 若 $e_i$ 未执行,则 $e_i \notin \Gamma_{current}$。
\item 此时,根据 Core 层的 \textbf{范围确定性 (Scope Determinism)},$e_j$ 中对 $e_i$ 的引用将产生一个“未定义变量”错误或“自由变量”逃逸。
\item 类型检查器将返回 $\textsf{Fail}$。
\end{itemize}

\textbf{步骤 4:阻塞机制的必然性。}

由于 KOS-TL 的 Runtime 强制执行 \textbf{类型安全栅栏 (Type-Safe Fence)},任何校验失败的操作无法改变 $\Sigma$ 事实库的状态。为了使执行继续,Runtime 调度器必须挂起 $e_j$,将其放入待处理池(Pending Pool),并发出 $\textsf{Requirement}(e_i)$ 信号。

\textbf{步骤 5:结论。}

只有当 $e_i$ 到达并成功规约进入 $\Gamma$ 后,$e_j$ 的上下文才满足校验条件。因此,最终被“接受”的序列必然满足因果序。
\end{proof}

基于仿真(Simulation)理论,KOS-TL Runtime 具备两个核心逻辑性质:精化保真性 (Refinement Fidelity) 和 实时可观察性 (Observational Adequacy)。

\begin{Theorem}{精化保真性(Refinement Fidelity)}

设 $\mathcal{S}$ 为物理硬件状态空间(如 FPGA 寄存器或传感器读数集合),$\mathcal{D}_{Core}$ 为逻辑论域。定义精化函数 $\mathcal{E} : \mathcal{S} \to \mathcal{D}_{Core}$。若 Runtime 捕获物理状态 $s \in \mathcal{S}$ 得到 $ku = \mathcal{E}(s)$,则满足:
\begin{enumerate}
\item \textbf{合法性 (Well-formedness)}:存在类型 $A \in \mathcal{U}$,使得 $\Gamma \vdash ku : A$ 恒成立。
\item \textbf{模拟一致性 (Simulation Consistency)}:对于物理迁移 $s \xrightarrow{hw} s'$,存在仿真关系 $R \subseteq \mathcal{S} \times \mathcal{D}_{Core}$ 使得:

$$(s, ku) \in R \implies \exists ku' . (s', ku') \in R \land (ku \xrightarrow{small}^* ku' \lor \textit{Invalidated}(ku'))$$

\end{enumerate}
\end{Theorem}

\begin{proof}
我们通过构造仿真关系 (Simulation Relation) 并结合硬件抽象层(HAL)的规约进行证明:

\textbf{1. 构造仿真关系 $R$:}
定义关系 $R$ 如下:

$$(s, ku) \in R \iff (\textsf{val}(ku) = \textsf{measure}(s)) \land (\textsf{proof}(ku) \models \textsf{Inv}_{HW}(s))$$

其中 $\textsf{measure}(s)$ 是对物理信号的量化,$\textsf{Inv}_{HW}(s)$ 是由硬件电路(如冗余校验位或看门狗状态)强制执行的物理不变性。

\textbf{2. 合法性映射证明:}
根据 TL-Lang 的运行时精化规则,$\mathcal{E}(s)$ 的构造式为:

$$\mathcal{E}(s) \triangleq \langle \textsf{quantize}(s), \textsf{synthesize\_witness}(s) \rangle$$

由于 $\textsf{synthesize\_witness}$ 是由硬件描述语言(HDL)定义的确定性算子,它根据硬件寄存器状态 $\textsf{Reg}_{status}$ 直接映射为 Core 层的引入项(Introduction Rules)。根据 $\Sigma$ 类型的构造原则,只要硬件信号在物理量程内,总能构造出一个良构的项 $ku$。若信号超出量程,精化函数根据完备性定义,将映射至预定义的错误类型项,依然保持良构。

\textbf{3. 模拟一致性证明:}
对物理状态迁移 $s \xrightarrow{hw} s'$ 进行分类讨论:
\begin{itemize}
\item \textbf{情况 A:合规迁移。}
若 $s'$ 满足所有硬件安全约束,则精化函数 $\mathcal{E}$ 会提取新的状态位并合成新的证明项 $p'$。由于硬件层保证了 $s'$ 是由 $s$ 经由合法逻辑门变换而来,在 Core 层,对应的映射项 $ku'$ 必然可以通过内核规约步(如 $\beta$ 或 $\iota$ 规约)从 $ku$ 演化而来,从而维持了仿真关系。
\item \textbf{情况 B:非法/异常迁移。}
若物理迁移导致状态违背了 $\textsf{Inv}_{HW}$(例如传感器断线),硬件状态位会发生翻转。此时精化映射 $\mathcal{E}(s')$ 无法构造出原始类型 $A$ 的项,转而构造出 $\textsf{Invalidated}(ku')$。这种从“正常项”到“失效项”的转变,在 Kernel 层体现为结论的非单调翻转,符合 KOS-TL 处理冲突的语义,仿真关系依然在“错误处理”的维度上得到保持。
\end{itemize}
综上所述,精化过程保证了物理世界的任何有效变动都能在逻辑世界找到对应的真理表示。
\end{proof}

性质讨论:对“软硬件鸿沟”的弥合精化保真性(Refinement Fidelity)解决了传统嵌入式系统中最危险的\textbf{“认知失调”}问题:物理真实性:它保证了你在屏幕(或本体层)看到的“压力正常”,不仅仅是一个 UI 上的数字,而是由硬件寄存器状态数学推导出来的结论。安全性下沉:通过这个性质,KOS-TL 的安全性不仅停留在软件逻辑,而是通过精化函数 $\mathcal{E}$ 直接“锚定”在了 FPGA 的物理门电路上。

\begin{Theorem}{实时可观察性(Observational Adequacy)}

设 $\textit{ctrl} \in \mathcal{D}_{Core}$ 为内核生成的逻辑控制项,其类型为指令集 $\textsf{Cmd}$。设 $\mathcal{G} : \textsf{Cmd} \to \Pi^*$ 为指令生成器,将逻辑项映射为硬件指令序列 $\pi$。若 $\Gamma \vdash \textit{ctrl} : \textsf{Cmd}$ 且逻辑层断言 $\textit{ctrl}$ 满足性质 $\phi$,则:

$$\forall s \in \mathcal{S}, \quad (\textit{ctrl} \vdash \phi) \implies (\textsf{Exec}(\mathcal{G}(\textit{ctrl}), s) \models \textsf{Refine}^{-1}(\phi))$$

其中 $\textsf{Refine}^{-1}(\phi)$ 是逻辑性质 $\phi$ 在物理状态空间 $\mathcal{S}$ 中的谓词解释。
\end{Theorem}

\begin{proof}
我们采用 Hoare 逻辑 (Hoare Logic) 与 代数精化演算 (Refinement Calculus) 结合的方法进行证明:

\textbf{1. 构造映射关系:}
定义逻辑谓词 $\phi$ 与物理状态谓词 $P$ 之间的映射 $\mathcal{M} : \textsf{Prop} \to \mathcal{P}(\mathcal{S})$。
根据 精化保真性 (Refinement Fidelity) 的逆映射,若 $\textit{ctrl}$ 的语义目标是使系统进入 $\phi$ 状态,则其对应的底层寄存器状态必须满足 $P = \textsf{Refine}^{-1}(\phi)$。

\textbf{2. 逆向推导 (Backward Derivation):}
对 $\textsf{Cmd}$ 类型的构造进行归纳:
\begin{itemize}
\item \textbf{基础操作(原子指令):}
若 $\textit{ctrl}$ 是一个原子操作(如 $\textsf{SetValve(open)}$),其在 TL-Lang 的底层规约中被映射为特定的机器码序列 $\pi_a$。根据硬件抽象层(HAL)的 Hoare 三元组定义:

$$\{s \in \mathcal{S}\} \, \pi_a \, \{s' \in \textsf{Refine}^{-1}(\phi)\}$$

由于 $\mathcal{G}$ 在构建时已通过了基于 HAL 公理的静态验证,指令生成的正确性由 HAL 的完备性保证。

\item \textbf{复合操作(序列与分支):}
若 $\textit{ctrl}$ 由多个子项复合而成,则根据 Hoare 逻辑的组合规则:
若 $\{P\} \, \pi_1 \, \{Q\}$ 且 $\{Q\} \, \pi_2 \, \{R\}$,则 $\{P\} \, \pi_1; \pi_2 \, \{R\}$。
由于 $\textsf{Cmd}$ 类型在 Core 层满足\textbf{强规范化},生成的指令序列 $\pi$ 长度有限且路径确定,不存在逻辑层未定义的副作用。

\end{itemize}

\textbf{3. 原子性与干扰分析:}
在物理执行过程中,若发生中断,Runtime 必须维持观察一致性。
KOS-TL 的 Runtime 采用了 事务性 I/O (Transactional I/O) 机制。每一组由 $\mathcal{G}(\textit{ctrl})$ 生成的 $\pi$ 被包裹在一个逻辑原子块中。根据内核的“进度性”证明,该序列要么完全执行并达成 $s' \models \textsf{Refine}^{-1}(\phi)$,要么在失败时回滚并向内核提交一个 $\textsf{Invalidated}$ 证明项。在此机制下,不存在“指令执行了但未达成目标”的中间模糊状态。

\textbf{结论:} 逻辑层的语义目标 $\phi$ 能够无损地投影到物理状态空间。
\end{proof}

对“指令漂移”的防御实时可观察性(Observational Adequacy)在实际高安全场景中的意义在于:消除语义断层:在传统 C/C++ 开发中,编译器优化或驱动错误可能导致代码执行效果与设计意图不符(如指令重排导致的竞态)。在 KOS-TL 中,由于指令生成器 $\mathcal{G}$ 是经由形式化证明的,这种“意图与行为”的背离在逻辑上被消除了。可验证的物理效果:如果一个金融账户在逻辑上被封禁,实时可观察性保证了其在底层数据库中的对应记录也被执行了锁止,且该操作具有原子性保证。


我们定义系统的状态空间为 $\mathcal{S}$,并将系统状态拆分为两个视图:

逻辑视图 ($\mathcal{S}_L$):内核内存中的类型上下文 $\Gamma$ 和已规约的事实库 $\Sigma$。

物理视图 ($\mathcal{S}_P$):存储介质(磁盘或固态存储)中持久化的位流。

定义映射函数 $\textsf{Encode}: \mathcal{S}_L \to \mathcal{S}_P$,将逻辑证明项转化为物理存储格式。


\begin{Theorem}{持久化原子性与可见性(Durability Atomicity and Visibility)}

设系统在时刻 $t$ 执行状态转移 $\delta: \mathcal{S}_L \to \mathcal{S}'_L$。Runtime 层保证存在一个原子算子 $\textsf{Commit}$,满足:
\begin{enumerate}
\item \textbf{原子性}:$\mathcal{S}'_L$ 的逻辑确认(Acknowledgment)当且仅当 $\textsf{Encode}(\mathcal{S}'_L)$ 在 $\mathcal{S}_P$ 中完成持久化。
\item \textbf{可见性}:对于任何后续的读取操作 $\textsf{Recover}$,若 $\textsf{Commit}$ 已成功,则必然有 $\textsf{Recover}(\mathcal{S}_P) \equiv \mathcal{S}'_L$。
\end{enumerate}
即:不存在一个状态,使得逻辑上证明已成立,但物理重启后该证明丢失。
\end{Theorem}
\begin{proof}
我们通过构造 ``逻辑-物理同步锁'' (Logic-Physical Sync Lock) 和 幂等规约 (Idempotent Reduction) 机制来证明。
首先,我们定义系统的状态空间为 $\mathcal{S}$,并将系统状态拆分为两个视图:
\begin{itemize}
\item 逻辑视图 ($\mathcal{S}_L$):内核内存中的类型上下文 $\Gamma$ 和已规约的事实库 $\Sigma$。
\item 物理视图 ($\mathcal{S}_P$):存储介质(磁盘或固态存储)中持久化的位流。
\end{itemize}
定义映射函数 $\textsf{Encode}: \mathcal{S}_L \to \mathcal{S}_P$,将逻辑证明项转化为物理存储格式。

\textbf{步骤 1:构造证明项的持久化序列化。}

每一个逻辑变更 $\Delta \Sigma$ 在 KOS-TL 中都是一个带有 $\textsf{Id}$ 类型的证据。设 $\Delta \Sigma = (p : T)$。持久化过程被建模为一个依存对:
$ \textsf{Record} \equiv \Sigma(p : T). \textsf{Persist}(p) $
其中 $\textsf{Persist}(p)$ 是一个硬件底层的原语,只有物理写入完成才会返回见证。

\textbf{步骤 2:证明原子性。}

Runtime 维护一个写前日志 (WAL) 机制,其条目本身是 Core 层的一个项。
\begin{itemize}
\item 若系统在 $\textsf{Write}(\mathcal{S}_P)$ 过程中崩溃,由于 $\textsf{Persist}(p)$ 尚未生成有效见证,根据 Core 层的 \textbf{合流性 (Confluence)},重启后的恢复算子 $\textsf{Recover}$ 会发现该事务不满足 $\Sigma$-类型的完备性,从而自动回滚。
\item 只有当物理层返回 $p_{stored}$,逻辑层才会将 $\Gamma$ 更新为 $\Gamma \cup \{p\}$。
\end{itemize}

\textbf{步骤 3:证明可见性(一致性恢复)。}

假设系统重启。由于 KOS-TL Core 具有 \textbf{强规范化 (Strong Normalization)} 性质:
\begin{itemize}
\item 存储在 $\mathcal{S}_P$ 中的每一个证明项 $p$ 都是自包含且已化简的。
\item $\textsf{Recover}$ 算子通过重新执行类型检查 $\textsf{check}(\Gamma, p, T)$ 来重建逻辑视图。
\item 由于 Core 层是可判定的,且规约路径受合流性保护,恢复出的逻辑状态 $\mathcal{S}_L^{rec}$ 必然与崩溃前的最后一次有效 Commit 状态 $\mathcal{S}'_L$ 逻辑等价($\mathcal{S}_L^{rec} \equiv \mathcal{S}'_L$)。
\end{itemize}

\textbf{结论:}

物理存储的原子写入保证了逻辑状态的不可撤销性,而逻辑层的强规范化确保了物理数据在任何时刻重新载入后都能产生唯一的、确定的逻辑解释。
\end{proof}

\begin{Definition}{半可判定性}

一个集合 $S \subseteq \mathbb{N}$(或一个命题语言 $L$)被称为是\textbf{半可判定}的,如果存在一个图灵机(或算法)$M$,使得对于任意输入 $x$:
\begin{itemize}
\item 若 $x \in S$,则 $M(x)$ 停机并接受;
\item 若 $x \notin S$,则 $M(x)$ 可能停机并拒绝,也可能永久运行(停机问题不可知)。
\end{itemize}
\end{Definition}

\begin{Theorem}{证明搜索的半可判定性 (Semi-decidability of Proof Search)}

设 $\Gamma$ 为有限上下文,$P$ 为一命题。判定“是否存在证明项 $p$ 使得 $\Gamma \vdash p : P$”的问题是半可判定的。
\end{Theorem}
\begin{proof}
在 KOS-TL 这种包含依存类型和高阶逻辑的系统中,如果不加 $Fuel$ 限制,其证明搜索问题具有半可判定性。我们通过构造一个通用的枚举器(Enumerator)来证明该定理。
\textbf{步骤 1:证明项的可枚举性。}
KOS-TL Core 层的所有良构证明项(Proof Terms)是由一套有限的语法规则(如 $\lambda$-抽象、应用、对偶构造等)生成的。我们可以按照项的长度(或结构复杂度)对所有可能的证明项进行字典序枚举,记为序列 $\{p_1, p_2, p_3, \dots\}$。
\textbf{步骤 2:构造判定算法 $\mathcal{A}$。}
对于给定的命题 $P$ 和上下文 $\Gamma$,算法 $\mathcal{A}$ 执行以下步骤:
\begin{enumerate}
\item 开启一个循环,依次取出一个证明项 $p_i$。
\item 调用 Core 层的类型检查器(Type Checker)验证 $\textsf{check}(\Gamma, p_i, P)$。由于 Core 层具有 \textbf{强规范化性质},该步骤必然在有限时间内返回 $True$ 或 $False$。
\item 若返回 $True$,算法 $\mathcal{A}$ 停机并输出“$P$ 可证”。
\item 若返回 $False$,继续循环,检查下一个项 $p_{i+1}$。
\end{enumerate}
\textbf{步骤 3:分析停机行为。}
\begin{itemize}
\item \textbf{情形一:$P$ 确实是可证的。} 那么必然存在某个证明项 $q$ 满足条件。由于我们的枚举是完备的,在有限步内必然会遇到 $p_k = q$,此时算法停机。
\item \textbf{情形二:$P$ 是不可证的。} 算法将永远在循环中枚举并检查新的项,永远不会停机。
\end{itemize}
\textbf{结论:}
算法 $\mathcal{A}$ 能够识别所有“真”的命题(可证命题),但在面对“假”的命题(不可证命题)时无法保证停机。根据定义,该问题是半可判定的。
\end{proof}


\begin{Definition}{判定性证明的一些定义}

\begin{itemize}

    \item 命题空间 $P$:所有良构的 KOS-TL Core 命题。
    \item 证明算法 $\mathcal{A}$:Runtime 层尝试寻找命题 $p:P$ 的自动化过程。
    \item 资源向量 $\vec{\Delta} = \langle f, d, \tau \rangle$:
    \begin{itemize}
        \item $f \in \mathbb{N}$ (Fuel): 最大 $\beta$-规约步数。
        \item $d \in \mathbb{N}$ (Depth): 递归搜索的最大深度。
        \item $\tau \in \mathbb{R}^+$ (Timeout): 物理墙钟时间上限。
    \end{itemize}
\end{itemize}
\end{Definition}

\begin{Theorem}{KOS-TL Runtime 有界判定性}

设 $P$ 为 Runtime 待判定的逻辑命题。存在一个判定程序 $\mathcal{R}(P, \vec{\Delta})$,对于任意 $P$ 和有限的 $\vec{\Delta}$,$\mathcal{R}$ 在有限时间内必然停机,且其输出空间为:
$ \mathcal{O} = \{ \textsf{True}, \textsf{False}, \textsf{Unknown} \} $
其中 $\textsf{Unknown}$ 是确定性的“资源耗尽”状态。
\end{Theorem}
\begin{proof}
证明通过对执行步数 $k$ 的结构归纳法 (Structural Induction) 以及度量函数 (Measure Function) 的单调性完成。

\textbf{步骤 1:度量函数的定义。}

定义 Runtime 执行状态的度量函数 $\mu(\sigma)$,其中 $\sigma$ 是当前执行快照:

$$ \mu(\sigma) = \langle \text{fuel}, \text{depth}, \text{remaining\_time} \rangle $$

在逻辑执行的每一步(一个原子状态转移 $\sigma \to \sigma'$),该度量函数按照字典序(Lexicographical order)严格递减:

$$ \mu(\sigma') <_{lex} \mu(\sigma) $$

\textbf{步骤 2:状态转移的完备性分类。}

对于 Runtime 的单步动作,其逻辑只有三种可能:
\begin{enumerate}
\item \textbf{逻辑终结}:找到证明 $p$ 或冲突 $\pi$。此时算法直接返回 $\textsf{True}$ 或 $\textsf{False}$。
\item \textbf{继续归约}:资源尚未耗尽 ($\mu > 0$)。算法进入 $\sigma_{k+1}$,由于 $\mu$ 是良序的(Well-founded),此路径不可能无限延伸。
\item \textbf{触碰边界}:$\mu(\sigma)$ 的任一分量归零。算法立即停止并返回 $\textsf{Unknown}$。
\end{enumerate}

\textbf{步骤 3:停机性证明 (Termination)。}

由于 $\mu(\sigma)$ 的取值范围是有限的自然数集合(或受限的实数区间),根据 \textbf{良序原理 (Well-ordering Principle)},任何严格递减的序列必然在有限步内达到极小值。
在 KOS-TL Runtime 中,极小值点对应于输出集合 $\mathcal{O}$。

\textbf{步骤 4:判定性验证。}

判定性的定义是算法对所有输入均停机。由于:
\begin{itemize}
\item 每一个原子规约步是 Core 层保证可判定的;
\item 总步数受 $\vec{\Delta}$ 强制限制。
\end{itemize}
因此,Runtime 不再具有半可判定性中的“无限搜索”特征,程序变为对输入命题和资源边界的全函数 (Total Function)。
\end{proof}

\section{应用示例:乱序日志的因果修复}

在制造业场景中,若设备异常信号 $s_{ES}$ 因延迟晚于质检信号 $s_{QI}$  到达:

\begin{itemize}
\item \textbf{精化阻塞}:当$s_{QI}$ 到达时,精化算子发现无法构造出“已有设备异常证明”的 $p$,该事件被 Runtime 置入挂起队列。
\item \textbf{证据补齐}:$s_{ES}$ 到达后,运行层更新$\Sigma$。此时调度器检测到环境变化,重新触发 $s_{QI}$的精化。
\item \textbf{逻辑物化}:原本断裂的因果链在逻辑证据补齐后,由内核层完成原子转移,最终由运行层在物理数据库中插入溯源结论。
\end{itemize}


\begin{Example}{工业传感器触发的安全停机}

1. 配置状态 (Configuration)

当前的运行时状态 $\langle \sigma, Q, \textsf{Env} \rangle$ 如下:

$\sigma$ (逻辑快照):设备状态为 Running,温度阈值为 $80^{\circ}C$。

$Q$ (事件队列):$[ \dots ]$(当前为空)。

$\textsf{Env}$ (外部环境):连接着一个 Modbus 协议的温度传感器。

2. 外部流 (External Stream) 与 inject

传感器向系统发送了一个原始比特流:$s$ (Raw Signal): 0x4A 0x02 (代表温度读取值为 $82^{\circ}C$)。

动作:inject(s, Q) 将该十六进制信号推入待处理队列。

3. 精化过程:elaborate(s)

Runtime 层尝试将这个“无意义”的数字转换为 Kernel 层认可的“语义事件”:

精化逻辑:elaborate 查找配置规则,发现 0x4A 是温度警报。

映射结果:映射到 L1 层的事件 $e_{stop}$。

$e_{stop}.\textsf{pre}$:当前状态必须是 Running。

$e_{stop}.\textsf{act}$:将状态改为 Stopped。

$e_{stop}.\textsf{post\_prf}$:证明该操作符合“过温强制保护公理”。

4. 调度与判定 (Scheduling)

按照你给出的调度算法,系统执行如下:Pop: 从 $Q$ 中取出 $s$。Elaborate: $s$ 成功精化为 $e_{stop}$。

Kernel\_Check: Runtime 调用 Kernel 层判断式 $\Gamma \vdash e_{stop} : \textsf{Event}$。

验证通过:该事件携带了正确的 $\textsf{post\_prf}$(即使在 $82^{\circ}C$ 时停机也是符合安全定义的)。

Step: 逻辑状态更新:$\sigma_{new} = \textsf{STEP}(\sigma, e_{stop})$。

5. 持久化:commit 与 Materialize

动作:commit($\sigma_{new}$)。

 物化存储 $\mathcal{M}$:将更新后的状态写入物理数据库(如 PostgreSQL),并触发物理硬件的继电器断开电流。

\end{Example}

\chapter{KOS-TL系统}

将KOS-TL的内核层、核心层和运行层融合在一起便形成了KOS-TL(Knowledge Operating System - Type Logic),也称之为“知行逻辑”。知行逻辑是一个基于直觉依赖类型论并融合小步操作语义的完整逻辑系统。它通过分层架构实现了知识的静态约束、动态演化与环境精化的统一。

\section{总体架构}

KOS-TL 的语法由三层嵌套的表达式构成,涵盖了从抽象类型到物理配置的完整范畴。
\subsection{Core 层:类型定义与逻辑基座 (The Denotational Foundation)}

Core 层是系统的“大脑”,它将领域本体映射为依存类型论。

本体集成:将领域公理定义为基础类型(Base Types)和谓词。

验证机制:基于 $BHK$ 解释的类型检查器,确保每一个 $t:A$ 都是一个合法的知识构造。

职责:提供静态约束。它规定了系统“能理解什么”以及“什么是真理”。

\subsection{Kernel 层:动态演化与意图调度 (The Operational Engine)}

Kernel 层是系统的“心脏”,它负责状态的受控迁移。
状态模型:维护三元组 $\sigma = \langle \mathcal{K}, \mathcal{T}, \mathcal{P} \rangle$(知识、时间、意图)。
演化机制:执行小步操作语义(Small-step Semantics)。它调用 Core 层的判定能力来验证每一次状态跳转。

职责:提供动态一致性。它规定了系统“如何从当前真理演化到下一个真理”。


\subsection{Runtime 层:环境精化与物理执行 (The Physical Interface)}

Runtime 层是系统的“感官与肢体”,它处理与物理世界的边界交互。

精化 (Refinement):通过 $\textsf{elab}$ 算子将模糊的物理信号(Signals)“提升”为 Core 层认可的证明项。

物化 (Materialization):通过 $\mathcal{M}$ 映射将逻辑结论“降解”为持久化存储或硬件指令。

职责:提供保真性。它规定了逻辑指令如何可靠地作用于物理实体。


\subsection{架构全局不变式 (Global Invariant)}

KOS-TL 的 Grand Map 揭示了一个核心法则:

$\forall \text{ 物理变更 } \delta \in \mathcal{M}, \quad \exists \text{ 逻辑证明 } p \in \textsf{Core} \quad \text{s.t.} \quad \textsf{TypeCheck}(p, \textsf{Ontology}) = \textsf{Pass}$

\section{KOS-TL的最小内核}

KOS-TL 的最小内核是一个\textbf{形式化的决策内核},它将整个决策过程压缩为五个核心接口。这些接口紧密协作,以确保每个事件、每个状态变化和每个决策都符合严格的规范约束。

\subsection{事件入口接口 (Event Intake Interface)}

事件是系统状态变化的唯一触发器,KOS-TL 对事件有严格的定义,确保每个事件都具备合法性和可验证性。

\subsection{世界状态接口 (World State Interface)}

世界状态表示系统在某一时刻的\textbf{合法配置},状态只能通过合规的事件进行转移,且每个状态都必须满足合法性约束。

\subsection{规则与合规接口 (Normative Rule Interface)}

KOS-TL 的核心思想在于\textbf{规范性规则}:每个行动都必须满足合规性要求。规则通过形式化推理提供合法性验证,每个行动都可追溯至具体的规范。

\subsection{执行与轨迹接口 (Trace Interface)}

执行轨迹是 KOS-TL 核心的可重放证据,它记录了事件与规则的执行路径,并能够证明某一状态的构造过程。

\subsection{决策输出接口 (Decision Interface)}

KOS-TL 提供了一套决策系统,所有决策都基于执行轨迹进行验证,确保每个决策都是合法、可追溯的,并且有明确的责任归属。


\section{全局交互协议 (Global Interaction Protocol)}

该协议描述了一个物理脉冲如何穿越四层架构,最终固化为全域公认的真理。

\subsection{阶段 I:精化与注入 (Refinement \& Injection)}

\begin{enumerate}[label=(\arabic*)]

    \item 触发方

    Runtime Layer (External Environment)
    \item 动作
    \begin{itemize}
    \item 物理传感器产生原始信号 $s \in Q_{raw}$。
    \item Runtime 调用核心算子 $\textsf{elab}(s, \textsf{Env})$。
    \item 跨层交互:$\textsf{elab}$ 引用 Ontology 层 定义的谓词模板,并在 Core 层 构造出一个依存对证明项 $p : \textsf{Pre}(e, \Sigma)$。
    \item 结果:生成一个合法的意图项 $\langle e, p \rangle$。
    \end{itemize}
\end{enumerate}
\subsection{阶段 II:内核入队与排序 (Kernel Enqueue \& Sequencing)}

\begin{enumerate}[label=(\arabic*)]
    \item 触发方
        Kernel Layer

    \item 动作
    \begin{itemize}
        \item 内核接收来自 Runtime 的意图项。
        \item 调用 Kernel 算子 $\textsf{schedule}(\Sigma, e)$,将事件挂载至意图队列 $\mathcal{P}$。
        \item 此时,系统时钟 $\mathcal{T}$ 保持不变,但 $\Sigma$ 的配置已发生逻辑预占。
    \end{itemize}
\end{enumerate}
\subsection{阶段 III:逻辑规约与判定 (Reduction \& Judgment)}
\begin{enumerate}[label=(\arabic*)]
    \item 触发方
    Kernel Layer (Core Engine)

    \item 动作
    \begin{itemize}
        \item 内核循环调用 $\textsf{peek}(\Sigma)$ 取出队首事件。
        \item 核心校验:依据 Core 层 的类型检查规则进行判定:

  $$\Gamma, \mathcal{K}, \mathcal{T} \vdash p : \textsf{Pre}(e, \Sigma)$$

        \item 规约计算:执行 $\textsf{Op}(e)$。此时,Core 层执行 $\beta$ 与 $\iota$ 规约,计算出新状态的备选项 $\Sigma_{try}$。
        \item 后置闭环:验证 $\Sigma' \vdash p' : \textsf{Post}(e)$。
    \end{itemize}
\end{enumerate}
\subsection{阶段 IV:原子物化与持久化 (Materialization \& Persistence)}
\begin{enumerate}[label=(\arabic*)]
  \item 触发方

  Runtime Layer (Storage Subsystem)
  \item 动作
  \begin{itemize}
  \item 内核将校验通过的 $\Sigma'$ 下发至 Runtime。
  \item Runtime 调用物化映射 $\mathcal{M} \vdash \Sigma' \Downarrow \mathcal{M}'$。
  \item 物理确认:底层数据库返回 ACK,逻辑时钟执行 $\textsf{tick}$,正式完成状态跳转。
  \item 因果锚定:在物理存储中记录转移项 $\delta = \langle \Sigma \xrightarrow{e} \Sigma' \rangle$。
  \end{itemize}
\end{enumerate}
\begin{table}[h]
\centering
\caption{系统演化流程中的实体属性表}
\label{tab:entity-attributes}
\begin{tabular}{|c|l|l|l|l|}
\hline
步骤 & 实体 & 数据形态 & 负责层级 & 属性 \\
\hline
1 & 信号 & 原始位流 (Raw Bits) & Physical & 非确定性 \\
\hline
2 & 证明 & 依存对 $\langle e, p \rangle$ & Runtime/Core & 构造性 \\
\hline
3 & 意图 & 挂起队列 $\mathcal{P}$ & Kernel & 有序性 \\
\hline
4 & 规约 & $\lambda$-项演化 & Core/Kernel & 确定性 \\
\hline
5 & 事实 & 持久化知识 $\mathcal{K}$ & Runtime & 不可篡改性 \\
\hline
\end{tabular}
\end{table}

协议的一致性保证 (Global Invariant)

该协议强制执行一个全局不变式:

“物理存储的任何比特翻转,必须存在一条从 Ontology 延伸至 Core 的完整证明链条。”

这意味着 KOS-TL 系统不存在“未定义的行为”。任何不满足该协议路径的操作(如非法注入、证明缺失、时钟倒流)都会在各自的层级被自动拦截,并回滚至上一个良构状态 $\Sigma_{last}$。


\section{交互界面}

\subsection{Core 与 Kernel 的交互界面:类型判定接口 ($\textsf{Logic-Kernel Interface}$)}

\begin{itemize}
\item \textbf{方向}:Kernel 调用 Core。

\item \textbf{交互内容}:Kernel 将当前的意图 $e$ 及其携带的证明项 $p$ 提交给 Core。

\item \textbf{界面原语}:$\textsf{check}(\Gamma, p, \textsf{Pre}(e))$ 与 $\textsf{reduce}(\textsf{Op}(e), \sigma)$。

\item \textbf{属性}:内涵性。它是纯逻辑的,不感知物理时间或硬件状态。
\end{itemize}

\subsection{Kernel 与 Runtime 的交互界面:演化驱动接口 ($\textsf{Kernel-Runtime Interface}$)}

\begin{itemize}
\item \textbf{方向}:双向。

\item \textbf{交互内容}:
  \begin{itemize}
  \item \textit{Upward} (Runtime $\to$ Kernel):提供精化后的事件对 $\langle e, p \rangle$ 压入队列。
  \item \textit{Downward} (Kernel $\to$ Runtime):下发经过校验的新状态 $\sigma'$ 请求物化。
  \end{itemize}

\item \textbf{界面原语}:$\textsf{schedule}(e, p)$ 与 $\textsf{commit}(\sigma')$。

\item \textbf{属性}:原子性。确保逻辑状态的跳转与物理存储的更新同步。
\end{itemize}

\subsection{Core 与 Runtime 的横向依存:精化模板接口 ($\textsf{Refinement Interface}$)}

\begin{itemize}
\item \textbf{方向}:Runtime 引用 Core。

\item \textbf{交互内容}:Runtime 的 $\textsf{elab}$ 算子需要引用 Core 层定义的本体(Ontology)模板来构造合法证明。

\item \textbf{属性}:构造性。保证从物理信号提取的数据符合逻辑定义的排序(Sorts)。
\end{itemize}

\begin{figure}[h]
\centering
\begin{tikzpicture}[
    scale=0.65, transform shape,
    node distance=1.5cm,
    layer/.style={rectangle, rounded corners, draw=black, very thick, minimum width=8cm, minimum height=2.5cm, fill=white},
    interface/.style={fill=blue!10, draw=blue!50, dashed, thick, rounded corners, minimum width=6cm, minimum height=0.8cm},
    dataflow/.style={thick, color=orange!80!black},
    font=\sffamily
]
% --- Layers ---
\node[layer, fill=red!5] (core) {
    \begin{tabular}{c}
    \textbf{Core Layer (逻辑基座)} \\
    \footnotesize 依存类型检查 / $\beta,\iota$-规约 / 本体定义 ($\Gamma$)
    \end{tabular}
};
\node[layer, fill=green!5, below=of core] (kernel) {
    \begin{tabular}{c}
    \textbf{Kernel Layer (演化内核)} \\
    \footnotesize 状态三元组 $\sigma \langle \mathcal{K}, \mathcal{T}, \mathcal{P} \rangle$ / 意图调度 / 小步转移
    \end{tabular}
};
\node[layer, fill=blue!5, below=of kernel] (runtime) {
    \begin{tabular}{c}
    \textbf{Runtime Layer (执行环境)} \\
    \footnotesize 信号精化 (elab) / 物理存储 (M) / 硬件交互
    \end{tabular}
};
% --- Interfaces ---
\node[interface] (int1) at ($(core.south)!0.5!(kernel.north)$) {\footnotesize \textbf{Interface: Type Judgment} (判定项 $p$, 谓词 $Pre$)};
\node[interface] (int2) at ($(kernel.south)!0.5!(runtime.north)$) {\footnotesize \textbf{Interface: Commit/Schedule} (新状态 $\sigma'$, 事件 $e$)};
% --- Data Flows ---
% Runtime to Kernel
\draw[dataflow] (runtime.west) -- ++(-1,0) |- node[pos=0.25, left, align=right] {精化事件\\$\langle e, p \rangle$} (kernel.west);
% Kernel to Core
\draw[dataflow] (kernel.east) -- ++(1,0) |- node[pos=0.25, right, align=left] {逻辑校验请求\\$\Gamma \vdash p : A$} (core.east);
% Core to Kernel
\draw[dataflow] ($(core.south west)!0.3!(core.south)$) -- ($(kernel.north west)!0.3!(kernel.north)$)
    node[midway, left] {\scriptsize 规约结果};
% Kernel to Runtime
\draw[dataflow] ($(kernel.south east)!0.3!(kernel.south)$) -- ($(runtime.north east)!0.3!(runtime.north)$)
    node[midway, right] {\scriptsize 物化指令};
% External World
\node[draw, right=2cm of runtime, fill=gray!10] (env) {物理世界};
\draw[dashed, <->] (runtime) -- (env) node[midway, above] {\scriptsize 信号/IO};
\end{tikzpicture}
\caption{KOS-TL分层交互界面图}
\label{figure:KOSinterface}
\end{figure}

KOS-TL 通过$\Sigma$ -类型将“知”(静态知识与证明)与“行”(动态状态转换)耦合。在整体视角下,它是一个自洽的、可计算的逻辑实体:Core 提供语义框架,Kernel 提供演化动力,Runtime 提供环境映射。这种架构使得复杂系统不仅能够存储数据,更能通过逻辑归约实现因果追溯与合规性自我验证。

\section{系统性质}

\begin{Theorem}{知识单调性 - Monotonicity}

设 $\sigma$ 为内核知识库(即已接受的事实集),$ku$ 为一良构的知识对象使得 $\Gamma \vdash ku : A$。若 $\sigma$ 满足 $ku$ 的引入条件(记作 $\sigma \vdash ku$),则对于任何满足演化相容性的后续状态 $\sigma'$,若不存在针对 $ku$ 的冲突证明 $\pi$ (即 $\sigma' \nvdash \textsf{refute}(ku)$),则:

$$\sigma \subseteq \sigma' \implies (\sigma' \vdash ku)$$

即:已确立的真理在知识库的有效扩张下保持不变。
\end{Theorem}

\begin{proof}
我们通过 Kripke 语义框架 (Kripke Semantics) 和 构造性逻辑的弱化引理 (Weakening Lemma) 进行证明:

\textbf{1. 建立框架扩张模型:}
我们将内核状态演化定义为一个 Kripke 框架 $\langle W, \le, \Vdash \rangle$,其中:
\begin{itemize}
\item $W$ 是所有可能的知识库状态集合。
\item $\le$ 是定义在 $W$ 上的偏序关系,$\sigma \le \sigma'$ 表示 $\sigma'$ 是 $\sigma$ 的一个合法演化后继。
\item $\Vdash$ 是强迫关系,$\sigma \Vdash ku$ 表示在状态 $\sigma$ 下,知识对象 $ku$ 的证明项是可构造的。
\end{itemize}

\textbf{2. 证明核心层的持久性 (Persistence):}
KOS-TL 的 Core 层基于直觉主义类型论(Intuitionistic Type Theory)。在直觉逻辑中,所有算子($\Pi, \Sigma$ 等)都满足持久性。我们对 $ku$ 的证明结构进行归纳:
\begin{itemize}
\item \textbf{基础项}:若 $ku$ 是一个原子事实(如物理常数或已验证的 ID),根据 Kripke 模型定义,若 $\sigma \Vdash ku$ 且 $\sigma \le \sigma'$,由于 $\sigma \subseteq \sigma'$,则 $ku$ 及其原始证明证据在 $\sigma'$ 中依然存在。
\item \textbf{复合项}:若 $ku = \langle v, p \rangle$ 是一个依存对。根据归纳假设,$v$ 的值在扩张中保持不变。对于证明项 $p$,由于 $\sigma'$ 仅仅增加了新的事实而未引入针对 $p$ 的反证(由定理前提保证),根据类型论的 \textbf{弱化引理 (Weakening Lemma)},$\Gamma, \sigma \vdash p : P \implies \Gamma, \sigma' \vdash p : P$ 依然成立。
\end{itemize}

\textbf{3. 排除因果撤销 (Exclusion of Cancellation):}
在 KOS-TL 中,只有当内核显式构造出矛盾项 $\textsf{contra}(ku)$ 时,该项才会从“当前活跃库”移动到“历史存档区”。
若 $\sigma' \nvdash \textsf{refute}(ku)$,则说明在 $\sigma'$ 的搜索空间内没有能与 $ku$ 发生归约坍缩的反对证据。因此,逻辑演化算子 $\textsf{unify}$ 会保持 $ku$ 的可访问性。

\textbf{结论}:$\sigma' \Vdash ku$ 成立,知识具有单调性。
\end{proof}

性质讨论:对“因果溯源”的意义
知识单调性(Knowledge Monotonicity)解决了复杂系统中的“记忆不一致”问题:

证据持久性:它保证了如果银行系统在 T1 时刻证明了一笔交易是合规的,除非在 T2 时刻发现了证据造假(反证),否则该合规性结论永远不会因为数据库清理或其他交易的增加而莫名消失。

决策一致性:这使得基于 KOS-TL 构建的无人系统(如自动驾驶)能够维持长期的环境认知,避免因为处理新传感信息而“遗忘”了之前的安全边界。

\begin{Theorem}{计算反射性 - Reflexivity}

在 KOS-TL 内核中,存在一个反射算子 $\textsf{reflect}$,使得对于任何良构的项 $t \in \mathcal{D}_{Core}$ 及其在 Kernel 层发生的演化步 $\textsf{step} : t \xrightarrow{small} t'$,系统能够自动合成一个内部证明项 $\pi$,满足:

$$\Gamma \vdash \pi : \textsf{EvalPath}(t, t')$$

其中 $\textsf{EvalPath}$ 是一个依存类型,记录了从 $t$ 到 $t'$ 的所有公理化推导序列。这意味着内核的每一次状态变迁都附带一份关于其自身合法性的“元证明”。
\end{Theorem}

\begin{proof}
我们通过 Martin-Löf 类型论中的恒等类型 (Identity Types) 和 元循环映射 (Meta-circular Mapping) 进行证明:

\textbf{1. 规约步的代数映射:}
由于 KOS-TL 的 Core 层基于纯粹的、无副作用的依存类型演算,其计算语义是引用透明的 (Referential Transparent)。每一个规约步 $t \xrightarrow{small} t'$ 并非内存的随机抹写,而是应用了一个具体的归约规则(如 $\beta$-reduction 或 $\iota$-reduction)。

\textbf{2. 证明项的自动合成 (Synthesis):}
对于内核执行的每一类基本规约,我们定义映射函数 $\mathcal{R}$:
\begin{itemize}
\item \textbf{Beta 规约}:当执行 $(\lambda x. M) N \to M[N/x]$ 时,内核利用内部公理 $\textsf{beta\_axiom}$ 构造 $\pi = \textsf{refl}_{\beta}(M, N)$。
\item \textbf{Iota 规约}:当执行 $\textsf{proj}_1 \langle a, b \rangle \to a$ 时,内核利用 $\textsf{proj\_axiom}$ 构造 $\pi = \textsf{refl}_{\iota}(a, b)$。
\end{itemize}
由于所有的规约规则都在 Core 层有对应的公理定义,内核在执行计算的同时,可以同步记录所使用的公理序列。

\textbf{3. 利用 $J$-消解算子建立等价性:}
在依存类型论中,等价类型 $\textsf{Id}_A(t, t')$ 的唯一构造子是 $\textsf{refl}$。根据 $J$-算子(Identity Elimination),如果两个项在逻辑规约意义下是等价的,则它们在所有逻辑谓词下是不可区分的。
通过将 Kernel 的每一步执行动作 $t \to t'$ 映射为 $J$-算子的应用过程,内核实际上是在不断构造一个关于“我为何从 $t$ 变到 $t'$”的数学证词。

\textbf{4. 元循环自审:}
存在一个子程序 $\textsf{Audit} \subset \textsf{Kernel}$,该程序接受证明项 $\pi$ 和路径 $\textsf{EvalPath}$ 作为输入。由于 KOS-TL 具备强规范化性质,$\textsf{Audit}$ 能够在有限步内验证 $\pi$ 是否确实支撑了从 $t$ 到 $t'$ 的转换。
\end{proof}

性质讨论:对“自主系统”的意义
计算反射性(Computational Reflexivity)将 KOS-TL 提升到了\textbf{“自觉系统”}的高度:

全时自动审计:传统系统需要外部审计日志,而 KOS-TL 的日志就是它的执行路径。这意味着审计不是“事后烟”,而是“事前证明”。

决策透明化:在自动驾驶或金融交易中,当系统做出一项决策(如紧急避障或拦截转账)时,反射性保证了系统能够立即输出一份人类可读且数学有效的“合规性解释报告”。

自修复的逻辑依据:当系统检测到硬件精化映射出现偏差时,它能通过反射性对比“预期路径”与“实际路径”的逻辑差异,从而精确定位导致冲突的逻辑算子。

\begin{Theorem}{全系统安全性 - System-Wide Safety}

设 $\mathcal{S}$ 为系统的物理状态空间,$\textsf{Safe} \subseteq \mathcal{S}$ 为预定义的物理安全子集。若 KOS-TL 系统的初始化状态 $s_0 \in \textsf{Safe}$,则对于任何物理演化序列 $s_0 \xrightarrow{hw} s_1 \xrightarrow{hw} \dots \xrightarrow{hw} s_n$,始终满足:

$$\forall i \ge 0, \quad s_i \in \textsf{Safe}$$

前提条件:
\begin{enumerate}
\item Core 层满足一致性(Consistency)。
\item Kernel 层满足进度性(Progress)与演化相容性。
\item Runtime 层满足精化保真性(Refinement Fidelity)。
\end{enumerate}
\end{Theorem}

\begin{proof}
证明采用分层归纳法,将物理演化映射为逻辑证明项的规约。

\textbf{1. 基础情形 (Base Case)}
对于初始状态 $s_0$,根据 Runtime 的精化保真性:

$$\mathcal{E}(s_0) = ku_0 \quad \text{且} \quad \Gamma \vdash ku_0 : \textsf{Qualified}(s_0)$$

由于 $s_0 \in \textsf{Safe}$,在 Core 层中对应的谓词 $\textsf{is\_safe}(\textsf{proj}_1(ku_0))$ 的证明项 $p_0$ 存在。

\textbf{2. 归纳步骤 (Inductive Step)}
假设系统在第 $i$ 步处于 $s_i \in \textsf{Safe}$。考虑向第 $i+1$ 步的迁移:

\textbf{A. 物理扰动与精化:}
当物理环境发生改变 $s_i \xrightarrow{hw} s_{i+1}$(如传感器数值变化或硬件故障),Runtime 立即捕捉该变化并尝试构造新的知识对象 $ku_{i+1}$:

$$\mathcal{E}(s_{i+1}) = ku_{i+1}$$

\textbf{B. 内核逻辑判定:}
内核将 $ku_{i+1}$ 提交给 unify 算子。此时产生两种分支:
\begin{itemize}
\item \textbf{分支 1:$s_{i+1}$ 仍属于 Safe}:
内核能够基于 Core 层规则成功构造出 $p_{i+1} : \textsf{is\_safe}(s_{i+1})$。根据 Kernel 的演化相容性,状态 $\sigma$ 更新为包含 $ku_{i+1}$ 的新状态,安全性得以保持。
\item \textbf{分支 2:$s_{i+1}$ 试图越过 Safe 边界}:
此时,在 Core 层中无法构造出类型为 $\textsf{is\_safe}(s_{i+1})$ 的证明项。
根据 Core 层一致性(不能证明伪命题),内核的逻辑引擎会产生一个规约阻塞(或类型冲突)。
\end{itemize}

\textbf{C. 闭环自愈 (Self-healing Loop):}
根据 Kernel 进度性,内核不会卡死,它会转而执行 find\_root\_cause 并触发 analyze。
Runtime 接收到内核发出的安全指令 $\pi$,根据实时可观察性,该指令在物理层强制执行(如熔断、切换冗余路径),将物理状态拉回到 $s'_{i+1} \in \textsf{Safe}$。

\textbf{3. 矛盾归约 (Reductio ad Absurdum)}
假设存在某个 $s_j \notin \textsf{Safe}$:

\begin{enumerate}
\item 这意味着 Runtime 必须精化出一个逻辑项 $ku_j$,使得 $\Gamma \vdash ku_j : \textsf{is\_safe}$。
\item 而 $s_j \notin \textsf{Safe}$ 意味着 $\textsf{is\_safe}(s_j)$ 在 Core 层等价于 $\bot$(伪命题)。
\item 那么推导出 $\Gamma \vdash ku_j : \bot$。
\item 这违反了 Core 层一致性定理(系统中不存在伪命题的项)。
\item 故 $s_j \notin \textsf{Safe}$ 在逻辑上不可构造。
\end{enumerate}
\end{proof}

\begin{table}[htbp]
\centering
\caption{KOS-TL 逻辑性质全图谱 (The Logical Spectrum of KOS-TL)}
\footnotesize
\setlength{\tabcolsep}{4pt}
\begin{tabularx}{\linewidth}{@{} >{\raggedright}p{2.0cm} >{\centering}p{1.5cm} >{\raggedright}p{2.0cm} >{\raggedright\arraybackslash}X @{}}
\toprule
\textbf{性质名称} & \textbf{归属层级} & \textbf{核心价值} & \textbf{形式化隐喻 / 定义} \\
\midrule
一致性 (Consistency) & Core & 根绝逻辑矛盾 & $\sigma \nvdash \bot$ \\
强规范化 (Normalization) & Core & 确保实时响应 & $\forall t,\, \exists v{:}\,\text{NormalForm},\; t \twoheadrightarrow v$ \\
进度性 (Progress) & Kernel & 持续自愈运行 & $\mathcal{C} \notin \text{Final} \Rightarrow \exists \mathcal{C}'{:}\; \mathcal{C} \xrightarrow{\text{small}} \mathcal{C}'$ \\
演化相容性 (Evolutionary) & Kernel & 安全状态演化 & $\sigma \xrightarrow{T} \sigma' \Rightarrow \text{TypeCheck}(\sigma') = \text{Success}$ \\
单调性 (Monotonicity) & \textbf{全系统}/Kernel & 因果证据持久 & $\sigma \subseteq \sigma' \Rightarrow (\sigma \Vdash ku \Rightarrow \sigma' \Vdash ku)$ \\
保真性 (Fidelity) & Runtime & 物理映射不失真 & $(s, ku) \in \text{SimulationRelation}$ \\
可观察性 (Adequacy) & Runtime & 指令下达无损 & $\text{Exec}(\mathcal{G}(\textit{ctrl}), s) \models \text{Refine}^{-1}(\phi)$ \\
反射性 (Reflexivity) & \textbf{全系统} & 全路径审计追踪 & $\forall t \to t',\; \exists \pi{:}\,\text{Id}(t, t')$ \\
\bottomrule
\end{tabularx}
\end{table}


\section{Trace = Proof 定理}

\begin{theorem}{Trace as Proof of Legitimacy}

设 $\mathcal{K}$ 为满足 KOS-TL 接口定义的演化内核。对于任意状态 $\sigma \in \mathsf{State}$ 与行动 $a \in \mathsf{Action}$,以下命题等价:

\begin{enumerate}
    \item 行动 $a$ 在状态 $\sigma$ 中是逻辑合法的,即:$\mathsf{decide}(\sigma, a) \neq \mathsf{Forbidden}$;
    \item 存在一条有限执行轨迹 $\mathcal{T} = \sigma_1 \xrightarrow{e_1} \sigma_2 \xrightarrow{e_2} \dots \xrightarrow{e_{n-1}} \sigma_n$,满足 $\sigma_n = \sigma$,且 $a$ 在该轨迹的演化步骤中被实例化,并满足谓词 $\mathsf{valid\_trace}(\mathcal{T})$;
    \item 行动 $a$ 所诱导的后继状态 $\sigma'$ 在内核 $\mathcal{K}$ 中是可构造的证明项(Constructible Term)。
\end{enumerate}

反之,若在当前知识库下不存在满足上述条件的轨迹 $\mathcal{T}$,则对应行动或状态在逻辑上归约为底类型($\bot$),即不可构造。
\end{theorem}

\section{KOS-TL的核心适用场景}

KOS-TL 作为一种基于事件驱动的决策逻辑系统,适用于以下核心场景:

\subsection{根因追溯 (Root Cause Analysis)}

在面对复杂系统时,KOS-TL 能通过回溯\textbf{执行轨迹}来精确定位导致问题的“根因”。KOS-TL 不仅回答“为什么发生了某件事”,而且通过形式化的 \textbf{trace} 证明了每一个操作和事件的合规性与合法性。

\subsection{反事实推理 (Counterfactual Reasoning)}

反事实推理通常要求对一个系统状态进行假设性分析,KOS-TL 能提供“合法假设”,并通过可构造性推导出假设的后果。这一点与传统的统计推理(基于概率的假设)有所区别,KOS-TL 使得反事实推理具有“形式化验证”的能力。

\subsection{合规性决策系统 (Normative Decision Systems)}

KOS-TL 使得决策不再是“简单的预测模型”,而是“合规性推理”。系统不仅依赖数据和算法,“每个决策”背后都可以通过合法的“trace”被证明其合法性。这使得 KOS-TL 特别适合于高风险领域,如金融、医疗、军事等。

\subsection{审计与问责 (Audit and Accountability)}

KOS-TL 的执行轨迹可以为每一个决策提供“可重放的证明”,这些证明能够准确揭示出每个操作的责任归属及合法性,支持高效的“审计与问责”。例如,在监管合规和法律系统中,KOS-TL 的这种特性尤为重要。

\subsection{复杂系统运行与治理 (Complex Systems Governance)}

对于复杂的基础设施和多层次系统,KOS-TL 提供了一个完整的“状态演化与决策系统”。无论是电网、交通管理、还是供应链管理,KOS-TL 都能精确定义每一步操作的合法性,并追溯每一个决策。

\subsection{AI 治理与可信性 (AI Governance and Trustworthy AI)}

在 AI 系统中,KOS-TL 可以被用于确保“AI 决策的可信性”,不仅仅是“为什么 AI 给出这个建议”,而是“在规范范围内,AI 能够做出的所有可能建议”都可以通过 KOS-TL 的形式化证明进行审查与验证,避免 AI 系统产生不合规或不合法的建议。


\section{KOS-TL 在知识全生命周期中的本体论角色}

这是一个高度“本体论级别”的命题。从最抽象的层面审视,KOS-TL(Knowledge Operation System Type Logic)并非是在替代传统的知识理论,而是在重新规定知识在系统中的存在方式与生命周期机制。

在 KOS-TL 的视角下,核心命题从“系统知道什么(Knowing what)”转向了“知识如何活着(How knowledge lives)”。

\subsection{核心定义:作为“操作宪法”的执行内核}
KOS-TL 在知识全生命周期中扮演的是一种\textbf{“知识操作宪法 + 执行内核”}的角色。它规定了什么样的知识可以被承认为知识、知识如何被生成、演化、失效以及如何被追责。

简而言之:KOS-TL 关心的不是\textbf{认识论真理(Epistemic Truth)},而是\textbf{操作合法性(Operational Legitimacy)}。

\subsection{知识发现阶段:作为合法性过滤器}
在传统视角中,知识发现通常遵循“数据 $\to$ 模型 $\to$ 结论”的路径,关注统计显著性或经验支持。

而在 KOS-TL 视角下,发现并不等于进入知识系统。任何“候选知识”必须满足一个构造性前提:
\begin{quote}
    \textit{该知识能否被构造成一个 $\Sigma$-对象(即:数据 $d$ + 证明项 $p$)?}
\end{quote}

\textbf{角色定位:知识准入控制(Epistemic Admission Control)}
\begin{itemize}
    \item 原始信号通过 $\mathsf{elab}$ 算子进行提炼。
    \item 只有能构造出类型正确、约束满足且证据可携带的项,才能被“提升”为知识事件。
    \item \textbf{本质转变:} 将“发现”从认知问题转化为构造问题——不是“我认为这是知识”,而是“我能否构造它”。
\end{itemize}

\subsection{知识生成阶段:作为知识生成机制本身}
这是 KOS-TL 最具原创性的部分。传统系统中,知识生成表现为推理结果或规则触发的“新断言”。

在 KOS-TL 中,\textbf{生成即演化}。没有凭空新增的事实,每一条新知识都是一次受控的状态迁移:
$$\sigma \xrightarrow{\langle e, p \rangle} \sigma'$$
该过程伴随着前置条件的验证、后置证明的自动合成以及因果链的实时记录。

\textbf{角色定位:知识生成的物理定律}
\begin{itemize}
    \item 规定了哪些生成路径是允许的,哪些在类型层面上“不存在”。
    \item 知识不是被“推出”的,而是通过 $\mathsf{STEP}$ 算子“执行”出来的。
\end{itemize}

\subsection{知识演化阶段:作为时间化的知识本体}
传统知识观倾向于认为知识是超越时间的(Timeless),演化仅表现为版本覆盖。KOS-TL 则认为知识天然嵌入在时间与因果之中,不存在“脱离轨迹的真理”。

\textbf{角色定位:知识因果历史学家 + 执行法官}
\begin{itemize}
    \item 一个命题的意义由其“如何到达此处”、“依赖了哪些事件”以及“在什么条件下成立”共同定义。
    \item 回滚(Rollback)不是状态修改,而是轨迹(Trace)重演。
    \item 冲突(Conflict)不是逻辑不一致,而是演化的不可执行。
\end{itemize}

\subsection{知识失效与判定阶段:作为裁决系统}
在 KOS-TL 中,知识不会被简单地“删除”。它通过轨迹的演进发生“逻辑死亡”:
\begin{itemize}
    \item \textbf{失效:} 知识在新的轨迹下变得不可达,或其前提条件不再可构造。
    \item \textbf{否定:} 表现为新事件使得原轨迹不再可延续。
    \item \textbf{反事实(Counterfactual):} 表现为构造另一条可执行但未发生的轨迹。
\end{itemize}

\textbf{角色定位:状态机裁决者}
否定不是逻辑反驳,而是执行失败;责任归因是在因果链中定位断点。

\subsection{知识全生命周期抽象图景总结}
\begin{table}[h]
\centering
\caption{KOS-TL 与传统系统的角色对比}
\begin{tabular}{|l|l|l|}
\hline
\textbf{生命周期阶段} & \textbf{传统系统} & \textbf{KOS-TL 的角色} \\ \hline
发现 (Discovery) & 认知 / 统计模型 & 合法性过滤器 ($\mathsf{elab}$) \\ \hline
生成 (Generation) & 推理结论 / 断言 & 可执行的构造过程 \\ \hline
演化 (Evolution) & 版本变化 / 覆盖 & 小步轨迹验证 ($\mathsf{STEP}$) \\ \hline
一致性 (Consistency) & 周期性检查 & 类型检查 + 证明合成 \\ \hline
失效 (Expiration) & 删除 / 覆盖 & 路径不可执行 \\ \hline
追责 (Traceability) & 外部审计日志 & 内生因果证明 \\ \hline
\end{tabular}
\end{table}

\subsection{结论:哲学层面的重构}
如果总结 KOS-TL 在认识论(Epistemology)上的立场:
\begin{quote}
    \textbf{知识不再是“被相信的命题”,而是“被证明可执行的状态演化”。}
\end{quote}
KOS-TL 成功地将认识论下沉为操作系统设计(Operating System Design)。

\section{KOS-TL 对知识非单调性的支持与重构}

KOS-TL 不仅支持知识的非单调性,而且把“非单调性”从一种补丁式的逻辑特性,提升为知识系统的结构性事实。本节将从“表现形式”、“内生必然性”以及“与经典非单调逻辑的本质差异”三个维度进行阐述。

\subsection{非单调性的内生表现}
在 KOS-TL 中,非单调性是内生的而非附加的。以下现象在系统中是常态而非异常:
\begin{quote}
    \textit{已经成立的知识,在引入新的事件、规则或约束后,不再成立、不再可用,甚至不再“可构造”。}
\end{quote}
关键点在于:在 KOS-TL 中,知识不是被“逻辑否定”,而是\textbf{“失去可执行性”}。这一区别构成了 Zhi-Xing Logic 的核心。

\subsection{KOS-TL 非单调性的必然性}
KOS-TL 必然是非单调的,因为它从根本上否定了“知识是静态命题集合”和“推理是集合闭包运算”的传统假设。其基本对象是状态演化序列:
$$\sigma_0 \xrightarrow{e_1} \sigma_1 \xrightarrow{e_2} \cdots \xrightarrow{e_n} \sigma_n$$

其必然性源于以下三个层面:
\begin{enumerate}
    \item \textbf{前提的可变性:} 知识的成立依赖于其前提的可构造性。一旦新事件 $e_{n+1}$ 改变了可用规则或引入了更强约束,原有的知识可能直接失去构造路径。这在语义上表现为“此路不通”。
    \item \textbf{小步演化的时间性:} KOS-TL 假设世界是一步步被“执行”出来的。新事件不等于新事实,而是对未来可达状态空间的重塑。旧结论在新状态空间中可能不再可达,这是一种时间化的非单调性。
    \item \textbf{否定作为“执行终止”:} 系统中没有中心化的“真值翻转”机制。旧知识仍存在于历史中,但在当前轨迹下无法再现、无法延续、无法作为前提。它不是 \textit{false},而是 \textit{inapplicable}。
\end{enumerate}

\subsection{与经典非单调逻辑的根本差异}
KOS-TL 与传统非单调逻辑(如默认逻辑、AGM 信念修正等)有着本质不同:

\subsubsection{与默认逻辑(Default Logic)的区别}
\begin{itemize}
    \item \textbf{默认逻辑:} 关注结论是否可撤销,以规则冲突为中心,将非单调视为一种“推理策略”。
    \item \textbf{KOS-TL:} 关注路径是否可再执行,以轨迹可达性为中心,将非单调视为一种“时间事实”。它不问结论是否应成立,而问从当前状态出发是否还能到达该结论。
\end{itemize}

\subsubsection{与 AGM 信念修正的区别}
\begin{itemize}
    \item \textbf{AGM:} 核心是集合论的,关注如何在最小代价下修改信念集合(“改想法”)。
    \item \textbf{KOS-TL:} 核心是动力学的,关注在新事件发生后,哪些轨迹仍然合法(“改世界线”)。
\end{itemize}

\subsubsection{与 ASP(Answer Set Programming)的区别}
ASP 的非单调性来自全局稳定模型的重新选择;而 KOS-TL 是局部、前向、因果敏感的小步变化。系统不需要重新计算整个模型,而是在当前状态继续演化。

\subsection{结论:非单调性的本质}
若概括 KOS-TL 的非单调性本质:
\begin{quote}
    \textbf{KOS-TL 的非单调性不是“结论被推翻”,而是“世界线被剪枝”。}
\end{quote}
更技术化地表述:非单调性在 KOS-TL 中表现为\textbf{可执行轨迹集合随事件单调收缩},而非知识集合的反复修正。

\subsection{哲学落点:责任驱动的知识论}
传统非单调逻辑隐含的是“人的无知导致推理必须保守”;KOS-TL 隐含的则是“世界的演化要求知识必须承担时间与责任”。因此,KOS-TL 是一种\textbf{“责任驱动的非单调知识论”},它将认识论(Epistemology)彻底下沉到了系统执行(Execution)的维度。

\section{特性与应用}

在跨国银行的合规审计中,处理数亿条 Swift 流水不仅是“大数据”挑战,更是“逻辑准确性”挑战。传统系统通常在“统计异常检测”(如发现大额频繁转账)和“规则硬编码”之间挣扎,容易产生海量误报。引入 KOS-TL 后,合规审计从“概率性黑盒”转变为“形式化因果系统”。

下面我们详细展开这一融合案例的逻辑架构与执行流程。

1. 逻辑抽象:定义“反洗钱公理” (AML Axioms)在 KOS-TL 中,合规性不是数据库的一个 flag 字段,而是一个证明目标(Proof Goal)。

A. 资金流向的不变性(Invariants)我们通过 Core 层的依存类型定义合规转账。一个合规转账 $T$ 必须满足:$$\textsf{ValidTx} \equiv \Sigma(t : \textsf{TxData}). \Sigma(e : \textsf{Evidence}). \textsf{CheckCompliance}(t, e)$$其中:$t$: Swift 报文数据(汇款人、收款人、金额)。$e$: 业务逻辑证明(贸易合同、报关单的 Hash 等)。$\textsf{CheckCompliance}$: 一个逻辑函数,它要求 $t$ 与 $e$ 必须在语义上对齐(例如:货品价值与转账金额在逻辑误差范围内)。

B. 拓扑公理:无环性(Acyclicity)洗钱的核心特征是“分拆与整合(Layering \& Integration)”,通常表现为资金在多个实体间兜圈子最终回到原点。公理定义:定义一个路径类型 $\textsf{Path}(A, A)$,如果能构造出该类型的项且资金属性未发生本质变化,则判定为逻辑矛盾(Contradiction)。

2. 融合执行过程:从海量数据到逻辑证据

第一阶段:大规模筛选(Database 侧 - 效率优先)底层数据库(如 ClickHouse 或图数据库 Neo4j)利用其强大的并发能力,执行初步的图算法。

任务:在数亿条记录中寻找“疑似环路”或“高风险节点关联”。结果:筛选出 10,000 条可疑链条。此时,这些链条仅具有“统计学嫌疑”,尚未定性。

第二阶段:证据请求与精化(Runtime 侧 - 保真性)KOS-TL 内核接管这 10,000 条链条。对于每一条链条,Runtime 模块向相关业务系统发起证据回填请求。

操作:请求该笔 Swift 交易对应的底层合同(Contract)和提单(Bill of Lading)。

保真性体现:证据被精化为 $ku_{evidence}$,并带上不可篡改的时间戳和来源证明。第三阶段:依存类型校验(Core/Kernel 侧 - 严谨性)这是 KOS-TL 的核心步骤。内核尝试为每一条可疑链条构造一个\textbf{“合规性证明项”} $p$。

\begin{Theorem}{交易合规性判定 - Transaction Compliance}

 对于可疑链条 $L = \{t_1, t_2, \dots, t_n\}$,若要标记为 \textsf{Verified},必须构造出一个总证明项:$$P_{total} = \langle p_1, p_2, \dots, p_n \rangle$$使得每一项 $p_i$ 都能证明该笔交易对应的证据 $e_i$ 能够消解该链条形成的“环路猜想”。

\end{Theorem}

逻辑拦截:如果某笔转账是虚假贸易(金额与合同 Hash 对不上),Core 层将无法生成证明项。后果:由于 一致性定理,内核无法将该链条状态演化为 Verified。

场景描述:冷却系统异常物理状态 $s$:冷却泵 A 的压力传感器输出电压异常波动。安全目标 $\textsf{Safe}$:压力必须维持在 $[P_L, P_H]$ 区间内,且传感器必须具有有效的校准证明。

第一步:Runtime 层的精化与保真 (Refinement \& Fidelity)当硬件产生信号时,Runtime 并不直接转发数值,而是执行精化函数 $\mathcal{E}$。 物理捕捉:传感器产生原始电平 $2.4V$。构造知识对象:Runtime 根据硬件寄存器中的单位配置,构造出:$$ku_{press} = \langle 120kPa, p_{calib} \rangle : Press$$ 性质体现:精化保真性确保了 $120kPa$ 与其校准证据 $p_{calib}$ 被强绑定。如果没有 $p_{calib}$,Runtime 无法构造出类型为 $Press$ 的项。

第二步:Kernel 层的演化与单调性 (Evolution \& Monotonicity)内核接收到 $ku_{press}$,开始更新全局状态 $\sigma$。知识合一 (Unify):内核尝试将新数据并入知识库。$$\sigma_{new} = \textsf{unify}(\sigma, ku_{press})$$性质体现:知识单调性确保了之前记录的“冷却泵 A 处于开启状态”这一事实不会消失。新旧知识在逻辑空间中共存,形成完整的因果链。

第三步:Core 层的规约与一致性 (Reduction \& Consistency)内核发起安全审计,调用 analyze 函数检查压力是否在安全阈值内。规约执行:内核执行 $\beta$-规约,将 $120kPa$ 代入谓词 $\textsf{is\_safe}(v) \equiv (P_L \le v \le P_H)$。逻辑拦截:假设 $P_H = 110kPa$。由于 $120 > 110$,Core 层判定证明项 $p_{safe} : \textsf{is\_safe}(120)$ 不可构造。性质体现:一致性定理保证了系统不能假装它是安全的。内核无法生成“安全报告”,规约步被导向了“根因分析”路径。

第四步:反射性审计与自愈 (Reflexivity \& Self-healing)内核生成自愈指令 $\textit{ctrl}$ (如:开启备用泵 B,关闭故障泵 A)。反射证明:内核生成一份证据 $\pi$。$$\pi : \textsf{Id}(\sigma_{fault}, \sigma_{recovery})$$性质体现:计算反射性让内核能向人类操作员或审计日志证明:“我关闭 A 泵是因为其压力 $120kPa$ 违反了 Core 层定义的 $110kPa$ 上限,整个决策符合逻辑宪法。”

第五步:Runtime 层的可观察执行 (Observational Adequacy)指令 $\textit{ctrl}$ 到达 Runtime。指令映射:Runtime 将逻辑指令 $\textsf{Close(Pump\_A)}$ 精化为特定的物理总线信号。性质体现:实时可观察性保证了逻辑层预期的“流量切断”物理效果能够百分之百发生。


% 本章:Core 层的实现
% 整合 kos-core 的架构、设计与核心实现

\chapter{Core 层的实现}

本章介绍 KOS-TL 的 Core 层在工程上的实现形式。当前实现采用 Haskell 语言,在 \texttt{kos-core/} 目录下形成独立的形式化内核,作为系统的“宪法”层:所有进入 Kernel/Runtime 的类型与项均须经解析与类型检查通过,非法构造在类型层面不可表示。

\section{设计原则与架构}

\subsection{设计原则}

kos-core 遵循以下设计原则,与本书核心层(第 2 章)的角色定位一致:

\begin{itemize}
\item \textbf{Parser 即门卫}:所有 Term 仅能通过解析 Core DSL(.kos 源文件)产生,非法语法无法生成 AST。
\item \textbf{归纳构造}:代数数据类型(ADT)保证每个 Term 变体都有明确定义,无 $\bot$ 构造子。
\item \textbf{类型检查即合法性}:\texttt{parseAndCheckTerm} / \texttt{parseAndCheckModule} 通过后才产出合法项或模块,失败则 \texttt{Left} 报错。
\item \textbf{双轴 Universe}:$\mathcal{U}_i$(计算轴)、$\textsf{Type}_i$(逻辑轴),对应本书 Core 层语法。
\item \textbf{证明即构造}:Proof 模块提供目标驱动的证明搜索(\texttt{prove}、\texttt{proveDepth}),与“证明构造与小步演化”对应,支持依赖型构造子与候选列表(\texttt{tryWithCandidates})。
\end{itemize}

\subsection{整体架构与数据流}

内核采用“单入口、分层校验”的管线:源码 $\to$ Parser $\to$ AST $\to$ 类型检查/证明搜索 $\to$ 合法项。对外 API 集中在 \texttt{KosCore.hs}:\texttt{parseAndCheckTerm}(解析并推断类型)、\texttt{parseAndCheckModule}(解析并检查模块内所有声明)、\texttt{typeWellFormed}(类型良构判定)、以及 Proof 模块的 \texttt{prove}、\texttt{proveDepth}、\texttt{tryWithCandidates}、\texttt{exactTerm} 等。上下文(Context)维护变量绑定与定义,供 TypeCheck 与 Proof 共享;定义等价由 \texttt{normalize}(Reduction)+ $\alpha$ 等价(Alpha)判定。

\subsection{目录结构}

\begin{verbatim}
kos-core/
├── src/
│   ├── KosCore.hs          -- 主模块,对外 API(含 Proof 再导出)
│   └── KosCore/
│       ├── AST.hs          -- Term 代数数据类型(含 Triv、值依赖谓词)
│       ├── Universe.hs     -- 双轴 Universe
│       ├── Context.hs      -- 类型上下文、ctxNames
│       ├── Substitution.hs -- 变量替换
│       ├── Reduction.hs    -- β/ι/δ/ζ/η 归约
│       ├── TypeCheck.hs    -- 双向类型检查、expandTypeSynonym
│       ├── Proof.hs        -- 证明自动化(策略、深度限制、tryWithCandidates)
│       ├── Parser.hs       -- 语法分析
│       ├── Alpha.hs        -- α 等价
│       └── JSON.hs         -- JSON 序列化
├── app/Main.hs             -- CLI 入口
├── examples/               -- .kos 示例
└── docs/                   -- PROOF_TOOLS_COMPARISON、PROOF_REVIEW_KNOWLEDGE_OPS 等
\end{verbatim}

\section{语法与类型构造}

\subsection{类型构造(Types)}

与 Kos.pdf 的 Core 层语法对应,kos-core 实现的类型构造如表\ref{tab:core-types}所示。

\begin{table}[h]
\centering
\caption{Kos.pdf 类型构造与 kos-core 实现对照}
\label{tab:core-types}
\begin{tabular}{lll}
\toprule
\textbf{Kos.pdf 定义} & \textbf{kos-core 实现} & \textbf{状态} \\
\midrule
$\textsf{Prop}$ & \texttt{Prop Text} & 已实现 \\
$\textsf{Type}_i$ & \texttt{Universe Logical i} & 已实现 \\
$\mathcal{U}_i$ & \texttt{Universe Computational i} & 已实现 \\
$\textsf{Val}, \textsf{Time}, \textsf{ID}$ & \texttt{Val}, \texttt{Time}, \texttt{Ident} & 已实现 \\
$\Pi(x:A).B$ & \texttt{Pi Text Term Term} & 已实现 \\
$\Sigma(x:A).B$ & \texttt{Sigma Text Term Term} & 已实现 \\
$A + B$ & \texttt{Sum Term Term} & 已实现 \\
$\textsf{Id}_A(a, b)$ & \texttt{Id Term Term Term} & 已实现 \\
\bottomrule
\end{tabular}
\end{table}

\subsection{项构造(Terms)}

\begin{table}[h]
\centering
\caption{Kos.pdf 项构造与 kos-core 实现对照}
\label{tab:core-terms}
\begin{tabular}{lll}
\toprule
\textbf{Kos.pdf 定义} & \textbf{kos-core 实现} & \textbf{状态} \\
\midrule
$x$(变量) & \texttt{Var Text} & 已实现 \\
$\lambda x.t$, $t\,u$ & \texttt{Lam}, \texttt{App} & 已实现 \\
$\langle t, u \rangle$ & \texttt{Pair Term Term} & 已实现 \\
$\textsf{split}(t, x.y.u)$ & \texttt{Split Term Text Text Term} & 已实现 \\
$\textsf{inl}(t)$, $\textsf{inr}(t)$ & \texttt{InL}, \texttt{InR};\texttt{InLS}, \texttt{InRS} & 已实现 \\
$\textsf{case}(t, x.u, y.v)$ & \texttt{Case Term Text Text Text Term} & 已实现 \\
$\textsf{refl}$(Id 引入) & \texttt{Refl Term} & 已实现 \\
$\textsf{let } x=u \textsf{ in } t$ & \texttt{Let Text Term Term Term} & 已实现 \\
规范证明项(值谓词成立时) & \texttt{Triv} & 已实现(Proof 产出,check 接受) \\
\bottomrule
\end{tabular}
\end{table}

\subsection{值依赖谓词(扩展)}

kos-core 在核心层语法基础上增加了值依赖谓词(Value-Dependent Predicates),使类型构造可依赖项的具体数值;对应本书“证明构造与小步演化”中的可计算验证(如 \textsf{TimeOK} 的数值比较):

\begin{itemize}
\item \texttt{gt(a, b)}:$a > b$,当 $a$、$b$ 可解析为数值时成立则类型良构。
\item \texttt{ge(a, b)}:$a \ge b$
\item \texttt{lt(a, b)}:$a < b$
\item \texttt{le(a, b)}:$a \le b$
\item \texttt{eq(a, b)}:$a = b$(支持数值或字符串相等)
\end{itemize}

例如 \texttt{gt(val "200", val "180")} 表示命题“200 > 180”,类型检查时会对两边的值进行求值比较,成立则接受。

\section{.kos 语言语法}

\subsection{模块结构}

\begin{verbatim}
module ModuleName where
  type Name : Kind
  def name : Type := term
\end{verbatim}

\texttt{type} 声明类型别名,\texttt{def} 声明项定义(支持 $\delta$ 归约)。

\subsection{原子类型与 Base Sorts}

\begin{itemize}
\item \texttt{Prop P}:命题,$\textsf{Prop} : \textsf{Type}_1$
\item \texttt{U0}, \texttt{U1}, \texttt{U}:计算轴 Universe
\item \texttt{Type0}, \texttt{Type1}, \texttt{Type}:逻辑轴 Universe
\item \texttt{val "x"}:值字面量
\item \texttt{time "2025-01-01"}:时间字面量
\item \texttt{id "BATCH1"}:标识符字面量
\end{itemize}

\subsection{Π 类型与 λ}

\begin{itemize}
\item \texttt{Pi(x:A). B} 或 \texttt{Π(x:A). B}:依赖函数类型
\item \texttt{A -> B}:箭头简写(非依赖)
\item \texttt{lam (x:A) . t}:$\lambda$ 抽象
\item \texttt{f a}:函数应用
\end{itemize}

\subsection{Σ 类型}

\begin{itemize}
\item \texttt{Sigma(x:A). B}:依赖对类型
\item \texttt{<d, p>}:对构造
\item \texttt{split (p) as x y in body}:$\Sigma$ 消除
\end{itemize}

\subsection{Sum 类型}

\begin{itemize}
\item \texttt{A + B}:不交并
\item \texttt{inl(A,B,v)}, \texttt{inr(A,B,v)}:显式类型注入
\item \texttt{inl v}, \texttt{inr v}:单参数注入(需 check 模式)
\item \texttt{case s of inl x -> t1; inr y -> t2}:Sum 消除
\end{itemize}

\subsection{Id 类型与 Let}

\begin{itemize}
\item \texttt{Id(A,a,b)}:恒等类型
\item \texttt{refl w}:自反证明
\item \texttt{let x : A := v in body}:Let 绑定($\zeta$ 归约)
\end{itemize}

\section{推导规则与归约}

\subsection{推导规则实现}

TypeCheck.hs 实现双向类型检查(infer/check),对应 Kos.pdf 的推导规则:

\begin{itemize}
\item \textbf{Π 引入}:\texttt{Lam} 分支,\texttt{check ctx' bodyTerm bodyTy}
\item \textbf{Π 消除}:\texttt{App} 分支,\texttt{Right (substitute x a body)}
\item \textbf{Σ 引入}:\texttt{Pair},\texttt{check ctx d dom},\texttt{check ctx p (substitute x d body)}
\item \textbf{Σ 消除}:\texttt{Split (Pair d p) x1 x2 body} $\to$ \texttt{substitute x2 p (substitute x1 d body)}
\item \textbf{Conversion}:\texttt{check} 结构失败时 fallback:\texttt{infer} + \texttt{definitionallyEqual}
\end{itemize}

\subsection{归约规则实现}

Reduction.hs 实现 $\beta$、$\iota$、$\delta$、$\zeta$、$\eta$:

\begin{itemize}
\item $\beta$:\texttt{App (Lam x \_ body) arg -> substitute x arg body}
\item $\iota$(Σ):\texttt{Split (Pair d p) x1 x2 body -> substitute x2 p (substitute x1 d body)}
\item $\iota$(Sum):\texttt{Case (InL ...)} 与 \texttt{Case (InR ...)} 分支
\item $\delta$:\texttt{Var x -> ctxLookupDef x ctx}
\item $\zeta$:\texttt{Let x \_ty val body -> substitute x val body}
\item $\eta$:\texttt{Lam x ty (App f (Var y))} 当 $y=x$ 且 $x \notin \text{FV}(f)$ 时归约为 \texttt{f}
\end{itemize}

定义等价 \texttt{definitionallyEqual} = \texttt{normalize} + $\alpha$ 等价。

\section{证明自动化(Proof 模块)}

Proof.hs 实现目标驱动的证明搜索,与第 8 章“核心层:证明构造与小步演化”及证明搜索机(PSM)对应,并对标 Coq/Lean 的常见 tactic 能力。

\subsection{策略与类型}

\begin{itemize}
\item \textbf{Tactic}:\texttt{Context -> Term -> Maybe Term},给定上下文与目标类型,尝试产生证明项。
\item \textbf{prove}:主入口,按固定顺序尝试各策略直至成功;内部调用 \texttt{proveDepth defaultProveDepth}(默认深度 100),防止无限递归。
\item \textbf{proveDepth}:带深度上限的证明搜索;递归子目标时传入 \texttt{depth-1},\texttt{depth $\le$ 0} 时立即返回 \texttt{Nothing}。
\end{itemize}

\subsection{策略顺序与含义}

\begin{enumerate}
\item \textbf{tryValuePred}:目标为 \texttt{Gt/Ge/Lt/Le/Eq(a,b)} 时求值比较,成立则返回 \texttt{Triv}(对应 checkTime/checkSpace/checkBatch 的可计算验证)。
\item \textbf{tryRefl}:目标为 \texttt{Id A a b} 且 $a \equiv b$ 时返回 \texttt{Refl a}。
\item \textbf{tryAssumption}:目标与上下文中某变量类型定义等价时返回 \texttt{Var name}(对应 PSM 的见证使用)。
\item \textbf{tryIntroSigma}:目标 $\Sigma(x:A).B$ 时证 $A$ 得 $d$、证 $B[d/x]$ 得 $p$,返回 \texttt{Pair d p}。
\item \textbf{tryIntroPi}:目标 $\Pi(x:A).B$ 时在 $\Gamma,x:A$ 下证 $B$ 得 \texttt{Lam x A body}。
\item \textbf{tryIntroSumLeft/Right}:目标 $A+B$ 时证 $A$ 得 \texttt{InL A B proof} 或证 $B$ 得 \texttt{InR A B proof}。
\item \textbf{tryConstructor}:在上下文中查找类型形如 $\Pi\ldots \to \text{goal}$ 的定义,对参数做\textbf{依赖顺序}证明(\texttt{proveArgs}:证完 $A_i$ 将 $d_i$ 代入后续类型再证),构造 \texttt{App(..App(d,p1)..,pn)},对应 \textsf{mkCausalProof} 等小步构造。
\end{enumerate}

\subsection{候选与 exact}

\begin{itemize}
\item \textbf{tryWithCandidates ctx goal candidates}:对每个候选 $c$,用 \texttt{infer ctx c} 得类型,\texttt{addDef "\_\_e" ty c ctx} 将 $c$ 加入上下文,再对 \texttt{goal} 调用 \texttt{prove};返回第一个成功的证明。对应 monograph 的 \textsf{deriveCausal}(a,f) = firstSome (map $\lambda e.\,$tryWithProc(a,f,e), getProcSteps(f,$\sigma$))。
\item \textbf{exactTerm ctx goal t}:若 \texttt{check ctx t goal} 通过则返回 \texttt{Just t},否则 \texttt{Nothing}(对标 Lean \texttt{exact})。
\item \textbf{firstSuccess}、\texttt{orElse}:组合多策略为“第一个成功”。
\end{itemize}

\subsection{与知识操作的衔接}

Proof 模块不维护知识库 $\sigma$;\textsf{getProcSteps}/\textsf{getAnomalies} 由上层(Kernel/C)实现,将候选列表与目标传入 \texttt{tryWithCandidates} 或多次 \texttt{prove} 即可。详见 kos-core 文档 \texttt{PROOF\_REVIEW\_KNOWLEDGE\_OPS.md} 与 \texttt{PROOF\_TOOLS\_COMPARISON.md}。

\section{双轴 Universe 系统}

Universe.hs 实现:

\begin{itemize}
\item $\textsf{Prop} : \textsf{Type}_1$
\item $\textsf{Type}_i : \textsf{Type}_{i+1}$,$\mathcal{U}_i : \mathcal{U}_{i+1}$
\item $\mathcal{U}_i : \textsf{Type}_{i+1}$(计算轴可提升到逻辑轴)
\item $\textsf{Prop} \hookrightarrow \mathcal{U}_1$(\texttt{propEmbedToData})
\item Impredicativity:$\Pi(x:A).B : \textsf{Prop}$ 当 $B : \textsf{Prop}$
\end{itemize}

\section{与 C Runtime 的集成}

kos-core 作为独立可执行文件,通过 C Bridge(\texttt{kos\_core\_bridge.h})与 Kernel/Runtime 层集成:

\begin{itemize}
\item \texttt{kos\_core\_bridge\_check\_expr}:对 .kos 表达式进行类型检查
\item \texttt{kos\_core\_bridge\_term\_from\_kos}:解析并校验后返回 JSON 序列化
\item \texttt{kos\_core\_bridge\_infer\_from\_kos}:类型推断
\end{itemize}

C 层在添加类型定义、精化信号时,可调用上述接口,确保仅合法类型进入系统。

\section{实现状态与 Kos.pdf 对照}

表\ref{tab:core-comparison}总结 kos-core 与 Kos.pdf 的对照情况。

\begin{table}[h]
\centering
\caption{kos-core 与 Kos.pdf 实现对照}
\label{tab:core-comparison}
\begin{tabular}{lll}
\toprule
\textbf{维度} & \textbf{Kos.pdf 要求} & \textbf{kos-core 实现} \\
\midrule
类型构造 & Prop, Type$_i$, U$_i$, Val, Time, ID, Π, Σ, +, Id, Gt/Ge/Lt/Le/Eq & 已实现 \\
项构造 & λ, app, pair, split, inl, inr, case, refl, let, Triv & 已实现 \\
Universe & 双轴、Prop↪U₁、Impredicativity & 已实现 \\
归约 & $\beta$, $\iota$, $\delta$, $\zeta$, $\eta$ & 已实现 \\
Conversion & $\Gamma \vdash t:A$, $A \equiv B$ $\Rightarrow$ $\Gamma \vdash t:B$ & 已实现 \\
证明自动化 & 目标驱动搜索、深度限制、Sigma/Pi/Sum/构造函数、tryWithCandidates & 已实现(Proof.hs) \\
元理论 & SN, Subject Reduction, Confluence & 无形式化证明 \\
本体管理 & 业务本体 $\to$ Sorts/类型 & 未实现 \\
计算自反性 & 自动合成 Id 证明 & 部分:Id/Refl + Proof 中 tryRefl;归约时自动合成未实现 \\
\bottomrule
\end{tabular}
\end{table}

\subsection{已知简化}

\begin{itemize}
\item \textbf{Π 类型推断}:\texttt{infer} 对 Π 构造统一返回 \texttt{Type₁};核心层规定 predicative 情形为 $\Pi(x:A).B : \textsf{Type}_{\max(i,j)}$,实现做了简化。
\item \textbf{Pair 类型推断}:对依赖 $B(x)$ 的 $\Sigma$ 类型推断做了简化。
\end{itemize}

\subsection{进一步改进方向}

\begin{itemize}
\item J 消除子:Id 类型的依赖消除(J rule),当前仅支持 refl 引入。
\item 元理论形式化:SN、Subject Reduction、Confluence 的机器可验证证明。
\item 本体管理器:业务本体 $\to$ Sorts/类型 的 TypeOntology 模块。
\item 计算自反性自动合成:归约时自动合成 $\textsf{Id}(t,t')$ 证明。
\item Proof 扩展:rewrite、hint 数据库、多子目标/repeat 组合子等(见 kos-core 文档 \texttt{PROOF\_TOOLS\_COMPARISON.md})。
\end{itemize}



\chapter{KOS-TL的应用示例}

设想一家大型离散制造企业,其核心问题为:
\begin{quote}
\emph{当某批次产品出现严重质量问题时,系统是否能够自动追溯其生产过程,
识别潜在的设备、人员或原材料异常,并给出可执行、可解释的因果链?}
\end{quote}

该问题具有以下典型特征:
\begin{itemize}
  \item 数据来源多样(工单、设备日志、人员排班、质检记录);
  \item 强时间顺序与因果约束;
  \item 推理结果需直接支持生产决策与责任界定。
\end{itemize}

系统中涉及如下核心表(来自不同系统):
\begin{enumerate}[label=(\arabic*)]
  \item 生产记录:Product(ProductID, Model)
  \item 批次记录:Batch(BatchID, ProductID, ProduceDate)
  \item 生产线:ProductionLine(LineID, Factory)
  \item 工艺路径表:ProcessRoute(Model, StepName, StepOrder, TargetLineType)
  \item 工艺阈值表:ProcessThreshold(Model, StepName, ParamName, MinValue, MaxValue)
  \item 步骤执行详表:StepExecution(WOID, StepName, StartTime, EndTime, EquipID)
  \item 传感器序列表:SensorTimeSeries(EquipID, ParamName, Value, Timestamp, DeviceStatus)
  \item 工单;WorkOrder(WOID, BatchID, LineID)
  \item 操作员:Operator(OperatorID, Name, Role)
  \item 操作日志:OperationLog(LogID, WOID, OperatorID, Time)
  \item 装配:Equipment(EquipID, LineID)
  \item 装配状态:EquipmentStatus(EventID, EquipID, Status, Time)
  \item 过程参数:ProcessParam(LogID, ParamName, Value)
  \item 质量审核:QualityInspection(InspectID, BatchID, Result, Time)
  \item 缺陷:DefectReport(ReportID, BatchID, DefectType)
  \item 供应链:SupplierPart(PartID, SupplierID, BatchID)
\end{enumerate}

为了举例说明,我们跟踪一个轴承生产质量追溯中的因果推理流程(如表\ref{tab:causal-reasoning}所示)。
\begin{table}[h]
\centering
\caption{轴承生产质量追溯因果推理}
\label{tab:causal-reasoning}
\begin{tabular}{>{\centering\arraybackslash}p{2cm} >{\centering\arraybackslash}p{3cm} >{\raggedright\arraybackslash}p{9cm}}
\toprule
\textbf{步骤} & \textbf{系统动作} & \textbf{具体数据实例} \\
\midrule
输入 & 质检系统上报异常 &Batch\_202310-01 在 2023-10-10 10:00 检测到“硬度不均”。 \\
类型实例化 & 构造$f_{fail}$ & $f_{fail} : \mathsf{FailureEvent} = \langle \text{"B2310"}, \text{"HARD\_ERR"}, \text{10:00} \rangle$ \\
内核推理 & 搜索因果证据 & 检索到该批次在 08:00 经过 HeatTreatment\_03 炉,且该炉在 07:55 有个温控波动 $a_{temp}$。 \\
逻辑合成 & 构造因果链 & $r = \langle f_{fail}, a_{temp}, \text{prf}_{causal} \rangle$。此时证明项 $\text{prf}_{causal}$ 自动校验了 $07:55 < 10:00$。 \\
\bottomrule
\end{tabular}
\end{table}

通过使用KOS-TL的推理,最终输出给用户的不是一条SQL查询结果,而是一个逻辑证明包。用户点击报告时,系统可以展开 $\text{prf}_{causal}$,直接定位到该温度波动的原始PLC原始日志,因为该日志是构造报告 $r$ 的组成部分。



\section{KOS-TL的应用过程}

KOS-TL在应用过程中主要涉及到如下环节:
\begin{enumerate}
    \item 定义初始的原子类型、谓词类型和事件与约束。这部分是属于内核层(Core)。这种定义本质上是规定我们所刻画的逻辑系统的合法性边界。
    \item 运行时层(runtime)是系统与外部的界面,通过运行时,KOS-TL系统获取到数据,并将其精化为类型对象(逻辑可操作对象)。
    \item 核心层(Kernel)负责具体的知识操作。
\end{enumerate}


\subsection{内核层:规则定义与逻辑约束}


    针对“轴承生产质量追溯因果推理”问题,内核层定义相应的类型和约束。

(1)基础原子类型

基础原子类型如表\ref{tab:domain-types-refinement}所示。包括:$\mathsf{BatchID}, \mathsf{Machine}, \mathsf{Time}$ 。

\begin{table}[h]
\centering
\caption{领域概念与类型精化关系表格}
\label{tab:domain-types-refinement}
\begin{tabular}{>{\centering\arraybackslash}p{2cm} >{\raggedright\arraybackslash}p{3.5cm} >{\raggedright\arraybackslash}p{3.5cm} >{\raggedright\arraybackslash}p{5cm}}
\toprule
\textbf{领域概念} & \textbf{逻辑层类型 (Core层定义)} & \textbf{内核底层原子类型 (Kernel实现)} & \textbf{精化逻辑 (Refinement)} \\
\midrule
Time & Time & Float / UInt64 & 直接映射,表示 Unix 时间戳或逻辑时钟。 \\
BatchID & BatchID & Val / String & $\Sigma(s:Val).\text{Proof}(isIDFormat(s))$ \\
Machine & Machine & Val / Enum & $\Sigma(v:Val).\text{Proof}(v\in EquipRegistry)$ \\
\bottomrule
\end{tabular}
\end{table}

(2)谓词类型

谓词类型包括:

\begin{itemize}

    \item $\mathsf{InRoute}(b, m)$

    定义了“批次 $b$ 是否允许在机器 $m$ 上加工”。

    \item $\mathsf{Overlap}(t, dur)$

    定义时间点 $t$ 是否落在区间 $dur$ 之内。
\end{itemize}

(3)事件与约束

\begin{enumerate}[label=(\roman*)]

    \item 失效事件类型 ($\mathsf{FailEvt}$)

    $$\mathsf{FailEvt} \equiv \Sigma(b: \mathsf{BatchID}). \Sigma(err: \mathsf{ErrorCode}). \Sigma(t: \mathsf{Time}). \mathsf{Proof}(t \in \text{Shift}_{QA})$$

    该类型不仅记录了哪个批次坏了,还强制要求携带一个“检测时间必须在质检班次内”的证明。

    \item 生产过程类型 ($\mathsf{ProcStep}$)

    $$\mathsf{ProcStep} \equiv \Sigma(b: \mathsf{BatchID}). \Sigma(m: \mathsf{Machine}). \Sigma(dur: \mathsf{Time} \times \mathsf{Time}). \mathsf{Proof}(\mathsf{InRoute}(b, m))$$

    这里通过 $\mathsf{InRoute}$ 约束,确保了该批次轴承在 $m$ 机器上的记录是符合工艺路径定义的。

    \item 环境异常类型 ($\mathsf{Anomaly}$)

    $$\mathsf{Anomaly} \equiv \Sigma(m: \mathsf{Machine}). \Sigma(p: \mathsf{Param}). \Sigma(v: \mathsf{Val}). \Sigma(t: \mathsf{Time})$$

    \item 因果有效性约束($\mathsf{CausalProof}(a, f)$)

    $$\mathsf{isBefore}(t(a), t(f)) \land \mathsf{isSameResource}(\text{location}(a), \text{process}(f))$$

    其中工艺一致性约束 $\mathsf{IsValidRoute}$定义一个谓词,强制要求生产记录 $e$ 中的 MachineID 必须属于该产品定义的 StandardRoute。

    追溯被定义为一个证明搜索问题:
    $$\forall f : \mathsf{Failure}, \exists (a, \pi) : \Sigma(a:\mathsf{Anomaly}) . \mathsf{CausalProof}(a, f)$$

    对于每一个失效,必须能构造出一个对应的异常及其因果证明。

    \item 因果证明($\mathsf{CausalProof}$)

    因果证明被定义为一个依存乘积类型($\Sigma$-Type),它要求同时满足时序、位置和工艺逻辑的一致性。

    $$\mathsf{CausalProof}(a, f) \equiv \Sigma(e : \mathsf{ProcStep}). \text{Prop}_{causal}(a, e, f)$$

    其中,$\text{Prop}_{causal}$ 是一个复合谓词,要求:
    \begin{itemize}
        \item 时序逻辑:异常 $a$ 发生在过程 $e$ 之内,且过程 $e$ 完成于失效 $f$ 之前。$a.t \in e.dur \land e.dur.end < f.t$

        \item 空间逻辑:异常设备 $a.m$ 正是生产该批次的过程设备 $e.m$。$a.m = e.m$

        \item 批次一致性:过程 $e$ 所处理的批次正是失效的批次 $f.b$。$e.b = f.b$
    \end{itemize}

    因果证明的构造函数mkCausalProof:

$$
\begin{aligned}
\mathsf{mkCausalProof} : \; & \Pi(a : \mathsf{Anomaly}). \\
& \Pi(f : \mathsf{FailEvt}). \\
& \Pi(e : \mathsf{ProcStep}). \\
& \underbrace{(a.t \in e.dur \land e.dur.end < f.t)}_{\text{时序一致性证明项 } p_{time}} \to \\
& \underbrace{(a.m = e.m)}_{\text{空间一致性证明项 } p_{space}} \to \\
& \underbrace{(e.b = f.b)}_{\text{批次一致性证明项 } p_{batch}} \to \\
& \mathsf{CausalProof}(a, f)
\end{aligned}
$$
    \item 因果报告($\mathsf{RootCauseReport}$ )

    $$\mathsf{RootCauseReport} \equiv \Sigma(f : \mathsf{FailEvt}) . \Sigma(a : \mathsf{Anomaly}) . \mathsf{CausalProof}(a, f)$$

    这个定义在逻辑语义上表达了三层含义:
    \begin{itemize}

        \item 失效存在性 ($f$):必须指明一个已经发生的质量失效(如:轴承硬度不均)。

        \item 异常存在性 ($a$):必须指明一个在生产过程中发生的物理异常(如:电压跌落)。

        \item 因果证明 ($\mathsf{CausalProof}$):最关键的部分。它不是一个布尔值,而是一个证明项。只有当 Kernel 层能成功构造出满足时空约束的证据链时,这个类型才被视为“非空”(Inhabited)。
    \end{itemize}
\end{enumerate}

\subsection{运行时层:数据抓取与对象精化(Elaboration)}


    运行时层从外部数据库提取数据,并将其转化为 KOS-TL 内核可理解的对象项,从而实现“数据逻辑化”。

    运行时层依赖的原始数据表包括:
\begin{itemize}

    \item Product\_Master: 维护工艺路径(Batch\_202310-01 $\to$ 轴承A型 $\to$ 需经过“热处理”工序)。

    \item Execution\_Log: 记录工单执行(Batch\_202310-01 在 M\_03 热处理炉上生产,时间 08:00-09:30)。

    \item IoT\_Sensor\_Stream: 记录 M\_03 的传感器流(08:15 电压波动 $15\%$)。

    \item Quality\_Report: 质检记录(10:00 检测到“硬度不均”)。
\end{itemize}

    运行时层将上述数据“精化”为带证明的项,此外

    \begin{enumerate}[label=(\roman*)]

        \item 失效事件$f_0$:

         $$f_0 = \text{mkFailure}(\text{Batch}\_{202310-01}, \text{Hardness\_Issue}, 10:00, \pi_{QA\_Sign})$$

        精化逻辑:从 Quality\_Report 读取,$\pi_{QA\_Sign}$ 是对该记录真实性的逻辑背书。

        \item 生产记录项$e_{proc}$:

        $$e_{proc} = \text{mkStep}(\underbrace{\text{Batch}\_{202310-01}}_{b}, \underbrace{\text{M\_03}}_{m}, \underbrace{\langle 08:00, 09:30 \rangle}_{dur}, \underbrace{\pi_{route}}_{\mathsf{Proof}(\mathsf{InRoute}(b,m))})$$

        精化逻辑:关联工单与设备日志,确保时间段闭合。

        \item 设备异常项$a_{volt}$:

        $a_{volt}$:$$a_{volt} = \text{mkAnomaly}(\text{M\_03}, \text{Voltage\_Drop}, 08:15, \pi_{iot\_hash})$$

    \end{enumerate}

\subsection{核心层:证明构造与小步演化}


核心层是执行层,它接收运行时层的对象,按照内核层的规则进行计算。其证明构造和演化如算法\ref{alg:kostl-kernel}所示。


\begin{algorithm}
\caption{KOS-TL 因果证明内核(类型驱动 / 依赖类型形式)}
\label{alg:kostl-kernel}
\begin{algorithmic}[1]

\REQUIRE 失效事件 $f : \mathsf{FailEvt}$,知识库 $\sigma$
\ENSURE  $\mathsf{Option}(\Sigma a:\mathsf{Anomaly}.\;\mathsf{CausalProof}(a,f))$

\vspace{0.5em}

\STATE \textbf{逻辑谓词(作为类型)}
\STATE \hspace*{\algorithmicindent} $\mathsf{TimeOK}(a,e,f) \;\triangleq\; (a.t \in e.dur) \land (e.dur.end < f.t)$
\STATE \hspace*{\algorithmicindent} $\mathsf{SpaceOK}(a,e) \;\triangleq\; (a.m = e.m)$
\STATE \hspace*{\algorithmicindent} $\mathsf{BatchOK}(e,f) \;\triangleq\; (e.b = f.b)$

\vspace{0.5em}

\STATE \textbf{因果证明构造器}
\STATE \hspace*{\algorithmicindent}
$\mathsf{mkCausalProof} :
\Pi(a:\mathsf{Anomaly})\,
\Pi(f:\mathsf{FailEvt})\,
\Pi(e:\mathsf{ProcStep})\,
\mathsf{TimeOK}(a,e,f) \to
\mathsf{SpaceOK}(a,e) \to
\mathsf{BatchOK}(e,f) \to
\mathsf{CausalProof}(a,f)$

\vspace{0.5em}

\STATE \textbf{局部证明搜索(以单个工序为见证)}
\STATE \hspace*{\algorithmicindent}
$\mathsf{tryWithProc}(a,f,e) \triangleq$
\STATE \hspace*{\algorithmicindent}\hspace*{1em}
$\textbf{match }\mathsf{checkTime}(a,e,f)\textbf{ with}$
\STATE \hspace*{\algorithmicindent}\hspace*{2em}
$\textbf{None} \Rightarrow \textbf{None}$
\STATE \hspace*{\algorithmicindent}\hspace*{2em}
$\textbf{Some }p_t \Rightarrow$
\STATE \hspace*{\algorithmicindent}\hspace*{3em}
$\textbf{match }\mathsf{checkSpace}(a,e)\textbf{ with}$
\STATE \hspace*{\algorithmicindent}\hspace*{4em}
$\textbf{None} \Rightarrow \textbf{None}$
\STATE \hspace*{\algorithmicindent}\hspace*{4em}
$\textbf{Some }p_s \Rightarrow$
\STATE \hspace*{\algorithmicindent}\hspace*{5em}
$\textbf{match }\mathsf{checkBatch}(e,f)\textbf{ with}$
\STATE \hspace*{\algorithmicindent}\hspace*{6em}
$\textbf{None} \Rightarrow \textbf{None}$
\STATE \hspace*{\algorithmicindent}\hspace*{6em}
$\textbf{Some }p_b \Rightarrow
\textbf{Some}(\mathsf{mkCausalProof}(a,f,e,p_t,p_s,p_b))$

\vspace{0.5em}

\STATE \textbf{异常级证明搜索($\exists e$ 的程序化实现)}
\STATE \hspace*{\algorithmicindent}
$\mathsf{deriveCausal}(a,f) \triangleq
\mathsf{firstSome}
\bigl(
\mathsf{map}(\lambda e.\,\mathsf{tryWithProc}(a,f,e),
             \mathsf{getProcSteps}(f,\sigma))
\bigr)$

\vspace{0.5em}

\STATE \textbf{最终分析函数(目标类型驱动)}
\STATE \hspace*{\algorithmicindent}
$\mathsf{analyze}(f) \triangleq
\mathsf{firstSome}
\bigl(
\mathsf{map}
(\lambda a.\,
 \mathsf{map}(\lambda p.\,\langle a,p\rangle,
              \mathsf{deriveCausal}(a,f)),
 \mathsf{getAnomalies}(f,\sigma))
\bigr)$

\vspace{0.5em}

\STATE \textbf{return } $\mathsf{analyze}(f)$

\end{algorithmic}
\end{algorithm}
小步操作的输入是初始配置$\mathcal{C}_0 \leftarrow \langle \Gamma_0, \sigma_0, f_0 \rangle$ ,其中$\Gamma_0$是小步操作过程中的上下文,包含了所有预定义的类型、公理和函数签名。$\sigma_0$是一个事实集合,包含了此时此刻系统已知的所有已物化的碎片(Fragments)。$f_0$是本次演算的初始项,通常是一个新发生的、需要解释的观测事实。

上述流程可以总结为表\ref{tab:stage-actions}。

\begin{table}[h]
\centering
\caption{KOS-TL应用总结}
\label{tab:stage-actions}
\begin{tabular}{>{\centering\arraybackslash}p{2cm} >{\raggedright\arraybackslash}p{6cm} >{\raggedright\arraybackslash}p{8cm}}
\toprule
\textbf{阶段} & \textbf{动作内容} & \textbf{数据/逻辑产出} \\
\midrule
Runtime & 从数据库读取 10:00 硬度异常和 08:15 电压异常。 & 实例化:$f_0$ 和 $a_{volt}$ 对象被创建。 \\
Kernel & 发现两者存在关联,启动 CausalSearch 指令。 & 演化:状态从“发现问题”向“寻找证据”迁移。 \\
Core & 检查 (08:15$<$10:00) 以及该批次是否在 M\_03 生产。 & 验证:通过类型检查,准许构造因果链。 \\
终态 & 合成并物化 RootCauseReport。 & 结论:输出一份包含原始数据指针和逻辑路径的硬因果报告。 \\
\bottomrule
\end{tabular}
\end{table}

\section{KOS-TL 应对业务规则的动态演化}

在航空制造等高精密场景中,业务规则常因工艺改进或失效分析而发生变更。我们以“轴承热处理”为例,假设业务规则发生如下演化:
\begin{quote}

\emph{为防止回火脆性,若热处理过程中发生电压异常,必须同时核查该时段的“冷却水循环压力”。只有当电压异常与水压异常同时存在时,才判定为“严重质量缺陷”。}
\end{quote}

在 KOS-TL 中,这种规则变更不需要重写系统代码,只需在 \textbf{L0 Core 层} 对因果证明的类型定义进行局部精化,利用类型系统的驱动力实现全链路逻辑更新。
\subsection{类型定义的重构与精化}
\begin{itemize}
\item \textbf{原始类型定义(单一异常判定):}系统最初仅关注电压异常 $a$ 与失效 $f$ 之间的时序因果:

$$\mathsf{CausalProof}(a, f) \equiv \Sigma(e : \mathsf{ProcStep}). \mathsf{Prop}_{time}(a, e, f)$$

\item \textbf{精化后的类型定义(双重异常注入):}

根据新标准,我们将 $\mathsf{CausalProof}$ 重定义为必须包含“水压异常证据”的依存乘积类型:

$$\mathsf{CausalProof}(a, f) \equiv \Sigma(e : \mathsf{ProcStep}). \Sigma(w : \mathsf{WaterPressureAnomaly}). \mathsf{Prop}_{joint}(a, w, e, f)$$

其中,水压异常类型 $\mathsf{WaterPressureAnomaly}$ 被严格定义为水压值 $w$ 对正常区间 $[w_{\text{min}}, w_{\text{max}}]$ 的违反证明:
$$\mathsf{WaterPressureAnomaly} \equiv \Pi(w_{val} : \mathbb{R}). \; (w_{val} < w_{\text{min}} \vee w_{val} > w_{\text{max}}) \to \mathsf{Anomaly}$$

\item \textbf{逻辑约束:}

新的复合谓词 $\mathsf{Prop}_{joint}$ 不仅要求电压异常 $a$ 发生在过程 $e$ 中,还强制要求存在一个同一过程内的水压异常证据 $w$,从而在逻辑层实现了“双重异常”的共存约束。
\end{itemize}

\subsection{Kernel 层的链式反应与自适应规约}

一旦 Core 层的类型签名发生变更,运行时的 \textbf{L1 Kernel 层} 会通过小步语义(Small-step Semantics)自动触发以下演化保护:
\begin{enumerate}
\item \textbf{构造函数$mkCausalProof$的失效隔离}

原有的构造函数应用 $\mathsf{mkCausalProof}(a, f, e)$ 由于参数项与新定义的 $\Sigma$ 类型不匹配,在 Kernel 的类型检查器(Type Checker)中会立即产生 \textbf{Type Mismatch}。这意味着所有仅包含电压异常的旧逻辑路径在当前状态下变得“不可达”。

\item \textbf{环境搜索与证据自动合成}

当 Kernel 尝试满足新的目标类型时,它会自动在当前状态 $\sigma$ 的知识集 $\mathcal{K}$ 中启动检索。
\begin{itemize}
    \item \textbf{Case 1(证据缺失):} 若 $\sigma$ 中仅有电压数据,Kernel 无法构造出 $\mathsf{WaterPressureAnomaly}$ 的实例,导致因果链闭合失败,生产流程将因“证据链不完整”而被逻辑锁止。
    \item \textbf{Case 2(双重异常触发):} Kernel 自动提炼(elab)水压传感器信号,若符合异常判定,则自动将电压、水压证据与生产过程 $e$ 进行“逻辑缝合”,合成新的复杂证明项 $p_{joint}$,完成状态迁移。
\end{itemize}
\end{enumerate}

\subsection{总结:从逻辑变更到合规执行}

通过对 $\mathsf{CausalProof}$ 的重新定义,KOS-TL 展示了其在业务规则变更下的核心优势:

\begin{itemize}
\item \textbf{确定性演化:}

规则的变化直接体现为逻辑类型的约束增强,而非脆弱的 if-else 分支。

\item \textbf{因果溯源性:}

每一份最终生成的“严重质量缺陷”报告中,都内生地包含了水压与电压双重异常的数学证明。

\end{itemize}

这种机制确保了制造系统不仅是“在运行”,而且是“在符合最新逻辑宪法的前提下运行”。
这种能力的本质是类型驱动开发 (Type-Directed Development) 的极端应用,如表\ref{tab:kos-tl-mechanisms}所示

\begin{table}[htbp]  % 浮动表格,位置建议:h=here, t=top, b=bottom, p=page
\centering
\caption{KOS-TL 处理机制特性表}
\label{tab:kos-tl-mechanisms}  % 用于交叉引用
\begin{tabular}{p{2cm}p{5cm}p{4cm}}  % 三列:特性3cm,处理机制6cm,意义6cm;| 为竖线分隔
\toprule
\textbf{特性} & \textbf{处理机制} & \textbf{意义} \\
\midrule
自愈性 (Self-Healing) & 如果数据源(Runtime)没有提供水压数据,Kernel 会报“类型缺失错误”而非给出错误结论。 & 强制保证结论的安全性,杜绝盲目追溯。 \\
\hline
\addlinespace[0.5ex]  % 行间小间距(可选)
逻辑下推 (Push-down) & 新规则通过 mkCausalProof 签名向下传递,Kernel 的搜索算法会自动感知到新的参数需求。 & 开发者不需要重新编写搜索算法,算法随类型自适应。 \\
\hline
\addlinespace[0.5ex]
零冗余 & 旧的追溯代码不需要删除,只要它们引用的类型被更新,它们的行为就自动改变。 & 实现了真正的“配置即逻辑”。 \\
\bottomrule
\end{tabular}
\end{table}

在这个例子中,我们并没有修改 analyze 函数,也没有修改 getProductionContext 函数。你只是通过修改 Core 层的类型签名,重新定义了“什么是真相”的物理边界。

Kernel 层就像一个全自动的拼图机器,你改变了拼图模板(Core),它就会自动改变寻找拼图碎片(Data)的策略和最终拼出的图案(Result)。这正是 KOS-TL 在应对复杂、多变的工业环境时的核心价值,即让系统的逻辑演进与业务规则的变更保持原子级的一致。

\section{KOS-TL 跨领域逻辑一致性与因果闭环求解}

在复杂工业生态中,知识的价值往往体现在跨领域的协同约束上。KOS-TL 通过“共享逻辑内核”与“类型交叉引用”(Cross-domain Type Reference),将原本孤立的业务部门转化为逻辑耦合的统一体。本节以“质量-财务”协同为例,展示如何通过类型系统实现跨域的逻辑锁定与证据驱动的闭环解锁。

\subsection{跨域逻辑锁定:从质量异常到财务冻结}

在 KOS-TL 架构下,跨领域的一致性并非通过数据库触发器实现,而是通过 Core 层定义的\textbf{跨域依赖类型}强制约束。

\subsubsection{跨域本体定义 (L0 Core 层)}
假设热处理批次 $b$ 发生异常,我们定义财务凭证类型 $\mathsf{AuditVoucher}$,使其高度依赖质量域的证明项:

\begin{itemize}
    \item \textbf{质量域异常证明:}
    $$\mathsf{QualIssue}(b) \equiv \Sigma(a: \mathsf{Anomaly}). \Sigma(e: \mathsf{ProcStep}). \mathsf{CausalProof}(a, e, b)$$

    \item \textbf{财务域待审计凭证:}
    $$\mathsf{AuditVoucher}(s, b) \equiv \Sigma(v: \mathsf{Voucher}). (\mathsf{QualIssue}(b) \to \mathsf{FrozenStatus}(v, s))$$
\end{itemize}

\textbf{逻辑含义:} 该定义规定,若逻辑上下文 $\sigma$ 中存在批次 $b$ 的质量异常证明,则对应的货款凭证 $v$ 在逻辑上必须处于 $\mathsf{FrozenStatus}$(冻结状态)。若无此证明,该财务对象在类型层面上无法被构造为“待审计凭证”。

\subsubsection{内核演化与类型归约 (L1 Kernel 层)}
当电压异常通过 $\mathsf{elab}$ 提炼出证明项 $p_{qual} : \mathsf{QualIssue}(b)$ 时,Kernel 启动演化:
\begin{enumerate}
    \item \textbf{状态迁移:} $\mathsf{STEP}(\sigma_n, \langle e_{qual}, p_{qual} \rangle) \to \sigma_{n+1}$。
    \item \textbf{自动重构:} 财务域通过类型交叉引用,自动订阅到 $p_{qual}$,导致凭证对象从 $\mathsf{Payable}$ 类型重构为 $\mathsf{AuditVoucher}$ 类型。
\end{enumerate}

\subsection{逻辑闭环解锁:整改证明驱动的演化}

要解除上述逻辑锁定,系统要求必须构造出一个\textbf{整改证明(Rectification Proof)},以在演化轨迹中逻辑性地消解异常效应。

\subsubsection{定义整改证明类型}
解锁并非简单的开关翻转,而是一个复杂的构造过程:
$$\mathsf{QualRectified}(b, p_{old}) \equiv \Sigma(a: \mathsf{Action}). \Sigma(p_{new}: \mathsf{QualPass}(b)). \mathsf{CausalResolve}(a, p_{old}, p_{new})$$
其中 $p_{old}$ 是原异常证明,$a$ 是整改动作,$p_{new}$ 是新合格证明。$\mathsf{CausalResolve}$ 证明了动作 $a$ 确实在逻辑上解决了 $p_{old}$ 识别的失效风险。

\subsubsection{定义解锁转换算子}
财务解锁被定义为一个严格的单向函数,强制要求整改证明作为输入:
$$\mathsf{unlockPayment} : \mathsf{AuditVoucher}(s, b) \to \mathsf{QualRectified}(b, p_{old}) \to \mathsf{NormalVoucher}(s, b)$$

\begin{algorithm}
\caption{KOS-TL 跨域一致性与闭环解锁内核}
\label{alg:kostl-cross-domain}
\begin{algorithmic}[1]

\REQUIRE 批次 $b : \mathsf{Batch}$,财务凭证 $v : \mathsf{Voucher}$,当前状态 $\sigma$
\ENSURE  $\mathsf{State}$(更新后的演化状态)

\vspace{0.5em}

\STATE \textbf{跨域关联定义(作为依存类型)}
\STATE \hspace*{\algorithmicindent} $\mathsf{QualIssue}(b) \triangleq \Sigma(a: \mathsf{Anomaly}). \Sigma(e: \mathsf{ProcStep}). \mathsf{CausalProof}(a, e, b)$
\STATE \hspace*{\algorithmicindent} $\mathsf{AuditVoucher}(v, b) \triangleq \Sigma(p: \mathsf{QualIssue}(b)). \mathsf{FrozenStatus}(v, p)$
\STATE \hspace*{\algorithmicindent} $\mathsf{QualRectified}(b, p_{old}) \triangleq \Sigma(a: \mathsf{Action}). \Sigma(p_{new}: \mathsf{QualPass}(b)). \mathsf{CausalResolve}(a, p_{old}, p_{new})$

\vspace{0.5em}

\STATE \textbf{跨域逻辑锁定(质量 $\to$ 财务)}
\STATE \hspace*{\algorithmicindent} $\mathsf{applyLock}(b, v, \sigma) \triangleq$
\STATE \hspace*{\algorithmicindent}\hspace*{1em} $\textbf{match } \mathsf{searchProof}(\sigma, \mathsf{QualIssue}(b)) \textbf{ with}$
\STATE \hspace*{\algorithmicindent}\hspace*{2em} $\textbf{None} \Rightarrow \sigma \quad \text{// 证据不足,维持原状}$
\STATE \hspace*{\algorithmicindent}\hspace*{2em} $\textbf{Some } p_{qual} \Rightarrow$
\STATE \hspace*{\algorithmicindent}\hspace*{3em} $\text{let } v_{new} = \mathsf{mkAuditVoucher}(v, p_{qual})$
\STATE \hspace*{\algorithmicindent}\hspace*{3em} $\textbf{return } \mathsf{updateState}(\sigma, v, v_{new}) \quad \text{// 财务状态随因果证据单调演化}$

\vspace{0.5em}

\STATE \textbf{因果闭环解锁算子}
\STATE \hspace*{\algorithmicindent} $\mathsf{mkUnlockProof} : \Pi(v_{aud}: \mathsf{AuditVoucher}). \Pi(p_{rect}: \mathsf{QualRectified}). \mathsf{NormalVoucher}$
\STATE \hspace*{\algorithmicindent} $\mathsf{tryUnlock}(v_{aud}, \sigma) \triangleq$
\STATE \hspace*{\algorithmicindent}\hspace*{1em} $\text{let } p_{old} = \mathsf{proj}_1(v_{aud}) \quad \text{// 通过反射提取凭证中关联的原始异常证据}$
\end{algorithmic}
\end{algorithm}

\subsection{求解过程:知行合一的因果轨迹}

最终,该业务过程在 KOS-TL 中被求解为一条完整的、带有数学证明的轨迹 $\mathcal{T}$:
$$\sigma_{init} \xrightarrow{p_{old} : \mathsf{QualIssue}} \sigma_{frozen} \xrightarrow{p_{rect} : \mathsf{QualRectified}} \sigma_{unlocked}$$

\textbf{结论与优势:}
\begin{enumerate}
    \item \textbf{强约束执行:} 任何不具备 $\mathsf{QualRectified}$ 证据项的解锁尝试都将在 Kernel 层触发 \textit{Type Mismatch} 错误,消灭了人为绕过规则的可能。
    \item \textbf{审计透明度:} 财务状态的变更不再是孤立的数值修改。通过 $\Sigma$-Type,每一个财务动作背后都内生地关联了不可伪造的质量证据,实现了真正的“逻辑穿透”。
    \item \textbf{非单调性的体现:} 随着 $p_{rect}$ 的引入,原本因 $p_{old}$ 导致的“不可支付”状态在新的状态空间中被剪枝,代之以新的、经过验证的可支付路径。
\end{enumerate}

\section{KOS-TL 的反事实推理与根因判定}

在 KOS-TL 架构中,反事实推理(Counterfactual Reasoning)并非基于概率密度的经验猜测,而是基于\textbf{构造性逻辑的必要性验证}。它通过内核在平行上下文(Parallel Contexts)中尝试重新实例化证明项,来精确判定各事实项对结果的因果贡献度。

\subsection{反事实推理的形式化定义}

在 KOS-TL 的公式化表达中,反事实判定被定义为对证明项存在性的否定推导:
\begin{equation}
\mathcal{C} \vdash \neg a \implies \neg (\exists \pi : \mathsf{Proof}(f))
\end{equation}
该公式表达了在当前系统配置 $\mathcal{C}$ 下的逻辑断言:若原子事实 $a$ 不存在,则失效事件 $f$ 的证明项 $\pi$ 无法被实例化。

\subsection{实现机制:虚拟上下文与环境切片}

KOS-TL 通过内核对\textbf{平行状态空间(Parallel State Spaces)}的模拟来完成推理,其核心在于环境 $\Gamma$ 的分支(Branching)机制:

\begin{enumerate}
    \item \textbf{构造影子状态 ($\sigma \setminus \{a\}$):} 内核生成当前知识库 $\sigma$ 的一个逻辑切片,有针对性地移除或修改某个事实项(例如移除电压异常证明 $a_{volt}$)。
    \item \textbf{假设性小步演化:} 在新的虚拟配置 $\langle \Gamma, \sigma \setminus \{a\}, f \rangle$ 下重新启动 Small-step 演化规约。
    \item \textbf{归约与引理对比:}
    \begin{itemize}
        \item 若演化结果归约为 $\bot$(底类型/空集),则判定 $a$ 是 $f$ 的\textbf{必要因果}。
        \item 若演化依然能生成新的合法证明项 $\pi'$,则说明存在\textbf{冗余因果}或 $a$ 仅为干扰噪声。
    \end{itemize}
\end{enumerate}

\subsection{根因判定的逻辑深度:构造与验证的解耦}

利用反事实推理,Kernel 可以从逻辑层面精确判定根本原因(Root Cause),实现从“相关性”向“因果性”的跨越:

\begin{itemize}
    \item \textbf{必要性判定(Elimination Rule):}
    利用类型论中的消除规则计算贡献度(Contribution):
    \begin{equation}
    \mathsf{Contrib}(a, f) \iff (\sigma \vdash f) \wedge (\sigma \setminus \{a\} \nvdash f)
    \end{equation}
    这里的 $\nvdash$ 表示在剥离 $a$ 后,无法从剩余知识库中构造出任何指向 $f$ 的合法证明。

    \item \textbf{通过否定来肯定:}
    当反事实轨迹 $\langle \Gamma, \sigma \setminus \{a\}, f \rangle \implies \bot$ 时,逻辑演算的坍缩反而加强了现实轨迹的因果权威性。这在逻辑上排除了“即使电压不波动,硬度也会因其他原因降低”的替代解释。
\end{itemize}

\subsection{应用场景分析}

\begin{enumerate}[label=(\arabic*)]
    \item \textbf{根因敏感度分析 (Sensitivity Analysis):}
    通过修改 $\sigma$ 中 $a_{volt}$ 的数值(如将波动阈值从 $10\%$ 模拟降至 $2\%$),观察 $\mathsf{mkCausalProof}$ 在 Core 层定义的物理约束中是否仍能通过,从而判定事故的临界边界。

    \item \textbf{责任归属判定 (Liability Attribution):}
    在虚拟环境中移除“供应商 A 的原材料”事实项。若证明链断裂,则支撑了对供应商的追偿逻辑;若证明链完整,则说明缺陷源于内部工序(如 M\_03 炉)的内生波动。

    \item \textbf{预防性模拟 (Preemptive Simulation):}
    构造“前瞻性反事实”:若未来状态提升功率 $5\%$,会诱发失效证明的实例化吗?这允许系统在物理事故发生前,在逻辑空间内完成“虚拟事故”的预演与剪枝。
\end{enumerate}

\begin{algorithm}
\caption{KOS-TL 反事实根因判定算法(基于平行上下文)}
\label{alg:kostl-counterfactual}
\begin{algorithmic}[1]

\REQUIRE 现实状态 $\sigma$,已证失效 $f : \mathsf{FailEvt}$,待检事实 $a : \mathsf{Anomaly}$
\ENSURE  $\mathsf{ContribDegree}$(因果贡献度判定)

\vspace{0.5em}

\STATE \textbf{1. 现实轨迹验证 (Real-world Baseline)}
\STATE \hspace*{\algorithmicindent} $\text{let } \pi_{real} = \mathsf{searchProof}(\sigma, \mathsf{CausalProof}(a, f))$
\STATE \hspace*{\algorithmicindent} \textbf{if } $\pi_{real} = \mathsf{None}$ \textbf{ then return } $\mathsf{Irrelevant}$ \quad \text{// $a$ 甚至不在当前的因果链中}

\vspace{0.5em}

\STATE \textbf{2. 构造影子上下文 (Shadow Context Construction)}
\STATE \hspace*{\algorithmicindent} $\sigma' \gets \sigma \setminus \{ a \}$ \quad \text{// 从知识库中逻辑移除原子项 $a$}
\STATE \hspace*{\algorithmicindent} $\Gamma' \gets \mathsf{branch}(\Gamma, \sigma')$ \quad \text{// 创建平行演化分支}

\vspace{0.5em}

\STATE \textbf{3. 反事实实例化尝试 (Counterfactual Re-instantiation)}
\STATE \hspace*{\algorithmicindent} $\mathsf{findAlternative}(\sigma', f) \triangleq$
\STATE \hspace*{\algorithmicindent}\hspace*{1em} $\text{let } \text{Goals} = \{ \pi' \mid \pi' : \mathsf{CausalProof}(x, f) \text{ s.t. } x \in \sigma' \}$
\STATE \hspace*{\algorithmicindent}\hspace*{1em} $\textbf{return } \mathsf{searchProof}(\sigma', \text{Goals})$

\vspace{0.5em}

\STATE \textbf{4. 逻辑对比与归约 (Logical Reduction)}
\STATE \hspace*{\algorithmicindent} $\text{let } \pi_{alt} = \mathsf{findAlternative}(\sigma', f)$
\STATE \hspace*{\algorithmicindent} $\textbf{match } \pi_{alt} \textbf{ with}$
\STATE \hspace*{\algorithmicindent}\hspace*{1em} $| \mathsf{None} \Rightarrow$
\STATE \hspace*{\algorithmicindent}\hspace*{2em} \textbf{assert } $(\sigma \vdash f) \land (\sigma' \nvdash f)$
\STATE \hspace*{\algorithmicindent}\hspace*{2em} \textbf{return } $\mathsf{NecessaryCause}$ \quad \text{// 演化坍缩至 $\bot$,证明 $a$ 是根因}
\STATE \hspace*{\algorithmicindent}\hspace*{1em} $| \mathsf{Some}(\pi'_{other}) \Rightarrow$
\STATE \hspace*{\algorithmicindent}\hspace*{2em} \textbf{if } $\mathsf{confidence}(\pi'_{other}) > \theta$ \textbf{ then}
\STATE \hspace*{\algorithmicindent}\hspace*{3em} \textbf{return } $\mathsf{RedundantCause}$ \quad \text{// 存在替代路径, $a$ 非唯一必要条件}
\STATE \hspace*{\algorithmicindent}\hspace*{2em} \textbf{else}
\STATE \hspace*{\algorithmicindent}\hspace*{3em} \textbf{return } $\mathsf{PrimaryContributor}$ \quad \text{// 替代路径强度弱, $a$ 仍是主因}

\vspace{0.5em}

\STATE \textbf{5. 敏感度探测(可选扩展)}
\STATE \hspace*{\algorithmicindent} $\sigma_{eps} \gets \mathsf{mutate}(a, \epsilon) \quad \text{// 扰动 $a$ 的参数值}$
\STATE \hspace*{\algorithmicindent} \textbf{if } $\mathsf{searchProof}(\sigma_{eps}, \mathsf{CausalProof}) = \bot$ \textbf{ then return } $\mathsf{CriticalThreshold}$

\end{algorithmic}
\end{algorithm}

\subsection{结论:从“改想法”到“改世界线”}

KOS-TL 的反事实推理将认识论中的怀疑转化为了内核中的确定性计算。它证明了知识不仅是关于“发生了什么”的被动记录,更是关于“如果不发生会怎样”的严密逻辑图谱。这种能力使得 KOS-TL 在处理高价值制造中的多因果复合故障时,具有极高的诊断精度与法律效力级的可追溯性。



\section{决策即合规性推理:基于 Trace 的可证明合法性}
\label{sec:decision-as-compliance}

前文(质量追溯、跨域锁定、反事实根因判定)中,系统会产出多类“决策”:例如认定某异常 $a$ 为失效 $f$ 的\textbf{必要原因}、对某批次的货款凭证执行\textbf{冻结}或\textbf{解锁}、或出具一份\textbf{根因报告} $r$。本节延续这些例子,说明此类决策在 KOS-TL 中何以不再是“简单的预测模型”的结论,而是\textbf{合规性推理}的产物;并给出具体示例,展示\textbf{每个决策}背后如何通过合法的 \textbf{trace}(轨迹)被证明其合法性。

\subsection{从预测到合规性推理}

在传统数据驱动系统中,决策通常由“输入数据 + 预测/规则模型 $\to$ 输出结论”构成。结论的正确性依赖模型与数据质量,但\textbf{为何该结论在逻辑上被允许}往往缺乏显式、可验证的根据。在 KOS-TL 中,决策被刻画为状态演化过程中的\textbf{合法步骤}:合法性边界由 Core 层类型与约束显式定义;决策即合法构造的产出(只有通过类型检查与证明项构造才能物化);每个决策对应一条可展示的 trace,从 $\sigma_0$ 到终态步步可追溯。因此,系统不仅依赖数据和算法,更依赖\textbf{逻辑宪法}与\textbf{依宪演算};决策的可接受性由“是否存在于某条合法 trace 的终点”判定。

\subsection{Trace 与可证明合法性}

“每个决策背后都可以通过合法的 trace 被证明其合法性”在形式层面体现为:(1)轨迹的良构性——每一步 $\mathsf{STEP}(\sigma_{i-1}, e_i) = \sigma_i$ 均由内核与类型系统定义;(2)决策即轨迹上的物化项——最终决策是某类型 $T$ 的实例,其存在性即“存在合法 trace 到达该决策”的证明;(3)可审计与可解释——审计方可要求系统展开 trace,从原始数据引用到证明项再到构造子应用,逐环验证。这与“预测模型 + 解释工具”有本质区别:在 KOS-TL 中,\textbf{合法性证明与决策生成是同一过程}。

\subsection{具体示例:三类决策及其合法 trace}
\label{sec:decision-examples}

下面沿用本章的轴承质量追溯与质量-财务跨域场景,对三类典型决策分别给出“决策内容—对应 trace—合规性含义”的具体示例。

\subsubsection{示例一:根因判定决策——“$a_{volt}$ 为批次 B2310 硬度失效的必要原因”}

\textbf{决策内容:} 系统输出“电压异常 $a_{volt}$ 是批次 B2310 硬度失效 $f_0$ 的必要原因”(即算法\ref{alg:kostl-counterfactual} 返回 $\mathsf{NecessaryCause}$)。这不是“模型给出的相关性排序”,而是一条\textbf{合规性推理}的结论。

\textbf{对应的合法 trace(摘要):}
\begin{enumerate}[label=(\roman*)]
  \item \textbf{现实轨迹验证:} $\pi_{real} = \mathsf{searchProof}(\sigma, \mathsf{CausalProof}(a_{volt}, f_0)) = \mathsf{Some}(\pi_{causal})$,即当前 $\sigma$ 下已存在因果证明。
  \item \textbf{构造影子上下文:} $\sigma' = \sigma \setminus \{ a_{volt} \}$,在平行分支 $\Gamma'$ 下演化。
  \item \textbf{反事实实例化:} $\mathsf{findAlternative}(\sigma', f_0) = \mathsf{None}$——在移除 $a_{volt}$ 后无法再构造出 $f_0$ 的因果证明。
  \item \textbf{逻辑归约与输出:} 由 $(\sigma \vdash f_0) \wedge (\sigma' \nvdash f_0)$ 得 $\mathsf{Contrib}(a_{volt}, f_0)$,算法返回 $\mathsf{NecessaryCause}$。
\end{enumerate}

\textbf{合规性含义:} 该决策的合法性由上述整条 trace 保证。审计方可逐步核验:$\pi_{causal}$ 来自 Core 层定义的 $\mathsf{CausalProof}$ 类型;影子上下文与 $\mathsf{findAlternative}$ 的语义由内核规则固定;返回“必要原因”仅当类型论意义上的必要性条件被满足。因此,这不是“黑箱给出结论再事后解释”,而是\textbf{只有在这条 trace 存在时},“必要原因”这一决策才会被系统接受并输出。

\subsubsection{示例二:财务冻结决策——“对批次 $b$ 的货款凭证 $v$ 执行冻结”}

\textbf{决策内容:} 系统将凭证 $v$ 从“可支付”变为“待审计冻结”状态(即物化 $\mathsf{AuditVoucher}(v, b)$)。该决策直接延续前文“跨域逻辑锁定”一节中的质量-财务协同场景。

\textbf{对应的合法 trace(摘要):}
\begin{enumerate}[label=(\roman*)]
  \item \textbf{质量域证据存在:} $\mathsf{searchProof}(\sigma, \mathsf{QualIssue}(b)) = \mathsf{Some}(p_{qual})$,即当前状态 $\sigma$ 中已存在批次 $b$ 的质量异常证明 $p_{qual} : \mathsf{QualIssue}(b)$。
  \item \textbf{跨域锁定应用:} $\mathsf{applyLock}(b, v, \sigma)$ 匹配到 $p_{qual}$,调用 $\mathsf{mkAuditVoucher}(v, p_{qual})$,得到 $v_{new} : \mathsf{AuditVoucher}(v, b)$。
  \item \textbf{状态更新:} $\sigma \xrightarrow{\langle e_{lock}, v_{new} \rangle} \sigma'$,其中 $v$ 在 $\sigma'$ 中已被替换为 $v_{new}$,财务状态变为冻结。
\end{enumerate}

\textbf{合规性含义:} “冻结”决策的合法性由“存在 $p_{qual}$”与“$\mathsf{AuditVoucher}$ 的构造子仅接受 $\mathsf{QualIssue}(b)$ 的证明项”共同保证。Core 层定义规定了“无质量异常证明则无法构造待审计凭证”;因此,不存在“绕过质量证据、仅凭人工或规则引擎标记”的合法冻结。审计方可展开 trace:从 $p_{qual}$ 追溯到质量域的因果链与原始数据,再追溯到 $\mathsf{mkAuditVoucher}$ 的输入,从而在逻辑上证明该冻结决策的合法性。

\subsubsection{示例三:根因报告决策——“出具 RootCauseReport $r$”}

\textbf{决策内容:} 系统向用户输出一份根因报告 $r$,其中包含因果证明 $\text{prf}_{causal}$,并可定位到原始 PLC 日志等数据。这对应本章开篇轴承案例中“用户拿到的逻辑证明包”。

\textbf{对应的合法 trace(摘要):}
\begin{enumerate}[label=(\roman*)]
  \item \textbf{事件与数据摄入:} Runtime 层将质检异常与设备日志精化为 $f_0 : \mathsf{FailEvt}$、$a_{temp} : \mathsf{Anomaly}$ 等项,并写入 $\sigma$。
  \item \textbf{因果证据检索与证明合成:} Kernel 在 $\sigma$ 中检索到该批次在 08:00 经 HeatTreatment\_03 炉,且该炉在 07:55 存在温控波动 $a_{temp}$;通过类型检查与 $\mathsf{mkCausalProof}$ 合成 $\text{prf}_{causal} : \mathsf{CausalProof}(a_{temp}, e_{proc}, f_0)$,并校验 $07:55 < 10:00$ 等约束。
  \item \textbf{物化报告:} 构造 $r = \langle f_0, a_{temp}, \text{prf}_{causal} \rangle : \mathsf{RootCauseReport}$,并交付用户;用户点击可展开 $\text{prf}_{causal}$ 并定位到原始 PLC 日志。
\end{enumerate}

\textbf{合规性含义:} 报告 $r$ 的存在性本身即“存在一条从数据摄入到证明合成再到物化的合法 trace”的证明。若业务规则变更(如要求“双重异常”才认定严重缺陷),Core 层精化 $\mathsf{CausalProof}$ 后,只有满足新约束的 trace 才能产出新报告;旧规则下的 $r$ 在新类型下将无法通过检查。因此,每个“出具报告”的决策都可被其对应的 trace 完整证明,满足可追溯、可解释、可审计的合规需求。

\subsection{小结}

上述三例表明:根因判定、财务冻结、根因报告等决策在 KOS-TL 中均对应一条\textbf{显式的、良构的 trace},且该 trace 的每一步均由 Core 层类型与 Kernel 小步语义定义。系统不仅依赖数据和算法,更依赖逻辑宪法与依宪演算;\textbf{每个决策}背后都可以通过这条合法 trace 被证明其合法性,从而将决策从“简单预测”升维为“合规性推理”。


\section{应用示例:AI 决策的可信性与合规边界}
\label{sec:ai-decision-trustworthiness}

在 AI 系统中,KOS-TL 可被用于确保“AI 决策的可信性”:不仅回答“为什么 AI 给出这个建议”,更可刻画“在规范范围内,AI \textbf{能够做出的所有可能建议}”的边界,使这些建议均可通过 KOS-TL 的形式化证明被审查与验证,从而避免 AI 系统产生不合规或不合法的建议。本节沿用本章的轴承质量追溯与质量-财务跨域案例,给出具体到类型与项的应用示例。

\subsection{从“单一解释”到“合规建议空间”}

传统可解释 AI 往往在模型输出某一建议 $y$ 后,再事后生成“解释”(如特征重要性或反事实说明)。这能回答“为什么这次给出了 $y$”,但无法回答:\textbf{系统是否可能在其他输入或内部状态下输出一个不合规的 $y'$?} 在 KOS-TL 中,决策被建模为类型论中的\textbf{可构造项};“所有可能建议”即所有在给定规范(Core 层类型与约束)下可被构造的、属于某“建议类型”的项。不合规建议对应的是\textbf{无法通过类型检查的构造},因此系统在逻辑上不可能将其物化为输出。

\subsection{建议类型与合规性索引}

沿用本章类型,将“AI 工艺/质量顾问”可输出的建议与既有决策类型对齐,并显式给出“建议类型”对合规证据的索引关系。

\begin{itemize}
  \item \textbf{冻结建议} $\mathsf{RecLock}(b, v)$:表示“建议对批次 $b$ 的凭证 $v$ 执行冻结”。在 Core 层规定:
  \[
  \mathsf{RecLock}(b, v) \equiv \Sigma(p : \mathsf{QualIssue}(b)).\, \mathsf{AuditVoucher}(v, b).
  \]
  即:只有存在 $p : \mathsf{QualIssue}(b)$ 时,$\mathsf{AuditVoucher}(v, b)$ 才可被 $\mathsf{mkAuditVoucher}(v, p)$ 构造;进而只有此时 $\mathsf{RecLock}(b, v)$ 才可被 inhabited。AI 若输出“建议冻结 $(b, v)$”,则该输出在类型论上必须是一个依赖 $p$ 的项 $\mathsf{recLock}(b, v, p, v_{audit})$,其中 $v_{audit} = \mathsf{mkAuditVoucher}(v, p)$。无 $p$ 则无法构造该类型,\textbf{不合规的冻结建议在类型层面被排除}。

  \item \textbf{根因报告建议} $\mathsf{RecReport}$:表示“建议出具根因报告”。与既有 $\mathsf{RootCauseReport}$ 一致:
  \[
  \mathsf{RecReport} \equiv \Sigma(f : \mathsf{FailEvt}).\, \Sigma(a : \mathsf{Anomaly}).\, \mathsf{CausalProof}(a, f).
  \]
  任意可输出的报告建议 $r : \mathsf{RecReport}$ 必须包含 $f$、$a$ 以及 $\pi : \mathsf{CausalProof}(a, f)$;$\pi$ 的构造又依赖 Kernel 的 $\mathsf{mkCausalProof}$ 与时空约束等。因此,“建议出具报告”与“存在一条可构造因果证明的 trace”一一对应;无法构造 $\mathsf{CausalProof}(a, f)$ 时,类型 $\mathsf{RecReport}$ 不可 inhabited,系统不会产出该建议。

  \item \textbf{必要原因判定建议} $\mathsf{RecNecessaryCause}(a, f)$:表示“建议将异常 $a$ 判定为失效 $f$ 的必要原因”。与算法\ref{alg:kostl-counterfactual} 的返回类型一致,可定义为依赖反事实 trace 的证明项类型(例如 $\mathsf{NecessaryCause}$ 仅当 $\mathsf{Contrib}(a, f)$ 在反事实验证下成立时才可构造)。形式上有:
  \[
  \mathsf{RecNecessaryCause}(a, f) \rightarrow \mathsf{CausalProof}(a, f) \times (\sigma' \nvdash f),
  \]
  其中 $\sigma'$ 为移除 $a$ 后的影子状态。因此,AI 无法输出“将某异常判定为必要原因”而不提供相应因果证明与反事实不可达证明。
\end{itemize}

\subsection{具体场景:轴承批次 B001 的“可证明建议空间”}

设当前状态为 $\sigma$,对应轴承批次 B001 已发生电压异常、质量域已存在 $p_{fail} : \mathsf{QualIssue}(\mathtt{B001})$,且财务凭证 $v_0$ 尚未冻结。在此状态下:

\begin{enumerate}[label=(\roman*)]
  \item \textbf{可被构造的建议(合规):}
  \begin{itemize}
    \item 冻结建议:因 $\mathsf{searchProof}(\sigma, \mathsf{QualIssue}(\mathtt{B001})) = \mathsf{Some}(p_{fail})$,可调用 $\mathsf{mkAuditVoucher}(v_0, p_{fail})$ 得到 $v_{audit} : \mathsf{AuditVoucher}(v_0, \mathtt{B001})$,进而得到 $\mathsf{recLock}(\mathtt{B001}, v_0, p_{fail}, v_{audit}) : \mathsf{RecLock}(\mathtt{B001}, v_0)$。AI 若输出“建议对 B001 的 $v_0$ 执行冻结”,则该输出即该项,且其 trace 可追溯到 $p_{fail}$ 及质量域的因果链。
    \item 根因报告建议:若 Kernel 已在 $\sigma$ 中构造出 $\pi_{causal} : \mathsf{CausalProof}(a_{volt}, f_0)$($f_0$ 为 B001 的硬度失效事件),则可物化 $r : \mathsf{RootCauseReport}$,对应“建议出具根因报告”的项 $r : \mathsf{RecReport}$。该建议的合法性由 $\pi_{causal}$ 及 $\mathsf{mkCausalProof}$ 的输入约束保证。
  \end{itemize}
  \item \textbf{不可被构造的“建议”(系统不会产出):}
  \begin{itemize}
    \item 对\textbf{无质量异常}的批次 $b'$ 建议冻结其凭证:类型 $\mathsf{RecLock}(b', v)$ 要求存在 $p : \mathsf{QualIssue}(b')$。若 $\sigma \vdash \neg \mathsf{QualIssue}(b')$ 或证明搜索返回 $\mathsf{None}$,则无法构造 $p$,从而 $\mathsf{RecLock}(b', v)$ 不可 inhabited,AI 系统在类型检查阶段即无法产出“建议冻结 $b'$ 的凭证”这一输出。
    \item 在\textbf{无因果证明}的情况下建议“$a$ 为 $f$ 的必要原因”:$\mathsf{RecNecessaryCause}(a, f)$ 依赖 $\mathsf{CausalProof}(a, f)$ 与反事实验证;若 $\mathsf{searchProof}(\sigma, \mathsf{CausalProof}(a, f)) = \mathsf{None}$,则无法构造该建议类型,系统不会给出该判定。
  \end{itemize}
\end{enumerate}

因此,在给定 $\sigma$ 下,“AI 能够做出的所有可能建议”恰好等于类型系统中所有在当前状态可构造的、属于上述建议类型的项的集合;该集合由 Core 层类型与 Kernel 的证明搜索与构造子\textbf{显式界定},不合规建议因类型不可构造而无法出现。

\subsection{审查与验证:Trace 即合规证明}

监管或审计方若要求验证“AI 的每个建议是否合规”,在 KOS-TL 中可采取两种等价方式:(1)\textbf{按建议追溯}:给定 AI 输出的一项建议(如 $\mathsf{recLock}(\mathtt{B001}, v_0, p_{fail}, v_{audit})$),展开其依赖的 $p_{fail}$ 与 $v_{audit}$ 的构造 trace,回溯到质量域证据与 $\mathsf{mkAuditVoucher}$ 的调用,逐步核验;(2)\textbf{按类型枚举}:对“所有可能建议类型”做证明搜索(如对所有批次 $b$ 检查 $\mathsf{QualIssue}(b)$ 是否可证),得到的可构造项集合即为“在规范范围内 AI 能够做出的所有可能建议”,可与实际输出做包含关系检查,确保无超范围输出。两种方式均依赖同一套 Trace 与类型构造,从而将“AI 决策的可信性”落实为可证明合法性,避免不合规或不合法的建议被系统产出。


\section{轴承案例中的轨迹 (Trace) 归纳与形式化分析}

在“轴承热处理”案例中,轨迹(Trace)不仅是生产数据的线性记录,更是一条由证据项驱动的、跨领域耦合的因果演化链。它记录了系统从正常运行到异常锁定,再到反事实诊断与最终逻辑解锁的全生命周期。

\subsection{轨迹的形式化总览}
一条完整的演化轨迹 $\mathcal{T}$ 被定义为状态 $\sigma$ 在事件 $e$ 与证明 $p$ 联合驱动下的单调增长过程:
\begin{equation}
\begin{split}
\mathcal{T} &= \sigma_{init} \xrightarrow{\langle e_{proc}, p_{ok} \rangle} \sigma_{produce} \xrightarrow{\langle e_{volt}, p_{fail} \rangle} \sigma_{anomaly} \\
&\quad \xrightarrow{\langle e_{sys}, p_{lock} \rangle} \sigma_{frozen} \xrightarrow{\langle e_{rect}, p_{rect} \rangle} \sigma_{unlocked}
\end{split}
\end{equation}

\subsection{轨迹阶段的构造性分析}

\begin{enumerate}
    \item \textbf{生产初始化轨迹 ($\sigma_{init} \to \sigma_{produce}$)} \\
    轴承批次(Batch \#B001)进入热处理炉。证明项 $p_{ok}$ 满足 Core 层定义的工艺本体 $\Sigma$ 约束。状态 $\sigma_{produce}$ 中包含了该批次“合法加工中”的构造性命题。

    \item \textbf{异常演化轨迹 ($\sigma_{produce} \to \sigma_{anomaly}$)} \\
    传感器捕捉到电压波动(10\% 偏移)。内核通过 $\mathsf{elab}$ 算子提炼出满足 $\mathsf{CausalProof}$ 谓词的证明项。此时发生\textbf{非单调演化}:原本在 $\sigma_{produce}$ 中成立的“批次合格”推论在当前状态下变得不可构造。

    \item \textbf{跨域联动轨迹 ($\sigma_{anomaly} \to \sigma_{frozen}$)} \\
    Kernel 调用跨域引用算子 $\mathsf{applyLock}$。质量域的异常证明项 $p_{fail}$ 被注入财务域的 $\mathsf{AuditVoucher}$ 构造函数中。此时,财务状态在逻辑上被强制锚定在质量证明上,实现了生产与财务的原子级耦合。

    \item \textbf{反事实诊断分支 (Parallel Branching)} \\
    内核在平行上下文内创建影子状态 $\sigma' = \sigma \setminus \{a_{volt}\}$。通过尝试在不含电压异常的环境中重新构造失效证明,内核证实了演化坍缩(Collapse to $\bot$),从而在主轨迹中锁定了电压波动为硬度不达标的必要根因。

    \item \textbf{闭环解锁轨迹 ($\sigma_{frozen} \to \sigma_{unlocked}$)} \\
    质量部提交重检合格报告,构造出 $\mathsf{QualRectified}$ 证明项。该项通过 $\mathsf{CausalResolve}$ 逻辑性地遮蔽了旧有的 $p_{fail}$。财务凭证随之发生类型转换(Type Refinement),回归至 $\mathsf{NormalVoucher}$。
\end{enumerate}

\subsection{轨迹核心价值总结}

\begin{table}[h]
\centering
\caption{轴承案例中 Trace 的核心逻辑特性}
\begin{tabular}{@{}ll@{}}
\toprule
\textbf{特性} & \textbf{在轴承案例中的具体表现} \\ \midrule
因果透明度 & 财务冻结凭证内含逻辑指针,直接指向质量域的异常证明节点。 \\
单调增长性 & 系统未“删除”错误,而是通过叠加 $p_{rect}$ 实现了逻辑层面的修正与闭回。 \\
非单调表现 & 电压异常事件导致原有“合格”路径被剪枝,迫使系统进入“审计”分支。 \\
自审计能力 & 轨迹本身即为可计算的合规报告,无需额外生成离线的外部审计日志。 \\ \bottomrule
\end{tabular}
\end{table}

\textbf{结论:} 轴承案例的 Trace 证明了 KOS-TL 能够将碎片化的物理事件转化为逻辑严密的因果链。在这种机制下,任何状态变更都必须携带数学上的证据支撑,确保了复杂制造环境下“知”与“行”的绝对一致性。

\section{逻辑作为系统内核}

通过上述推理过程,KOS-TL 展示了其作为逻辑系统的核心优势:

\begin{enumerate}

\item \textbf{构造即证据}

    生成的$\mathsf{RootCauseReport}$不仅仅是文本,它包含了指向原始设备日志和质检记录的指针及逻辑证明链,实现了“可解释性”的硬保证。
\item \textbf{状态保持性}

    在 Small-step 每一跳中,内核均验证$Post$ 约束。如果追溯结果违反了物理逻辑(如原因发生在结果之后),该步转移将失败,确保系统状态  始终处于逻辑一致的空间。
\item \textbf{计算完备性}

    与描述逻辑的 Open World Assumption 不同,KOS-TL 在内核层利用强规范化性质,保证溯源程序在有限步内必然给出确定的因果结论或报告证据不足。
\end{enumerate}

在该实例中,KOS-TL 将质量追溯从“事后审计”转变为“实时逻辑演算”。系统不是在询问“为什么坏了”,而是在尝试构造一个关于“坏了”的完整逻辑证明。这种从 \textbf{Truth-judgement}(真假判断)向 \textbf{Proof-construction}(证明构造)的范式转移,使得制造业的复杂知识管理具备了操作系统级的严谨性。

本节通过大型制造业的质量异常分析场景,展示 KOS-TL 如何将异构业务表映射为依存类型,并通过 Small-step 操作语义实现知识的自动派生与因果追溯。

\subsection{状态$\sigma$与轨迹$\mathcal{T}$的详细演化路径}

在轴承生产质量追溯案例中,系统的演化过程体现为状态$\sigma$的序列化迁移和轨迹$\mathcal{T}$的逐步构造。本节详细分析从初始状态到最终解锁状态的完整演化路径。

\subsubsection{状态$\sigma$的形式化结构}

在KOS-TL中,状态$\sigma$被定义为三元组$\sigma = \langle \mathcal{K}, \mathcal{TS}, \mathcal{P} \rangle$,其中:
\begin{itemize}
    \item $\mathcal{K}$:当前生效的知识集合(Knowledge Set),包含所有已物化的类型项和证明项
    \item $\mathcal{TS}$:逻辑时间戳(Logical Timestamp),记录状态的逻辑时钟
    \item $\mathcal{P}$:事件队列(Event Queue),存储待处理的事件-证明对$\langle e, p \rangle$
\end{itemize}

\subsubsection{完整演化轨迹的详细分解}

轴承案例的完整演化轨迹$\mathcal{T}$包含五个关键状态和四个状态迁移步骤:

\begin{enumerate}
    \item \textbf{初始状态 $\sigma_{init}$}

    状态内容:
    \begin{align*}
    \sigma_{init} &= \langle \mathcal{K}_{init}, \mathcal{TS}=0, \mathcal{P}=\emptyset \rangle \\
    \mathcal{K}_{init} &= \{ \Gamma_0, \text{工艺路径定义}, \text{设备注册表} \}
    \end{align*}

    其中$\Gamma_0$包含所有预定义的类型、公理和函数签名(如$\mathsf{FailEvt}$、$\mathsf{ProcStep}$、$\mathsf{mkCausalProof}$ 等)。

    \item \textbf{状态迁移 $\sigma_{init} \xrightarrow{\langle e_{proc}, p_{ok} \rangle} \sigma_{produce}$}

    \textbf{事件$e_{proc}$}:批次Batch\_202310-01进入M\_03热处理炉,开始生产。

    \textbf{证明项$p_{ok}$}:通过$\mathsf{elab}$算子从Runtime层精化得到:
    $$p_{ok} = \pi_{route} : \mathsf{Proof}(\mathsf{InRoute}(\text{Batch\_202310-01}, \text{M\_03}))$$

    \textbf{状态更新}:
    \begin{align*}
    \sigma_{produce} &= \langle \mathcal{K}_{produce}, \mathcal{TS}=1, \mathcal{P}=\emptyset \rangle \\
    \mathcal{K}_{produce} &= \mathcal{K}_{init} \cup \{ e_{proc}, p_{ok} \}
    \end{align*}

    其中$e_{proc} = \text{mkStep}(\text{Batch\_202310-01}, \text{M\_03}, \langle 08:00, 09:30 \rangle, \pi_{route})$,类型为$\mathsf{ProcStep}$。

    \textbf{小步操作验证}:Kernel层验证$e_{proc}$的类型,确保$\pi_{route}$满足$\mathsf{InRoute}$谓词,通过后状态迁移成功。

    \item \textbf{状态迁移 $\sigma_{produce} \xrightarrow{\langle e_{volt}, p_{fail} \rangle} \sigma_{anomaly}$}

    \textbf{事件$e_{volt}$}:传感器在08:15检测到M\_03炉电压波动(15\%偏移)。

    \textbf{证明项$p_{fail}$}:通过$\mathsf{elab}$算子提炼异常项:
    $$a_{volt} = \text{mkAnomaly}(\text{M\_03}, \text{Voltage\_Drop}, 08:15, \pi_{iot\_hash})$$

    然后Kernel层启动因果搜索,调用$\mathsf{analyze}(f_0)$函数(其中$f_0$是10:00检测到的硬度异常):
    \begin{align*}
    \mathsf{deriveCausal}(a_{volt}, f_0) &= \mathsf{tryWithProc}(a_{volt}, f_0, e_{proc}) \\
    &= \mathsf{mkCausalProof}(a_{volt}, f_0, e_{proc}, p_{time}, p_{space}, p_{batch})
    \end{align*}

    其中:
    \begin{itemize}
        \item $p_{time} : \mathsf{TimeOK}(a_{volt}, e_{proc}, f_0)$ 证明 $(08:15 \in [08:00, 09:30]) \land (09:30 < 10:00)$
        \item $p_{space} : \mathsf{SpaceOK}(a_{volt}, e_{proc})$ 证明 $\text{M\_03} = \text{M\_03}$
        \item $p_{batch} : \mathsf{BatchOK}(e_{proc}, f_0)$ 证明 $\text{Batch\_202310-01} = \text{Batch\_202310-01}$
    \end{itemize}

    \textbf{状态更新}:
    \begin{align*}
    \sigma_{anomaly} &= \langle \mathcal{K}_{anomaly}, \mathcal{TS}=2, \mathcal{P}=\emptyset \rangle \\
    \mathcal{K}_{anomaly} &= \mathcal{K}_{produce} \cup \{ a_{volt}, f_0, prf_{causal}, p_{time}, p_{space}, p_{batch} \}
    \end{align*}

    其中$prf_{causal} : \mathsf{CausalProof}(a_{volt}, f_0)$是完整的因果证明项。

    \textbf{非单调演化}:此时,原本在$\sigma_{produce}$中可构造的"批次合格"命题在$\sigma_{anomaly}$中变得不可构造,因为存在$prf_{causal}$证明该批次存在质量风险。

    \item \textbf{状态迁移 $\sigma_{anomaly} \xrightarrow{\langle e_{sys}, p_{lock} \rangle} \sigma_{frozen}$}

    \textbf{事件$e_{sys}$}:系统检测到质量异常,触发跨域联动机制。

    \textbf{证明项$p_{lock}$}:Kernel层调用$\mathsf{applyLock}$算子:
    \begin{align*}
    p_{qual} &: \mathsf{QualIssue}(\text{Batch\_202310-01}) \\
    &= \Sigma(a_{volt}:\mathsf{Anomaly}).\Sigma(e_{proc}:\mathsf{ProcStep}).\mathsf{CausalProof}(a_{volt}, e_{proc}, \text{Batch\_202310-01})
    \end{align*}

    财务域通过类型交叉引用,自动订阅到$p_{qual}$,导致凭证对象从$\mathsf{Payable}$类型重构为$\mathsf{AuditVoucher}$类型:
    $$v_{audit} = \mathsf{mkAuditVoucher}(v_{original}, p_{qual}) : \mathsf{AuditVoucher}$$

    \textbf{状态更新}:
    \begin{align*}
    \sigma_{frozen} &= \langle \mathcal{K}_{frozen}, \mathcal{TS}=3, \mathcal{P}=\emptyset \rangle \\
    \mathcal{K}_{frozen} &= \mathcal{K}_{anomaly} \cup \{ p_{qual}, v_{audit} \}
    \end{align*}

    \textbf{跨域锁定}:此时财务凭证$v_{audit}$在类型级别被锁定,无法支付,因为其类型定义要求必须存在$\mathsf{QualRectified}$ 证明才能解锁。

    \item \textbf{反事实诊断分支(平行上下文)}

    在状态$\sigma_{anomaly}$时,Kernel层创建影子状态进行反事实推理:
    $$\sigma' = \sigma_{anomaly} \setminus \{ a_{volt} \}$$

    在$\sigma'$中重新尝试构造失效证明:
    $$\mathsf{analyze}(f_0) \text{ in } \sigma' = \mathsf{None}$$

    演化结果归约为$\bot$(底类型),证明$a_{volt}$是$f_0$的\textbf{必要因果},从而在主轨迹中锁定了电压波动为硬度不达标的根因。

    \item \textbf{状态迁移 $\sigma_{frozen} \xrightarrow{\langle e_{rect}, p_{rect} \rangle} \sigma_{unlocked}$}

    \textbf{事件$e_{rect}$}:质量部提交重检合格报告,执行整改动作。

    \textbf{证明项$p_{rect}$}:构造整改证明:
    \begin{align*}
    p_{rect} &: \mathsf{QualRectified}(\text{Batch\_202310-01}, p_{qual}) \\
    &= \Sigma(a_{rect}:\mathsf{Action}).\Sigma(p_{new}:\mathsf{QualPass}(\text{Batch\_202310-01})).\mathsf{CausalResolve}(a_{rect}, p_{qual}, p_{new})
    \end{align*}

    其中$\mathsf{CausalResolve}$证明整改动作$a_{rect}$在逻辑上解决了$p_{qual}$识别的失效风险。

    \textbf{状态更新}:
    \begin{align*}
    \sigma_{unlocked} &= \langle \mathcal{K}_{unlocked}, \mathcal{TS}=4, \mathcal{P}=\emptyset \rangle \\
    \mathcal{K}_{unlocked} &= \mathcal{K}_{frozen} \cup \{ p_{rect}, v_{normal} \}
    \end{align*}

    财务凭证通过$\mathsf{unlockPayment}$算子发生类型转换:
    $$v_{normal} = \mathsf{unlockPayment}(v_{audit}, p_{rect}) : \mathsf{NormalVoucher}$$

    \textbf{非单调性体现}:随着$p_{rect}$的引入,原本因$p_{qual}$导致的"不可支付"状态在新的状态空间中被剪枝,代之以新的、经过验证的可支付路径。
\end{enumerate}

\subsubsection{完整轨迹的形式化表示}

完整的演化轨迹$\mathcal{T}$可以形式化表示为:

\begin{equation}
\begin{aligned}
\mathcal{T} &= \sigma_{init} \xrightarrow{\langle e_{proc}, p_{ok} \rangle} \sigma_{produce} \xrightarrow{\langle e_{volt}, p_{fail} \rangle} \sigma_{anomaly} \\
&\quad \xrightarrow{\langle e_{sys}, p_{lock} \rangle} \sigma_{frozen} \xrightarrow{\langle e_{rect}, p_{rect} \rangle} \sigma_{unlocked} \\
&= \Sigma(n=4, e_1=e_{proc}, e_2=e_{volt}, e_3=e_{sys}, e_4=e_{rect}, \\
&\quad \sigma_1=\sigma_{produce}, \sigma_2=\sigma_{anomaly}, \sigma_3=\sigma_{frozen}, \sigma_4=\sigma_{unlocked}, \\
&\quad \prod_{i=1}^{4} \mathsf{Step}(\sigma_{i-1}, e_i, \sigma_i))
\end{aligned}
\end{equation}

其中每个$\mathsf{Step}(\sigma_{i-1}, e_i, \sigma_i)$都包含:
\begin{itemize}
    \item \textbf{前置条件证明}:证明事件$e_i$在状态$\sigma_{i-1}$下满足触发约束
    \item \textbf{后置条件证明}:证明状态$\sigma_i$满足演化后的逻辑约束
    \item \textbf{类型检查证明}:证明所有新构造的项都满足其类型定义
\end{itemize}

\subsubsection{状态$\sigma$的详细内容表}

表\ref{tab:sigma-evolution}详细列出了每个状态$\sigma_i$中包含的知识项:

\begin{table}[htbp]
\centering
\caption{状态$\sigma$的详细演化内容}
\label{tab:sigma-evolution}
\small
\begin{tabular}{>{\raggedright\arraybackslash}p{2cm} >{\raggedright\arraybackslash}p{3cm} >{\raggedright\arraybackslash}p{9cm}}
\toprule
\textbf{状态} & \textbf{逻辑时钟} & \textbf{知识集合$\mathcal{K}$内容} \\
\midrule
$\sigma_{init}$ & $\mathcal{TS}=0$ & $\Gamma_0$(类型上下文)、工艺路径定义、设备注册表 \\
\midrule
$\sigma_{produce}$ & $\mathcal{TS}=1$ & $\mathcal{K}_{init} \cup \{ e_{proc}:\mathsf{ProcStep}, p_{ok}:\mathsf{Proof}(\mathsf{InRoute}) \}$ \\
& & $e_{proc} = \text{mkStep}(\text{Batch\_202310-01}, \text{M\_03}, \langle 08:00, 09:30 \rangle, \pi_{route})$ \\
\midrule
$\sigma_{anomaly}$ & $\mathcal{TS}=2$ & $\mathcal{K}_{produce} \cup \{ a_{volt}:\mathsf{Anomaly}, f_0:\mathsf{FailEvt}, prf_{causal}:\mathsf{CausalProof}, p_{time}, p_{space}, p_{batch} \}$ \\
& & $a_{volt} = \text{mkAnomaly}(\text{M\_03}, \text{Voltage\_Drop}, 08:15, \pi_{iot\_hash})$ \\
& & $f_0 = \text{mkFailure}(\text{Batch\_202310-01}, \text{Hardness\_Issue}, 10:00, \pi_{QA\_Sign})$ \\
& & $prf_{causal} = \mathsf{mkCausalProof}(a_{volt}, f_0, e_{proc}, p_{time}, p_{space}, p_{batch})$ \\
\midrule
$\sigma_{frozen}$ & $\mathcal{TS}=3$ & $\mathcal{K}_{anomaly} \cup \{ p_{qual}:\mathsf{QualIssue}, v_{audit}:\mathsf{AuditVoucher} \}$ \\
& & $p_{qual} = \Sigma(a_{volt}:\mathsf{Anomaly}).\Sigma(e_{proc}:\mathsf{ProcStep}).\mathsf{CausalProof}(a_{volt}, e_{proc}, \text{Batch\_202310-01})$ \\
& & $v_{audit} = \mathsf{mkAuditVoucher}(v_{original}, p_{qual})$ \\
\midrule
$\sigma_{unlocked}$ & $\mathcal{TS}=4$ & $\mathcal{K}_{frozen} \cup \{ p_{rect}:\mathsf{QualRectified}, v_{normal}:\mathsf{NormalVoucher} \}$ \\
& & $p_{rect} = \Sigma(a_{rect}:\mathsf{Action}).\Sigma(p_{new}:\mathsf{QualPass}).\mathsf{CausalResolve}(a_{rect}, p_{qual}, p_{new})$ \\
& & $v_{normal} = \mathsf{unlockPayment}(v_{audit}, p_{rect})$ \\
\bottomrule
\end{tabular}
\end{table}

\subsubsection{轨迹$\mathcal{T}$的核心特性}

通过上述详细分析,轨迹$\mathcal{T}$展现了以下核心特性:

\begin{enumerate}
    \item \textbf{单调增长性}:知识集合$\mathcal{K}$在每个状态迁移中单调增长,系统未"删除"错误,而是通过叠加新证明项实现逻辑修正。

    \item \textbf{非单调表现}:虽然知识集合单调增长,但\textbf{可构造的命题集合}可能发生非单调变化。例如,$\sigma_{produce}$ 中可构造"批次合格",但$\sigma_{anomaly}$中该命题变得不可构造。

    \item \textbf{因果透明度}:每个状态迁移都携带明确的证明项,如$p_{time}$、$p_{space}$、$p_{batch}$等,这些证明项直接指向原始数据(如传感器日志、工单记录)。

    \item \textbf{跨域耦合}:在$\sigma_{frozen}$状态中,财务凭证$v_{audit}$通过类型依赖直接关联到质量证明$p_{qual}$,实现了跨域的逻辑锁定。

    \item \textbf{自审计能力}:轨迹$\mathcal{T}$本身即为可计算的合规报告,每个$\mathsf{Step}$证明都记录了状态迁移的合法性,无需额外生成外部审计日志。
\end{enumerate}

\subsubsection{小步操作语义的形式化}

每个状态迁移$\sigma_i \xrightarrow{\langle e, p \rangle} \sigma_{i+1}$都遵循小步操作语义:

$$\frac{\Gamma \vdash e : \mathsf{Event} \quad \Gamma, \sigma_i \vdash p : \mathsf{Proof}(\mathsf{Pre}(e, \sigma_i)) \quad \Gamma, \sigma_i, e, p \vdash \sigma_{i+1} : \mathsf{State} \quad \Gamma, \sigma_i, \sigma_{i+1} \vdash p_{post} : \mathsf{Proof}(\mathsf{Post}(\sigma_i, \sigma_{i+1}))}{\mathsf{Step}(\sigma_i, e, \sigma_{i+1})}$$

其中:
\begin{itemize}
    \item $\mathsf{Pre}(e, \sigma_i)$:事件$e$在状态$\sigma_i$下的前置条件
    \item $\mathsf{Post}(\sigma_i, \sigma_{i+1})$:状态迁移的后置条件
    \item $p_{post}$:后置条件的证明项
\end{itemize}

这种形式化确保了每一步状态迁移都是经过验证的、合法的逻辑操作,而非简单的数据更新。

\subsubsection{轨迹演化的可视化表示}

图\ref{fig:sigma-trace-evolution}展示了状态$\sigma$和轨迹$\mathcal{T}$的完整演化过程:

\par\noindent
\refstepcounter{figure}%
\textbf{图~\thefigure:}状态$\sigma$与轨迹$\mathcal{T}$的演化路径\label{fig:sigma-trace-evolution}\par\medskip
{\centering
\begin{tikzpicture}[scale=0.72, every node/.append style={transform shape},
    node distance=2cm and 3cm,
    state/.style={rectangle, draw=blue!50, fill=blue!10, thick, minimum width=2.5cm, minimum height=1.5cm, text centered, font=\footnotesize},
    event/.style={ellipse, draw=green!50, fill=green!10, thick, minimum width=2cm, minimum height=0.8cm, text centered, font=\tiny},
    proof/.style={diamond, draw=orange!50, fill=orange!10, thick, minimum width=1.5cm, minimum height=0.8cm, text centered, font=\tiny},
    arrow/.style={->, thick, blue!70},
    label/.style={font=\tiny, above, sloped}
]

% 状态节点(用 \parbox 包裹多行,避免 \\ 与 enumitem 冲突导致 missing \item)
\node[state] (sigma0) {\parbox{2.2cm}{\centering\footnotesize$\sigma_{\mathrm{init}}$\\[1pt]$\mathcal{TS}=0$\\[1pt]$\mathcal{K}=\{\Gamma_0\}$}};
\node[state, right=of sigma0] (sigma1) {\parbox{2.2cm}{\centering\footnotesize$\sigma_{\mathrm{produce}}$\\[1pt]$\mathcal{TS}=1$\\[1pt]$\mathcal{K} \cup \{e_{\mathrm{proc}}, p_{\mathrm{ok}}\}$}};
\node[state, right=of sigma1] (sigma2) {\parbox{2.2cm}{\centering\footnotesize$\sigma_{\mathrm{anomaly}}$\\[1pt]$\mathcal{TS}=2$\\[1pt]$\mathcal{K} \cup \{a_{\mathrm{volt}}, f_0, \mathrm{prf}_{\mathrm{causal}}\}$}};
\node[state, right=of sigma2] (sigma3) {\parbox{2.2cm}{\centering\footnotesize$\sigma_{\mathrm{frozen}}$\\[1pt]$\mathcal{TS}=3$\\[1pt]$\mathcal{K} \cup \{p_{\mathrm{qual}}, v_{\mathrm{audit}}\}$}};
\node[state, right=of sigma3] (sigma4) {\parbox{2.2cm}{\centering\footnotesize$\sigma_{\mathrm{unlocked}}$\\[1pt]$\mathcal{TS}=4$\\[1pt]$\mathcal{K} \cup \{p_{\mathrm{rect}}, v_{\mathrm{normal}}\}$}};

% 事件和证明
\node[event, below=1cm of sigma0] (e1) {$e_{proc}$};
\node[proof, below=0.3cm of e1] (p1) {$p_{ok}$};
\node[event, below=1cm of sigma1] (e2) {$e_{volt}$};
\node[proof, below=0.3cm of e2] (p2) {$p_{fail}$};
\node[event, below=1cm of sigma2] (e3) {$e_{sys}$};
\node[proof, below=0.3cm of e3] (p3) {$p_{lock}$};
\node[event, below=1cm of sigma3] (e4) {$e_{rect}$};
\node[proof, below=0.3cm of e4] (p4) {$p_{rect}$};

% 状态迁移箭头
\draw[arrow] (sigma0) -- node[label] {$\langle e_{proc}, p_{ok} \rangle$} (sigma1);
\draw[arrow] (sigma1) -- node[label] {$\langle e_{volt}, p_{fail} \rangle$} (sigma2);
\draw[arrow] (sigma2) -- node[label] {$\langle e_{sys}, p_{lock} \rangle$} (sigma3);
\draw[arrow] (sigma3) -- node[label] {$\langle e_{rect}, p_{rect} \rangle$} (sigma4);

% 事件到状态的连接
\draw[->, dashed, green!50] (e1) -- (sigma0);
\draw[->, dashed, green!50] (e2) -- (sigma1);
\draw[->, dashed, green!50] (e3) -- (sigma2);
\draw[->, dashed, green!50] (e4) -- (sigma3);

% 证明到状态的连接
\draw[->, dashed, orange!50] (p1) -- (sigma0);
\draw[->, dashed, orange!50] (p2) -- (sigma1);
\draw[->, dashed, orange!50] (p3) -- (sigma2);
\draw[->, dashed, orange!50] (p4) -- (sigma3);

% 反事实分支
\node[state, above=2cm of sigma2, draw=red!50, fill=red!10] (sigma2prime) {\parbox{2.5cm}{\centering\footnotesize$\sigma' = \sigma_{\mathrm{anomaly}} \setminus \{a_{\mathrm{volt}}\}$\\[1pt]反事实分支}};
\draw[->, dashed, red!50] (sigma2) to[out=90, in=-90] node[label, left] {反事实推理} (sigma2prime);
\draw[->, dashed, red!50] (sigma2prime) to[out=180, in=90] node[label, above] {$\mathsf{analyze}(f_0) = \mathsf{None}$} (sigma2);

\end{tikzpicture}
\par}

图\ref{fig:sigma-trace-evolution}清晰地展示了:
\begin{itemize}
    \item \textbf{状态序列}:五个关键状态$\sigma_{init} \to \sigma_{produce} \to \sigma_{anomaly} \to \sigma_{frozen} \to \sigma_{unlocked}$
    \item \textbf{事件驱动}:每个状态迁移都由事件$e_i$触发
    \item \textbf{证明伴随}:每个状态迁移都携带证明项$p_i$,确保迁移的合法性
    \item \textbf{反事实分支}:在$\sigma_{anomaly}$状态时创建影子状态$\sigma'$进行反事实推理
    \item \textbf{知识增长}:每个状态的$\mathcal{K}$集合都单调增长
\end{itemize}

\subsubsection{轨迹演化的关键洞察}

通过上述详细分析,我们可以得出以下关键洞察:

\begin{enumerate}
    \item \textbf{状态即知识快照}:每个状态$\sigma_i$不仅记录了"数据",更记录了"可构造的知识"。状态迁移不是简单的数据更新,而是知识集合的逻辑扩展。

    \item \textbf{证明即迁移凭证}:每个状态迁移$\sigma_i \xrightarrow{\langle e, p \rangle} \sigma_{i+1}$中的证明项$p$不仅是"为什么可以迁移"的证据,更是"迁移后状态合法"的保证。

    \item \textbf{轨迹即审计日志}:完整的轨迹$\mathcal{T}$本身就是一条可计算的审计日志,每个$\mathsf{Step}$证明都记录了状态迁移的合法性,无需额外生成外部日志。

    \item \textbf{非单调性的体现}:虽然知识集合$\mathcal{K}$单调增长,但\textbf{可构造的命题集合}可能发生非单调变化,这体现了KOS-TL处理非单调逻辑的能力。

    \item \textbf{跨域耦合的原子性}:在$\sigma_{frozen}$状态中,财务凭证通过类型依赖直接关联到质量证明,这种耦合是类型级别的,无法绕过。
\end{enumerate}

这种基于状态$\sigma$和轨迹$\mathcal{T}$的演化机制,使得KOS-TL能够将碎片化的物理事件转化为逻辑严密的因果链,确保了复杂制造环境下"知"与"行"的绝对一致性。

为了全面展示本章中涉及的所有逻辑构造,我们将所有类型(包括谓词)和项整理如下:

\begin{table}[htbp]
\centering
\caption{KOS-TL应用示例中的类型(含谓词)汇总表}
\label{tab:types-predicates-summary}
\small
\begin{tabular}{>{\raggedright\arraybackslash}p{2cm} >{\raggedright\arraybackslash}p{3.5cm} >{\raggedright\arraybackslash}p{7cm}}
\toprule
\textbf{类别} & \textbf{名称} & \textbf{定义/说明} \\
\midrule
\multirow{8}{*}{\textbf{基础原子类型}}
& $\mathsf{BatchID}$ & 批次标识符类型 \\
& $\mathsf{Machine}$ & 机器/设备类型 \\
& $\mathsf{Time}$ & 时间类型(Unix时间戳或逻辑时钟) \\
& $\mathsf{ErrorCode}$ & 错误代码类型 \\
& $\mathsf{Param}$ & 参数类型 \\
& $\mathsf{Val}$ & 值类型 \\
& $\mathsf{Voucher}$ & 凭证类型 \\
& $\mathsf{Action}$ & 动作类型 \\
\midrule
\multirow{16}{*}{\textbf{谓词类型}}
& $\mathsf{InRoute}(b, m)$ & 批次 $b$ 是否允许在机器 $m$ 上加工 \\
& $\mathsf{Overlap}(t, dur)$ & 时间点 $t$ 是否落在区间 $dur$ 之内 \\
& $\mathsf{isBefore}(t(a), t(f))$ & 时间顺序谓词:$a$ 的时间早于 $f$ 的时间 \\
& $\mathsf{isSameResource}(\text{location}(a), \text{process}(f))$ & 资源一致性谓词 \\
& $\mathsf{IsValidRoute}$ & 工艺一致性约束谓词 \\
& $\mathsf{TimeOK}(a,e,f)$ & 时序一致性:$a.t \in e.dur \land e.dur.end < f.t$ \\
& $\mathsf{SpaceOK}(a,e)$ & 空间一致性:$a.m = e.m$ \\
& $\mathsf{BatchOK}(e,f)$ & 批次一致性:$e.b = f.b$ \\
& $\mathsf{Prop}_{causal}(a, e, f)$ & 复合因果谓词(时序+空间+批次一致性) \\
& $\mathsf{Prop}_{time}(a, e, f)$ & 时序因果谓词 \\
& $\mathsf{Prop}_{joint}(a, w, e, f)$ & 联合异常谓词(电压+水压) \\
& $\mathsf{QualIssue}(b)$ & 质量域异常证明:$\Sigma(a:\mathsf{Anomaly}).\Sigma(e:\mathsf{ProcStep}).\mathsf{CausalProof}(a,e,b)$ \\
& $\mathsf{FrozenStatus}(v, s)$ & 凭证 $v$ 处于冻结状态 $s$ \\
& $\mathsf{QualPass}(b)$ & 批次 $b$ 合格证明 \\
& $\mathsf{CausalResolve}(a, p_{old}, p_{new})$ & 因果解决:动作 $a$ 解决了旧证明 $p_{old}$,产生新证明 $p_{new}$ \\
& $\mathsf{Contrib}(a, f)$ & 因果贡献度:$(\sigma \vdash f) \land (\sigma \setminus \{a\} \nvdash f)$ \\
\midrule
\multirow{5}{*}{\textbf{事件与约束类型}}
& $\mathsf{FailEvt}$ & 失效事件类型:$\Sigma(b:\mathsf{BatchID}).\Sigma(err:\mathsf{ErrorCode}).\Sigma(t:\mathsf{Time}).\mathsf{Proof}(t \in \text{Shift}_{QA})$ \\
& $\mathsf{ProcStep}$ & 生产过程类型:$\Sigma(b:\mathsf{BatchID}).\Sigma(m:\mathsf{Machine}).\Sigma(dur:\mathsf{Time} \times \mathsf{Time}).\mathsf{Proof}(\mathsf{InRoute}(b,m))$ \\
& $\mathsf{Anomaly}$ & 环境异常类型:$\Sigma(m:\mathsf{Machine}).\Sigma(p:\mathsf{Param}).\Sigma(v:\mathsf{Val}).\Sigma(t:\mathsf{Time})$ \\
& $\mathsf{WaterPressureAnomaly}$ & 水压异常类型:$\Pi(w_{val}:\mathbb{R}).(w_{val} < w_{\text{min}} \vee w_{val} > w_{\text{max}}) \to \mathsf{Anomaly}$ \\
& $\mathsf{CausalProof}(a, f)$ & 因果证明类型:$\Sigma(e:\mathsf{ProcStep}).\mathsf{Prop}_{causal}(a,e,f)$ \\
\midrule
\multirow{4}{*}{\textbf{报告和凭证类型}}
& $\mathsf{RootCauseReport}$ & 根因报告:$\Sigma(f:\mathsf{FailEvt}).\Sigma(a:\mathsf{Anomaly}).\mathsf{CausalProof}(a,f)$ \\
& $\mathsf{AuditVoucher}(s, b)$ & 待审计凭证:$\Sigma(v:\mathsf{Voucher}).(\mathsf{QualIssue}(b) \to \mathsf{FrozenStatus}(v,s))$ \\
& $\mathsf{NormalVoucher}(s, b)$ & 正常凭证类型 \\
& $\mathsf{QualRectified}(b, p_{old})$ & 整改证明类型:$\Sigma(a:\mathsf{Action}).\Sigma(p_{new}:\mathsf{QualPass}(b)).\mathsf{CausalResolve}(a,p_{old},p_{new})$ \\
\bottomrule
\end{tabular}
\end{table}

\begin{table}[htbp]
\centering
\caption{KOS-TL应用示例中的项(Terms)汇总表}
\label{tab:terms-summary}
\small
\begin{tabular}{>{\raggedright\arraybackslash}p{2cm} >{\raggedright\arraybackslash}p{3.5cm} >{\raggedright\arraybackslash}p{7cm}}
\toprule
\textbf{类别} & \textbf{名称} & \textbf{定义/说明} \\
\midrule
\multirow{4}{*}{\textbf{事件实例}}
& $f_{fail}$ & 失效事件实例:$\langle \text{"B2310"}, \text{"HARD\_ERR"}, \text{10:00} \rangle$ \\
& $f_0$ & 失效事件实例:$\text{mkFailure}(\text{Batch\_202310-01}, \text{Hardness\_Issue}, 10:00, \pi_{QA\_Sign})$ \\
& $a_{temp}$ & 温度异常实例 \\
& $a_{volt}$ & 电压异常实例:$\text{mkAnomaly}(\text{M\_03}, \text{Voltage\_Drop}, 08:15, \pi_{iot\_hash})$ \\
\midrule
\multirow{1}{*}{\textbf{过程实例}}
& $e_{proc}$ & 生产记录项:$\text{mkStep}(\text{Batch\_202310-01}, \text{M\_03}, \langle 08:00, 09:30 \rangle, \pi_{route})$ \\
\midrule
\multirow{12}{*}{\textbf{证明项}}
& $prf_{causal}$ & 因果证明项(自动校验时序约束) \\
& $p_{time}$ & 时序一致性证明项:$(a.t \in e.dur \land e.dur.end < f.t)$ \\
& $p_{space}$ & 空间一致性证明项:$(a.m = e.m)$ \\
& $p_{batch}$ & 批次一致性证明项:$(e.b = f.b)$ \\
& $p_{qual}$ & 质量证明项:$\mathsf{QualIssue}(b)$ 的证明 \\
& $p_{old}$ & 旧异常证明项 \\
& $p_{new}$ & 新合格证明项 \\
& $p_{joint}$ & 联合证明项(电压+水压双重异常) \\
& $p_{rect}$ & 整改证明项 \\
& $\pi_{QA\_Sign}$ & QA签名证明(质检记录真实性背书) \\
& $\pi_{route}$ & 路径证明:$\mathsf{Proof}(\mathsf{InRoute}(b,m))$ \\
& $\pi_{iot\_hash}$ & IoT哈希证明 \\
& $\pi_{final}$ & 最终证明项(特化的因果证明函数) \\
\midrule
\multirow{1}{*}{\textbf{因果链}}
& $r$ & 因果链:$\langle f_{fail}, a_{temp}, prf_{causal} \rangle$ \\
\midrule
\multirow{6}{*}{\textbf{构造函数}}
& $\text{mkFailure}$ & 失效事件构造函数 \\
& $\text{mkStep}$ & 生产过程步骤构造函数 \\
& $\text{mkAnomaly}$ & 异常构造函数 \\
& $\mathsf{mkCausalProof}$ & 因果证明构造函数:$\Pi(a:\mathsf{Anomaly}).\Pi(f:\mathsf{FailEvt}).\Pi(e:\mathsf{ProcStep}).\mathsf{TimeOK}(a,e,f) \to \mathsf{SpaceOK}(a,e) \to \mathsf{BatchOK}(e,f) \to \mathsf{CausalProof}(a,f)$ \\
& $\mathsf{mkAuditVoucher}$ & 审计凭证构造函数 \\
& $\mathsf{mkUnlockProof}$ & 解锁证明构造函数:$\Pi(v_{aud}:\mathsf{AuditVoucher}).\Pi(p_{rect}:\mathsf{QualRectified}).\mathsf{NormalVoucher}$ \\
\bottomrule
\end{tabular}
\end{table}

从依赖类型论的视角,我们可以将KOS-TL应用示例中的所有类型和项组织成一个知识图谱。该图谱展示了类型之间的依赖关系(通过$\Sigma$和$\Pi$类型构造)、项到类型的归属关系,以及构造函数到类型的映射关系。

\begin{figure}[htbp]
\centering
\caption{KOS-TL类型与项的知识图谱(依赖类型论视角)}
\label{fig:knowledge-graph}
\resizebox{\textwidth}{!}{%
\begin{tikzpicture}[
    node distance=1.5cm and 2cm,
    atom/.style={ellipse, draw=blue!50, fill=blue!10, thick, minimum width=1.5cm, minimum height=0.8cm, text centered, font=\footnotesize},
    composite/.style={rectangle, draw=green!50, fill=green!10, thick, minimum width=2cm, minimum height=1cm, text centered, font=\footnotesize},
    predicate/.style={diamond, draw=orange!50, fill=orange!10, thick, minimum width=1.5cm, minimum height=1cm, text centered, font=\footnotesize},
    constructor/.style={trapezium, trapezium left angle=70, trapezium right angle=70, draw=purple!50, fill=purple!10, thick, minimum width=2cm, minimum height=0.8cm, text centered, font=\footnotesize},
    term/.style={circle, draw=red!50, fill=red!10, thick, minimum size=0.8cm, text centered, font=\tiny},
    sigma/.style={->, thick, blue!70},
    pi/.style={->, thick, green!70, dashed},
    instance/.style={->, thick, red!50},
    depends/.style={->, thick, orange!70, bend right=15}
]

% 基础原子类型层
\node[atom] (BatchID) {BatchID};
\node[atom, right=of BatchID] (Machine) {Machine};
\node[atom, right=of Machine] (Time) {Time};
\node[atom, below=0.8cm of BatchID] (ErrorCode) {ErrorCode};
\node[atom, right=of ErrorCode] (Param) {Param};
\node[atom, right=of Param] (Val) {Val};
\node[atom, below=0.8cm of ErrorCode] (Voucher) {Voucher};
\node[atom, right=of Voucher] (Action) {Action};

% 复合类型层
\node[composite, above=2cm of BatchID] (FailEvt) {FailEvt};
\node[composite, right=of FailEvt] (ProcStep) {ProcStep};
\node[composite, right=of ProcStep] (Anomaly) {Anomaly};
\node[composite, above=1.5cm of FailEvt] (CausalProof) {CausalProof};
\node[composite, right=of CausalProof] (RootCauseReport) {RootCauseReport};
\node[composite, above=1.5cm of RootCauseReport] (AuditVoucher) {AuditVoucher};
\node[composite, right=of AuditVoucher] (QualRectified) {QualRectified};
\node[composite, below=1cm of Anomaly] (WaterPressureAnomaly) {WaterPressureAnomaly};

% 谓词类型层
\node[predicate, above=3cm of Machine] (InRoute) {InRoute};
\node[predicate, right=of InRoute] (TimeOK) {TimeOK};
\node[predicate, right=of TimeOK] (SpaceOK) {SpaceOK};
\node[predicate, above=1cm of InRoute] (BatchOK) {BatchOK};
\node[predicate, right=of BatchOK] (PropCausal) {Prop$_{causal}$};
\node[predicate, above=1cm of PropCausal] (QualIssue) {QualIssue};
\node[predicate, right=of QualIssue] (CausalResolve) {CausalResolve};

% 构造函数层
\node[constructor, below=3cm of BatchID] (mkFailure) {mkFailure};
\node[constructor, right=of mkFailure] (mkStep) {mkStep};
\node[constructor, right=of mkStep] (mkAnomaly) {mkAnomaly};
\node[constructor, below=1cm of mkFailure] (mkCausalProof) {mkCausalProof};
\node[constructor, right=of mkCausalProof] (mkAuditVoucher) {mkAuditVoucher};
\node[constructor, right=of mkAuditVoucher] (mkUnlockProof) {mkUnlockProof};

% 项实例层
\node[term, below=2cm of mkFailure] (f0) {$f_0$};
\node[term, right=of f0] (eproc) {$e_{proc}$};
\node[term, right=of eproc] (avolt) {$a_{volt}$};
\node[term, below=1cm of f0] (prfcausal) {$prf_{causal}$};
\node[term, right=of prfcausal] (ptime) {$p_{time}$};
\node[term, right=of ptime] (pspace) {$p_{space}$};
\node[term, below=1cm of prfcausal] (pqual) {$p_{qual}$};
\node[term, right=of pqual] (prect) {$p_{rect}$};

% Σ类型依赖关系(蓝色实线)
\draw[sigma] (FailEvt) -- (BatchID);
\draw[sigma] (FailEvt) -- (ErrorCode);
\draw[sigma] (FailEvt) -- (Time);
\draw[sigma] (ProcStep) -- (BatchID);
\draw[sigma] (ProcStep) -- (Machine);
\draw[sigma] (ProcStep) -- (Time);
\draw[sigma] (Anomaly) -- (Machine);
\draw[sigma] (Anomaly) -- (Param);
\draw[sigma] (Anomaly) -- (Val);
\draw[sigma] (Anomaly) -- (Time);
\draw[sigma] (CausalProof) -- (ProcStep);
\draw[sigma] (RootCauseReport) -- (FailEvt);
\draw[sigma] (RootCauseReport) -- (Anomaly);
\draw[sigma] (RootCauseReport) -- (CausalProof);
\draw[sigma] (AuditVoucher) -- (Voucher);
\draw[sigma] (QualRectified) -- (Action);
\draw[sigma] (WaterPressureAnomaly) -- (Anomaly);

% Π类型依赖关系(绿色虚线)
\draw[pi] (mkCausalProof) to[out=90, in=-90] (Anomaly);
\draw[pi] (mkCausalProof) to[out=90, in=-90] (FailEvt);
\draw[pi] (mkCausalProof) to[out=90, in=-90] (ProcStep);
\draw[pi] (mkCausalProof) to[out=90, in=-90] (CausalProof);
\draw[pi] (mkUnlockProof) to[out=90, in=-90] (AuditVoucher);
\draw[pi] (mkUnlockProof) to[out=90, in=-90] (QualRectified);
\draw[pi] (WaterPressureAnomaly) to[out=180, in=0] (Anomaly);

% 谓词依赖关系(橙色)
\draw[depends] (ProcStep) -- (InRoute);
\draw[depends] (CausalProof) -- (PropCausal);
\draw[depends] (PropCausal) -- (TimeOK);
\draw[depends] (PropCausal) -- (SpaceOK);
\draw[depends] (PropCausal) -- (BatchOK);
\draw[depends] (QualIssue) -- (Anomaly);
\draw[depends] (QualIssue) -- (ProcStep);
\draw[depends] (QualIssue) -- (CausalProof);
\draw[depends] (AuditVoucher) -- (QualIssue);
\draw[depends] (QualRectified) -- (CausalResolve);

% 项到类型的归属关系(红色)
\draw[instance] (f0) -- (FailEvt);
\draw[instance] (eproc) -- (ProcStep);
\draw[instance] (avolt) -- (Anomaly);
\draw[instance] (prfcausal) -- (CausalProof);
\draw[instance] (ptime) -- (TimeOK);
\draw[instance] (pspace) -- (SpaceOK);
\draw[instance] (pqual) -- (QualIssue);
\draw[instance] (prect) -- (QualRectified);

% 构造函数到类型的映射
\draw[instance] (mkFailure) -- (FailEvt);
\draw[instance] (mkStep) -- (ProcStep);
\draw[instance] (mkAnomaly) -- (Anomaly);
\draw[instance] (mkCausalProof) -- (CausalProof);
\draw[instance] (mkAuditVoucher) -- (AuditVoucher);
\draw[instance] (mkUnlockProof) -- (QualRectified);

% 图例
\node[below=0.5cm of Action, anchor=north west, font=\tiny] (legend) {\textbf{图例:}};
\node[below=0.2cm of legend, anchor=west, font=\tiny] (legend1) {\textcolor{blue!70}{$\Sigma$类型依赖} \textcolor{green!70}{$\Pi$类型依赖} \textcolor{orange!70}{谓词依赖} \textcolor{red!50}{实例化关系}};

\end{tikzpicture}%
}
\end{figure}

图\ref{fig:knowledge-graph}展示了KOS-TL应用示例中的知识图谱结构,从依赖类型论的视角揭示了以下关系:

\begin{enumerate}
\item \textbf{$\Sigma$类型依赖(蓝色实线)}:表示依存和类型,其中复合类型依赖于其组成部分的基础类型。例如,$\mathsf{FailEvt}$依赖于$\mathsf{BatchID}$、$\mathsf{ErrorCode}$和$\mathsf{Time}$。

\item \textbf{$\Pi$类型依赖(绿色虚线)}:表示依存积类型(函数类型),构造函数通过$\Pi$类型定义其类型签名。例如,$\mathsf{mkCausalProof}$的类型签名是$\Pi(a:\mathsf{Anomaly}).\Pi(f:\mathsf{FailEvt}).\Pi(e:\mathsf{ProcStep}).\ldots \to \mathsf{CausalProof}(a,f)$。

\item \textbf{谓词依赖(橙色曲线)}:表示类型对谓词的依赖关系。例如,$\mathsf{ProcStep}$依赖于谓词$\mathsf{InRoute}$来确保工艺路径的有效性。

\item \textbf{实例化关系(红色实线)}:表示项(terms)到类型的归属关系,以及构造函数到其构造类型的映射关系。
\end{enumerate}

该知识图谱清晰地展示了KOS-TL中“知识即类型,证明即程序”的核心思想:每个复合类型都通过$\Sigma$类型携带证明项,每个构造函数都通过$\Pi$类型定义其逻辑约束,从而实现了从数据到逻辑的完整映射。

\begin{table}[htbp]
\centering
\caption{描述逻辑 (DL) 与 KOS-TL 的根本差异}
\label{tab:dl-kostl-diff}
\begin{tabular}{l|l|l}
\hline
\textbf{维度} & \textbf{描述逻辑 (DL)} & \textbf{KOS-TL} \\ \hline
核心范式 & 静态本体一致性 & 动态操作语义 \\
逻辑性质 & 真值判定 (Truth) & 证明构造 (Proof) \\
时间处理 & 外部扩展 (Temporal DL) & 内生时间排序约束 \\
应用目标 & 知识描述与查询 & 知识操作系统内核 \\ \hline
\end{tabular}
\end{table}

\begin{quote}
\textit{“KOS-TL 并非用于简单地描述世界,而是用于运行一个以知识为内核的、逻辑闭环的操作系统。”}
\end{quote}

在 KOS-TL 中,知识的派生(Derivation)本质上就是一种高阶函数的构造与应用过程。

传统系统是“用硬编码的函数去处理数据”,而 KOS-TL 体现的是“数据驱动了逻辑项的合成,进而形成了具备推理能力的派生函数体”。我们可以从以下三个维度来理解这种“数据派生函数”的机制:
\begin{enumerate}

    \item 从“数据元组”到“证明项”的升维

    在 Runtime 层,原始数据(Batch\_202310-01, 10:00)只是被动的信息。但在精化(Elaboration)过程中,它被封装成了带有逻辑签名的 $\Sigma$-项。实质:这些项在 Kernel 层看来,不再是单纯的“值”,而是一个个小型的、待组合的函数片段。例如,$e_{proc}$ 携带的 $\pi_{route}$ 实际上是一个能够证明“路径合法性”的逻辑函数。

    \item 动态生成的“因果链”,即函数合成

    analyze 函数或 RootCauseReport 的构造过程,实际上是 Kernel 根据实时数据,动态地合成了(Compose)一个新的逻辑函数。

    当 Kernel 发现 $a_{volt}$ 和 $f_0$ 的数据匹配时,它并不是简单地把它们连起来,而是构造了一个新的 Lambda 项:

    $$\lambda (t) . \text{proof\_of\_overlap}(t, e_{proc}.dur)$$

    这个新生成的逻辑项 $\pi_{final}$ 实际上就是一个特化(Specialized)的函数。它专门用于处理“在这个特定时间、这台特定机器上发生的这一类硬度问题”。

    \item 三层架构中的“派生”分工

    轴承案例展示了函数是如何在不同层级间“流动”并被“物化”的(如表\ref{tab:layer-derived-behavior}所示)。

\begin{center}
\captionof{table}{层次与派生行为本质在轴承案例中的表现}
\label{tab:layer-derived-behavior}
\begin{tabular}{>{\centering\arraybackslash}p{1.5cm} >{\raggedright\arraybackslash}p{3cm} >{\raggedright\arraybackslash}p{6cm}}
\toprule
\textbf{层次} & \textbf{派生行为的本质} & \textbf{轴承案例中的表现} \\
\midrule
Core & 定义函数的“范畴” & 定义了 mkCausalProof 这个高阶函数模板。它规定了函数必须满足的输入输出类型。 \\
Runtime & 提供函数的“算子” & 从 SQL 表中提取具体数值,并将其精化为具有逻辑含义的原子项(如 $f_0, a_{volt}$)。 \\
Kernel & 实现函数的“演算” & 通过 Small-step 语义,将碎片化的数据项填充进 Core 层的模板中,物化出一个具体的因果证明函数 $R$。 \\
\bottomrule
\end{tabular}
\end{center}

    在传统软件中,如果你想增加一种“电压导致硬度不均”的追溯逻辑,你需要手动写一个 checkVoltage() 函数。但在 KOS-TL 中:你只需在 Core 层定义一般的因果原理(即 $t(a) < t(f)$ 且物理关联)。当 Batch\_202310-01 的特定电压数据进入系统时,Kernel 会利用这些数据\textbf{派生(Derive)}出一个针对该批次的、特化的证明函数实例。

    函数不是预先写死的,而是根据当前的系统状态 $\sigma$ 和捕获的事实 $f$ 动态推导出来的逻辑结果。总结这个例子完美诠释了 "Proof as Program" (证明即程序) 的哲学:推导出一个质量根因的证明,等同于派生出了一个能够解释该失效的逻辑函数。数据(电压、时间、批次)不再仅仅是被处理的对象,它们成为了构造这个“逻辑函数”的零部件。
\end{enumerate}


\chapter{总结}

\section{哲学范式:从“真理理论”转向“可执行规范”}
自弗雷格以来,传统逻辑(一阶逻辑、描述逻辑)的核心是静态真值($\mathcal{M} \models \varphi$),其目标是刻画世界“是什么样”。KOS-TL 明确拒绝了这种“单一模型中心论”,实现了逻辑角色的根本转变:

\begin{itemize}
\item \textbf{状态语义(State-based Semantics)}

语义的基本单位不再是永恒不变的模型,而是动态演化的状态序列($\sigma_0 \to \sigma_1 \to \dots$)。逻辑判断的中心从“命题是否为真”转向了“状态转移是否合法”。

\item \textbf{构造主义立场(Constructivism)}

继承 Martin-Löf 类型论的精髓,将“命题即类型,证明即程序”推广为“知识即类型,操作即程序,事件即构造子”。在 KOS-TL 中,知识的存在性由其是否可被构造决定,这使得系统在哲学上是自洽且透明的。

\item \textbf{知行合一}

逻辑不再仅仅用于描述,而是成为一种可执行的规范系统。它不仅刻画真理,更刻画了知识如何被操作、更新与执行,弥合了逻辑推理与物理操作之间的断层。
\end{itemize}

\section{逻辑特性:事件驱动与操作语义的融合}
KOS-TL 的本质创新在于将操作语义引入逻辑核心,使其具备了“计算即推理”的能力:

\begin{itemize}
\item \textbf{事件作为一等公民(Events as First-class Constructors)}

不同于描述逻辑中被动的事实记录,事件在 KOS-TL 中是连接理解与操作的枢纽。它是类型论中的构造子,规定了对象生成的合法机制,为系统提供了处理时间敏感性、因果关系和状态迁移的对象基础。

\item \textbf{操作性派生规则(Operational Derivation)}

派生规则不再是纯粹的真值保持,而是一种操作性规范。它定义了系统在特定上下文和时间约束下,如何“合法地”产生新语义标注(如风险预警)。这种规范性立场允许系统在不同场景下表现出不同的推理策略,具备极高的灵活性与可解释性。

\item \textbf{小步操作语义(Small-step Operational Semantics)}

逻辑判断的形式演变为 $\langle \Sigma, c \rangle \to \Sigma'$。这种微观层面的演化刻画,使得 KOS-TL 能够精准管理知识库的单调性与资源消耗,使其更接近“因果推理计算机”。
\end{itemize}

\section{系统能力:计算反射性驱动的自治与审计}
反射性(Reflexivity)在 KOS-TL 中从编程技巧提升到了形式化自省的高度:

\begin{itemize}
\item \textbf{计算即证明的“自证”}

基于柯里-霍华德同构,内核在执行每一步计算(规约)时,都会同步合成等价性证明项($\text{Id}$ 证明)。这意味着系统每走一步都会留下不可篡改的逻辑脚印,证明其行为符合预设公理。

\item \textbf{全路径逻辑审计(Auditability)}

由于存在反射机制,审计不再依赖外部文本日志,而是变成了一种实时数学校验。证明链的完备性直接决定了系统状态的合法性,强制实现了“透明治理”。

\item \textbf{逻辑自愈(Self-Inspection \& Healing)}

反射性允许内核“回看”决策路径,在遇到逻辑矛盾时能够定位公理冲突并触发自愈算子,为系统的自主运行提供了决策依据。
\end{itemize}

\section{工程范式:类型程序设计(TDD)的边界拓宽}
KOS-TL 将类型系统从“内存安全工具”升华为“系统自治公理”,为现代系统设计提供了核心启示:

\begin{itemize}
\item \textbf{从“类型安全”到“逻辑确定性”}

通过物理-逻辑双向精化类型,将物理定律和业务规则内生为类型的属性。这种“带有证据的类型”确保了非法状态在系统设计层面就是不可表示的。

\item \textbf{持久化依存类型存储}

打破了数据库作为原始字节堆填的现状,使存储成为类型系统的运行时延伸。利用类型的单调性约束管理数据生命周期,确保了知识演化的因果一致性。

\item \textbf{跨层级逻辑透镜(Refinement Lenses)}

通过严密的双向精化映射,解决了高层业务实体(如“合规转账”)与底层物理存储(如 SQL 条目)之间的断层,保证了每一步物理动作都忠实于高层逻辑意图。
\end{itemize}

\section{KOS-TL 的本质}
KOS-TL 是一个将直觉主义类型论的构造语义、操作语义与知识工程实践完美融合的逻辑系统。它不仅仅是一套算法,更是一套\textbf{“系统宪法”}。
它证明了:通过 依存类型($\Pi$ \& $\Sigma$)+ 事件构造子 + 反射性,可以构建出一种逻辑自洽、因果可追溯、且能与物理世界严密对齐的智能自治系统。它将证明从单纯的程序正确性校验,进化到了驱动复杂现实世界决策的核心动力。


\backmatter

% 参考文献(使用手动列表,避免 BibTeX 的 Buffer size exceeded;引用仍用 \cite{key})
\begin{thebibliography}{99}
\bibitem{Baader2003}
Baader, F., Calvanese, D., McGuinness, D., Nardi, D., \& Patel-Schneider, P. (Eds.). (2003). \textit{The description logic handbook: Theory, implementation and applications}. Cambridge University Press.

\bibitem{Chenpeng2026}
Chen, P. (2026). KOS-TL (Knowledge Operation System Type Logic): A Constructive Foundation for Executable Knowledge Systems. Technical Report.

\bibitem{Davidson1967}
Davidson, D. (1967). The logical form of action sentences. In N. Rescher (Ed.), \textit{The logic of decision and action} (pp. 81--95). University of Pittsburgh Press.

\bibitem{Girard1972}
Girard, J.-Y. (1972). Interprétation fonctionnelle et élimination des coupures de l'arithmétique d'ordre supérieur. Thèse de doctorat, Université Paris VII.

\bibitem{Girard1989}
Girard, J.-Y., Lafont, Y., \& Taylor, P. (1989). \textit{Proofs and types}. Cambridge University Press.

\bibitem{MartinLof1984}
Martin-Löf, P. (1984). \textit{Intuitionistic type theory}. Bibliopolis.

\bibitem{McCarthy1963}
McCarthy, J. (1963). Situations, actions and causal laws. Technical Report, Stanford University. (Reprinted in M. Minsky (Ed.), \textit{Semantic information processing}, MIT Press, 1968.)

\bibitem{Reiter2001}
Reiter, R. (2001). \textit{Knowledge in action: Logical foundations for specifying and implementing dynamical systems}. MIT Press.

% === 文献综述补充:类型论与相关体系核心文献 ===
\bibitem{Barendregt1992}
Barendregt, H. (1992). Lambda calculi with types. In S. Abramsky, D. Gabbay, \& T. Maibaum (Eds.), \textit{Handbook of logic in computer science} (Vol. 2, pp. 117--309). Oxford University Press.

\bibitem{Coquand1988}
Coquand, T., \& Huet, G. (1988). The calculus of constructions. \textit{Information and Computation}, \textbf{76}(2/3), 95--120.

\bibitem{Luo1994}
Luo, Z. (1994). \textit{Computation and reasoning: A type theory for computer science}. Oxford University Press.

\bibitem{Harper1993}
Harper, R., Honsell, F., \& Plotkin, G. (1993). A framework for defining logics. \textit{Journal of the ACM}, \textbf{40}(1), 143--184. (Preliminary version: LICS 1987.)

\bibitem{Pfenning1999}
Pfenning, F., \& Schürmann, C. (1999). System description: Twelf — A meta-logical framework for deductive systems. In H. Ganzinger (Ed.), \textit{CADE-16} (LNCS 1632, pp. 202--206). Springer.

\bibitem{Harper2016}
Harper, R. (2016). \textit{Practical foundations for programming languages} (2nd ed.). Cambridge University Press.

\bibitem{HoTTBook2013}
The Univalent Foundations Program. (2013). \textit{Homotopy type theory: Univalent foundations of mathematics}. Institute for Advanced Study. \url{https://homotopytypetheory.org/book}.

\bibitem{Voevodsky2014}
Voevodsky, V. (2014). Univalent foundations of mathematics. In \textit{Proceedings of the Logic in Computer Science (LICS) 2014} (pp. 1--2). IEEE.

\bibitem{Nanevski2008}
Nanevski, A., Morrisett, G., \& Birkedal, L. (2008). Hoare type theory, polymorphism and separation. \textit{Journal of Functional Programming}, \textbf{18}(5--6), 865--911.

\bibitem{Reynolds2002}
Reynolds, J. C. (2002). Separation logic: A logic for shared mutable data structures. In \textit{LICS 2002} (pp. 55--74). IEEE.

\bibitem{Girard1987}
Girard, J.-Y. (1987). Linear logic. \textit{Theoretical Computer Science}, \textbf{50}(1), 1--102.

\bibitem{Harel2000}
Harel, D., Kozen, D., \& Tiuryn, J. (2000). \textit{Dynamic logic}. MIT Press.

\bibitem{Fowler2005}
Fowler, M. (2005). Event sourcing. \url{https://martinfowler.com/eaaDev/EventSourcing.html}. (Retrieved as of 2024.)
\end{thebibliography}

\end{document}
